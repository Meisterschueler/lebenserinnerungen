\section{Fortsetzung des Studiums in Breslau}
Rektor und Senat der Breslauer Universität kamen den aus dem Felde zurückkehrenden Studenten entgegen insbesondere durch Errichtung von Zwischensemestern sowie Sonderlehrgängen und -vorlesungen für Kriegsteilnehmer. An den Aufnahmebedingungen für die Seminare -- unerlässlich für die Hauptfächer -- hielten jedoch die meisten Professoren nach wie vor fest. Ich musste also in Germanistik schleunigst Gotisch und Althochdeutsch zur Examensreife weiter betreiben, desgleichen Altfranzösisch, was ich autodidaktisch tat. Zu Ostern 1919 gelang mir die Aufnahme in das germanistische Seminar bei Prof. Siebs und schließlich auch in das romanische Seminar des gestrengen Prof. Appel. Er legte mir -- die Prüfung fand in seiner Villa im Scheitninger Park statt -- eine Seite aus einem höfischen Ritterroman Chrestien de Troyes', also in gebundener Sprache, vor; der Übersetzung ins Deutsche, die befriedigend verlief, folgten schwierigere grammatische Fragen, von denen ich zwei nicht beantworten konnte. Ergebnis: Wegen Versagens in der altfranzösischen Grammatik kann ich Sie nicht aufnehmen, zumal ich schon 16 Teilnehmer im Seminar habe und über diese Zahl ungern hinausgehe. Ich äußerte meine Bestürzung, es treffe mich als Kriegsteilnehmer, der 4,5 Studienjahre eingebüßt habe, sehr schwer und bat ihn, die Prüfung noch fortzusetzen. Er fragte mich, ob ich noch ein anderes romanisches Fach anzubieten habe; ich nannte Italienisch. Er ging darauf ein, fragte ob er mir eine Zeitung oder einen bestimmten Schriftsteller vorlegen solle. Ich erbat nach einigem Zögern Baccaccio. Er gab mir eine Seite aus dem Decamerone; ich las die halbe Seite -- er bezeichnete meine Aussprache als gut und die anschließende Übersetzung ins Deutsche als richtig. Die zweite Hälfte des Textes musste ich auf italienisch erzählen. Auch hiermit war Appel zufrieden und entschied, mich \enquote{aufgrund meiner gemeinromanischen Kenntnisse ausnahmsweise noch aufzunehmen}. Ich atmete auf.

Ablenkung vom Studium durch politische Betätigung lehnte ich ab. Im Januar 1919 fürchteten manche einen linkssozialistischen-kommunistischen Handstreich zur Verhinderung der bevorstehenden Wahl zur Nationalversammlung. Der stramm deutschnationale Professor Max Koch, Professor der deutschen und vergleichenden Literaturgeschichte, angeblich getaufter Jude, rief in der Uniform eines Majors die Studentenschaft zum bewaffneten Widerstand gegen einen Linksputsch auf und verteilte Infanteriegewehre und Karabiner, deren er ein stattliches Arsenal zur Verfügung hatte -- ich folgte seinem Rufe nicht. Als der nationale Studentenbund VDST durch meinen Klassenkameraden Mahling mich bat, meine Unterschrift unter eine Geburtstagsgratulation an Exkaiser Wilhelm~II. in Doorn zu setzen, lehnte ich ab. Die \enquote{Alte Breslauer Burschenschaft der Raczeks}, die ein schönes Haus an der Oder unweit der Gneisenaubrücke besaßen, luden mich zu einem Bierabend im berühmten Breslauer Ratskeller ein -- ich weiß nicht, wie sie auf mich gefallen waren. Ich folgte der Einladung und ließ mich -- natürlich ohne den gewünschten Erfolg -- keilen. Wie kindlich schien mir Gespräche und Ansichten dieser jungen Leute, die, wenn überhaupt, meist nur wenige Wochen in der Heimat Wehrdienst geleistet hatten! Man sprach von Paukboden, Bestimmungsmensur, Konvent, Stiftungsfesten u.ä. Ich kam mir vor wie ein Student, der in einen Kreis von Tertianern geraten war und verabschiedete mich zeitig.

Von meinen Konabiturienten waren mehr als die Hälfte gefallen, darunter Hansen und Anders. Ich sah Menzel, Mahling und Richter wieder; Menzel intelligent aber willensschwach, hatte das Theologiestudium aufgegeben und versuchte sein Heil, ohne zu Stuhl zu kommen, in Jura und Zahnmedizin. Nach Jahren traf ich ihn wieder als Verkäufer in der Breslauer Buchhandlung Max \& Co. Von befreundeten Breslauer Kommilitonen sah ich wieder den Mediziner Willy Kauf, den Germanistik-Historiker Albert Engmann, den Kunstmaler und Graphiker Leo Loch und den Philosophen und Germanisten Eduard Rogier, mit dem mich eine längere Freundschaft verband.

Ich wohnte zunächst, wie schon kurze Zeit 1913 in der Martinistraße neben dem winzigen Martinikirchlein, im Schatten der mächtigen gotischen Backsteinkirche, der Kreuzkirche, einer Doppelkirche, in deren oberem Chor sich der wundervolle hochgotische Sarkophag Heinrichs~IV., des Minnesängers befindet; hochoben am Turm prangte noch aus der Zeit der polnischen Piasten ein erzener polnischer Adler. Die Martinistraße mündete am geschmackvoll barocken Stiftsgebäude für Damen des schlesischen katholischen Adels in die schöne klassizistische Domstraße, deren Abschluss die Vorderfront des gotischen Domes und rechts das klassizistische Palais des Fürstbischofs von Breslau bildeten. Vom kleinen Park des Palais aus, der gelegentlich für die Öffentlichkeit zugängig war, bot sich über die Oder hinweg ein herrlicher Blick auf die turmreiche Altstadt Breslaus, die englische Reisende des 18. Jahrhunderts als sehr geräuschvoll und \enquote{dirty but well churched} bezeichneten. Hier beim Fürstbischof Kopp pflegte Wilhelm~II. bei seinen Besuchen Breslaus zu übernachten, sofern er nicht beim vornehmen Breslauer Kurassierregiment abstieg. Dagegen ignorierte er stets den tüchtigen demokratischen Oberbürgermeister Dr. Wagner.

Man traf sich wieder wie vor dem Kriege, beim Mittag- oder auch beim Abendessen in den behaglich-barocken Räumen des Studentenheimes gegenüber der Universitätsbibliothek auf der Sandinsel mit der gleichnamigen gotischen Kirche. Das Gebäude des Studentenheims gehörte ursprünglich zum angrenzenden St. Annen-Kirchlein, das man der kleinen ca. 200 Mann starken polnischen Kolonie Breslaus eingeräumt hatte. Das Essen, das in der ersten Nachkriegszeit nur gegen Lebensmittelmarken gereicht wurde, war kalorienarm, Fleisch war selten, Dörrgemüse spielte eine große Rolle, im Volksmunde in der Kriegszeit \enquote{Drahtverhau} genannt.

Im Juni zog ich um in den Villen-Vorort Kleinburg, wo ich in der Luxus-Etagenwohnung des begüterten Direktors einer Feuer-Versicherungs\-ge\-sell\-schaft ein Gratis-Zimmerchen innehatte: dadurch, dass er mich aufgenommen hatte, entging er der Zwangsbelegung durch die links-sozialistische (damals un\-abh\-ängig-sozialistische) städtische Wohnungskommision. Dort ließ es sich gut arbeiten, es war sehr ruhig, nur hin und wieder hörte man das Auto eines Kriegsgewinnlers. Die Elektrischen verkehrten auf fast allen Vorkriegsstrecken. Omnibusse gab's damals noch nicht. Der Nachteil Kleinburgs machte sich bei Straßenbahnerstreiks, die damals erstmalig auftraten, bemerkbar: ich benötigte bis zur Universität zu Fuß 45 Minuten.

Meinen Lebensunterhalt bestritt ich mit meiner in der Kriegszeit im wesentlichen aufgesparten Offizierslöhnung, die über \num{6000} Mark ausmachte. Auch die 600 Mark während meiner Schulzeit gesparten Gelder aus Nachhilfeunterricht betrugen nun etwa 800 Mark. Die Gelder trug ich auf die Deutsche Bank und legte sie großenteils in Aktien an, in der Hoffnung, auf diese Weise dem Währungsverfall begegnen und womöglich noch Realgewinne erzielen zu können.

Nach dem Tode meines Vaters mussten wir den Haushalt auflösen. Ich veranstaltete eine Auktion, bei der ich selbst der Auktionär war und hatte das Glück, dass ein im Nachbardorf Tschirne gebürtiger, in Berlin verankerter Ankäufer auftrat, der die aus Bunzlau, Liegnitz und Görlitz gekommenen Händler stets überbot, so dass wir etwas über \num{10000} Mark vereinnahmten, was -- trotz der inflationsbedingten Geldentwertung -- damals immerhin dem Jahreseinkommen eines Studienrates entsprach. 

Mit diesen Mitteln konnte Ella sich ihrem Wunsch gemäß in Darmstadt als Krankenschwester ausbilden lassen und anschließend sich eine Beschäftigung in der höchst modernen und sehr kultivierten Odenwaldschule unter Geheb, dem Nachfolger, leisten. Glücklicherweise hatten meine Eltern einen Teil des kleinen Vermögens, das meine Mutter in die Ehe eingebracht hatte, vor dem Kriege -- sage und schreibe auf Anraten einer Pariser Bank -- in einer 6\%igen mexikanischen Staatsanleihe angelegt, die meine Mutter jetzt, Frühjahr 1920, zum 7-fachen Preise des Nennwertes verkaufte. Auch diesen Betrag legte ich in Aktien und z.T. -- törichterweise -- in ausländischen Banknoten an. Meine Mutter ließ sich im schönen Dorfe Klitschdorf a. Queis, dem Sitz des Fürsten Solms / Barath nieder. Später, 1922 f, nahm ich nicht unerhebliche Anleihen bei einem Gastwirt und einem Mühlenbesitzer auf, spekulierte mit ihm an der Breslauer Börse mit (Schein-)Erfolgen und zahlte das aufgenommene Geld nach der Deflation schrittweise zum Dollarwert des Aufnahmetages zurück. Nach dem damaligen höchst ungerechten Gesetz \enquote{Mark gleich Mark}, das einen Stinnes zum Milliardär machte, wäre ich dazu nicht verpflichtet gewesen.

\section{Berlin -- Zwischensemester und Kapp-Putsch}
\marginpar{408}
Während des Zwischensemesters im Frühherbst 1919 lernte ich Dr. Hede Drescher kennen, die schon mit einer mittelhochdeutschen Arbeit bei Siebs promoviert worden war und die sich jetzt auf das Staatsexamen in den Hauptfächern Deutsch und Englisch vorbereitete. Uns verband bald eine engere Freundschaft. Sie regte mich zu häufigen gemeinsamen Theaterbesuch an. Im Herbst zog ich nach der zentraler gelegenen Monhauptstraße, in die Wohnung des Ehepaares Prause (später Dr. phil. Prause) mit dem ich von 1921 ab in freundschaftlichem Verhältnis stand. Von Prof. Siebs hatte ich ein Thema für eine Dissertation angenommen. Es lautete: Einflüsse der niederdeutschen Mundart in der Sprache Gustav Frenssens.

Ich sammelte -- soweit mir hierfür Zeit verblieb -- Material aus sämtlichen Schriften Frenssens, erhielt in einigen schwierigen Fragen von Frenssen selbst brieflich Auskunft, erhielt meine Examensarbeit für Deutsch als Hauptfach aus diesem Stoffgebiet, arbeitete auch von 1922-24 hauptsächlich in den \enquote{Großen Ferien} weiter an der Dissertation, schloss sie aber trotz des Zuredens Siebs' nicht ab, da sich meine Interessen stark der russischen Sprache und Literatur zuwandten und ich mich außerdem veranlasst sah, zur Verbesserung meiner Qualifikation im dritten Unterrichtsfach -- Englisch -- mich auf einen Studienaufenthalt in England vorzubereiten.

Eines Morgens im Winter 1920 verließ ich wie gewöhnlich das Haus, um in die nahe Universitätsbibliothek zu gehen. An der Straßenecke Monhaupt-Sternstraße klebte ein kleiner weißer schwarz bedruckter Zettel mit den schlichten Worten: \enquote{Wer auf der Straße stehen bleibt, läuft Gefahr, beschossen zu werden. Unterschrift: 5. Marinebrigade.} Beim Weitergehen hörte ich einzelne MG-Stöße. Was ist in der Nacht geschehen? Die Unzufriedenheit war allgemein. Streiks, Schlamperei und Angestellteninflation in der Verwaltung, schwarzer Markt und Schiebertum blühten, kommunistische Putschversuche in Thüringen, Bayern u.a., politische Morde von Rechtsextremisten\dots ich war inzwischen in die Stadtmitte gelangt. An allen vier Ecken des \enquote{Ringes}, des großen Rathausplatzes, die gleichzeitig wichtige Straßenkreuzungen bildeten, standen mit Stacheldraht eingeigelte Marineinfanteristen im Stahlhelm um ein oder zwei Maschinengewehre, hin und wieder gelegentlich ein Warnschuss über die Köpfe der Schaulustigen feuernd. Es verkehrte keine Straßenbahn, am frühen Morgen war als Gegenmaßnahme gegen den Putsch der Rechtsextremisten und Großagrarier unter Führung Kapp's der Generalstreik ausgerufen und allgemein befolgt worden, einschließlich der gesamten Angestellten- und Beamtenschaft der Verwaltung bis hin in die höchsten Spitzen. Allen voran hatte Oberbürgermeister Dr. Wagner, einer der Spitzenvertreter der demokratischen Partei, jegliche Mitarbeit mit den Putschisten abgelehnt. Der gesamte Eisenbahnverkehr ruhte, bis auf die allerwichtigsten Milch- und Kartoffeltransporte. Äußerste Zwangsmaßnahmen wie später die Faschisten und Nationalsozialisten -- Erschießungen der Resistenten -- wagte Kapp nicht -- und so musste dieser Putschversuch nach wenigen Tagen zusammenbrechen. Die Regierung Ebert-Scheidemann kehrte aus Süddeutschland nach Berlin zurück.

Auch die Kompläsant-unterwürfige Erfüllungbereitschaft Erzbergers, des Leiters der deutschen Friedensabordnung gegenüber den wahnsinnig überspannten \enquote{Reparations}forderungen der Alliierten, besonders Frankreichs, entfachte Empörung, nicht nur in den Kreisen der Rechten.

Eines Morgens hallte die Mohnhauptstraße wieder von den Rufen: Extrablatt, Extrablatt, Erzberger ermordet! Im Nachbarhaus öffnete sich das Fenster eines Majors a.D.: \enquote{Bengel, brüll nicht so -- Gott sei Dank, dass das Schwein tot ist!}

Aus jener Zeit stammte folgender Witz: ein Amerikaner steht in Breslau auf dem Ring und fragt einen Bürger auf das Rathaus zeigend: \enquote{Uas ein Haus ist das?} \enquote{Das Rathaus} \enquote{Uas machen in die Haus?} \enquote{Man arbeitet.}\enquote{Uieviel arbeiten in die Haus?} \enquote{Ungefähr die Hälfte.}

Um Ostern herum meldete ich mich zum wissenschaftlichen Staatsexamen. Ende Juli war der Termin für die Ablieferung der damals handschriftlich abzufassenden schriftlichen Prüfungsarbeit. Ich schaffte die Reinschrift nicht ganz, 70 Seiten waren fertig, 15 fehlten noch, obwohl mein Freund Rogier und Hede Drescher am letzten Abend zur Seite standen (Durchsicht der fertigen DIN A4-Bögen und mit Erfrischungen). Um 4 Uhr morgens konnte ich nicht mehr weiter. Nach ein paar Stunden Schlaf begab ich mich zur maßgeblichen Behörde, dem Provinzialschulkollegium am Neumarkt, erklärte dem zuständigen alten Bürodiener, dass ich die Arbeit erst mit 24 Stunden Verspätung abliefern könne und drückte ihm knisternd die Hand, worauf er, wie immer in ähnlichen Fällen sagte: \enquote{Geht in Ordnung, ich spreche mit dem Geheimrat!}

Im Oktober nötigte ich dieses alte Faktum erneut meinetwegen \enquote{mit dem Geheimrat, d.h. dem Vorsitzenden des wissenschaftlichen Prüfungsamtes zu sprechen}, um den Plan für die mündliche Prüfung, die für alle drei Fächer an einem einzigen Tag stattfinden sollte, zeitlich um 2-3 Wochen auseinanderzulegen. Die Sache gelang nach Wunsch. Es begann mit Englisch, meinem schwächsten Fach, hatte ich doch Englisch nicht in der Schule sondern als Autodidakt nach Toussaint-Langenscheidt gelernt, Lektoren nur wenig genutzt und keinen Studienaufenthalt in England genommen. Meine Klausurübersetzung aus dem Deutschen ins Englische \enquote{weise u.a. einige sehr archaische Formen der unregelmäßigen Verben vor}, sagte leicht fragend-vorwurfsvoll der Prüfer zu Beginn; ich erwiderte, dass sei auf meine Shakespeare-Lektüre im Urtext zurückzuführen. Ich fand damit Verständnis, das Endergebnis war \enquote{gut}.

Als Klausurarbeit in Germanistik hatte ich im Arbeitszimmer Siebs' eine längere Stelle aus Hans Sachs sprachlich zu interpretieren. Die mündliche Prüfung erstreckte sich auf Lesen und Übersetzen a) eines gotischen, b) eines althochdeutschen Textes mit anschließenden grammatischen Fragen, auch aus dem mittelhochdeutschen Sprachgebiet. Die literaturgeschichtliche Prüfung ging nur wenig über die von mir angegebenen Schriftsteller: Opitz, Klopstock, Goethe, Eichendorff hinaus und endete mit dem Vortrag eines auswendig gelernten Gedichtes nach meiner Wahl. Ergebnis: mit Auszeichnung.

Ein gutes Resultat ergab auch die Prüfung im 2. Hauptfach -- Französisch. Prof. Appel meinte anerkennend, meine Klausur weise französisches Stilgefühl auf. Meine altfranz. Sprachkenntnisse waren jetzt gut, dann hatte ich ein längeres modernes Gedicht zu lesen und zu übersetzen und Fragen über franz. Metrik zu beantworten. Die weitere Prüfung beschränkte sich auf eine freie Unterhaltung über Molière und Voltaire, wobei Appel Wert auf mein persönliches Urteil legte, was seine Zustimmung fand.

Zusammenfassendes Endergebnis der wissenschaftlichen Prüfung lautete: \enquote{gut} bestanden.

\section{Referendariat}
Als Kriegsteilnehmer hatte ich die Vergünstigung, die Staatsprüfung in Philosophie -- damals obligatorisch für alle \enquote{Anwärter des höheren Schuldienstes} -- um maximal 1 Jahr zu verschieben. Dies hatte den Vorteil, dass ich bereits am 1.1.1921 den für Kriegsteilnehmer auf ein Jahr begrenzten Vorbereitungsdienst in der Schule beginnen konnte, jedoch den Nachteil, dass sich im Referendarjahr sehr vieles zusammendrängte: philosophische und pädagogische Theorie, pädagogische Praxis und zwei umfangreiche Hausarbeiten, eine philosophische und eine pädagogische, daneben ein pädagogisches und ein philosophisches Seminar, abgesehen von den Fachseminaren in Deutsch und neueren Sprachen. Und die Lehrproben mit ihren detaillierten Plänen, die in mehrfacher Ausfertigung vorgelegt werden mussten. Ich glaube nicht, dass von den Studienreferendaren damals viele mit einer 12-stündigen täglichen Arbeitszeit ausgekommen sind -- ich jedenfalls in der Regel nicht.

Mein Philosophielehrer war seit 1919 ausschließlich der Neukantianer Prof. Hönigswald, ein scharfer, kritischer Denker, den besonders die Problemgeschichte der Philosophie, nicht die Geschichte der Philosophen interessierte, der Herkunft nach ungarischer Jude und Dr. med. Aus Zeitmangel musste ich auf kunstgeschichtliche Vorlesungen und das sehr interessante Seminar von Professor Wilhelm Pinder verzichten. (Zu Beginn der ersten Sitzung -- wir waren 7 Mann -- leuchtete der Witz auf: Gehen Sie an die Kunst nicht heran wie ein deutscher Oberlehrer! -- ?? -- Der legt sich bei der Betrachtung von Kunstgegenständen zwei Fragen vor: 1. Was hat der Künstler darstellen wollen? 2. Inwiefern hat er sein Ziel nicht erreicht?)

Zum 1. Januar 1921 wurde ich zur pädagogischen Ausbildung an die Realschule II überwiesen, deren oberste Klasse U II\footnote{Untersekunda, d.h. Klasse 10, siehe Anhang \ref{tab:jahrgangsstufen}} war. Der Direktor, der die allgemeine Pädagogik behandelte, erschien mir sehr pedantisch und unbedeutend. Mein Fachausbilder war Studienrat Müthing, den ich im Verdacht hatte, dass er sich vom Militärdienst mit Hilfe von \enquote{Rheumatismus} -- er war damals noch nicht 30 Jahre alt -- gedrückt hatte. Er wollte offenbar mit seinen Stiftungsprotokollen bei der Behörde Eindruck machen; zu diesem Zweck ließ er zunächst eine vollständige Niederschrift ausführen, korrigierte und ergänzte sie und ließ sie dann noch einmal als Reinschrift anfertigen. Als er mir auch eine Anzahl \enquote{stilistischer Verbesserungen}, die ich z.T. für Verböserungen hielt, mit roter Tinte für die Reinschrift vorschrieb, lehnte ich mich, der ich in den Kriegsjahren an Sicherheit und Selbstbewusstsein gewonnen hatte, dagegen auf, indem ich ihn vor den Kollegen ersuchte, mir zu erläutern, inwiefern seine stilistischen Bemängelungen Verbesserungen seien, worauf er erwiderte: \enquote{ich werde Ihre Entwürfe überhaupt nicht mehr verbessern}, worauf ich sagte: \enquote{Herr Kollege, auf Ihre stilistischen Verbesserungen verzichte ich gern.}

Das war natürlich meinerseits eine schlimme Entgleisung: 1. meinen Ausbilder und quasi Vorgesetzten mit \enquote{Herr Kollege} -- wie er mich -- anzureden 2. die Brüskierung des Dr. Müthing Coram Collegam! Ich zog die Konsequenzen, begab mich zum \ac{psk} und bat um meine Überweisung an ein anderes Ausbildungsseminar und schlug das Realgymnasium am Zwinger vor, von dem ich Gutes gehört hatte. Ich begründete meine Bitte mit unzureichender Förderung an der Realschule II, zumal dort die ganze Oberstufe (O II - I\textsuperscript{a})\footnote{Obersekunda bis Prima Abitur} fehle. Mein Wunsch wurde sofort erfüllt.

Da der Winter 1920-21 früh einsetzte und sehr streng war, wurde der Schulbetrieb einige Wochen wegen Kohlenmangels eingestellt. Ende März machte ich mit meinem Freund Rogier eine mehrtägige Wanderung durch die Glatz-Reinerzer Berge, von wo wir jedoch durch wieder einsetzendes Winterwetter verscheucht wurden.

Das Realgymnasium am Zwinger war eine Anstalt mit einer gewissen nationalen Tradition. In den ersten Tagen meines Dortseins wurde eine bronzene Gedenktafel für die im ersten Weltkrieg gefallenen Lehrer und Schüler der Anstalt eingeweiht. Auf ihr stand der lapidare lateinische Satz: Exoriare aliquis nostris ex ossibus ultor -- Möge aus unseren Gebeinen ein Rächer entstehen -- ein Satz, der von den Linksparteien und pazifistischen Organisationen angefochten wurde. Es bedurfte geschicktester dialektischer Klimmzüge, um die Entfernung der Tafel bzw. Tilgung des Spruches zu verhindern. Die Festrede hielt der Historiker und Germanist Dr. Jenthe, temperamentvoller und geistvoller Redner: er verlieh in ihr der Tatsache, dass die Bismarckbüste von der Morgensonne, die Gefallenentafel des verlorenen Großen Krieges jedoch von der untergehenden Sonne angestrahlt wurde, symbolische Bedeutung.

Anstalts- und Seminarleiter war Dr. Aust, ein gewandter, tüchtiger Mann, Alter Herr vom A.T.V (Akademischen Turnverein), wie die Mehrzahl der Breslauer Direktoren (Ich war seinerzeit aus dem Greifswalder A.T.V ausgetreten). An dieser Schule waren mehrere tüchtige Lehrer und Wissenschaftler tätig, wie vor allem der Indogermanist Nehring, der Philosoph und pädagogische Theoretiker Kynast, die beide nach einigen Jahren zu Universitätsprofessoren aufstiegen.

Als Tutoren fungierten für Deutsch der quicklebendige, nur allzu redefreudige Dr. Dohn und für Französisch der alte Durchschnittslehrer, damals noch mit dem Titel \enquote{Professor}, Wende. Einer von Dr. Austs Kernsätzen lautete: \enquote{Studieren Sie fleißig das empfohlene pädagogische und fachmethodische Schriftenmaterial, aber kritisch. Dann wird sich zeigen, ob Sie nur nachbeten können oder die Schule von Ihnen Eigenes und Neues zu erwarten hat.} Dieser Satz beeindruckte mich, doch zweifle ich, dass ich im ersten Jahr schon Eigenes geleistet habe. Dr. Aust wurde nach Ostern 6 Wochen zur Kur geschickt, ich vertrat ihn in seinem Unterricht -- Französisch in O II und O I -- er erklärte sich nach seiner Rückkehr mit dem Gelernten zufrieden. Die Löhnung für die $6 * 8$ Wochenstunden erhielt ich, wie damals üblich, erst am Quartalsende -- und sie reichte infolge der 1921 verschärft einsetzenden Inflation gerade zum Kauf eines Pfundes Butter aus!

Der Sommer 1921 war trocken und heiß; die ältliche Miss Hussel, bei der ich wöchentlich eine englische Konversationsstunde nahm, fürchtete sehr um die Wasserversorgung Breslaus. Körperpflege, Turnen etc. hatte ich nach dem Kriege \enquote{aus Zeitmangel}, wie ich mir selbst einredete, vernachlässigt. Rogier und Prauses regten mich zum Schwimmen an, und ich kaufte mir ein stark gebrauchtes Fahrrad für 600 M, nachdem ich ein Jahr vorher das tadellose, moderne Rad meines Vaters für 500~M verschleudert hatte. Mit Prauses, gelegentlich auch zusammen mit ihnen und den beiden tschechischen Untermiedern Rudik und Balcar, machte ich Sonntagsausflüge per Rad in die Umgebung Breslaus. Das war nötig und anregend, bewahrte mich aber nicht vor wiederholter Infektion mit der in jenem Jahre fast epidemisch auftretenden Grippe.

Als Hausarbeit in Philosophie erhielt ich \enquote{Prinzipien der Locke'schen Pädagogik} und in Pädagogik und Methodik \enquote{Behandlung des Wortschatzes und der Wortbildung im Französischunterricht}, eine Arbeit, der bestimmungsgemäß eigene Unterrichtserfahrungen zugrunde zu liegen hatten. Die mündliche Philosophieprüfung fand im Dezember statt, zusammenfassendes Ergebnis war \enquote{gut}, während die auf Anfang Februar 1922 verschobene Pädagogische Prüfung das Prädikat \enquote{Mit Auszeichnung bestanden} erhielt.

In Deutsch hielt ich eine Lehrprobe in U III (Balladenbehandlung), in Französisch eine Lektürestunde in O II. Schon am Tage nach der \enquote{mit Auszeichnung} bestandenen Assessorenprüfung musste ich mich nach Laubau begeben zur Verwaltung der Stelle eines plötzlich verstorbenen alten Neusprachlers, des Dr. Herrtrich. Ich erteilte hauptsächlich neusprachlichen Unterricht auf der Mittelstufe und gewann guten Kontakt mit den Schülern, so dass ich sogar in U II eine nachmittägliche AG in Französisch auf die Beine stellte. Als ich Ende April die Anstalt verließ, hielt der Sprecher der U II sogar eine kleine Abschieds- und Dankesrede für mich, ich sei sehr anregend gewesen. Ich verspürte keine Lust, weiter auf einer Kleinstadtschule mit einem nur durchschnittlichen Kollegium fern von dem geistigen Zentrum Breslau schlecht und recht im Schulalltag zu vegetieren.

In Breslau hatte mir der amtliche Vertreter der Assessoren beim \ac{psk} gesagt, dass die Anstellungsaussichten im höheren Schuldienst als Folge des verlorenen Krieges ungünstig seien, vor 10 Jahren könne ich nicht mit einer Anstellung rechnen und auch eine volle Beschäftigung in Breslau werde sich in nächster Zeit für mich nicht finden lassen. \marginpar{426} Das waren deprimierende Aussichten. Ich fragte, ob ich durch Erwerb der Lehrbefähigung in Russisch die Berufsaussichten verbessern könne -- das sei nicht auszuschließen, ich müsse jedoch bedenken, dass es nur sehr wenige Schulen mit diesem Unterrichtsfach gebe. Ich kam auf den Gedanken, die Lehrbefähigung in Mathematik zu erwerben und fragte deshalb den Vorsitzenden des wissenschaftlichen Prüfungsamtes, Geheimer Rat Klan, seines Zeichens Mathematiker, der, wie ich wusste, mich schätzte: Nein, davon rate er ab, ich würde als Mathematiker nie für voll genommen werden, ich sei Linguist -- Schuster bleib bei deinem Leisten -- war seine Antwort. Mein Entschluss war gefasst: Ich entsage einer Beschäftigung im Lehrberuf 1922/23, auch wenn mir in Breslau oder außerhalb eine solche angeboten würde, bleibe ich in Breslau mit nur 2 Wochenstunden \enquote{Am Zwinger} -- wie die Bestimmung lautete, wenn man nicht aus der Berufsliste gestrichen werden wollte \enquote{in unterrichtlichen Beziehungen} -- setzte sofort die bisher als Hobby getriebene Beschäftigung mit Russisch intensiv an der Universität fort und beginne als Ausweichstudium Volkswirtschaft, für das ich als Spekulant natürlich Interesse hatte.

Ich hörte bei Prof. Paul Diels Slawistik, arbeitete im Seminar aktiv mit, hielt im Sommer- ein umfangreiches sprachliches und im Wintersemester ein literarisches Referat (Puschkin und Goethe sowie \enquote{das Verhältnis von Puschkins Boris Godunow zu Schillers Demetrius}.) In den Seminarübungen waren leider nicht großrussische sondern im Sommer Polnisch (Lyrik von Mickiewicz), im Winter Ukrainische Lyrik behandelt worden; beides zog natürlich für mich erhebliche Mehrarbeit nach sich. Außerdem hörte ich bei Prof. Obst, einem tüchtigen Wissenschaftler und erfolgreichen Geschäftsmann, \enquote{Einführung in das Bank- und Börsenwesen}, sehr interessant, doch zog ich mich nach einiger Zeit zurück aus Zeitmangel: Russisch hatte den unbedingten Vorrang, und schließlich wollte ich in knapp einem Jahre die Lehrbefähigung in dieser schwierigen Sprache erwerben. Und Prof. Obst riet von einem Studium der Volkswirtschaft ab, sofern man nicht eine abgeschlossene Ausbildung in einem Wirtschaftszweige besitze, also in erster Linie im Handel, in der Industrie, im Bankfach. Wer diese Voraussetzungen nicht mitbringe, der solle wenigstens in den akademischen Ferien in einem dieser Wirtschaftszweige volontieren. Bis Ostern 1923 betrieb ich aber als Hobby die Lektüre des Wirtschaftsteils der Zeitung, regelmäßiges Studium des Kurszettels und Gespräche am Bankschalter, sogar Schlings Börsenbuch zog ich zu Rate; ich verwandte täglich 1-2 Stunden hierauf -- und spekulierte mit eigenen und aufgenommenen Geldern. Die Commerzbank war dicht neben der sehr reichhaltigen Breslauer Stadtbibliothek; beide Institute besuchte ich monatelang fast täglich. Ich beobachtete an mir selbst, dass ich mich in der Atmosphäre der Arbeitsräume der wissenschaftlichen Bibliothek oft besser zu konzentrieren vermochte als wenn ich zu Hause allein am Schreibtisch säße.

Dann und wann besuchte ich Mama und Ella, die von April bis Ende September im gesunden von Kiefernwald umgebenen Klitschendorf lag. Ihre nicht sehr feste Gesundheit war auf die Dauer der Tätigkeit als Krankenschwester -- den Nachtwachen bei völlig unzureichender Ruhe am Tage -- nicht gewachsen. Als ich sie im März 1922 in Breslau von einem tüchtigen Arzt untersuchen ließ, stellte er beidseitige Tbc in vielen kleinen Herden fest. Die Infektion ging wahrscheinlich auf die Zeit zurück, in der sie meinen Vater gepflegt hatte. Die Heilungsaussichten wurden als wenig günstig beurteilt. \enquote{Viel Ruhe in frischer Luft, gute Verpflegung, Rotwein mit Ei.} \enquote{Ja geht es denn nicht auch mit Mehlsuppe und etwas Butter?} fragte meine Schwester. \enquote{Natürlich, das ist sogar besser, aber welcher Dame kann ein Arzt dies raten oder gar verschreiben, ohne zu gewärtigen, dass die Patientin dem Arzt den Rücken kehrt?} war die aufschlussreiche Antwort.

Ella lag eisern den langen sonnigen Sommer über hinter dem Mietshaus meiner Mutter, bis zum nach Harz duftenden Kiefernwalde waren nur wenige Schritte. Sie lebte hauptsächlich von Mehlsuppe mit einem guten Stück Butter, wurde dabei rund und pausbäckig, aber das tückische leichte Fieber wollte zunächst nicht weichen. Erst Anfang September ging es zurück, um in wenigen Wochen gänzlich zu verschwinden. Mit Staunen stellte der Breslauer Arzt Ziesche fest: völlige Heilung.

Vor Ostern 1923 unterzog ich mich der Prüfung in Russisch als Zusatzfach für alle Stufen, erreichte das Prädikat \enquote{gut}, was den Vorsitzenden der Prüfungskommission veranlasste, das zusammenfassende Ergebnis der wissenschaftlichen Prüfungen auf \enquote{mit Auszeichnung bestanden} heraufzusetzen.

Nun nahm ich zu Beginn des neuen Schuljahres 1923/24 eine Assessorenstelle in Reichenbach am Eulengebirge wahr, ohne jedoch die Verbindung mit der geistigen und wirtschaftlichen Metropole Schlesien aufzugeben: ich behielt meine Studentenbude bei Prauses, fuhr oft zum Wochenende hinüber, war am Sonnabendabend oft im Theater, d.h. meist im Lobetheater, aber auch im Thalia, sah Stücke Hauptmanns, Sudermanns, Zuckmayers (Der fröhliche Weinberg, Schinderhannes), Frank Wedekinds (Frühlingserwachen, Lulu, Büchse der Pandora), Georg Kaisers, Arthur Schnitzlers, abgesehen von Klassikern.

Reichenbach, Kleinstadt von weniger als \num{20000} Einwohnern, mit schönem Blick auf das nahe, leicht mit der Kleinbahn zu erreichende Eulengebirge, an dessen Fuß die großen, berühmten Weberdörfer Langenbielau und Peterswaldau liegen. Dank der beachtlichen Textilindustrien vor allem der über \num{8000} Mann Mitarbeitern zählenden, sehr modernen Dierig AG von europäischem Ruf kam in die Kleinstadt ein leichter Hauch von Großzügigkeit. Es gab Industrielle, deren Syndici und einige andere Akademiker, die zum Wochenende gern die Metropole Breslau, einzelne gelegentlich sogar die Dresdner Oper, aufsuchten und dieser Horizont über die Kleinstadt hinausragte.

\marginpar{Abschrift von Heft 3 beendet am 04-08-99 in Barnave}

\section{Die große Inflation}
Im Lehrerkollegium des Staatlichen Reform Realgymnasiums mit Französisch als 1. Fremdsprache gab es einige geistig bewegliche und anregende Kollegen wie den Zeichenlehrer und Kunstmaler besonders Portraitisten Ahrend, mit dem ich bald freundschaftlichen Umgang pflegte, den demokratischen Parteipolitiker und Stadtrat Prof. Klein, routinierten Kleinstadtpolitiker mit besonderem Geltungsbedürfnis, den Altmeister des Skisportes Studienrat Senkpiel, seines Zeichens Mathematiker und Physiker, den Neusprachler und Vorsitzenden der Deutschen Volkspartei Gallasch und den intelligenten, abenteuerlustigen, nie langweiligen Physiker der Mädchenoberschule Sürig aus Bremen, dessen Frau sich hatte von ihm scheiden lassen. Mit mehreren von ihnen traf man regelmäßig beim Mittag zunächst im Restaurant des Hotel zum Schwarzen Adler, dann im Ratskeller zusammen. Vater Jahn hatte im Winter 1922/23 sein Hotel an das Finanzamt für 2 Millionen \enquote{Reichsmark} verkauft und musste es zum 1. Juli räumen -- als armer Mann infolge der rasend fortgeschrittenen Geldentwertung und des grausamen Gesetzes Mark = Mark, unabhängig vom jeweiligen Realwert.

An diese Kleinstadtschule war ich versetzt worden, weil für den hier vor zwei Jahren eingeführten Russischunterricht ein neuer Fachlehrer gebraucht wurde. Russisch war an dieser Schule Wahlpflichtfach, d.h. mit der Versetzung in die O II hatte der Schüler die Wahl, entweder Französisch als 1. Fremdsprache bis zum Abitur weiterzuführen oder statt dessen Russisch zu wählen, das nun als Hauptfach in 4 Wochenstunden gelehrt wurde und in dem beim Abitur eine schriftliche Prüfungsarbeit geleistet werden musste. Ich muss sagen, dass ich mich diesem in Deutschland neuen Schulfach hingebend widmete, was u.a. intensive Stundenvorbereitung und Verbesserung der eigenen Sprechfertigkeit bedeutete, wobei mir die nach Reichenbach verschlagene sehr gebildete Witwe eines während des Bürgerkrieges in Russland gefallenen russischen Generals gute Dienste leistete.

Ich hatte ein komfortables möbliertes Zimmer in der weißen Villa italienisch-klassizistischen Stils der Witwe Cohn, des Besitzers einer mittelgroßen Spinnerei, deren lange Halle mit dem Dach im Sägenprofil nur eine Steinwurfweite hinter der Villa stand. Die alte Dame lehnte es ab, das Auto, das \enquote{neumodische Fahrzeug} eines ihrer beiden Söhne zu besteigen; sie fuhr oft im Zweispänner mit Kutscher nachmittags in die Berge.

Die Inflation nahm schon vor den großen Ferien hektische Formen an und störte sogar den Schulbetrieb: anfänglich einmal, dann zwei Mal in der Woche, später fast täglich wurden zwischen 11 und 12 Uhr im Lehrerzimmer an die Kollegen Banknoten ausgegeben -- \num{100} -- \num{1000} -- \num{10000} -- \num{100000} -- Millionen -- Milliarden und schließlich über Billionen lautende Scheine. Man rannte, wenn es der Dienst gestattete, bis 12 Uhr Mittags zu Geschäften, um das Papier in irgend einen Sachwert umzusetzen: ein Paar Schuhe, ein Hemd, eine Krawatte oder Lebensmittel, denn um 12 Uhr schlossen alle Geschäfte, durch Radio erfuhr man zwischen 13 und 14 Uhr den neuen, oft um 50, 100 auch 300\% gestiegenen Kurs, dem entsprechend die Preise sofort geändert wurden und die Waren -- soweit vorhanden bzw. vom Kaufmann nicht gehortet -- entsprechend der jeweiligen Entwertungsrate teuer verkauft wurden.

Ich ergatterte einmal ein Nachthemd für mich für \num{30000} Mark, ab 15 Uhr kostete das gleiche Hemd bereits \num{90000} M! Einmal kaufte ich -- nur um überhaupt einen realen Sachwert zu ergattern -- ein Paar braune Ledergamaschen für ich weiß nicht mehr welche vierstellige Summe, die ich niemals getragen habe. Angesichts der gallopierenden Schwindsucht der Mark forderte ich schon im Mai 1923 für eine Nachhilfestunde 1 Goldmark -- dieser Preis wirkte auf manche Eltern abschreckend und die Kollegen rümpften die Nase.

Die Inflation führte zur Verelendung der Rentner, der Rentiers, der alten Leute. Die Unzufriedenheit breiter Volksschichten führte zu Aufruhr und Plünderungen der Geschäfte und Warenhäuser. So erlebte ich in Breslau während der Großen Ferien -- ich saß wieder über meiner Frenssen-Arbeit -- den \enquote{Schwarzen Freitag} vom 23. Juli 1923: In den Mittagsstunden wurden plötzlich die Schaufenster vieler größerer Geschäfte, darunter zahlreicher jüdischer, eingeschlagen und die Waren, vor allem die Auslagen, geplündert. Im Jahre 1923 war Deutschland das Eldorado der Ausländer. Sie lebten spottbillig und kauften Sachwerte aller Art en masse. In den Weinlokalen gaben die Fremden den Ton an: sie zahlten für die besten Weine weit weniger als in ihrer Heimat für die billigste Limonade. Der allgemeine Ausverkauf war so weit fortgeschritten, dass ich in Breslau keine kleine Schreibmaschine mit deutschen Lettern auftreiben konnte -- ich kaufte eine Erika mit polnischen Lettern! Ich zählte zu denen, die auf die Jagd nach ausländischen Banknoten gingen, am Wochenende trieb ich mich auf dem Hauptbahnhof herum, in Reichenbach rief mich der Ober des führenden Hotels, des \enquote{Kaiserhof}, wenn Ausländer in ihrer Währung bezahlt hatten und ich nahm ihm die Noten zum Tageskurs ab.

Eines Morgens im Herbst 1923 war dieser schaurige Spuk plötzlich mit der Schaffung der Rentenmark vorbei; auf der Strecke blieben alle diejenigen, die ihr Geld auf Sparkonten und in auf Mark lautende Anleihen, wie besonders Kriegsanleihen, angelegt hatten. Der Volksmund scherzte: zu den bisherigen Sparten der Geschäftsbücher, Soll und Haben, sei eine dritte getreten: soll gehabt haben\dots

Dagegen gab es eine dünne Schicht von Kriegs- und Inflationsgewinnern, deren Spitzenrenner Goldmillionen und, wie Hugo Stinnes, sich sogar mit Hilfe von Reichsmitteln -- die zu dem fast völlig entwerteten Nennwert zurückgezahlt wurden -- ein ganzes Wirtschaftsimperium geschaffen hatten. Der Typus des Kriegsgewinnlers und Neureichen war Raffke. Die gute Berliner Wochen-Illustrierte stellte die Preisaufgabe: Was sagt Raffke vor dem Colosseum? Die preisgekrönte Antwort lautete: Baut nicht, wenn ihr keen Jeld habt!

In den Weihnachtsferien fuhr ich mit meinem Schulfreund Hans Menzel, jetzt Angestellter in der Buchhandlung Max \& Co, in den Schnee des Riesengebirges. Wir übernachteten auf der Reifträgerbaude oberhalb von Oberschreiberhau. Während er sich dem Rodelsport hingab, lieh ich mir Skis und übte unermüdlich die Abfahrt auf einer kleinen Strecke. In Reichenbach erstand ich ein Paar gebrauchte Skis, fuhr sonntags in das Eulengebirge, machte die Ski-Wandertage mit den Schülern mit. Der Skilauf begeisterte mich, er gab mir ein neues Lebensgefühl.

\marginpar{439}
Die Deflation erfasste nicht nur den monetären sondern auch den gesamten Wirtschaftsbereich einschließlich des Staatshaushaltes mit seinen Beamten- und Angestelltenapparat. Auch die Schulen blieben von Sparmaßnahmen nicht verschont. Aus Sparsamkeitsgründen sollte zum 1. April -- dem Beginn des neuen Schuljahres -- der Russischunterricht wieder eingestellt werden. Zahlreiche Beamte, so auch zwei Lehrer und Direktor Varges wurden in den Ruhestand bzw. \enquote{einstweiligen Ruhestand} versetzt.

Ich war sofort entschlossen, mit allen mir zu Gebote stehenden Mitteln gegen den Abbau des erst vor kurzem eingeführten und mit Erfolg betriebenen Unterrichtsfaches anzukämpfen. Zunächst mobilisierte ich -- sozusagen privatim -- zwei Mitglieder des Elternbeirates, deren Jungen ich im Russischunterricht hatte. Es gelang mir, unterstützt vom Stadtrat Prof. Klein, Direktor Varges zur sofortigen Einberufung des Elternbeirates zu veranlassen, der dann auch eine energische Protestresolution an die Schulaufsichtsbehörde in Breslau richtete. Ich suchte ferner persönlich führende Persönlichkeiten der Stadt und des Wirtschaftsdreiecks Reichenbach-Langenbielau-Peterswaldau auf und gewann sie für die Unterstützung in meinem Kampfe: Bürgermeister, Direktoren der Reichsbank und der Deutschen Bank, Fabrikbesitzer, insbesondere den einflussreichen Aufsichtsratvorsitzenden der weltbekannten Textilfirma Dienig in Langenbielau sowie die zuständigen Syndici; Syndicus Reh fuhr sogar nach Breslau, um seinem schriftlichen Einspruch durch persönlichen Besuch beim verantwortlichen Dezernenten Oberschul- und Geheimrat Dr. Jantzen Nachdruck zu verleihen. Durch einen Artikel im \enquote{Reichenbacher Tagesblatt}, in dem ich nicht vor einem politischen Angriff gegen den strammen deutschnationalen Geh. Rates Jantzen, meinen Dezernenten, zurückschreckte, hatte ich das Interesse der breiten Öffentlichkeit geweckt.

Der geschickte Redakteur der \enquote{Reichenbacher Zeitung}, der selbst noch eine kleine Nachrichtenagentur betrieb, brachte ihn natürlich ohne Namen des Verfassers unter der Schlagzeile \enquote{Ein Schulbeispiel unverständlicher Maßnahmen}. Dieser Artikel wurde, meist ungekürzt, von rund 50 deutschen Zeitungen übernommen, die auf den Schreibtisch des zuständigen Geheimrates Jantzen niederprasselten -- wie dieser mir selbst später mit misstrauisch fragendem Seitenblick mitteilte.

In der Kleinstadt lassen sich manche Dinge leichter arrangieren als in der Großstadt\dots Die Wirkung blieb nicht aus; zu Beginn des neuen Schuljahres verfügte die Behörde die vorläufige Weiterführung des Russischunterrichts in Reichenbach.

Die Anstellungsaussichten für Studienassessoren wurden behördlicherseits nach wie vor als ungünstig bezeichnet. Um wenigstens einem Teil der Assessoren ein verbrieftes Recht auf spätere Anstellung durch den Staats zu geben, hatte das Kultusministerium einen Numerus clausus eingeführt, der wissenschaftlich und pädagogisch besonders qualifizierte Assessoren ohne Rücksicht auf das Dienstalter besonders begünstigte. Zu meinem Erstaunen war ich nicht in den Numerus clausus aufgenommen worden. War daran der Bericht des Oberschulrates schuld, der meinen Unterricht zu Jahresbeginn revidiert hatte? fragte ich mich. Vom Vertreter der Assessoren am \ac{psk} -- eine solche Stelle gab es damals -- erhielt ich die Auskunft, der Bericht des Oberschulrates über meine unterrichtliche Leistung stehe nicht im Widerspruch zu meinen sehr guten Prüfungsleistungen. Ich machte ferner ausfindig, dass ein Oberschulrat, der an der Universität einen Lehrauftrag für praktische Pädagogik innehatte, die Aufnahme sämtlicher Assessoren, die einmal seine Hörer waren, in den Numerus clausus durchgesetzt hatte. Darunter befand sich auch ein Kollege, mein Konkurrent in Russisch, der die wissenschaftliche Staatsprüfung ein halbes Jahr vor mir abgelegt hatte, sie aber, wie ich von Prof. Diels, dem Slawisten, erfahren hatte, ohne Prädikat bestanden hatte.

Aufgrund dieser Information richtete ich eine Beschwerde an den Kultusminister, die mit dem Satze schloss: \enquote{Ich bitte Ew. Exzellenz prüfen zu lassen, ob das \ac{psk} in Breslau sich bei der Auswahl für den Numerus clausus an die ministerielle Verordnung Nr\dots vom\dots gehalten hat}.

Dieses Gesuch überreichte ich persönlich dem Direktor des \ac{psk} Dr. Müller. Er war wegen seiner Strenge und der hohen Anforderungen, die er als Vorsitzender von Prüfungsausschüssen bei der Reifeprüfung und der Pädagogischen Prüfung für das \enquote{Lehramt an höheren Schulen} stellte, allgemein gefürchtet; man nannte ihn den \enquote{Leichenmüller}. Er überflog das Gesuch und verharrte einige Augenblicke beim letzten Satz; er verfärbte sich leicht.

Dann verließ er wortlos das Amtszimmer und kehrte und kehrte nicht zurück.

Vom Amtsdiener, der Akten auf den Schreibtisch legte, hörte ich, der Herr Direktor habe die Herren Dezernenten in den Sitzungssaal zu einer Sondersitzung berufen.

Nach 50 Minuten betrat er endlich das Zimmer und ging zwei oder drei Minuten schweigend auf und ab. \enquote{Es sind}, begann er dann nicht unfreundlich, \enquote{zunächst einige Feststellungen gemacht worden. Nehmen Sie Ihr Schreiben wieder an sich}, es mir überreichend. Als ich es nur zögernd und mit fragend-misstrauischem Blick tat, wurde er etwas freundlicher: \enquote{Nehmen Sie es mit und gehen Sie nach Hause\dots warten Sie ab, warten Sie getrost ab, es steht nicht schlecht für Sie.} Jetzt war ich beruhigt, steckte das Schriftstück ein und erhob mich. \enquote{Noch eine Mahnung möchte ich Ihnen mit auf den Weg geben: Sie haben ein sehr gewagtes Spiel getrieben, tun Sie ähnliches nicht noch einmal!} -- 8 Tage darauf hatte ich die Verfügung meiner Aufnahme in den Numerus clausus in den Händen.

Im Herbst 1923 schaffte man sich einen \enquote{Detektor} und Kopfhörer an, um auf einem kleinen Bleikristall mit einer Nadel den Punkt zu \enquote{entdecken}, der einem ganz leise das geheimnisvolle \enquote{Tüt tüt} des Breslauer Radio Senders und die Töne der Geigen und Flöten und das leise Hämmern des Klaviers der Kapelle Aal erlauschen ließ. Im Sommer 1924 tauchten zunächst noch sehr teure, etwa das 3-4 fache des damaligen Monatsgehalts eines Studienassessors oder jungen Regierungsrates kostende Rundfunkgeräte mit Lautsprecher auf. Die \enquote{Spitzen der Gesellschaft} Reichenbachs gründeten einen Klub, um im \enquote{Hôtel Kaiserhof} mit einem solchen Gerät die Radiosendungen aus Breslau zu hören. Ich wandte mich an Stadtrat Prof. Klein mit der Bitte um Aufnahme in diesen privilegierten Hörerkreis. \enquote{Nein, meinte er, das geht nicht; dieser Kreis umfasst nur Persönlichkeiten in irgendwie leitender oder hervorragender Stellung, Fabrikbesitzer, Chefs anderer Unternehmungen, Direktoren der Reichsbank, der Deutschen Bank u.s.f. -- und Sie sind einfacher Studienassessor} Ich entgegnete, dass ich schließlich nicht nur einfacher Studienassessor, sondern \underline{der} Vertreter der osteuropäischen Kultur für das ganze Industriegebiet am und im Eulengebirge sei. Stadtrat Klein dachte nach. Er entschied, in dieser, aber nur in dieser meiner Eigenschaft werde er die Aufnahme in den Radioklub erwirken. So durfte ich im erlauchten Kreise den großen Augenblick miterleben, in dem um 13:40 Uhr der Klub der Reichenbacher Rundfunkhörer vom Breslauer Sender begrüßt und beglückwünscht wurde.

In den Großen Ferien befasste ich mich noch einmal mit der Frenssen-Arbeit, brachte sie jedoch nur zu einem vorläufigen, den Stil Frenssens nicht berücksichtigenden Abschluss, gab aber den Gedanken an Promotion -- leider! -- auf, weil inzwischen, vom Unterricht her, andere Ziele in den Vordergrund getreten waren: Russisch und Englisch. In beiden Sprachen waren meine Kenntnisse und Fertigkeiten nicht ausreichend, um einen wirklich guten Unterricht auf fortgeschrittener Stufe erteilen zu können; ich erfüllte nicht die fundamentale pädagogische Forderung, \enquote{fachlich aus dem Vollen zu schöpfen}. Mein Nahziel war ein Studienaufenthalt in England. Dieser Gedanke kam mir, wie gute oder gar produktive Gedanken oder Pläne fast immer, in den Ferien und zwar auf einer Riesengebirgswanderung mit Freunden. Übrigens konnte ich von damals an bei sommerlichen Bergwanderungen oder Aufenthalten im Gebirge lange Jahre hindurch den Gedanken an den vielgeliebten Skisport nicht ausschalten; das Relief der Berg\-land\-schaft sah ich stets vorwiegend unter dem Gesichtswinkel der wintersportlichen Möglichkeiten. Meine Merkwelt hatte sich, wenigstens im Gebirge, etwas verändert. Sank das Thermometer bei regnerischem Wetter im September oder Oktober in Breslau oder Reichenbach auf +8° und darunter, so frohlockte ich, wusste ich doch, dass es jetzt im Riesengebirge in der Höhe der Bradlerbaude, 1220 m, schon schneit.

Zwischen Großen- und Herbstferien nahm ich in Breslau an einem Lehrgang für Rettungsschwimmen teil: um etwas für meine körperliche Fitness zu tun, um wieder mal 8 Tage aus dem Reichenbacher Schulalltag herauszukommen und schließlich schien es für mich als Jugenderzieher ein kleines Plus zu bedeuten.

Zu Beginn des Winterhalbjahres fand die feierliche Einführung des neuen Direktors in Gegenwart der ganzen Schule, des Elternbeirates und der Prominenz von Reichenbach und Umgebung statt. \enquote{Man bemerkte} u.a. auch Kommerzienrat Christian Dierig. Mehrere Persönlichkeiten, vor allem der einführende Geheimrat Dr. Jantzen und Studiendirektor Mittag, waren im Ordengeschmückten Frack, das Kollegium und die übrigen Gäste im Cut erschienen. Dr. Jantzen, mit teutonischem Vollbar, damals eine Seltenheit, tat sich in seiner Rede viel zugute, dass es ihm gelungen sei, den leidenschaftlichen Wünschen weiter Bevölkerungskreise von Reichenbach und Umgebung zu entsprechen und den hier praktizierten russischen Unterricht am Leben zu erhalten. Die Eröffnungsrede des 38-jährigen Mittag, der vorher die deutsche Schule in Bukarest geleitet hatte -- eine Mamutschule mit 2000 Schülern -- zeugte von Klugheit und großem Selbstbewusstsein: er behandelte eingehend die Frage, was die Schule von der vorgesetzten Schulaufsichtsbehörde, dem \ac{psk}, erwarte. Jantzen meinte nachher im Ratskeller im kleinen Kreise, der neue Direktor habe schon ministerielle Allüren an sich\dots

Im Herbst 1924 kam neues Leben in die höheren Schulen Deutschlands mit dem Erscheinen der neuen Richtlinien des Ministerialrates Richert, der früher, wenige Jahre zuvor, Direktor des Realgymnasiums in Reichenbach gewesen war. Kehrten in ihm auch manche Gedanken Gaudigs und Kerschensteiners wieder wie die des Arbeitsunterrichts und überhaupt der starken Aktivierung der Schüler, so war neu das zentrale Prinzip der Deutschkunde, das für alle Fächer, einschließlich der mathematisch-naturwissenschaftlichen Gruppe, verbindlich war. Hierin, wie in manchen der sehr hochgespannten Bildungsziele der \enquote{Richtlinien} schoss Richert auch über das optimal Erreichbare hinaus. Auch unter tüchtigen und geistig regen Kollegen wurden manche Einzelheiten angefochten, ganz zu schweigen von vielen alten Lehrern, die nicht willens waren, noch umzulernen. Einen von ihnen fragte ich, was er in diesem Belange dem visitierenden Schulrat sagen werde: \enquote{Ich stehe zwar nicht auf dem Boden der Richtlinien, doch meine Richtlinien stehen auf dem Boden}, war die Antwort.

Richerts \enquote{Richtlinien} waren für längere Zeit das Thema von Allgemeinen und von Fachkonferenzen. Auch Mittag gab bereits in der ersten Allgemeinen Konferenz eine Einführung in das \enquote{Gedankengut} der \enquote{Richtlinien}. Ein neuer, besonders ein junger Direktor, hat es oft nicht ganz leicht, mit einem Dutzend oder mehr alter auf Lebenszeit angestellten Lehrern fertig zu werden. Aber Mittag zeigte sich den alten Herren gegenüber geschmeidig. Neben ihm saß Prof. Kilian, 60 Jahre alt, Philologenverbandsobmann der Schule. Er las während der Ausführungen Mittags seine Tageszeitung. Mittag: \enquote{Herr Kollege Kilian, darf ich Sie bitten, Ihre Zeitung beiseite zu legen und meinen Ausführungen Aufmerksamkeit zu schenken?} Kilian: \enquote{Das werde ich in dem Augenblick tun, wo Sie etwas mir Neues zur Sprache bringen; bisher paraphrasierten Sie lediglich die Richtlinien Seite 1-8. Und im übrigen möchte ich Sie darauf aufmerksam machen, dass an unserer Schule hier bei Konferenzen nicht geraucht wird.} Mittag: \enquote{Das wusste ich nicht, ich werde mit dieser Tradition nicht brechen.} Mittag drückte seine Zigarette aus; Kilian las ungerührt weiter Zeitung.

\marginpar{452}
Was einem Lehrer so ein einer Kleinstadt passieren kann, zeigt folgender Vorfall: Von den Eltern eines meiner Primaner wurde ich zu einem Hausball nach Langenbielau eingeladen. Ich fand nette Gesellschaft vor, und man blieb noch bis zum Morgen zusammen. Um 7:10 Uhr morgens traf ich mit dem Zuge wieder in Reichenbach ein, eilte in meine Wohnung, wechselte die Kleidung, trank Kaffee und gelangte, wegen meines Eiltempos den wenig begangenen Weg an der Stadtmauer wählend, Punkt 8 Uhr zur Schule, wo ich hintereinander sechs Stunden zu erteilen hatte.

In den ersten beiden Stunden war ich noch \enquote{aufgekratzt} und gut in Form, in der 3. Stunde machte sich aber die schlaflose Nacht schon störend bemerkbar. Zu Beginn der 4. Stunde, Russisch in Unterprima, trat der Sprecher auf mich zu und machte mir im Namen der Klasse den Vorschlag, heute, angesichts des schönen Wetters und des Wochenendes, den Unterricht in Form eines Klassenspazierganges nach draußen zu verlegen. Kurzentschlossen ging ich auf den Vorschlag ein, stellte jedoch die Bedingung, dass während des Spaziergangs nur in der Fremdsprache gesprochen werden dürfe. \enquote{\textcyr{Хорошо, мы соцлаены.}} [Charascho, müi ssozlajenüi] \enquote{Gut, einverstanden}, war die Antwort. Die frische Märzluft tat mir wohl. Lehrer und Schüler blieben brav bei der Stange, nur in den letzten 5 Minuten brach die Muttersprache durch. Ins Klassenbuch trug ich ein: \enquote{freie Sprechübungen} statt \enquote{Sprechübungen im Freien}.

Als ich mich am Wochenende ausgeschlafen hatte und den \enquote{Freiluft-Vor\-schlag} der Unterprima bedachte, kam er mir etwas seltsam vor; was mochte sie dazu veranlasst haben? Am Montag kam alles heraus: der Unterprimaner, in dessen Elternhaus der Ball stattfand, hatte strenges Stillschweigen gewahrt. Dafür war durch seinen jüngeren Bruder in der ganzen Schule verbreitet worden: Busse ist erst kurz vor 8 Uhr nach Hause gekommen. Na, das wird ja ein schöner Unterricht werden, u.ä. Meine Klassen behaupteten, sie hätten mir so gut wie nichts angemerkt und die U I habe mir zu Hilfe kommen wollen!

Die Reichenbacher \enquote{Gesellschaft} führte ihre Winterveranstaltung in der \enquote{Loge}, vor allem aber im \enquote{Kaufmannsheim} durch, in dessen oberen Räumen übrigens mein Freund Arendt, der Maler, der die Töchter und Frauen der Fabrikanten portraitierte, sich ein Atelier hatte einrichten dürfen. Auf einem der hier stattfindenen Bälle stellte mich der Fabrikbesitzer Herr Cohn, ich wohnte in der Villa seiner Mutter, seiner hübschen Nichte aus Frank\-furt/Main vor. Als ich sie sofort zum Tanzen engagierte, rief mir Herr Cohn nach: \enquote{Herr Assessor, aber bringen Sie sie mir genauso zurück, wie ich sie Ihnen übergeben habe.} Erst nachdem ich schon mit anderen jungen Damen getanzt hatte, engagierte ich, welcher Faux Pas!, die Tochter des Landrats Graf Degenfeld und erhielt -- einen Korb, während sie eine Sekunde später einem meiner Schüler von Prittwitz und Gaffrou, den Arm reichte. Noch heute glaube ich ihr \enquote{Voilà, Monsieur!} zu hören, mit dem sie, überlegen lächelnd an mir vorüberwalzte.

Mein Bildungsplan für Russisch fand bei Mittag Anklang: die ersten 1,5 Jahre intensiver Grammatikunterricht, der im wesentlichen zum Abschluss gebracht wird, natürlich mit kleinen interessanten und zu Sprechübungen geeigneten Lesestücken; dann vom 2. Halbjahr Unterprima bis zum Abitur 3 Wochenstunden russische Lektüre einschließlich des russischen Volksliedes, Text und Melodie; die 4. Wochenstunde blieb der Behandlung der großen russischen Literatur von Puschkin bis Gorki vorwiegend in deutscher Sprache vorbehalten. Zu diesem Zweck schuf ich eine russische Arbeitsbücherei vorwiegend aus deutschen Übersetzungen, aber auch aus Büchern über russische Kunst und einer kleinen Auswahl russischer Texte bestehend. Ich verfasste eine Schrift \enquote{Deutschkunde im Russischunterricht}, die Mittag dem Ministerialrat Richert übersandte. Zur Beschaffung der Bücher -- mir standen 800.- RM, damals soviel wie 3-4 Monatsgehälter eines Studienrates, zur Verfügung -- fuhr ich zwischen Weihnachten und Neujahr nach Berlin (Dieser Zeitpunkt erscheint angesichts meiner Ski-Leidenschaft seltsam, doch der Winter 1924/25 war in den Sudeten ungewöhnlich schneearm). 

Damals gab es in Berlin mehrere Emigrantenverlage, darunter den guten Ladyšnikow, und sieben russische Buchhandlungen allein in der Nähe des Ku'damms und der Gedächtniskirche. In den Jahre 1922/23 sollen damals gegen \num{300000} russische Emigranten in Berlin Unterschlupf gefunden haben.

\marginpar{454}
Trotz drängender Abitur-Korrekturen fuhr ich Anfang März zum Wochenende in die Grafschaft Glatz, um mit Freunden am Schneeberg endlich einmal Ski zu laufen. Aber Mitte März legte endlich auch die Eule\footnote{Anm. Helga: das Eulengebirge}, viel zu spät, einen 30 cm dicken Schneepelz an, der die Schule sofort zu einem schönen Skiwandertag herausforderte. Und die Osterferien verbrachte ich mit Prauses auf unserer Stammbaude, der Bradlerbaude, im Riesengebirge; in 14 Tagen wurden wir mit viel Sonne und Schnee reichlich für den schneearmen Winter entschädigt. Abends wurde bei \enquote{Hohenelber Bier} Skat oder Doppelkopf gespielt; doch wusste ich auch täglich etwas Zeit zur Vorbereitung auf den Studienaufenthalt in England abzuzweigen. Abgesehen von der sprachlich-literarischen Vorbereitung arbeitete ich auch mehrere Neuerscheinungen über England durch, von denen ich nur noch eines nennen kann, ein geistvolles, aber etwas einseitiges Buch von einem Schmidt \enquote{Das Land ohne Musik}.

Die Behörde bewilligte mir einen 3-monatigen Studienaufenthalt in England mit einer Dotierung von \num{1000}.- \enquote{Rentenmark}, das war mehr als das 3-fache Monatsgehalt. Mein Assessorengehalt lief weiter, jedoch musste ich einen Studienreferendar, der mich in den westlichen Fremdsprachen vertrat, selbst bezahlen. Für das Fach Russisch stand kein qualifizierter Vertreter zur Verfügung; auf meinen Vorschlag versuchte man es mit einer 40-jährigen ehemaligen Petersburger Lehrerin, die einen deutschen Förster geheiratet hatte, mit ihm nach der Oktoberrevolution nach Deutschland ausgewandert war, wo er -- kein Wunder infolge der Gebietsverluste Deutschlands nach 1918 -- in Reichenbach nur als Bademeister einen bescheidenen Unterhalt gefunden hatte. Im \enquote{Reichenbacher Tageblatt} las man am 5. Juni schmunzelnd die etwas boshafte Notiz: \enquote{Studienassessor Busse vom hiesigen Realgymnasium tritt am heutigen Tage einen 3-monatigen Studienaufenthalt nach England an. In der Zeit seiner Abwesenheit wird er von der Frau des Bademeisters vertreten.}\\

\marginpar{495 (Vati schrieb das 10/9/74)}

Im Nachtschnellzug gelangte ich am 25. Juni 1925 nach 13 Stunden im Abteil 3. Klasse auf hartem Holz sitzend (Schlafwagen, den ich für unerlaubten Luxus hielt, habe ich erst im Alter benutzt) von Breslau über Hannover nach Amsterdam. Erste Eindrücke, gleich ob von Menschen, Landschaften, Ländern oder Kunst sind immer die nachhaltigsten. Ich sehe noch heute vor mir links und rechts des Zuges die weiten sich bis zum Horizont erstreckenden saftig-grünen Wiesen, auf den hier und da weiße Segel langsam dahinglitten, dann und wann neben gepflegten großen Bauerngehöften einen hochragende holländische Windmühle, wie man sie von Bildern und Porzellan kennt.

Jetzt donnerte der Zug über Kanalbrücken, vorbei an gotischen Stadttoren aus roten Ziegeln und hielt schließlich vor einem großen langgestreckten Bahnhof -- Amsterdam -- dessen hellrote Ziegel noch jugendliche Frische atmeten. Auch der mächtige Bau der Amsterdamer Börse, den ich bald erblickte, der in seiner neuen Sachlichkeit geradezu Epoche gemacht hat, schien mir so recht in die frische Seeluft Amsterdams zu passen. Ich blieb drei Tage in dieser geschäftigen, weltoffenen Stadt, damals der Hauptstadt eines sehr wohlhabenden Landes, dessen einträglicher Kolonialbesitz das Mutterland um mehr als das dreißigfache an Ausdehnung übertraf. Ich bezog Quartier in einer kleinen an einer Gracht liegenden Pension. Schon um 4 Uhr morgens weckte mich das tacke-butt-butt der geschäftigen Schuten. Das Frühstück war üppig, schon an Englands hearty breakfast erinnernd. Auch die Kleidung der Menschen und manches andere deutete auf einen höheren Lebensstandard als den unsrigen. Im Vondelpark radelten Damen in seidenen Kleidern, das hübsch herausgeputzte Kind im Körbchen an der Lenkstange. In der Straßenbahn konnte der Schaffner auf meinen 10-Guldenschein nicht herausgeben -- da zahlte ein junges Mädchen beim Aussteigen für mich.

Das Reichsmuseum war ein starkes Erlebnis; neu waren für mich die prächtigen Interieurs aus Hollands großer Zeit. Als Slawist musste ich mir natürlich Zaandam und das Petershuisje ansehen. -- Damals existierte übrigens der mächtige Sperrdamm noch nicht, die Zuidersee, alias das Ijsselmeer, war bei weitem nicht so eingepoldert wie heute; die Schiffe verkehrten von Amsterdam nach Bremen und Hamburg noch durchs Ijsselmeer. Amsterdam war noch im vollen Wortsinn eine Seestadt. --

Auf dem Dampfer wurde während der Fahrt zur Ziehharmonika getanzt; in der behäbigen Fröhlichkeit und dem leicht schwerfälligen Tanzbewegungen schien sich mir das im Grunde bäurische Wesen dieses niederfränkischen Volksstammes zu offenbaren. Die Prinzessinnen des Hauses Oranien fallen hier offenbar nicht aus dem Rahmen, wofür auch die folgende Anekdote zeugte:

\begin{quote}
	Beatrix, die älteste Tochter der Königin Juliana, geht als Studentin die Treppe der Leydener Universität hinauf; hinter ihr machen sich Kommilitonen lustig über ihre plumpen, säulenhaften Beine. Beatrix dreht sich um und ruft ihnen zu: \enquote{Meine Herren, auf diesen Säulen ruht das Haus Oranien!}
\end{quote}

Eine flinke Straßenbahn brachte einen damals auf ziegelgepflasterter Straße in 50 Minuten über Haarlem zum Nordseebad Zandvoort; das Wasser war bereits angenehm temperiert. In Scheveningen hatte eine unruhige See Massen von Quallen, oft von 30 cm und mehr Durchmesser in die Nähe des Ufers gedrückt. Wir Badenden bewarfen uns gegenseitig lustig mit diesen Quallen, wie ich sie in dieser Größe und dieser Dichte weder vorher noch nachher gesehen habe.



\section{Englandfahrt}
Am 3. Tage schiffte ich mich am späten Abend in Hoek van Holland ein. Die Überfahrt nach Harwich kostete in der 2. Klasse 20 Mark. Im Vorderteil des Schiffes waren Liegen für ca. 50 männliche Fahrgäste eingerichtet, von denen etwa die Hälfte belegt war. Trotz des heftigen Stampfens der Maschine schlief ich sofort ein. Lange vor Sonnenaufgang -- die Erwartung des Neuen ließ mich nicht mehr schlafen -- ging ich an Deck, sprach mit einem englischen Frühaufsteher und einem Matrosen und ging gegen 7 Uhr in Harwich an Land. Die Pass- und Barmittelkontrolle war sehr eingehend; aufgrund der von mir vorgewiesenen Barschaft von 20 englischen Pfund (damals gleich 800 Mark was etwa 3 Monatsgehältern des Studienassessors bzw. jungen Studienrats entsprach) erhielt ich eine Aufenthaltsgenehmigung für 6 Wochen. Nach Ablauf dieser Frist musste ich im Londoner Polizeipräsidium mit meinen Geldmitteln erscheinen um eine Aufenthaltsverlängerung zu erreichen. Aus den an mich gerichteten Fragen entnahm ich, dass man vor allem befürchtete der Deutsche könnte zu Erwerbszwecken in London untertauchen, was, wie ich später hörte, vielen Kellnern, Bäckern, Frisören und anderen gelungen war, angesichts der damals stattlichen Zahl an Arbeitslosen in England jedoch unerwünscht war.

Unter einem strahlenden wolkenlosen Himmel hatte ich den schnellen boat-train bestiegen, doch je mehr wir uns dem Mittelpunkt des riesigen industriereichen Ballungszentrum London näherten, umso trüber und düsterer wurde die Atmosphäre unter dem nach wie vor wolkenlosen Himmel. Als ich die Liverpool-Street-Station verließ, warf die Sonne keinen Schatten mehr, sie stand als rötlich-gelbe Scheibe am verschmutzten Himmel. Man strömte, nein, man schritt, ohne Hast, in sehr aufrechter Haltung, den schwarzen Bowler auf dem Kopf zur Fabrik, zum Büro. Alle Männer, gleich ob Bürger oder Arbeiter trugen den Bowler hat. Ich merkte bald, dass allein schon der Hut mich dem Londoner als Ausländer kenntlich machte, desgleichen meine deutsche Aktentasche. Und da ich nicht auffallen wollte, waren Bowler und \enquote{case}, d.i. ein Miniaturköfferchen aus solidem Rindsleder meine ersten Erwerbungen in London.

Ich hatte meinen Koffer im Bahnhof Liverpool-Street Station gelassen und durchstreifte die City. Es war windstill. Die Luft war infolge des Kohlenrauchs unerträglich stickig bei einer Temperatur von höchstens 20°. Die Londoner Luft bedrückte meine Lungen derart, dass ich mir nicht vorstellen konnte, es hier wochen- oder gar monatelang auszuhalten. Vermischt mit etwas Herbstnebel hätte dieser Steinkohlerauch die berüchtigte Londoner \enquote{Pea soup} ergeben, aus Dickens Schilderungen nur allzu bekannt!

Die Menschen auf den Straßen bewegten sich gemächlicher als in Breslau oder Berlin; man war besser und modischer gekleidet, die Mehrzahl der Männer trug schon die weiten Oxford-trousers, die ich in Breslau erst bei einem oder zwei \enquote{gents} gesehen hatte.

Um mich im Sprechen und vor allem im Hörverstehen zu üben -- englische Rundfunksendungen zur Schulung des Gehörs gab es damals praktisch noch nicht -- ließ ich mir von den stattlichen, stets höflichen \enquote{Bobbies}, die an fast jeder Straßenkreuzung standen und meist mit \enquote{officer} angeredet wurden, den Weg zu diesem und jenem Gebäude, Platz oder Straße beschreiben. Alle Schutzleute sprachen \enquote{cockney}: \enquote{[gau this wai stait on], gentleman.} Ich etablierte mich schließlich in einem Boarding-House, d.i. einer Hotelpension in der Nähe des British Museum. Die Diele -- hall -- war ausgeschmückt mit Jadgtrophäen aus dem englischen Kolonialreich. \enquote{Oh, you have very nice corns}, bemerkte ich anerkennend zum Besitzerehepaar, was große Heiterkeit auslöste -- ich begriff, dass mir ein vokabularischer Schnitzer unterlaufen war: ich hatte corn (Hühnerauge) und horn (Geweih) verwechselt. Wenn ich glaubte, es hier mit Engländern zu tun zu haben, so wurde ich bald eines anderen belehrt: man betonte nachdrücklich und selbstbewusst: No, we are Irish people! In der Tat entsprach auch die lebhafte, sehr von Gebärden begleitete Gesprächigkeit nicht dem Bild, das ich mir von Engländern gemacht hatte.

Als ich nach dem Lunch wieder meine Streifzüge durch die City fortsetzte, atmete ich auf: eine frische Briese hatte den schweren Dunst verscheucht, die Sonne strahlte fast so hell wie in Berlin, und so blieb es auch an vielen Tagen jenes Sommers, der überdurchschnittlich trocken und warm war -- die Spitzentemperatur im Juli, die Schlagzeile machte und unter der die Londoner seufzten, war 27°C.

Während mir das englische \enquote{Hearty breakfast} zusagte, konnte ich mich mit lunch und dinner nicht befreunden: mehrere Tage hintereinander gab es zum Lunch cold mutton und zum dinner hot mutton with mint sauce, einer von kontinentalen Gaumen schwer zu verkraftenden gelben Tunke. Erst als ich zur Chefin die schockierende Äußerung gewagt hatte: \enquote{London seems to be full of sheep} wurde der Speisezettel etwas abwechslungsreicher. Die 10 Tische des Speiseraumes waren nur zur Hälfte besetzt, mit denen man ungezwungen während des Essens ins Gespräch kam. Ein 65-jähriger Gast erschien am ersten Sonntag in elegantem Cut, weißseidenen Gamaschen und Top-hat (Zylinder); ich fragte ihn, ob er im Gottesdienst gewesen sei? -- No -- Oder haben Sie einen Besuch gemacht? -- No -- Warum dann dieser Schick? -- Von 12-12:45 Uhr bin ich auf dem Piccadilly Circus auf- und abspaziert \enquote{and at this time I was the most elegant man on the circus}, erklärte er stolz. Ich fragte mich, ob man wohl auf dem europäischen Festland einem solchen Kauz stark vorgerückten Alters begegnen könnte\dots

M. Cooper, seines Zeichens clerk aus Pernambuco, ließ sich für einige Wochen im Boarding House nieder: ein unterkühlter Engländer wie er im Buche steht, mit einem touch von Langeweile, der zum Sprechen eines Anstoßes vom Partner bedurfte. Die beiden irischen Frauen machten sich über ihn und seine Steifheit lustig. Immerhin war er mir als Gesprächspartner und Compagnon bei Wanderungen durch die City und bei Einkäufen -- faute de mieux -- willkommen. Er hatte sportliche Interessen und ich besuchte mit ihm das berühmte Motorradrennen auf der Isle of Man in der Irischen See. Meine aus Deutschland mitgeführten dicht unter dem Knie geschlossenen Sporthose hatte ich durch die damals ausschließlich in London getragenen Knickerbocker ersetzt -- in London nannte man sie Plusfour, ich weiß nicht warum; in meiner deutschen Hose hätte ich ebenso wenig in die englische Landschaft gepasst wie ein weißer Sperling.

In die beiden in Liverpool-Street Station bereit stehenden Sonderzüge strömten die völlig einheitlich gekleideten ausschließlich männlichen Sportler bzw. Fans, in Plusfours und Sportmützen uniformer Machart, jeder trug die Zeitschrift Motor Cycling in der Hand. Im Speisewagen lernte ich ein für continental throats offenbar unzuträgliches, weil schrecklich kratzendes und brennendes Getränk kennen: ginger ale, d.i. Ingwer Bier.

Vor unserer Verschiffung in Liverpool gab es eine reichliche Stunde Aufenthalt: \num{2000} Menschen standen von Bobbies dicht zusammengepfercht in der Bahnhofshalle; staunenswert die gute Massendisziplin. Es war kein einziges Schimpfwort zu hören, alles blieb gutgelaunt. Die Masse sang und sang ein monotones Lied, die ganze Zeit hindurch, nur ein einziges, das wohl 25 mal wiederholt wurde, leider habe ich jetzt Text und Melodie vergessen\dots

Die Mehrzahl der Engländer, die wie wir beiden in den Kajüten keinen Zugang mehr gefunden hatten, legten sich an Deck hin -- und viele expektorierten sich liegend während der stürmischen Überfahrt auf dem 2500 t-Dampfers. Mein Mann aus Pernambuco hatte mich zwei Tabletten gegen Seekrankheit schlucken lassen, die ihren Zweck auch durchaus erfüllten.

Nach fünfstündiger Fahrt landeten wir in Douglas, der Hauptstadt dieser erst 1820 von den sogenannten Manx-Pionieren, die gaelisch sprachen, besiedelten Insel. Ein für mich überraschender Anblick: eine weite halbkreisförmige Meeresbucht, eingerahmt von einem Kranz 2-stöckiger Reihenhäuser, alle von der gleichen, an sich nicht ungefälligen Fassade, doch das Ganze von einer doch etwas störenden Einförmigkeit, die nur durch hier im Norden völlig überraschende Palmen zwischen den Häusern gemildert schien. Der Pernambuco-Mann sorgte dafür, dass ich damals blutiger und uninteressierter Motorlaie von der Rennleitung als Vertreter Deutschlands begrüßt wurde! Ich fuhr mit der Elektrischen auf die höchste, völlig kahle Erhebung der Insel, mit dem angelsächsischen Namen Snaefell, benutzte jede Gelegenheit zur conversation im englischen small-talk Stil -- und hatte im übrigen nicht viel vom Besuch der Isle of Man -- nach einer absolut schlaflos verbrachten Nacht.

Auf der Rückfahrt genoss ich die letzten 200~km der Strecke Liverpool-London: der Zug sauste rastlos durch herrliche Parklandschaften, hier und da mit einem Herrensitz oder einem schmucken Dorf im goldenen Licht der Morgensonne, doch landwirtschaftlich wenig genutzt, meist Besitzungen der Aristokratie. England deckte 1914 5/6 seines Lebensmittelbedarfs im Empire.

Nachdem ich mit Mr. Cooper ein paarmal sightseeing und -- ihn nur begleitend -- shopping in der City gewandert war, besuchten wir gemeinsam die große Empire Exhibition in Wembley. Man stieg in Mary-le-Bone um; der Schaffner rief: \enquote{Mälben, Mälben}

Ich schaute jetzt nach der Aussprachbezeichnung in meinem Metoula Sprachführer, sich als zuverlässig erwiesen hatte und las dort zu meinem Erstaunen [Märilbn] Ich fragte meine Wirtsleute, welche die richtige Aussprache sei und wurde belehrt, dass alle Aussprachevariationen üblich und damit richtig seien\dots

Aufgrund eines Inserats im Daily Telegraph erhielt ich über zwei Dutzend paying-guest Angebote. Ich besuchte die Mehrzahl der Bewerber aus westlichen und südwestlichen Londoner Stadtteilen, wobei ich einen Eindruck von der sehr ansprechenden Wohnkultur der middle-class und auch der higher middle-class people erhielt. Mir gefiel die stets gute Durchlüftung der Wohnräume und ihre Ausstattung mit bequemen chairs, in denen die Leute mehr lagen als saßen, so wie die oft die sehr guten dicken Teppiche, auf denen manche Familienmitglieder lagen, ohne sich durch den \marginpar{474} eintretenden Fremden stören zu lassen. Zweimal passierte es mir, dass mir Quartier und Leute zusagten und alle Fragen eine mich befriedigende Antwort erhielten, darunter auch die von mir gewünschte reichliche Sprechgelegenheit, bis schließlich die Übereinkunft an meiner \enquote{nationality} scheiterte. \enquote{German? Sorry, no, we can't take a German. Your nation has done too much harm to us.}

Ich landete im Hause des Dr. Ellis, in einer higher-middleclass-street eleganter Reihenhäuser unmittelbar westlich des Hide Park. In der 5. Nachmittagsstunde öffnete mir ein elegant im smoking gekleideter junger Mann von etwa 25 Jahren, führte mich nach kurzen, verbindlichen Worten ins Empfangszimmer, ließ mich Platz nehmen und holte seine Mutter, eine Dame von Welt von ca. 45. Jahren. Sie hatte bereits zwei paying guests, einen \enquote{swedish professor} und \enquote{a banker, director of a bank, from the Rhine}, die beide schon sehr gut englisch sprächen. Obwohl das Zimmer sehr schmal, der Pensionspreis sehr hoch waren -- er lag (monatlich) erheblich über einem Studienratsgehalt, nahm ich für begrenzte Zeit an. Alle Mahlzeiten wurden gemeinsam eingenommen, Mrs. Ellis präsidierte, die Köchin reichte ihr das Fleisch, das sie tranchierte. Es stellte sich heraus, dass der elegant gekleidete und gewandte junge Mann nicht ihr Sohn, sondern der Butler war. An manchen Abenden tanzten wir bei Grammophonmusik mit Frau Ellis und ihrer 20-jährigen Tochter. Beim Tanzen erfuhr ich von \enquote{Madam} Frau Ellis, sie habe früher -- in einem der Dominious -- 5 Bedienstete gehabt, jetzt müsse sie mit nur zweien auskommen, dem Butler und der Köchin; und wem habe sie das zu verdanken? Den Deutschen! Sie bemängelte, dass ich nicht Bridge spielte und machte mich an einigen Abenden mit den Spielregeln bekannt.

Einmal fuhren wir, d.h. Frau Ellis mit Tochter und einem englischen Bekannten, ins Theater. Frau Ellis legte mir nahe, dem Schofför [sic] 5 sh Trinkgeld zu geben. Für diesen Preis hatte man damals 5 gutbürgerliche Mittagessen im Ratskeller deutscher Städte\dots

Die Gespräche bei Tisch waren durchweg small-talk: weather, Hygiene u.ä. Dr. Ellis fragte mich, aus welcher Stadt ich komme; ich nannte Breslau. \enquote{I see, I already heard of, it's the capital of Tchekoslowakia.} Er empfand seinen Irrtum als völlig belanglos. Als Engländer interessierte ihn die Hauptwindrichtung in Deutschland. Meine Antwort, in Deutschland seien ähnlich wie in England Winde aus westlichen Richtungen vorherrschend, verwunderte ihn sehr. \enquote{But to the west of Germany are the Alps!} Als ich ihn, mein Befremden über diese geographische Unkenntnis mühsam verbergend, aufgeklärt hatte, meinte er nonchalent: \enquote{I know the European Continent only till Paris.}

Der nahe Hyde Park mit der schönbeuferten [sic] Wasserfläche des Serpentine, mit Rotten Row wo Sonntagvormittag während der \enquote{Season} die Cavalcarde der High Society paradierte und am Schabbesnachmittag und -abend die meist kleinen in schwarzer Seide gekleideten Jüdinnen aus Eastend am Wasserufer promenierten, verlockte zum Bummeln. Sonntag nachmittags bis 10 Uhr Abends, London hatte Sommerzeit und dies war die Zeit des Sonnenuntergangs, spielte eine starke Musicband. Wenn der 10. Glockenschlag vom Big Ben verklungen war, ertönte die Nationalhymne \enquote{God save the King}. Die Massen, die in Gruppen oder zu zweit oder einzeln auf den riesigen Rasenflächen herumlagen, erhoben sich sofort -- für mich sehr eindrucksvoll -- und hörten die Hymne stehend in guter Haltung an. Kein noch so im Grase herumdösender Stromer oder Kommunist, der liegen geblieben wäre. Dann verließen die Massen, viele Liebespaare nur zögernd aus den Gebüschen und kleinen Gehölzen das Freie betretend, gemächlich den Park. Als ich Frau Ellis die letzten Beobachtungen schilderte, meinte sie: \enquote{Nothing for me, I like to have my loving comfort.}

In einer Ecke des Hyde Park, unweit Marble Arch, war der Tummelplatz der Public speakers, gern \enquote{ranter und spontus} genannt. Besonders Sonnabend nachmittags sprachen oft ein Dutzend solcher Redner von ihren hölzernen Kanzeln, eine von der anderen manchmal kaum 50~m entfernt, zu den interessiert oder belustigt zuhörenden und vielfach von einem zum anderen fluktuierenden Zuhörermassen. Es war der Tummelplatz politischer und religiöser Sektierer, Propheten und Kämpfer für diese (z.B. Vegetarierer, Heilsarmee, Lernt Esperanto!) oder jene Heilslehre, für diesen oder jenen Verein. Ein leidenschaftlicher Abstinenzler bewies der Masse die Schädlichkeit des Alkohols am Experiment: er goss Wasser in ein Glas; einige kamen auf die Kanzel hinauf und überzeugten sich anhand einer starken Lupe, dass es im Wasser von Mikroben wimmelte. Dann goss er zwei Tropfen Alkohol hinein und die Mikroben starben. Einer der Zuhörer rief: \enquote{Jeder weiß, wie schädlich die Microben sind. Von jetzt an werde ich nie wieder ein Glas Wasser trinken, ohne vorher ein paar Tropfen Whisky hineinzugießen!} Alles lachte. Die meisten gingen weiter zu einem anderen Speaker und der zeternde Abstinenzler blieb allein im kleinen Kreis.

Das Familienoberhaupt einer lower middle-class family, in die ich dann übersiedelte, sprach durchaus normales Englisch, bis auf eine Absonderlichkeit: er sprach but nicht [bat], sondern [but] Als ich meine Verwunderung darüber ausdrückte, meinte er selbstbewusst: \enquote{Yes, the other people say [bat] and I say [but].}

Hier lernte ich eine Volksschullehrer-Familie kennen, von der ich eingeladen wurde. Die größte Merkwürdigkeit dieses Abends war folgendes Gespräch: ich gab meinem Bedauern Ausdruck, dass zwei so blutsverwandte westgermanische Völker wie Engländer und Deutsche sich im schrecklichen Weltkriege als Feinde gegenüber gestanden hätten. \enquote{Blutsverwandte? Wir gleicher germanischer Abstammung? Davon kann überhaupt keine Rede sein!} entgegnete man mir völlig selbstsicher und apodiktisch. \enquote{Sie können doch nicht bestreiten, dass Angeln und Juten aus der Gegend der Elbmündung, also aus Urgermanischen Gebiet im 5. Jahrhundert nach England gekommen sind.} \enquote{Das stimmt schon}, entgegnete der Lehrer, \enquote{aber damit ist keineswegs erwiesen, dass es sich bei den Angelsachsen um, rassisch gesehen, Germanen handelt. Im Gegenteil, es ist mit Hilfe der Bibel einwandfrei bewiesen, dass es sich bei den sogenannten Angelsachsen um einen rein jüdischen Stamm handelt, der nach der Zerstörung Jerusalems in das Nordseegebiet nahe der Elbmündung verschlagen wurde und dort, zugegeben, die germanische Sprache angenommen hat.} Und er stand auf, kramte im Bücherschrank und kam mit einem Stoß von Zeitschriften zurück mit dem Titel \enquote{British Israel World Federation}. Diese preudowissenschaftliche Zeitschrift, von der er mir einige Nummern zum Abschied schenkte, stand unter dem Protektorat eines Mitgliedes der königlichen Familie und hatte als Mitarbeiter vorwiegend Geistliche der High Church und Offiziere. Ihr war im Weltkriege eine wichtige propagandistische und \enquote{patriotische} Rolle zugefallen, der Masse der un- oder halbgebildeten Engländer einzusuggerieren, dass der Segen Gottes, den die Juden durch Hinrichtung des Gottessohnes verscherzt hatten, auf den unschuldigen 11. Stamm übergegangen sei und ihnen, den Engländern, damit der Sieg im Weltkriege als Frucht des Segens Gottes sicher sei. Auch besagter Lehrer war des Glaubens, dass Englands Sieg hierauf zurückzuführen sei.

Man bedauerte, dass ich die Poets' Corner in Westminster Abbey noch nicht besichtigt hatte. Schon am folgenden Tage holte ich das Versäumte nach und konnte beim nächsten Besuch von meinem Eindruck berichten. Der Lehrer und seine Frau baten mich nachdrücklich, ich solle ganz offen und rückhaltlos meine Meinung sagen. So eröffnete ich ihnen meine Enttäuschung: eine Ansammlung von Statuen höchstens 2. Güte, ich hätte fast geglaubt, mich in einem Second Hand Shop of Antiquities zu befinden. Wie konnte ich nur eine solche Dummheit begehen! Jetzt war es aus mit der Sympathie für den Ausländer; die -- ich muss schon sagen Zurechtweisung, die mir widerfuhr, gipfelte in dem Satze: \enquote{wir wissen genau, dass Deutschland etwas annähernd so Schönes wie unsere Poets' Corner nicht aufzuweisen hat.}

In Fulham nahm ich bei der schon ältlichen Miss Hurrell, deren sprachliches Lehrgeschick und gute allround-Kenntnisse mir schon in Breslau gute Dienste geleitet hatten und die jetzt in ihrem Reihenhaus in Fulham Sommerferien machte, Konversationsstunden. Sie machte mich mit ihrer 65-jährigen Freundin Miss Thatcher bekannt, die eine Rolle im englischen Flottenverein spielte und mir den Besuch einer Londoner Secondary School vermittelte. Sie war sehr ahnenstolz und führte ihr Geschlecht bis in die angelsächsische Zeit des Alfred the Great zurück. Sie erklärte stolz: My family never recognized William the Conqueror! (Wie schrecklich für Wilhelm den Eroberer!) Sie interessierte sich lebhaft für meinen neuen von einem City-Schneider gefertigten Cordanzug, betastete ihn hier und da, lobte und tadelte dies und das und suchte offenbar körperliche Berührung mit dem noch jungen Manne -- was mir aber lästig war. Nachher, auf unserem Conversationsspaziergang gab Miss Hurrell eine kurze Erklärung ab: meine Freundin ist manns\-toll.

Der Raum vor dem Amtszimmer des Headmasters der Secondary School war mit mehreren Photos des ca. 30 Lehrer starken Kollegiums der Schule geschmückt. Ich studierte die Gesichter -- es war schwer, wenn nicht unmöglich, nationale Unterschiede von deutschen Oberlehrer-Physiognomien zu entdecken. Als ich das Amtszimmer betrat, streifte der Headmaster gerade noch den rechten Arm in seinen Talar, der seiner ganzen Gestalt den nachhaltigen Eindruck von Würde verlieh. Nach \marginpar{483} einem kurzen sich aus der Situation ergebenden Gespräch wurde, meiner Bitte entsprechend, der Rundgang durch die Klassen angetreten. Wir verweilten in jeder ein paar Minuten. Einen Eindruck bekam ich vom Lateinunterricht: die lateinische Aussprache war englisch, die Unterrichtsmethode traditionsgebunden dosierend, der Lehrer saß auf dem Katheder, dozierte, die Schüler schrieben mit, aus der Klasse kamen keine Fragen. In den anderen Klassen und Fächern dasselbe Bild. Die Unterrichtsstunde näherte sich bereits ihrem Ende, als wir den Physikraum betraten: der Lehrer unterrichtete nur 6 ca. 16-jährige Schüler. Der headmaster wunderte sich über die geringe Schülerzahl und erfuhr vom Lehrer, dass die Klassenstärke eigentlich 22 sei, die anderen Schüler seien nicht gekommen weil sie im Tischtennisraum erst ihre Partien zu Ende führen wollten, was den Fachlehrer offenbar gar nicht verwunderte. Der headmaster bemerkte jedoch, ohne sich irgendwie zu erregen: \enquote{Oh, they shouldn't do that during the lesson}. Da an dieser Oberschule auch deutsch unterrichtet wurde, bat ich, auch diesem neusprachlichen Fach beiwohnen zu können. Doch der headmaster lehnte ohne Begründung ab. \enquote{No, I can't allow you that\dots}

Mit Beginn der Großen Ferien tauchte mein Freund Dr. Prause zu einem 8-wöchigen Studienaufenthalt in London auf und erhielt durch Miss Hurrell eine nette Pension in Fulham, ich glaube Wandsword Bridge Road, in einer gefälligen \enquote{One family houses}-Street bei der Witwe eines Offiziers, die auch zwei Gerichtsbeamte, Junggesellen, beherbergte. Dahin siedelte auch ich bald über. Prause und ich sprachen verabredungsgemäß nur englisch miteinander.

An der kleinen Tafelrunde hörte ich auch einiges zum Thema \enquote{London im Weltkriege}: Großen Schrecken hatten die nächtlichen Zeppelinangriffe auf London hervorgerufen. Die Badewannen mussten Nacht für Nacht ganz voll Wasser sein um durch Bombenabwurf entstandene Brände rasch löschen zu können. Man konnte abends nicht mehr baden! Der von den Zeppelinen angerichtete Schaden war, gemessen am 2. Weltkrieg, lächerlich gering, dagegen die Verluste an Zeppelinen, die der englischen Abwehr so große Ziele boten, waren erschreckend hoch, sodass der Einsatz dieser praktisch nur moralisch wirkenden Waffe bald gestoppt wurde. Und die Engländer sind zäh, das zeigte sich auch 1940, und haben bessere Nerven als wir.

Einmal erzählte ich bei Tisch den beiden Gerichtsmenschen, dass ich am vergangenen Sonntagnachmittag im schön an der Themse gelegenen Vorort Richmond in einem netten Lokal getanzt habe. Die Reaktion der beiden Tischgenossen war seltsam, wie mir schien ein klassisches Beispiel des englischen Cant: Man entrüstet sich über einen solchen Faux pas, den man mir nicht zugetraut habe, am Sonntag zu tanzen! Aus keiner Miene, keiner Tonnuance konnte ich, der danach forschte, entnehmen, dass es Scherz sei. \enquote{Cant} wurde während des Weltkrieges mit \enquote{Heuchelei} übersetzt, eher trifft schon \enquote{Unaufrichtigkeit}, eine Haltung, die einzunehmen, zur Schau stellen man sich aufgrund gesellschaftlicher Convenienz, besonders im Zeitalter der Queen, verpflichtet fühlt, auch wenn man sie im tiefsten Innern nicht teilt. Nichts deutete darauf hin, dass die beiden Engländer wirklich Puritaner waren, die es aus religiösen Gründen verurteilten. So muss ich also glauben, es war cant.

Übrigens aßen und sprachen wir, mein Freund Prause und ich, täglich mittags und abends zusammen mit den beiden Engländern -- aber auf der Straße kannte man sich nicht; das galt auch für alle anderen Pensionsbekanntschaften.

\section{Besichtigungen}
\marginpar{487}
Ich muss zunächst auf die große Empire Exhibition in Wembley zurückkommen. Damals war England mit seinem riesigen Anhang an dominions und colonies noch auf der Höhe seiner Macht. Wembley war eine imponierende Schau. Der Nachdruck war auf das Kommerzielle gelegt. Für mich war damals angesagt durch Wundt's Völkerpsychologie die englische Psyche der Hauptblickpunkt. Canada hatte u.a. im Erdgeschoss ein großes Wasserbassin mit allen Schiffahrtslinien ausgestellt: kleine Dampfer bewegten sich langsam von einem Hafen zum anderen. Oben an der Decke fuhren pausenlos die elektrischen Züge der Canadian Pacific Railway. Ein nettes Spielzeug, damals in Wembley, aber weniger für die Jugend als für die Erwachsenen, die eine halbe Stunde und länger dem schnellen Kreisen der Züge, dem Aufblinken und Verlöschen der Lichter u.s.w. gebannt zusehen konnten. Eine riesige Figur aus Canadischer Butter stellte den Prince of Wales dar, während Australien u.a. den King aus australischem Käse in kyklopischem Ausmaß den staunenden Betrachtern zumutete. Auf guten Geschmack ist in England weniger als in Frankreich Verlass.

Im Vergnügungspark war für grandiose thrilling sensations gesorgt. \enquote{The Water Chute} war wirklich toll: wir nahmen etwa 20 Mann in einem Boot Platz von der Form einer Fluss-Seil-Fähre, wurden in einem riesigen dunklen Fahrstuhl etwa 20 cm hoch gehievt, dann öffnete sich vor uns ein Tor, das Boot bewegte sich nach vorn, man schaute hinab in die Tiefe auf einen kleinen See, dann kippte das Boot nach vorn und sauste in die Tiefe. Alle stießen einen furchtbaren Schreckensschrei aus, das Boot klatschte hart auf die Wasserfläche und befand sich 2 Sekunden inmitten eines weißen Wasserwalles. Als wir ausgestiegen und wieder zu uns gekommen waren, hatten die Engländer nur ein Urteil: \enquote{It was awfully thrilling, wasn't it?} Ich musste ihnen beipflichten.

Zu den großen Londoner Attraktionen, die man einfach gesehen haben musste, zählte im Jahr 1925 noch der Crystal Palace, ein riesiger 50~m hoher an romanische Formen anklingender Bau aus Stahl und Glas, für die im Westen erste große Weltausstellung im Jahre 1851, eine technische Attraktion für die ganze Welt, wie rund 30 Jahre später der Pariser Eiffelturm. Die Räume des Cristal Palace beherbergten Ausstellungen und dienten allerlei Volksbelustigungen; manche Abende wurden gegen 10 Uhr beschlossen mit einem großen Feuerwerk, das kurz vor 10 Uhr endete. Aus der eingetretenen Finsternis tauchte mit dem letzten Glockenschlag Punkt 10 Uhr ein riesiges Portrait des King George~V. auf, gebildet von tausend und abertausend Glühbirnen. Alles erhob sich, applaudierte, die Kapelle intonierte die englische Nationalhymne, die man stehend in Andachtshaltung anhörte. Anschließend strebten die Massen ruhig und ohne Hast den meist öffentlichen Verkehrsmitteln zu.

Der große Zoo im Norden Londons am Regent Park, der wohl manche Anregung dem Hamburger Hagenbeck verdankt, war viel besucht von Jung und Alt. Für Kinder wurde viel getan: sie konnten auf Elefanten in den breiten Allen reiten, auch auf Kamelen und anderen Reittieren. In einem großen Käfig war ein prächtiges Löwenpaar, sie legte sich auf den Rücken und suchte mit zärtlichen Gesten die Aufmerksamkeit ihres männlichen Partners auf sich zu lenken. Der verharrte jedoch regungslos neben ihr stehend, den Blick gähnend auf die Menge der Zuschauer gerichtet. Da näherte sich die Löwin des Nachbarkäfigs, trat bis dicht an die trennenden Eisenstäbe heran, und im gleichen Augenblick, in dem der Löwenmann des ersten Käfigs ihr langsam den Kopf zuwandte, sprang meine Löwenfrau jäh auf an das trennende Gitter, und beide Rivalinnen brüllten sich fürchterlich an, in eifersüchtigem Hass sich mit der Tatze zu treffen suchend.

Eine prachtvolle Eifersuchtsszene dieser beiden Königinnen der Wüste, ich dachte an die Königinnen unserer Breitengrade Krimhild und Brünhilde, Elisabeth und Maria Stuart, doch die Stäbe verhüteten Blutvergießen.

Eine besondere Attraktion war die Affeninsel, mehr noch für weibliche als männliche Besucher des Zoos: auf einer Insel, die vielleicht durch einen 30-40~m breiten Wasserarm vom Zuschauer getrennt war, tummelten und balgten sich mindestens 100 männliche Affen herum, denen ihr erzwungenes Zölibat nur allzu sichtlich zu schaffen machte. \enquote{Was ist denn das für ein roter Stock, den die Affen alle am Bauche haben?} fragte ein kleines Mädchen ihre Mutter. \enquote{Oh, that's nothing}, war die ausweichende Antwort.

Im War Museum, im Kriegsmuseum, spürte man damals im 7. Nachkriegsjahr hier und dort noch Hassgefühle, die den Künstler zu Geschmacklosigkeiten verführten: ein Riesengemälde stellte eine schöne Rheinlandschaft dar; im Vordergrund auf einer Terrasse steht eine englische Haubitzbatterie, die Rohre auf das rechte Rheinufer gerichtet, mit Shagpfeife rauchenden Kanonieren. Darunter in großen Lettern: The Guard of the Rhine -- die Wacht am Rhein. Geschmacklos deshalb, weil dieses alte defensive Kampflied aus der Zeit Napoleons damals jedem Deutschen noch etwas bedeutete und überdies nicht gegen England sondern gegen das eroberungslästige Frankreich unter Napoléon gerichtet war, in einer Zeit, in der Deutsche und Engländer Verbündete waren.

Als geschmacklos empfand ich auch das Grabmal des Unknown Warrior -- des Unbekannten Soldaten in Westminster Abbey, das man dort nach dem französischen Vorbilde geschaffen hatte. Aber das englische Denkmal hält den Vergleich mit der Schlichtheit und Ehrlichkeit des französischen nicht stand. Ici repose un soldat français mort pour la Patrie. Das englische Monument ist umrahmt von einer wortreichen Inschrift, die auch Antwort gibt, wofür der unbekannte englische Soldat gefallen ist: für die Freiheit der kleinen Völker, für den Frieden und die Gerechtigkeit u.s.f. -- nur für Eines ist er nicht gestorben -- für England.

Das war meines Erachtens nicht mehr als Cant zu entschuldigen, das schien mir üble Heuchelei.

Beim Besuch des London Museum fiel mir ein Wesenszug des Engländers auf, der mich schon auf der Empire Exhibition in Wembley belustigt hatte: jungenhafte Schaulust und Freude am Spiel. In einem der Museumsräume war das London von 1666, dem Jahre des großen Brandes, im Verhältnis 10:1 nachgebildet. Durch Druck auf einen Knopf konnte der Besucher die City \enquote{in Brand} setzen und auf weitere Knöpfe die Ausbreitung der roten Feuersbrunst beobachten. Die Erwachsenen, besonders Männer, konnten sich nicht sattsehen am Wachstum des Riesenbrandes; wieder und immer wieder drückten dieselben Besucher auf die Knöpfe und genossen immer noch einmal das Schauspiel -- eine jungenhafte Freude am Spiel. Mir schien, dass Nietzsches Wort: \enquote{in jedem Manne steckt ein Kind, das will spielen} seine Richtigkeit nirgends deutlicher als in England erweise.

Vom Besuch des Towers, des Bloody Towers mit seinen vielen Schauergeschichten, mit den \enquote{Beefeaters} und den Kronjuwelen ist weniger in mir haften geblieben als der White Tower Wilhelms des Eroberers, imposanter, viereckiger und vierstöckiger Wehrbau, der wahrscheinlich aus denselben Gründen weiß angestrichen wurde wie die zahlreichen Fliehburgen auf slawischem Boden mit dem Namen Beograd, Bjelgorod u.s.f. Den stärksten Eindruck machte auf mich die stilreine, schlichtwuchtige normannische Hofkapelle; ich habe oft im Herbst 1974 an sie zurück gedacht, als ich die überraschenden normannischen Baudenkmäler Siziliens kennenlernte, in denen die geistvolle Stilsynthese des großen Rogers~II. mit der byzantinischen und arabischen Kunst glanzvollen Ausdruck gefunden hat.

Damals waren die zahlreichen Kirchen Christopher Wrens noch alle intakt, von denen nur einige gut gefielen, die stark an italienische Renaissanceformen erinnerten, deren Stilreinheit sie jedoch meines Erachtens nicht erreichten. Die gewaltige Kuppel von St. Paul's Cathedral, weithin sichtbar, war wie wohl auch heute noch das Wahrzeichen Londons und fehlte auf keiner Silhouette dieser Riesenstadt. Sie war geschmückt mit einigen frommen Bildern der Praeraffaeliten, besonders Holmar Hunt's, dessen harte Farben mich sehr enttäuschten. Ein alter, wohl zahnloser Kirchendiener machte mit mir das bekannte akustische Experiment in der Kuppel, der Whispering Gallery. Beim Sprechen bewegte er die Lippen so wenig, dass ich genau hinsehen musste, um überhaupt eine schwache Tätigkeit zu entdecken.

Als Deutschem zeigte er mir eines der ersten Exemplare von \enquote{Luther's Barby} wie er Luthers Bibel aussprach.

Konnte ich die englische Renaissance mit ihren Übergängen zum Barock einigermaßen genießen, so galt das in geringerem Maße für die englische Gotik. Die meisten Kirchen schienen mir unfertig, wenigstens ihre Türme, deren Aufwärtsstreben mir beizeiten -- wohl aus Sparsamkeitsgründen -- gebremst und symbolisch durch jeweils vier dürftige Türmchen an den Ecken ersetzt schien. Eigenartig und interessant fand ich, z.B. in Westminster Abbey und besonders in Windsor, die fächerartige Deckengestaltung.

Ein gotischer Bau, allbekannt, ist in seiner Art imposant und für das, wir müssen jetzt sagen einstige Weltreich repräsentativ: das prächtige erst in der Mitte des 19. Jahrhunderts erbaute, von der Themse aus besonders wirksame, Parlamentsgebäude. Auf seine Bauart und Ausdruckskraft sind die gewaltigen holländisch-flämischen Stadthallen, wie z.B. die von Ypern nicht ohne Einfluss gewesen.

Natürlich ist London trotz dieser Einschränkungen eine Kunststadt ersten Ranges aufgrund der Schätze, die allein aus dem riesigen Empire zusammengetragen wurden. Ich nenne das Indian Museum. Nirgends in Europa kann man sich besser als in diesem Museum, in dem ganze indische Tempel stehen, indische Kunst mit ihrer für den Europäer schwierigen Symbolik studieren oder ostasiatische Kunst, z.B. herrliche bis zu 2~m hohe chinesische Vasen aus rotem Lack im großartigen Victoria \& Albert Museum.

\marginpar{498}
Die National Gallery am Trafalgar Square, eine der größten der Welt, besuchte ich besonders gern an den Sonntagvormittagen, wo regelmäßig von jungen Kunstgelehrten anhand von Originalwerken Vorträge über einzelne Gemälde, einzelne Künstler oder Stilepochen gehalten wurden. Ihr Nutzen war für mich ein doppelter, der sprachliche Gewinn und die Schulung des Auges sowie kunsthistorische Belehrung. Hierbei handelte es sich meist um die italienische Renaissance sowie das 17. und 18. Jahrhundert.

Die großen englischen Meister des 19. Jahrhunderts lernte ich in der Tate Gallery kennen, und zwar zunächst die Originale einer Reihe farbig guter bis leidlicher Reproduktionen bekannter Bilder der Praeraffaelite Brotherhood. Ich erinnerte mich meiner Vorkriegssemester, als ich mit dem Gedanken spielte, eine Dissertation über die Beziehung von Malerei und Dichtung zu verfassen.

In der Tate Gallery fesselten mich sofort zwei große Maler: Constable, der Maler typisch englischer Landschaften spätromantischen und realistischen Charakters; kein Künstler hat wie er auch das englische Klima so ausgezeichnet dargestellt: zahlreich sind seine stets sehr gut in frischen Farben gemalten englischen Landschaften, in denen es eben geregnet hat, noch regnet oder bald regnen wird. In helle Begeisterung versetzte mich der geniale Turner, der weder in Deutschland noch Frankreich damals -- von Fachgelehrten abgesehen -- bekannt war. Auch mein Freund Arendt, der Maler, kannte von ihm kaum mehr als den Namen. Großartig ist bei ihm die Leuchtkraft der Farben, er malt über den Impressionismus weit hinausgehend Bilder wahren Farbenrausches, nicht nur, wie die französischen Impressionisten, die Spiegelung der Natur auf die Netzhaut des Künstlers bevor der Verstand eine Deutung der Farbflecken vollzogen hat. Turner ist der Vater des Impressionismus etwa zwei Jahrzehnte vor den großen Franzosen, er ist aber in seinem späteren Bildern der Wegbereiter kommender Kunst-ismen wie des Expressionismus und des Tachismus.

Enttäuschend war dagegen für mich die Ausstellung der Royal Academy of Arts: sehr akademisch, glatt-konventionell gemalte Portraits bekannter Persönlichkeiten, repräsentativ mit Ordensschmuck, Sportbilder, Jagdbilder.

Am darauffolgenden Tage hielt ich mich schadlos an einer trefflichen reich beschickten Cézanne-Ausstellung.

In einem Zweige der bildenden Kunst schienen mir die Engländer -- wie übrigens auch die Russen -- zu versagen: in der Skulptur. Fand ich schon den künstlerischen Wert der Poets' Corner recht mittelmäßig, so kann ich mich auch nicht einer wirklich erfreulichen Leistung eines Engländers auf diesem Gebiet erinnern, wohl aber einiger Geschmacklosigkeiten. Ein einziger Bildhauer, Henri Moore aus Yorkshire, ist in den letzten Jahrzehnten weithin berühmt geworden. Mein persönliches Verhältnis zu ihm ist \enquote{reserved}.

\marginpar{501}
Jetzt, ich schreibe im Januar 1975, im Kneipbad Lauterberg im Harz, liest man in den Zeitungen Vorkommnisse in England, die 1925 einfach unvorstellbar waren: wachsende Aggressivität der Fahrgäste gegenüber den Busschaffnern (Tötung zweier Schaffner), weil ihnen angeblich zu hohe Preise für die Fahrkarten abverlangt wurden, Ausschreitungen bei Umzügen, Widerstand gegen Polizisten.

Damals bestanden die öffentlichen Verkehrsmittel außer einem sehr ausgedehnten Netz von underground und tube mit ihrer in Deutschland noch kaum anzutreffenden Rolltreppe, aus ein paar letzten Trambahnen, aus sehr zahlreichen Doppelstockbussen; der obere Stock war völlig offen und ungeschützt, bei Regen konnte man sich vor allem die Beine mit einer an der Rückenlehne befestigten Lederdecke bedecken, wovon ich einige Male Gebrauch gemacht habe. Abends und nachts war oben kein Licht, der Schaffner nahm das Geld ohne es nachzuprüfen -- er wusste, dass es stimmte, das war damals in England Ehrensache.

Einmal fuhr ich abends mit meinem Freunde im Bus -- es versteht sich auf dem Oberdeck, wo man freien Blick nach allen Seiten hatte -- zum Picadilly Circus. Je mehr wir uns diesem Zentrum der City näherten, umso bunter und intensiver zuckten die Lichtreklamen, um am Picadilly Circus und dem benachbarten Leicester Square den Höhepunkt an Größe und Leuchtkraft zu erreichen. Manche der Riesenfarbbilder waren sogar schön, wie vor allem die Schnellzuglokomotive mit Tender und einem Wagen in mindestens natürlicher Größe, die in voller Fahrt befindlich und mit flackernder Rauchfahne am nächtlichen Himmel in kurzen Intervallen auftauchte. Da ich nicht die nötigen pennies bei mir hatte, zahlte ich mit einer 5 £-Note, damals etwa 100 Mark. Der Schaffner kehrte nach ein paar Minuten auf das dunkle Oberdeck zurück und übergab mir eine Handvoll Münzen. Am Picadilly Circus stiegen wir aus, zwängten uns durch die Menschenmassen und mein Freund rief, auf ein warnendes Plakat weisend: \enquote{Beware of pick-pockets!} (Achtung Taschendiebe) Ich tastete nach meiner Brieftasche -- oh Schreck! Ich war sie schon los! Rasch schaltend rannte ich zum gerade abfahrenden Bus zurück, erwischte noch die unterste Stufe des Trittbretts und stürzte die Treppe hinauf. Ich fand die Brieftasche vor den Beinen eines Engländers, der meinen Platz eingenommen hatte. Sie enthielt alle Ausweise und meine ganze Barschaft von £ 20! Nach 2 Schreckminuten holte ich beglückt Atem und stieg an der nächsten Haltestelle aus.

Die beiden Gerichtsbeamten in meiner Pension waren in Deutschland gewesen; besonders in Hamburg hatte es ihnen als Engländern gut gefallen bis auf\dots Schließlich rückten sie damit heraus: sie waren ein paar Mal von Kellnern geprellt worden. Sie fragten mich, ob mir denn in London eine Unkorrektheit widerfahren sei -- aus dem Ton der Frage war herauszuhören, dass sie fest mit einer verneinenden Antwort rechneten. Diesen moralischen Triumpf glaubte ich aus nationalen Gründen meinen beiden Engländern nicht einräumen zu dürfen und ich log, eine kleine Unredlichkeit sei mir auch schon passiert: man habe mir statt einer halben Krone (2,5 sh) nur 2 shilling herausgegeben. Meine beiden Gesprächspartner waren darob, so schien es mir jedenfalls damals, ehrlich niedergeschlagen.

Noch einen Vorzug des Engländers glaubte ich währen meines Englandaufenthaltes beobachtet zu haben: jegliches Fehlen des Neides, einer menschlichen Schwäche, die man in Deutschland so oft erleben konnte. Wenn Sonntag vormittags in Rotten Row die oberen Tausend zu Pferde ihren Luxus zur Schau stellten, konnte ich bei keinem einzigen aus den lowest classes, die zahlreich unter den Zuschauern waren, Worte oder Mienen des Neides beobachten.

Eine Merkwürdigkeit im Straßenbild, von den Engländern als etwas Alltägliches nicht weiter beachtet, fiel mir auf: richtige Dampflokomotiven mit Rauchfahnen transportierten schienenlos Güter in 1-3 Anhängern durch die Straßen. Mir wurde so bewusst, dass ich im Geburtslande der Dampfmaschinen war. Namen wie Stevenson, Stockton-Darlington tauchten aus den Kellern meines Gedächtnisses auf.

\marginpar{504}
Schauspiel und Kino habe ich während der season, die Anfang August zuende war, als die Frauen mit einem Karren oder großem Korbe bewaffnet in den Straßen \enquote{[buy] lavender, sweet lavender} ausriefen, nur ein- bzw. zweimal besucht. Mein Gehör war noch zu wenig geschult als dass ich das schnell gesprochene Englisch wirklich befriedigend hätte verstehen können. Auf der Bühne war gerade Tschechow aktuell; so kam mir zustatten, dass ich den Inhalt des Stückes aus der russischen Literaturgeschichte kannte.

Die Aufmachung in den Kinos war sehr nett: man wurde von einem ausgesprochen hübschen und mit einem eleganten Federhut geschmückten Girl auf seinen Orchester- oder Parkettplatz geleitet.

Um Sport habe ich mich nicht gekümmert, jedoch wurde von verschiedenen Seiten meine Unkenntnis des Cricketspiels bemängelt, und Miss Hurrell ruhte nicht eher, bis sie mir die Spielregeln erklärt und beigebracht hatte.

Allgemeiner Wertschätzung -- ohne Klassenunterschiede -- erfreute sich auch das Boxen. Gerade von weiblicher Seite wurde ich heftig attackiert, wenn ich diesen Sport als verrohend ablehnte; er sei für die Erziehung der Jugend, besonders der männlichen, außerordentlich wichtig.

In -- ich glaube Wimbledon -- schaute ich einmal dem vornehmsten und elegantesten englischen Sport, dem Polospiel, zu: Reiter und Pferdchen schienen in rasch wechselnder Haltung und Figur zu einer wahrhaft entzückenden Einheit verwachsen. Den sozial 2. Rang nahm das Golfspiel ein, während Tennis nur noch als das Spiel des mittleren und kleinen Bürgertums galt, während Rad- und Fußballspiel sich mit den untersten Sprossen der sozialen Rangleiter begnügen mussten. Entsprechend die Rangordnung der Klubs: der vornehmste der Poloclub, das Schlusslicht der Radfahrerclub.

Zu einem Abendtrunk ging der Engländer, aber auch die Engländerin, keineswegs immer in Gesellschaft in eines der zahlreichen, meist kleinen Kneipen, in das public house, kurz pub, einen nur gedämpft beleuchteten Raum mit nur ganz wenigen Tischen. Man setzte sich in der Regel nicht, stand an der Theke, um ein ale (Leichtbier), stout (Starkbier, bis zu 20\%) oder gin (gingerbier) für unsere Begriffe rasch zu trinken. In der Regel war um 10 Uhr abends Polizeistunde, sonntags auch 11 Uhr abends, die rücksichtslos innegehalten wurden. Diese Pubs trugen oft -- außer irgend einem Namen -- noch die Bezeichnung \enquote{private bar} oder \enquote{ladies only} über einem zweiten Eingang. Beide Bezeichnungen bedeuteten dasselbe: nur für Ladies. Ging man um 10 Uhr abends an solchen \enquote{private bars} vorbei, konnte man Frauen einzeln oder untergehakt zu zweit mit leichter Schlagseite den Heimweg antreten sehen. Wenn ich in englischer Gesellschaft vorsichtig auf meine Beobachtungen anspielte, erhielt ich prompt die Antwort, dieser Hang sei auf das sehr feuchte englische Klima zurückzuführen. Weder in den Pubs noch in den Restaurants und Hotel wurde, wie damals in Deutschland üblich, nach dem Ober gerufen oder wie anderwärts auf dem Kontinent üblich, in die Hände geklatscht, nein, das geschah ausschließlich geräuschlos: \enquote{catching the waiter's eye}.

Derselbe Brauch herrscht sogar an höchster Stelle, im Parlament. Will ein Abgeordneter sprechen, so muss er die Aufmerksamkeit des Präsidenten, des Mister Speaker, einzufangen suchen. \enquote{He must catch his eye}.

\marginpar{507}
Einer Londoner Besonderheit, die später vereinzelt auch auf dem Festland Schule machte, möchte ich hier noch kurz gedenken: der pavement artists, der Pflasterkünstler. Sie malten in bunter Kreide Bilder nach berühmten Vorbildern, hier und da auch aus eigener Phantasie, auf die Trottoirs breiter Straßen. Daneben lag der Hut zur Aufnahme von Geldspenden der schaulustigen Passanten.

Der deutsche Nibelungenfilm, eine künstlerisch wirklich gute Leistung, hatte damals in Deutschland und auch im Ausland, z.B. Frankreich, großen Erfolg. In England musste er nach kurzer Spielzeit abgesetzt werden. Ich fragte meinen Volksschulrektor, bei dem ich einige Konversationsstunden nahm, nach den Gründen. \enquote{Dieser Film ist für den Durchschnittsengländer zu anspruchsvoll} \enquote{it is too high-brewed for the English}, war die aufschlussreiche Antwort.\\

Die bauliche Struktur Londons hatte für mich etwas Überraschendes: der Kern der Stadt, die City, ein Gewirr von sehr engen Straßen mit hohen 7-8-stöckigen Häusern. Nach den Außenbezirken zu, in allen Himmelsrichtungen ging die Höhe der Häuser schrittweise, hier schneller, dort langsamer zurück und sank in den Wohnvierteln, die mir planlos mit kleinen Industriestädten durchsetzt schienen, rasch auf 3-2 Stockwerke ab, d.h. Erdgeschoss und 1-2 Stockwerke, sodass die Wohnhäuser kaum so hoch waren wie in unseren Kleinstädten. Es herrschte völlig das Einfamilienhaus als Reihenhaus vor, unser großstädtischer Mietskasernentyp, wo die ganze family \enquote{flat}, d.h. flach auf einer Ebene wohnte, war durchaus unbeliebt. Das One-family-house, das man in der Regel sein Leben lang bewohnte, war die beste Verkörperung des englischen \enquote{my home is my castle}. Abgeschlossenheit nach außen, Bequemlichkeit und Behaglichkeit im Innern. Dies galt für alle Schichten des Bürgertums und genau so für den Arbeiter. Diese Wohnform schien mir geradezu auf das englische Familienleben zugeschnitten, das selbst in der Weltstadt London dem Anschein nach patriarchalischer und geschlossener als bei uns war.

Die Häuser wurden von großen Baufirmen hergestellt, ganze Straßenzüge von Häusern einheitlichen Gepräges, was es bei uns damals seit dem Klassizismus am Ende des 18. Jahrhunderts nicht mehr gegeben hatte. Gewiss, die Reihenhäuser in den älteren Arbeitervierteln aus der zweiten Hälfte des 19. Jahrhunderts, so z.B. besonders rechts der Themse, waren schmucklos, aus billigem Baumaterial, eintönig und langweilig; doch die meisten jüngeren, nicht mit allzu billigen Mitteln gefertigte Straßenzüge wirkten in ihrer Geschlossenheit auch gefällig und waren besonders dann interessant, wenn sie nicht schnurgerade verliefen, sondern sich dem Gelände oder einer kurvigen Straße anpassten.

\marginpar{511}
Alle Einfamilienhäuser, auch die des gehobenen Bürgertums, waren schmal und vertikal strukturiert. Das One-family-house von Dr. Ellis, in der Nähe des Hyde Park gelegen, Typus des gehobenen Mittelstandes, war folgendermaßen angelegt: Im Souterrain, engl. basement, waren Küche, Keller und die Wohnräume des Hauspersonals, hier des Butlers und der Köchin, untergebracht; darüber im Erdgeschoss Salon und Esszimmer, im 1. Stock die Schlafzimmer für die Eltern und zwei Gästezimmer, darüber im 2. Stock Kinderzimmer bzw. Gastzimmer, darüber die Abstellräume des Dachbodens. Hinter dem Haus ein kleiner Garten.

In Fulham, einer Wohngegend der middle-class und lower middle-class waren die Häuser nur teilweise unterkellert, im Erdgeschoss befanden sich außer Küche und einem Nebengelass ein großes Wohn- und Esszimmer, darüber drei kleinere Wohn- bzw. Schlafzimmer und darüber im 2. Stock abermals 3 kleine Wohn- bzw. Schlafzimmer. Die Einfamilienhäuser waren in der Regel für vier Kinder mindestens berechnet. Jede Etage besaß WC, Dusche und Bad. Hinter dem Haus befand sich ein schmaler, durch hohe Ziegelmauern gegen die Nachbarn abgegrenztes Gärtchen.

Der Typus des Eigenhauses war auch in den ärmeren Vierteln, räumlich enger und über dem Erdgeschoss nur ein Stockwerk, anzutreffen. Die Bewohner war hier meist nicht Eigentümer, sondern zahlten niedrige Mieten. Auch in diesem untersten Bevölkerungsschichten zeigte sich die Abneigung gegen die flats.

Von meinem Reichenbacher Chef, Direktor Mittag, mit dem ich in Briefwechsel stand, hatte ich erfahren, dass der Kulturminister von Neuphilologen, die auf Staatskosten einen längeren Studienaufenthalt im Ausland durchführten, außer dem üblichen Kursbericht von 5-6 Schreibmaschinenseiten in deutscher Sprache eine wissenschaftlich fundierte Arbeit von etwa 25-30 Schreibmaschinenseiten in der betreffenden Fremdsprache über ein selbstgewähltes kulturkundliches Thema erwartete. Diese ministerielle Verfügung war natürlich für einen Neuphilologen, der zum 1. Mal im Ausland ist und sein Hauptaugenmerk auf seine fremdsprachliche Vervollkommnung zu richten hat, nicht sehr sinnvoll. Ich fügte mich und begann alsbald mit Notizen zu dieser Arbeit, für deren Fertigstellung in so kurzer Zeit ich notgedrungen englische Literatur, mehr als mir lieb war, herausziehen musste.

\marginpar{514}
Dass der Handel und mit ihm die Banken in England wohl eine noch größere Rolle als auf dem Kontinent spielen, fand und findet noch seinen Ausdruck in den Bank Holidays, den Bankfeiertagen, die jedoch wahre Volksfeiertage sind, an denen im Gegensatz zu den Sonntagen weder Theater noch irgendwelche anderen Vergnügungsstätten geschlossen sind. Der erste Montag im August war damals geradezu \underline{der} große Volksfeiertag des Sommers, der praktisch drei Tage, von Sonnabend bis Montag dauerte. Wie ein großer Teil der Londoner machte auch ich mit meinem Freunde einen Ausflug, und zwar nach Brighton, das \enquote{London by the Sea}, mit seinem steinigen Strand (pebbles), den die meisten Menschen nur mit Pantoffeln oder Strandschuhen betreten können und dessen an Rummelplätze erinnernde Architektur längs der Strandpromenade ich nicht in bester Erinnerung habe. Um 22 Uhr kamen wir durstig wieder in London Victoria Station an, aber es gelang uns nirgends noch zu einem Glas Bier zu kommen. Bobbies sagten uns, das könne man nach 10 Uhr abends nur in Clubs bekommen.

Halbtagsausflüge führten mich u.a. nach Hampton Court, wo ich den Tudor Stil kennenlernte, der merkwürdigerweise im historistischen 19. Jahrhundert auch bei den Schlössern des schlesischen Adels Eingang gefunden hatte: mit dem Anbau der Zuckerrübe studierte man deren Verarbeitung in England, nahm Kontakt zum englischen Adel auf und importierte außer Maschinen eine Neigung zum Lebens- und Baustil der englischen Aristokratie.

Ein Tag war Windsor und dem nahen Eton gewidmet, dessen berühmtes College auch baulich interessant ist. Der Pedell erläuterte uns u.a. den Vollzug der körperlichen Züchtigung an Schülern bis zum vollendeten 18. Lebensjahr und zwar durch einen \enquote{hierfür besonders geeigneten teacher} unter Hilfestellung des Pedells.

Meinen Studienaufenthalt beschloss ich gemeinsam mit meinem Freund Prause mit einem einwöchigen Besuch von Oxford, der wirklich entzückenden Universitätsstadt, von deren Colleges -- es war die Zeit der Sommerferien -- wir eine ganze Reihe besichtigten. Eindruck: beneidenswerter Reichtum an behaglichen Wohn- und Studienräumen, an Bibliotheken inmitten von greens und cricketgrounds -- aber eben für eine begüterte Oberschicht gedacht. Eindrucksvoll die Hauptstraße High Street, hochinteressant die Rodelian Library ihrem riesigen Schatz an Handschriften und Inkunabeln.

Auch in Oxford bemerkte ich, dass weniger die High Curch als die Low Churches, die Sekten Kirchen, Strahlkraft besaßen. Waren es in London die fröhlichen Melodien zu den Kirchenliedtexten, so beeindruckten mich in Oxford die Predigten, besonders eine ausgezeichnete Interpretation von Tschechows \enquote{Cherry Orchard}, ein Stück, das damals in England gerade aktuell war. Man war in einer wirklichen Gemeinde, wo jeder den anderen kannte. Der Pfarrer hatte mich als fremden Vogel schon während der Predigt erspäht und schoss nach dem Gottesdienst auf mich los, um mich in ein mir sehr willkommenes Gespräch zu verwickeln.

Im klassischen Lande des Sportes lockte es mich, wenigstens etwas Wassersport zu treiben: Die schon in den westlichen Außenbezirken Londons sehr schmale Themse mit der trägen Strömung eines Steppenflusses erfreut sich bildschöner Ufer. Der Mensch hilft hier in England der Natur mit großem Geschmack durch Bepflanzung mit Sträuchern und Bäumen; alles wirkt ganz natürlich ohne jede aesthetische Effekthascherei, von der man in Paris und Umgebung nicht frei ist. Viele Engländer, einzeln und in Familien, brachten besonders das Wochenende in ihren Houseboats oder in Booten liegend und lesend zu. Manche stakten auch in ihren flachen vorn und hinten abgeklappten Booten den nicht tiefen Fluss auf- und abwärts. Dieses punting machte mir Spaß; man musste Acht geben, dass man die Stange, den punt, an den oft sumpfigen Stellen nicht zu tief drückte, sonst konnte es einem passieren, dass man die Stange nicht herausbekam und dabei ins Wasser fiel.

Eine schöne Tagestour im Bus führte uns in die Shakespeare-Stadt Statford-upon-Avon und dann weiter zur mächtigen Feudalburg Warwick Castle, die schon damals nach dem 1. Weltkrieg im amerikanischen Besitz übergegangen war. Nirgends hatte ich so zahlreiche vorzügliche Porträts von der Hand Holbeins gesehen wie hier. Auch Rubens und van Dijck waren bestens vertreten.

Über den schönen alten, prächtig mit Blumen geschmückten Badeort Lea\-ming\-ton gelangten wir zur Endstation unseres Ausfluges, dem geschichtsträchtigen und durch Scott weithin bekannten Feudalschloss Kenilworth. Unterwegs kam man auf der Hin- und Rückfahrt durch saubere alte Dörfer, die massiven Häuser mit hohen großen Schornsteinen, die z.T. sogar architektonische Funktion hatten, eingebettet in eine anmutige, leicht wellige Landschaft, die weite Strecken den Charakter englischer, d.h. natürlicher, aufgelockerter Parks hatte. Die landwirtschaftlich genutzten Flächen traten demgegenüber zurück. Während des 1. Weltkrieges musste England 5/6 seines Bedarfs an Nahrungsmitteln aus den Dominions und Colonies beziehen!

\section{Rückkehr über Frankreich}
\marginpar{520}
Mich verlangte es, statt den für die Heimfahrt kürzesten Weg über Köln-Berlin zu wählen, den Umweg über Paris und Dijon zu machen. Bei der Passkontrolle in Boulogne bekam ich zu spüren, dass die Erinnerung an den 1. Weltkrieg noch nicht verklungen war: während die mit mir das Fährboot verlassenden Engländer und Franzosen rasch am Fließband abgefertigt wurden, stutzte der Kontrolleur beim Anblick des deutschen Passes, hielt mich mit \enquote{arrêtez!} zurück, fertigte erst alle anderen Passagiere ab und unterzog dann meinen Pass einer eingehenden Prüfung. Wenige Minuten später sauste der boat-train Boulogne-Paris durch die nordfranzösische Landschaft. Die Engländer meines Abteils zogen ihre Sandwiches hervor und meinten lachend, dies könne ihr \enquote{last meal}, ihre Henkersmahlzeit sein -- eine Anspielung auf die damals sehr häufigen Eisenbahnunfälle in Frankreich. Der Anlass: die Zugführer erhielten damals Prämien, wenn sie Verspätungen rasch wieder aufholten, was nicht selten zu überhöhter Geschwindigkeit und zum Überfahren des Haltesignals führte. In Frankreich kursierte damals die Scherzfrage: Was bedeutet RF (Republique Française) über dem Hauptportal der Bahnhöfe? Die Antwort: Rencontres fréquentes (häufige Zusammenstöße)

Beim Verlassen der Gare du Nord hätte auch ein Blinder an der stärkeren Intensität, der größeren Differenziertheit und der höheren Tonlage auch der Stimmen gemerkt, dass er nicht mehr in England ist. Während man sich dort schon um eine Dämpfung des Straßenlärms, besonders um Vermeidung des auf einen einheitlich tiefen Ton abgestimmten Hupens bemühte, herrschte in Paris ein lustiges Hupkonzert in allen Tonarten, und dem Sergent de ville gehorchte man keineswegs so widerspruchslos wie dem König Bobby in London. Gleich vor dem Bahnhof war ich Zeuge eines lautstarken, beiderseits von lebhaften Gesten begleiteten Auseinandersetzung zwischen einem \enquote{flic} (wie der Pariser Schutzmann oft geringschätzig genannt wurde) und einem Taxichauffeur.

Da in Frankreich infolge der dort eingerissenen Inflation die Lebenshaltung für Engländer und auch Deutsche dank der \enquote{festen} Rentenmark ziemlich billig war, stieg ich gleich am Bahnhof in einem besternten Hotel mit englischem Namen ab, fuhr in mein Zimmer hinauf, war beglückt, dass es bereits mit einem Zimmerklosett ausgerüstet war, benutzte dieses eiligst und\dots oh Schreck! es fehlte das Abflussrohr! Moral: das Bidet ist nicht für große Geschäfte zu benutzen!

In Paris fühlte ich mich wohler als in London, durchfuhr mit Métro, Straßenbahn oder Taxi (nur doppelt so teuer als die Breslauer Straßenbahn), sah was sich seit 1913 verändert hatte -- Spuren von den Angriffen der deutschen Gothas (Flugzeuge) und der Bertha (des weittragenden Geschützes) waren nicht zu erkennen. Ich hatte darüber gelesen in der rührend tragischen Liebesgeschichte Pierre et Luce von Romain Rolland und hörte einiges über kühne deutsche Flugzeuge, besonders die \enquote{Taube de 5 heures}\footnote{\enquote{Taube} war ein Flugzeugtyp des Flugpioniers Igo Etrich}, die an mehreren Nachmittagen sich aus großer Höhe bis dicht über die Innenstadt herunterschraubte und das Kriegsministerium mit MG beschoss, während die Straßenpassanten entsetzt Schutz in den Kirchen suchten.

\marginpar{524}
Ich besuchte die Große Internationale Kunstgewerbeausstellung und in ihr mit besonderem Interesse den Sowjetischen Pavillon, ließ mich noch einige wenige Tage im herrlichen Quartier Latin nieder, saß tags lange im Eckcafé an der Place de la Sorbonne, sofern ich nicht eines der bekannten Museen durchstöberte, machte mit Cook einen Tagesausflug nach Fontainebleau und Barbizon -- mit jener Reisegesellschaft, die solche amerikanischen Titel herausgebracht hatte wie \enquote{How to do the Louvre in one hour} und \enquote{How to do Europe in a week} -- und verließ diese großartige Stadt mit der Absicht, sie bald möglichst länger zu besuchen.

Nostalgiegefühle erweckte in mir auch der Besuch Dijons. Am ersten Morgen eilte ich in die Stadtbibliothek: hatte M. Regnandin den Weltkrieg überstanden, war er noch im Amt? Ja, er war es, man holte ihn mir herbei, er lief sofort zu seinem Chef und ließ sich den ganzen Tag vom Dienst befreien. Er zeigte mir, was sich seit 1912 in Dijon verändert hatte -- Dijon war gewachsen, Elektroindustrie hatte Wein, pain d'épices und moutarde de Dijon überflügelt; die Kunstakademie war mit der von Lyon zusammengelegt, einige der schönsten Bilder des reichhaltigen Kunstmuseums, besonders zwei Rubens und einige ausgezeichnete niederländische Genrebilder waren nach Paris gewandert. Der Cours du Parc, die Prachtstraße, die zu dem einst von Lenôtre angelegten, jetzt etwas verwilderten Park führte, war mit zwei monumentalen Denkmälern zur Erinnerung an Marneschlacht und Schlacht um Verdun geschmückt, beide die zähe französische Widerstandskraft gestaltend. Wir déjeunierten zusammen, aßen escargots de Bourgogne und tranken roten Burgunder -- aber die Gedanken Regnandins kehrten immer wieder auf den Weltkrieg zurück, der zwischen unseren beiden Völkern einen so tiefen Graben aufgerissen hatte, dass an ein wirkliches Vergessen des Geschehenen, eine Aussöhnung oder gar Freundschaft der beiden Nationen nicht zu denken sei.

Wir schieden am Abend in aller Herzlichkeit und doch -- besonders ich -- recht traurig.

Von den Bekannten von einst konnte ich nur noch Madame Beaugey ausfindig machen, die in einem bescheidenen Anwesen in der Nähe des berühmten Weinortes Gevrey-Chambertin wohnte. Wir sprachen pausenlos, ohne Erfrischung, über fünf Stunden miteinander: ihr vielgeliebter bayerischer Freund von einst, ein cand. phil., war in Frankreich gefallen. \enquote{Il est resté dans la terre de France}. Aber auch sie glaubte nicht an die Möglichkeit einer deutsch-französischen Verständigung, einer echten Aussöhnung von der ich träumte, -- nein, das seien romantische, wirklichkeitsfremde Ideen.

In Besançon übersprang ich einen Zug, stieg in eine der Pferdedroschken am Bahnhof und fuhr eine Stunde lang kreuz und quer durch die Stadt, das alte Vesontio, dessen geographische Lage Caesar im Bellum Gallicum so exakt und noch heute zutreffend beschrieben hat.

In Straßburg verweilte ich einige Stunden, die ich ausschließlich dem herrlichen Münster widmete. Auf dem Bahnhof kaufte ich die deutschsprachige elsässische Zeitung \enquote{Die Zukunft}. Als ich durch die Sperre ging, meinte der Fahrkartenkontrolleur auf die Zeitung weisend: \enquote{'S isch e güet Blättle!}, was mir wohltat.

Da ich Heidelberg nicht kannte, machte ich hier 24 Stunden Station und besuchte das Heidelberger Schloss. In der deutschen Bank wollte ich meine letzte Geldreserve, rumänische Noten aus dem Besitz meiner Mutter, einlösen -- ihr Wert war bei Antritt der Studienreise ca. 20 Rentenmark, wovon ich damals 3-4 Tage hätte leben können -- aber, oh Schreck! Diese Devisen waren gesperrt worden, ich erhielt nicht einen roten Heller dafür. So trat ich die Weiterreise nach Begleichung der Hotelrechnung mit weit weniger als einer Rentenmark an, wovon ich noch 15 Pf für die Elektrische in Breslau aufheben musste und fastete bis Leipzig, wo ich 2 Stunden Aufenthalt hatte. Der Magen knurrte hörbar, ich sog den verführerischen Duft der Würstelbuden ein, stellte jedoch betrübt fest, dass meine Barschaft nicht zu einer Bockwurst reichte und verzehrte gierig zwei altbackene Semmeln. In Breslau brachte mich nach Mitternacht der letzte Wagen der \enquote{Gürtelbahn} für 15 Pfg. in mein Quartier.

\marginpar{528}
\section{Zurück in Reichenbach}
In Reichenbach wartete viel Arbeit auf mich. Zunächst wohnte ich im \enquote{Gasthof zum Kynast}, da die alte Besitzerin der \enquote{Villa Cohn} mein Zimmer vermietet hatte, und fand schließlich Zimmer mit Vollpension in einer kleinen Villa -- in London würde man sagen \enquote{cottage} -- in der Unterstadt, die der Witwe eines kürzlich verstorbenen Majors a.D. gehörte, des mir bekannten Geschäftsführers des \enquote{D.O.B} (Deutschen Offiziers Bundes), dem ich im vergangenen Winter beigetreten war. Hier wohnte auch ein gebildeter, mir etwa gleichaltriger Angestellter der Deutschen Bank. Mein modischer Londoner Anzug mit Oxford Trousers gab Anlass zu Neckereien der Reichenbacher: man habe mich zunächst für einen Amerikaner gehalten.

Im Realgymnasium hatte ich viel nachzuholen, da meine Vertreterin, die ehemalige Petersburger Lehrerin, methodisch völlig versagt hatte und die Schüler sämtlicher Russischklassen mehr oder weniger verbummelt waren. Die 25 Wochenstunden in neueren Sprachen, zu denen noch meist zwei Vertretungsstunden kamen, beanspruchten mich mit ihrer Vorbereitung und den vielen Korrekturen, den schriftlichen Klassenarbeiten und Proben der schriftlichen Hausarbeiten, an sich vollauf, doch traten hierzu noch kurzfristig die Anfertigung 1) meines Englandberichtes (6 Schreibmaschinenseiten), 2) der Arbeit \enquote{The English One family-house} (ca. 30 Schreibmaschinenseiten) und die Vorbereitung eines 45-minütigen Vortrages über meinen Studienaufenthalt in England den ich auf Verlangen des Direktors des \ac{psk} auf dem Deutschen Philologentag zu halten hatte, der in der ersten Oktoberwoche während der Herbstferien in Breslau zusammentrat. Für diese drei Arbeiten standen mir kaum drei Wochen zur Verfügung, so dass ich mehr als die halbe Nacht zu Hilfe nehmen musste. Den Vortrag für die Philologentagung stenographierte ich vollständig, da ich noch nicht vor einem großen Gremium gesprochen hatte und das Lampenfieber fürchtete. Auf dem Wege zum Tagungsraum bekämpfte ich meine Unruhe erfolgreich mit zwei Cognacs. Mein Auftritt gelang, alte Herren sagten mir freundliche Worte über Vielfalt und Qualität meiner Beobachtungen und die angebliche Abgewogenheit meines Urteils über die Engländer und ihre Kultur. Auch der für mich zuständige Oberschulrat fand anerkennende Worte, worauf ich ihn sofort wegen der von mir erstrebten Anstellung in Breslau anzapfte!!!

Das \ac{psk} hatte während der großen Ferien meinem Vertreter in Französisch und Englisch, einem Studienreferendar, eine Vergütung von RM 80.- gezahlt, die von meinem Gehalt einbehalten worden war. Das Studium der einschlägigen Bestimmungen zeigte mir jedoch, dass der Studienreferendar keinen Anspruch darauf hatte. Ich bat nun das \ac{psk} nicht etwa um die Rückzahlung der zu Unrecht einbehaltenen 80.- RM, sondern -- was eine Behörde damals grundsätzlich nicht konnte -- mir die Richtigkeit meiner Auffassung der einschlägigen Verfügungen zu bestätigen. Meine Rechnung ging auf: der gestrenge Chef des \ac{psk}, auf eine unbedingt makellose weiße Weste bedacht, konnte mir nicht den von ihm begangenen Irrtum bescheinigen. Statt dessen kam ein von ihm persönlich gezeichnetes Schreiben: Angesichts Ihrer ungewöhnlichen wirtschaftlichen Notlage haben wir Ihnen eine einmalige Beihilfe von 250 RM bewilligt.

Direktor und Kollegium der Reichenbacher Schule staunten: kinderreiche alte Studienräte, die in der noch herrschenden Zeit der Deflation und der sehr knappen Beamtengehälter sich um eine einmalige Unterstützung an das \ac{psk} gewandt hatten, erhielten nur 80~M oder maximal 100 Mark, und dieser Studienassessor, Junggeselle, der sich in England neu eingekleidet hatte, erhielt 250 RM!! Dieser musste natürlich den Schulpapst zum Vetter haben!!

\marginpar{533}
Um den gesamten Englischunterricht des Reichenbacher Realgymnasiums einmal zu durchlüften, arrangierte ich einen Besuch des jugendlich frischen englischen Lektors, Mr. Wilson, von der Breslauer Universität; Direktor Mittag ging gern auf meine Anregungen ein. Vortrag und anschließende Aussprache mit den Schülern kamen überall gut an, die Schüler hatten das große Erfolgserlebnis, einen \enquote{richtigen Engländer} verstanden und mit ihm englisch gesprochen zu haben -- Auslandsreisen von Schülern in den Ferien gab es damals noch nicht.

Noch im Herbst überfielen mich im Unterricht, zufällig am gleichen Tag und ohne vorherige Ankündigung, der Stadtschulrat von Stettin und der Direktor des Realgymnasiums Trebniz, einer schön im Katzengebirge unweit Breslaus gelegenen Kleinstadt. Beide boten mir Anstellung zum 1.4.1926 an, ich zögerte jedoch meine feste Zusage hinaus in der Hoffnung auf Anstellung in dem Kultur- und Universitätszentrum Breslau. Schließlich attackierten mich -- mit 14 Tagen Abstand -- wiederum unangemeldet plötzlich im Unterricht erscheinend der Direktor des altberühmten Gymnasiums und Realgymnasiums zu St. Elisabeth in Breslau und darauf der Breslauer Stadtschulrat Dr. Lauterbach, die meine Anstellung an der genannten Breslauer Schule perfekt machten, zum Missvergnügen Direktor Mittags, der mich mit allen erdenklichen Versprechungen in Reichenbach zu halten versuchte. Er glaubte, ich wolle sicher bald Direktor werden und gab mir Tipps, wie ich diesem Ziel näher kommen könne. Ich entgegnete ihm jedoch -- so töricht war ich damals -- ich sei an einer solchen Stellung uninteressiert, ohne ihm den wahren Grund zu nennen: ich war ein so leidenschaftlicher Skiläufer geworden, dass ich nicht bereit war, die herrlichen Osterferien in Sonne und Schnee dem Beruf zu opfern -- und der Schulleiter hatte damals in den Osterferien den Eltern zur Verfügung zu stehen in Rücksicht auf das nach Ostern beginnende Schuljahr.

Der Übergang nach Breslau und an das Elisabethgymnasium verlief völlig reibungslos, umso mehr als ich die enge Verbindung mit Breslau allein schon durch das Behalten meines Studentenzimmers auf der Monkampstraße, durch das häufige Verleben des Wochenendes in Breslau und Teilnahme am kulturellen Leben dieser Stadt nicht abgerissen hatte. Am Kollegium der neuen Schule gab es Spannungen zwischen einigen Gruppen untereinander und zwischen Lehrern und dem Direktor; ich ließ mich natürlich nicht, wie versucht wurde, hineinziehen sondern blieb zunächst der Outsider.\\

Der Winter war schneearm gewesen und ich kaufte mir Schlittschuhe um auf dem kleinen Teich am Rande der Unterstadt zu laufen: zwei Versuche verliefen völlig unbefriedigend, ich war mittlerweile derart an die lange und breite Basis der Skier gewöhnt, dass ich mich auf der kurzen und so schmalen Schneide der Schlittschuhe hoffnungslos verunsichert fühlte. Herrliche Osterferien im Schneebereich der böhmischen Bradlerbaude mit Skiausflügen zur Kasselkoppe, nach Spindelmühle, zur Prinz-Heinrich-Baude, zur Wiesenbaude und ins Weißwassertal brachte Erholung und Kräftigung. In der Pfingstwoche zog ich mich auf die auf halbem Südhang gelegene Leierbaude zurück und versenkte mich in die Oberstufenlektüre für Deutsch. Der Südhang des Riesengebirgskammes war Ende Mai schon schneefrei, bis etwa \num{1000} m Höhenlage blühten schon die Anemonen zur Freude der ersten Sommergäste; um die über 1200~m hoch gelegene Bradlerbaude sah es noch wüst aus, aber das \enquote{Soaneschnietzel} der Mutter Hollmann schmeckte köstlich wie immer.

\section{Ferienkurs in Grenoble}
Schon am letzten Schultag vor den Großen Ferien trat ich die Fahrt nach der französischen Alpen-Universitätsstadt an, um in ihr an einem Ferienkurs für Romanisten teilzunehmen. Die erste Nacht verbrachte ich in einer Pension unweit des Münchner Hauptbahnhofs und eilte am nächsten Morgen zum Schnellzug nach Lindau. Kaum war dieser in Fahrt, fiel mir mein leichtes Gepäck auf -- ich hatte meinen Koffer auf dem Münchener Hauptbahnhof gelassen! Ich unterbrach die Fahrt auf der ersten Station nach 1,5 Stunden in Büchen, telefonierte und bekam den Koffer bis zum Abend nach Lindau gesandt. Im Wartesaal kriegte ich weder Kaffee noch Milch zu trinken. \enquote{Ja, was haben Sie denn überhaupt?} \enquote{No Bier!}, war die echt bairische Antwort. In Lindau am Bodensee verblieb mir noch Zeit zu einer Dampferfahrt nach der Zeppelinstadt Friedrichshafen. Auf dem Dampfer nach dem schweizerischen Romanshorn hatte ich mit dem Philosophen und späterem Universitätsprofessor in Kiel, Dr. Kienast, bei Selchenessen\footnote{selchen = räuchern von Fisch oder Fleisch; vermutlich besteht ein \enquote{Selchenessen} hier aus geräuchertem Fisch} (aus dem Bodensee) eine anregende Abendstunde.

Übernachtung in Luzern, am nächsten Morgen Fahrt über den Vierwaldstädter See, inmitten von Engländern, die, wie ich beobachten konnte, sogar kleine schweizer Kinder englisch anredeten, es als ganz selbstverständlich hinnahmen, dass sie vom Fahrkartenkontrolleur englisch angeredet wurden (tickets, please, ladies and gentlemen), auch mich englisch anredeten, worauf ich jedoch verärgert wortlos mit Achselzucken reagierte. Nach Fahrtunterbrechung in Lausanne und Genf landete ich am späten Abend des nächsten Tages in Grenoble. In der Gaststube erzählte ein biederer Franzose, der auch mit dem Zuge gekommen war, über nicht oder nur vereinzelt noch besetzte Tische hinweg von seiner Reise in die Schweiz, nach Genf. Das Erstaunlichste für ihn, wie anscheinend auch für mehrere Gäste war dies: man versteht die Schweizer tadellos! Denken Sie doch, die sprechen ein richtiges Französisch fast genauso wie wir hier! wussten Sie das? Ist das nicht fabelhaft?

Am nächsten Tag ließ ich mich, da die Pensionen schon alle besetzt waren, im Hôtel d'Angleterre nieder, hatte aber am nächsten Morgen Grund, mich bei der Concierge über unliebsame Mitbewohner meines Zimmers zu beklagen. Als Beweis zeigte ich eine Wanze, die ich in der Nacht verhaftet und in meinen Notizblock gesperrt hatte. Die Concierge nahm den Vorfall nicht tragisch: sie werde meine Sachen in ein anderes, garantiert sauberes Zimmer bringen lassen. Am nächsten Morgen sprach sie mich lachend an: Und wissen Sie, wen ich in ihr gestriges Zimmer gesteckt habe? Einen fetten curé, dem können die Tierchen ruhig einiges abzapfen!!

Diese interessanten klugen Insekten hatte ich erstmalig 1913 in Breslau in der Kohlenstraße, dann 1914 in der Lessingstraße des Hansaviertels (unweit des Stadtbahnhofs Bellevue\footnote{Bahnhof Berlin Bellevue im Bezirk Mitte}), dann ausgiebig während des Krieges in Polen und Russland kennengelernt, danach 1925 auch in London-Bloomsbury Woburnstreet und zuletzt 1927 in Paris in einem Hotel des Quartier Latin. Im Winter 1915/16 hatte ich mich mit dem Gedanken getragen, diesen begabten Tieren eine Monographie zu schreiben, in der ein besonderes Kapitel der Taktik und Strategie der beiden feindlichen Parteien, Tier und Mensch, im Kampfe gegeneinander gewidmet sein sollte.\\

Grenoble war damals noch nicht das große Elektrozentrum von heute, sondern noch eine ruhige Provinzstadt von \num{80000} Einwohnern, allerdings begann sie schon \enquote{La Capitale des Alpes} zu werden. Herrlich war in den Abendstunden das Alpenglühen auf der hohen und nahen Bergkette der Belle Donne. Die Isère, der bedeutendste Nebenfluss der Rhône, führte bis an den Rand im Juli reißende Schmelzwasser. Im engen Talkessel von Grenoble herrschte tagsüber drückende Hitze, was jedoch -- zum Erstaunen und Verdruss vieler ausländischer Studenten -- die Universitätsleitung nicht daran hinderte, das Betreten der Universitätsgebäude und Hörsäle ohne Jacket, \enquote{en bras de chemise} streng zu untersagen. Erstaunlich war für alle Nichtfranzosen, dass die Cafés von 10-12 Uhr vormittags recht gut mit männlichen Franzosen besetzt waren. Offenbar war es schon Gewohnheitsrecht auch für alle Angestellten, in den Dienststunden mal eine halbstündige Cafépause außerhalb der Dienstgebäude einzulegen. Daran nahm eine Gruppe polnischer Studenten, darunter der intelligente Sohn eines Kattowitzer Bergwerkdirektors namens Riger Anton -- ich kam mit ihm durch meine im Krieg erworbene Sprachkenntnisse in Kontakt. \enquote{Die Franzosen schreiben das Wort \enquote{Arbeit} im Gegensatz zu euch Deutschen und uns Polen mit kleinen Anfangsbuchstaben. Wir haben viel Ärger mit ihnen, sie sind kleinlich-bürokratisch, haben in Bergbau und Industrie die Oberleitung, wirken sich als Hemmschuhe aus; mein Vater muss zur Beschaffung von Schreibmaterial Gesuche an die französische Oberdirektion richten -- und die Genehmigung lässt auf sich warten!} -- Riger hatte im übrigen Respekt vor der deutschen Leistung. \enquote{Vous êtes une race intelligente et dure}, meinte er.

Dank der Inflation, die sich 1926 in der ersten Jahreshälfte verstärkt hatte, war für Deutsche das Leben in Frankreich ziemlich billig. So mietete ich mir einen französischen Studenten mit dem Auftrage, mich ein oder zwei Stunden auf Spaziergängen und bei Besichtigungen zu begleiten und mein Französisch zu überwachen und insbesondere meinen Sprechstil zu verbessern.

Eine herrliche Tagestour im Autobus durch die Hochalpen, u.a. über La Grave und den Col du Lautaret, kostete mich nur 2 RM. Einige Engländerinnen stießen Freudenschreie aus, wenn wir an blühendem Lavendel vorbeifuhren -- der Autobus musste halten und sie rupften die geliebten Blütenstengel.

Engländer, besonders weiblichen Geschlechts, stellten damals, vor der Zeit des Massentourismus, das Hauptkontingent an ausländischen Touristen in der Schweiz, Frankreich, Italien, kurz fast allen Ländern Europas.

So saß ich in Grenoble mehrfach am Mittagstisch zusammen mit Franzosen und Engländerinnen, die kaum ein einziges Wort Französisch konnten. So kam ich mehrfach in die Lage zwischen Franzosen und Engländern(-innen) zu dolmetschen, was mir Spaß machte, aber infolge des großen phonetischen Unterschieds einigermaßen anstrengend war.

Ich traf in Grenoble einen meiner früheren Russischschüler, der 1925 in Reichenbach Abitur gemacht hatte, in München studierte und jetzt mit seiner Mutter die Sommerferien in Grenoble verbrachte. Er hatte sich hier in eine hübsche Polin verliebt, was ihn jedoch nicht hinderte, mit mir einen mehrtägigen Ausflug in die Meije-Region zu machen. Er war einer der begabtesten Schüler, die ich erlebte; äußerlich bescheiden, zurückhaltend. Er hatte in Prima ein Referat über Dostojewskis \enquote{Schuld und Sühne} übernommen: Ich betrete ahnungslos die Klasse. Quer durch den Raum ist eine Wäscheleine gespannt, an der mit Klammern neun Guaschgemälde in DIN A4-Grösse befestigt sind: expressionistische en-face Bilder. Was bedeutet das? \enquote{Erläuterungen zu meinem Referat. Es sind die Bewusstseinszustände Raskolnikows.} Einige Minuten staunenden, langsam verstehenden Betrachtens. Es folgt ein subjektiv geprägtes, stellenweise sogar tiefgründiges Referat. Lehrer und Mitschüler zollen ihm einhellige Anerkennung. Mein Freund, der Zeichenlehrer und freischaffende Maler Arendt meinte: in dem Jungen steckt etwas. Noch während seiner Studienzeit hat er in München ein Buch über romantische Malerei veröffentlicht.

Die Hitlerzeit hemmte seine Entwicklung. Er wurde ein Opfer des 2. Weltkrieges.\\

Mit der Sekundärbahn, großer Rauchfahne und viel Gebimmel gings in das Romanchetal zur Endstation Bourg d'Oisans. Unterwegs zwischen zwei Stationen ein jähes Bremsen -- wir stiegen aus; der Zug hatte eine alte Bäuerin überfahren. Wir halfen, den vordersten Wagen auf die Seite zu kippen, sodass die Leichenteile hervorgeholt werden konnten. Eine Bäuerin aus unserem Abteil wiederholte unzählige Male: \enquote{Enfin, ça ne me regarde pas, mais ça me fait peur quandmême.}

Gegen 10 Uhr abends langten wir in einem der beiden Hotels des Städtchens an. Der Wirt sagte uns im Hotelrestaurant, dass er uns zu so später Stunde außer Getränken nur noch \enquote{sandwiches} anzubieten habe. Ich fragte nach \enquote{lait -- caillé} -- saurer Milch. Er sah mich erstaunt und kopfschüttelnd an und meinte: wenn an heißen Tagen wie heute die Milch sauer wird, so geben wir sie den Kühen -- aber saure Milch als menschliche Nahrung? Verschwand in der Küche und kam freudig mit sandwiches und einer Schüssel saurer Milch zurück. Voilà, Monsieur, eine Schüssel ist übrig geblieben -- wenn Sie sie essen wollen\dots aber wie? Auf mein Verlangen brachte er Zucker herbei und bat zusehen zu dürfen, wie ich diese Speise esse. Ich streute Zucker auf die Satte und begann zu löffeln -- der Wirt streckte abwehrend beide Arme mit erhobenen Handflächen aus und rief: \enquote{Mais ça, Monsieur, ça n'est pas français!}

Am nächsten Morgen fuhren wir im offenen char à bancs-Autobus auf der schmalen, in den Fels geschlagenen und gegen den senkrechten tiefen Absturz nicht einmal durch ein Geländer geschützten Straße zur Bérarde unterhalb des Meije-Massifs hinauf. Ich saß am rechten Rand, beim Blick in die Tiefe drohte mir schwindlig zu werden, zwei Frauen vor mir bekamen Schreikrämpfe, während der Fahrt wurden ihnen die Augen mit Tüchern verbunden. Den Fahrer kümmerte das überhaupt nicht, mit der linken Hand hielt er das Steuer, mit der rechten wies er nach oben auf weiße Bergesspitzen, Almen und einzelne winzig erscheinende Gehöfte, sie mit Namen benennend. Alle atmeten auf, als wir wohlbehalten in der Bérarde anlangten.

Da damals noch unbedingter als heute déjeuner und dîner ähnlich den Messen an feste Zeiten gebunden waren und noch nicht durch Picknicken ersetzt werden konnten, blieben alle Franzosen in der Bérarde beschäftigt mit dem Studium der Speisekarte und der Einrichtung im Wochenendzimmer. Auf der Bank vor der Tür saßen ein paar mit allem ausgerüstete Alpinisten im Planungsgespräch mit französischen Führern. Ich mischte mich in die Unterhaltung und fragte die Franzosen, ob sie mich wohl einen Tag zum Rande des Meije-Gletschers mitnehmen würden. Mit einem kritischen Blick auf meine Haferlschuh \enquote{Mit dieser Ausrüstung um keinen Preis!} war ihre Antwort.

Mein junger Freund und ich nahmen die in der klaren Hochgebirgsluft am Ende des Tals zum Greifen nahe vor uns liegende Gletscherzunge aufs Korn und rückten ab, vorbei an Feldern von blühenden Alpenrosen und Sturzbächen voller Gletschermilch, die in fast regelmäßigen Zeitabständen hier und da an den Felswänden auftauchenden und rasch vergehenden hellen Wölkchen beobachtend: es waren kleine Staublawinen, hervorgerufen durch hochoben sich lösende und auf Geröllhalden aufschlagende Steine. Nach gut drei Stunden Marsches, auf dem wir uns über neue deutsche Literatur unterhielten und er sich als begeisterter Freund Rilkes zeigte, waren wir auf der Gletscherzunge angelangt, einer dunklen, angelöcherten Eismasse, die lebendig schien: kleine und größere Steine bis zur Größe eines Menschenkopfes, die sich unter den Strahlen der Mittagssonne aus ihrer Eisfassung gelöst hatten, kamen zu Tale gehüpft -- man musste achthaben! In siebenstündigem Marsch -- wir unterbrachen ihn zur Nachtruhe in ? (nicht St. Christophe!) kehrten wir nach Bourg d'Oisans und von dort mit dem Zugerl nach Grenoble zurück.

Das Ende der Schulferien nahte; nach dem Regierungsantritt Poincarés war der französische Franc rapide gestiegen, das Leben in Frankreich für Ausländer von Tag zu Tag teurer geworden. Ich fuhr über Valence die schöne Strecke über Marseille-Cannes-Nizza-Cuneo-Turin-Mailand-Lecco-Comer See-Sondrio-Tirano und von hier mit der \enquote{Elektrischen} über den Bernina-Pass nach St. Moritz. Herrlich der Blick auf den Bernina See unten, der mit jeder Kehre der Bahn immer ferner und kleiner, und der Berninagletscher immer drohender und größer wurde. Auf eine Fahrt durch Oberitalien hatte ich mich nicht vorbereitet, die Zeit, in der ich mich mit der italienischen Sprache beschäftigt hatte, lag weit zurück. Um mich wieder aktiv in ihr zurechtzufinden, sagte ich während der Fahrt einst memorierte Gedicht- und Liedtexte auf, bzw. bastelte sie wieder zusammen, und siehe da, ich fühlte mich auch wieder den Sprachsituationen im Restaurant und im Hotel -- ich übernachtete in Turin -- gewachsen. Unvergesslich ist mir der herrliche Ozongeruch am See von St. Moritz -- oder was ich für Ozon hielt -- sowie der gelinde Schrecken, der mich oben im Hotelzimmer befiel, als ich den in vier Sprachen gedruckten \enquote{Avis} las: Gäste, die ihre Mahlzeiten nicht im Hof einnehmen, gewärtigen einen 80\%igen Aufschlag auf den Zimmerpreis. Da dieser schon an sich gepfeffert war, sagte ich mir als guter Wahlschlesier: \enquote{Suste nischt, ok hem.}

\marginpar{Ende des Heft 4 am 31.3.02 in Barnave, Heft 5 begonnen 2.4.02}

\section{Landheim in Strickerhäuser}
Einige Abwechslung in den Alltag des Schullebens am Elisabethgymnasium -- manche Lehrer empfanden es als Störung -- brachte das Landheim im Riesengebirge. Die Elternschaft hatte ein Berghotel mit 24 Zimmern in schöner Lage in Strickerhäuser (750~m ü.M.) gekauft, im Grenzbezirk zwischen Riesen- und Isergebirge. Das Schullandheim, ausgehend von den Landschulheimen, war nach dem ersten Weltkriege besonders (aus finanziellen Gründen) in der Oberschulpädagogik Mode geworden. Unser Landheim lag am Südhang einer in die Tschechoslowakei hineinragenden und von der Iser umflossenen Landzunge. Von der Südfront und der Terrasse des Heimes bot sich ein schöner Blick bis weit ins Sudetenland hinein. Längs der tschechischen Grenze, die z.T. weniger als 200~m vom Haus entfernt war, zog sich ein breiter Streifen, seit vielen Jahrhunderten deutsch besiedelten Gebietes schlesischer Mundart; doch war in der langen Zugehörigkeit zur Donaumonarchie ein leichter Wellenschlag der Wiener Kultur in Sprache, Lebensstil, Kunst und Mode bis an den Kamm der Sudeten gedrungen. Die tschechischen Behörden waren entgegenkommend, wir durften Sportplatz und Turnhalle der großen Nachbargemeinde Harrachsdorf jenseits der Grenze benutzen; erst in der Hitlerzeit verschlechterten sich die Beziehungen. Die Tschechen verbarrikadierten die Straßen mit Panzersperren, unser Auslauf wurde etwas eingeengt.

Die Klassen wurden von einem oder zwei Lehrern begleitet. Die unteren Klassen fuhren meist im Sommerhalbjahr hinauf, während die Wintermonate vorwiegend den skifreudigen älteren Schülern ab Obertertia vorbehalten blieb. Es fand täglich ein 4-stündiger Unterricht in den Fächern statt, die der begleitende Lehrer vertrat, meist vormittags, doch auch wechselnd nach der Wetterlage. Die andere Tageshälfte war der körperlichen Ertüchtigung, dem Sport und Wanderungen vorbehalten. Abends geselliges Beisammensein in Musikgruppen, Vorlesen, Tischtennis oder individuelle Beschäftigung im Haus; zuweilen auch Vorträge mit oder ohne Lichtbilder.

Ein pädagogischer Nutzen des Landheimaufenthaltes ist unbestritten: der Lehrer konnte in den drei Wochen Zusammenlebens mit seinen Schülern jeden einzelnen weit besser kennenlernen als es im Klassenunterricht in einem ganzen Jahre möglich ist.

Mit älteren Schülern wurden auch Ausflüge mit der Bahn in die weitere Umgebung mit kulturkundlicher Zielsetzung unternommen, so z.B. nach der alt berühmten deutschen Glasbläserei Josephinenhütte bei Schreiberhau, nach Gablonz, der böhmischen Glasindustriestadt oder nach der schönen und schön gelegenen fast rein deutsch bevölkerten Stadt Reichenberg mit \num{60000} Einwohnern. Mit einer interessierten Prima waren wir -- hier schon mit meiner Frau Margot -- eines Nachmittags bei Wilhelm Bölsche in Oberschreiberhau zu Gaste. Bölsche, der mir schon seit meiner Primanerzeit u.a. durch seine Essay- und Aufsatzsammlung \enquote{Stunden im All} bekannt war, zeigte und erläuterte uns seine naturwissenschaftliche Sammlungen und erzählte von seinen Reisen.

Lehrwanderungen unter Mitwirkung eines Försters des Grafen Schaffgotsch, dem der größte Teil der Waldungen des Riesengebirges gehörte, erwiesen sich als anregend. Er erzählte u.a. vom Leben des in Deutschland im Aussterben begriffenen und in den alten Fichtenbeständen des Riesengebirges hier und da noch vorkommenden Auerwildes und von der Jagd auf den Auerhahn während der Balzzeit.

\marginpar{556}
In der Regel fand für jeden Landheimdurchgang einmal evangelischer Gottesdienst im Heim statt, zu dem auch besonders die älteren Leute der \enquote{Kleinen Kolonie Strickerhäuser} erschienen. Pastor Dr. Opitz aus Oberschreiberhau wusste seine stets gehaltvoll und mit großer Aufmerksamkeit gehörte Predigt dem jeweiligen Klassenniveau mustergültig anzupassen. Er verkehrte im Hause Gerhard Hauptmanns in Agnetendorf, und so erfuhr ich im Gespräch manches über diesen von mir geschätzten Dichter.

Einmal lockte es mich, mit einer geistig regen und theaterfreudigen Prima die Aufführung einer Wandertruppe in Polaun, einem sudetendeutschen Dorf unweit der Grenze zu besuchen. Wir hatten die beiden vordersten Reihen besetzt. Das Stück erwies sich als Schmarrn voll falscher Sentimentalität und wurde noch dazu von den harmlosen Schauspielern mit unfreiwilligen Sprachschnitzern und missverständlichen Fremdwörtern gespickt, so dass viele von uns das Lachen nicht unterdrücken konnten. Als schließlich der tragische Held von beklagenswertem Krankenlager glücklich erstanden, jedoch noch mit großer weißer Binde um den verwundeten Schädel unter dem pathetischen Ruf \enquote{Licht! Luft! Sonne!!} die wohlgedeckte aristokratische Frühstückstafel ansteuerte, da hielt es meine Jungen nicht mehr. Wie auf Kommando brachen alle in ein homerisches Gelächter aus. Und da geschah es: die Frühstückstafel verstummte und es erhob sich der Großfürst mit Halsorden, der dunkelgrüne Uniformrock mit weiteren glänzenden Orden geschmückt, er erhob sich und schritt zur Rampe, schlenderte vernichtende Blicke auf uns herab und stieß die folgenden mundartlich gefärbten Worte hervor: \enquote{Meine Herrschaften! Entschuldigen Sie die Störung. Hier vorn sitzt ein Publikum, das für das Verständnis meines Trauerspiels noch nicht die nötige Reife besitzt. Wir werden erst dann weiter arbeiten, wenn völlige Ruhe eingekehrt ist.}

\marginpar{558}
Die zürnende Hoheit hatte mit der Ansprache sofort Erfolg. Meine Jungen bissen sich von nun an den kritischen Stellen mannhaft auf die Lippen und hielten wacker durch bis zum Schluss.

Der Beifall der zahlreich erschienenen Dorfbewohner war stark und echt; meine Jungen applaudierten stürmisch.

In den nächsten Tagen kursierten im Heim lustige Notizen über den Theaterabend. Kein Zweifel, der Abend in Polaun war für alle ein nicht zu wiederholendes Erlebnis. Keiner von uns konnte sich eine Chance ausrechnen, noch einmal im Leben ein Rencontre mit einem russischen Großfürsten zu haben.\\

Zu Beginn der Herbstferien fuhr ich, aus Strickerhäuser mit einer Tertia zurückgekehrt, zunächst über Glaz nach Prag, wo ich drei schöne Tage mit meiner Schwester Ella verlebte. Wir wohnten privat im hochgelegenen Stadtteil Vinohradz auf dem rechten Moldau-Ufer bei einem älteren Ehepaar, die aus der Zeit der Donaumonarchie noch gut deutsch sprachen. Überhaupt waren die Tschechen, wenn man sie deutsch ansprach, entgegenkommend und freundlich, oft fügten sie hinzu: \enquote{Ach, Sie sind Reichsdeutsche!} Ihre Feindschaft galt dagegen der deutschböhmischen Herrenschicht von einst, die mir in der Person des deutschböhmischen Burschenschafters Knaur 1912 in Dijon begegnet war.

Selbstverständlich schlenderten wir über die ehrwürdige Karlsbrücke zur Kleinseite mit dem mächtigen Hradschin, der herrlichen Promenade über der Moldau, wir besuchten das Wallensteinpalais und Barockkirchen. Unvergesslich im \enquote{Kleinen Theater} im Haus der Deutschen, Graben Nr. 26, eine vorzügliche Aufführung von Tolstois \enquote{Das Licht leuchtet in der Finsternis.}

\marginpar{3. Heft von Mama, noch zu gliedern}
In den Weihnachtsferien 1925 hatte ich mir gelegentlich eines Skiaufenthaltes bei Schlechtwetterlagen von einem aufgeweckten tschechischen Kellner die korrekte tschechische Aussprache beibringen lassen, sodass ich unter Beachtung gewisser gesetzlicher Lautunterschiede zwischen Russisch und Tschechisch mich mit Erfolg im Gebrauch des Tschechischen versuchen konnte. Dann fuhr Ella heim nach Breslau, ich weiter nach Wien, wo ich mich mit Prauses traf. Gemeinsamer Besuch des Burgtheaters, wo der \enquote{Aiglou} von Rostand (in Wien \enquote{Äglohn} gesprochen) gut, aber pathetisch wie damals in der Pariser Comédie Française gespielt wurde. Im Kunsthistorischen Museum beeindruckte mich besonders Velasquez und im Lichtensteinpalais die Niederländer. Uns gefiel die Wiener Atmosphäre, auch die gemütlichen Cafés, wo man bei guter Melange las und seine Briefe schrieb. Einen schönen Herbsttag verlebten wir auf der Raxalp; von der Drahtseilbahn sah ich zum ersten Mal Gemsen in freier Natur.\\

Der gesellschaftliche Verkehr im Kollegium des Elisabethgymnasiums und der akademischen \enquote{Donnerstag-Gesellschaft} des sehr komfortablen Kaufmannsheimes \enquote{Iwinger} behagte mir nicht, da mir recht unverhüllt, mit dem etablierten Studienrat, heiratsfähige und heiratswillige Töchter vor die Nase gesetzt wurden. Verließ das mir zugedachte Mädchen den Raum, überboten sich die verheirateten Damen sofort mit Lobpreisungen der charakterlichen, geistigen und hausfraulichen Tugenden des prächtigen Geschöpfes.

In der \enquote{Donnerstaggesellschaft} hatte ich an einem Abend dreimal mit einer jungen Dame getanzt und mich leidlich mit ihr unterhalten, als sie mich gegen Mitternacht in den Nebenraum führte und mich ihren Eltern, Vater Magistratssyndikus, vorstellte und ich sofort zu einer Flasche Sekt eingeladen wurde. Solche Eile war mir verdächtig -- und ich zog mich zurück, zumal ich auch nicht der erhoffte flotte, auch die mittle-aged ladies herumwirbelnde Tänzer war.

Im übrigen nahm mich die Schule mit ihren sehr starken Klassen hart in Anspruch. Ich hatte mehrere Primen und Obersekunda in Deutsch, Französisch und sogar Englisch, für das ich nur die Mittelstufen Fakultas besaß. Ferner eine starke Primaner Arbeitsgemeinschaft (abends), in der ich fast ausschließlich neuere französische Literatur las und sie mit den Teilnehmern -- unter möglichsten Vermeidung der Übersetzung ins Deutsche -- französisch besprach. Wir lasen u.a. Flaubert \enquote{un coeur simple}, \enquote{Mme Bovary}, Jules Romain \enquote{Knock}, Hémon, Maria Chapdelaine (Roman aus dem frz. Kanada). Ab Ostern '27 übernahm ich den neueingeführten wahlfreien Unterricht in Russisch (nachmittags, 2 Wochenstunden). Hinzu kam, dass ich fast jedes Jahr eine Oberprima in Deutsch und Französisch also auch in schriftl. Prüfung zum Abitur führte.

\marginpar{565}
Im Sommer 1927 mietete ich eine etwas komfortablere 2-Zimmerwohnung mit Vollpension auf der Goethe Straße, unweit des Elisabethgymnasiums. Ella hatte inzwischen eine feste und einflussreiche Dauerstellung in Haus und Firma Fache erworben: ich war oft mit ihr und allmählich auch mit ihrem Chef zusammen. Er war Inhaber einer Großbäckerei, -fleischerei und Likörfabrik, sowie Weinhandlung, mit deren Produkten er seine 25 eigenen Gaststätten -- ähnlich den Berliner Aschinger Gaststätten -- versorgte. Es kam vor, dass, wenn ich mittags aus der Schule kam, schon der Fache-Chauffeur vor meiner Haustür wartete, um mich sofort zur Likörfabrik zu fahren, wo ich mit verbundenen Augen Qualitätsurteile über Liköre und andere alkoholische Getränke abzugeben hatte. Später machten wir sonntags zu dritt Ausflugsfahrten in die Berge. Emil Fache war der Typus des äußerst aktiven Selfmademans, der sich vom Gedanken an sein Unternehmen nie völlig lösen konnte und der mitten im Gespräch über andere Themen plötzlich einen unternehmerischen oder organisatorischen Einfall hatte und von seinen Begleitern die restlose Konzentration auf diesen neuen Gedanken verlangte. Auch meine Mutter zog jetzt aus Niederschlesien in den Breslauer dörflichen Vorort Oswitz, der mit Breslau durch eine Straßenbahn verbunden war. Ein paar Jahre später, als sich bei ihr eine Herzschwäche störend bemerkbar machte und als Ella und Emil Fache ihre Beziehung durch Ehevertrag legalisiert hatten, nahmen Faches sie in ihre komfortable Appartementswohnung in Breslau-Kleinburg auf, wo sie 1936 am Herzschlag verstarb.

Ostern 1927 wurde Albert Engmann, Germanist und Historiker, den ich schon 1913 in der Breslauer Freien Studentenschaft kennengelernt hatte, am Elisabeth Gymnasium als Studienrat angestellt. Meine gutkameradschaftlichen Beziehungen zu ihm entwickelten sich zu einer festen Freundschaft. Mehrfach verbrachte ich mit ihm Sommer- auch Herbstferien, einmal in Paris und St. Malo, in Südfrankreich, auch am Gardasee und in Venedig.

Im Sommer 1927 zog es mich wieder zur Ville des Lumières an der Seine. Es gab damals noch im Quartier Latin die Familienpension für in- und ausländische Studenten. In einer solchen ließ ich mich nieder. Sie war von nur wenigen französischen und zahlreichen ausländischen, besonders deutschen, rumänischen, griechischen u.a. Studenten und Philologen besucht. Den Vorsitz der Tischrunde führte ein junger französischer Arzt, der Sohn des Inhabers der Pension. Die ausschließlich auf Französisch geführten Tischgespräche waren lebhaft und oft interessant. Einmal gab es eine Diskussion über Corneille und die anderen Klassiker des 17. Jahrhunderts. Als der junge Arzt für die klassische französische Literatur noch heute Weltgeltung und Aktualität beanspruchte, sah er sich einer Front der Balkanvölker gegenüber, die einhellig die Ansicht vertraten, dass diese Literatur für den Menschen des 20. Jahrhunderts überhaupt nicht mehr lesbar sei und ihm überhaupt nichts zu bieten hätte. Ich hielt mich in diesem Disput zurück, wertete aber die französische Klassik als Spiegel einer hohen, in sich geschlossenen Kultur des Zeitalters Ludwigs XIV. In dieser an Kunstdenkmälern so überreichen Stadt erlebte ich neu die Sainte-Chapelle, das Musée de Cluny, die römischen Arènes und die Place des Vosges, ein herrliches Beispiel eines restlos erhaltenen Platzes aus dem 18. Jahrhundert.

Die drei Rumänen in der Pension, von denen zwei während des 1. Weltkrieges in deutsche Kriegsgefangenschaft geraten waren -- der eine von ihnen als Offizier, jetzt als Gymnasialdirektor und Abgeordneter der Bauernpartei im Bukarester Abgeordnetenhaus -- hatten größte Hochachtung vor Deutschland und seiner Kultur, seinen Leistungen auf den verschiedensten Gebieten. Sie luden mich zu einem gemeinsamen Besuch des Eiffelturmes und des Dôme des Invalides ein. Es wurde ein durchaus angenehmer und gesprächsintensiver Nachmittag, zeitweise wie unter Kriegskameraden. Auch zwei Jugoslawen, die ihre Kroatische Nationalität mithin einstige Donaumonarchie betonten, suchten Kontakt mit uns. Von ihnen erfuhr ich, dass noch 1927 in kultivierten bürgerlichen Familien, so auch den ihren, neben einer kroatischen Zeitung die deutschsprachige Tageszeitung \enquote{Wiener Journal} vor allem wegen seines guten Feuilletons gehalten wurde; das wahre kroatische Kulturzentrum sei nach wie vor Wien. Rumänen wie Kroaten sprachen übrigens, wenn wir uns außerhalb der Pension begegneten, mit mir gern deutsch. Ich erinnere mich, dass damals, nach dem 1. Weltkrieg, die drei Nachfolgestaaten der Donaumonarchie Tschechoslowakei, Jugoslawien und Rumänien im politisch-militärischen Bunde der \enquote{Kleinen Entente} zusammengefasst waren, die mit antideutscher Tendenz im Kielwasser Frankreichs segelten.

Von Bühnenerlebnissen machte den nachhaltigsten Eindruck auf mich\linebreak \enquote{Knock}, eine Komödie von Jules Romains im Théâtre des Champs-Elysées. Diesem Stück, eine geistvolle Verspottung des geschäftstüchtigen, zur Charlatanerie neigenden Arzttyps, war mit Recht ein großer Bühnenerfolg: großartige Regie und schauspielerische Leistung, über 200 mal hintereinander im gleichen Theater gegeben.

\marginpar{569}
Der Kuriosität halber sei erwähnt, dass im literaturfreudigen Paris, zumal im Quartier Latin, einzelne Buchhandlungen selbst am späten Abend noch geöffnet hatten. So erwarb ich einmal, als ich von der Oper nach Hause schlenderte, nach längerem Gespräch mit dem Buchhändler noch 1 Uhr morgens Gide's \enquote{Faux-monnayeurs}.

Auch im Juli des nächsten Jahres (1928) war ich noch einmal in Paris, diesmal mit Prauses und Engmann. Ich erinnere mich eines gemeinsamen Besuches der Oper, wo Wagners Tannhäuser für meinen Geschmack etwas zu süßlich gegeben wurde, ferner der Feier des 14. Juli, am Vorabend Parade einer illuminierten Flotte auf der Seine, Militärparade auf den Champs-Elysées, Tanz auf den Straßen -- nachmittags flüchteten wir bei 36° aus der drückenden Stadt, um in der Marne zu baden.\\

Die zweite Hälfte der Großen Ferien verlebten wir in St. Malo im Grenzbereich zwischen Normandie und Bretagne, ich mit Engmann in nahen Seebad Paramé. Wir erlebten hier das eigenartige Schauspiel der riesigen 15~m erreichenden Differenz von Ebbe und Flut: wir kamen bei Flut an und sahen nur Wasser und Wasser vor uns -- und am nächsten Morgen, unseren Augen nicht trauend, ein wahres Archipel von kleinen, den Schären ähnlichen Inselchen, die 5-10~m aus dem Meer herausragten! Herrlich abends der Rundgang auf den damals unversehrten Stadtmauern. Ähnliches ist meines Erachtens nur in Ragusa (Dubrownik) und Aigues Mortes zu erleben. Im Restaurant Charpentier, wo wir uns zu den Mahlzeiten trafen, begeisterte ich mich für die täglich frisch von den nahegelegenen Bänken kommenden Austern, die ich täglich als Vorspeise wählte, mit herbem Cidre, und mein Freund Engmann sah es besonders auf den Camembert ab, der ihm à discretion zur Verfügung stand. Wir beide hatten im Seebad Paramé Quartier bezogen, 2,5~km von St. Malo entfernt und mit ihm durch ein Schmalspurbähnchen verbunden. Als ich vormittags einmal unter französischen Badegästen im Sande lag, rief ein arabischer Zeitungsverkäufer \enquote{Le Journal, Le Petit Parisien} etc. aus, blieb dann vor mir, der ich in Badehose und gebräunt wie die Franzosen ausgestreckt lag, stehen und rief wiederholt: \enquote{Die Vossische Zeitung}, worauf ich ihn französisch fragte, wie er darauf komme, mir eine deutsche Zeitung anzubieten. Seine Antwort lautete in gebrochenem Deutsch \enquote{Sèhè doch, is deitsches Gèsicht!}

Interessant war eine Dampferfahrt auf dem fjordartigen Flusse Rance, vorbei am vornehmen Seebad Dinard nach Dinau; die Bauweise der Häuser, die keltische Tracht. Ein großes Erlebnis war natürlich der berühmte Mont St. Michel mit seiner ehemaligen frühgotischen Benediktinerabtei und den mächtigen Befestigungen aus dem 12. Jahrhundert. Innerhalb der Gruppe französischer Touristen war ich durch Fragen dem sehr beschlagenen französischen Fremdenführer aufgefallen, der es verstand, während seiner Ausführungen für die Masse mir persönlich noch einige kunstgeschichtliche Sonderbrocken zuzuwerfen.

Auf der Rückfahrt machte ich kurz Halt in Rennes, der Hauptstadt des einstigen Herzogtums Bretagne. Ich besichtigte den prächtigen Renaissancebau des Palais de Justice und suchte vergeblich in Zeitungsständen und Buchläden nach Zeugnissen in keltischer Sprache. Einen ganzen Tag blieb ich noch in Chartres, ausschließlich in Banne seiner Kathedrale. In Paris nahm ich noch eine Matinée in der Comédie Française mit (Molière). Kurz vorher speiste ich noch in einem benachbarten Restaurant. Da ich nicht Zeit hatte zu einem vollen Déjeuner, schlug mir der Kellner vor, einfach nur Hors-d'Oeuvres zu essen; er brachte eine riesige 16-teilige Schüssel mit recht erlesenen Dingen wie Spargelspitzen, Krebssalat etc.; als ich von allem mehr oder weniger ausgiebig gekostet hatte, war ich gesättigt und zahlte den Gegenwert von 0.80 DM. Tempora mutantur\dots!\\

Noch einmal verlebte ich 1929 die Sommerferien in Frankreich -- weniger Erholungs- als Studienaufenthalt wie immer. Meine Reise ging über das liebe alte Dijon, über Mâcon-Lyon-Vienne, das ich mir etwas genauer ansah und das einzelne Reiseschriftsteller als den Beginn der Provence -- wegen \enquote{zahlreicher römischer Altertümer} -- bezeichnen. Zu meinem Erstaunen war das Wasser der Rhône noch im Juli so kalt (12°-13°), dass ich an Baden nicht denken konnte. Ich zahlte damals für eine Nacht in einem durchaus guten komfortablen Hotel 2,50 frs. Weiter ging es mit ca. 2-stündigen Unterbrechungen in Valence und Montélimar, wo ich die wohl nördlichsten Ölbäume erblickte, nach Orange, in dessen mächtigem römischen Theater ich von einem mittleren Platz aus Racine's Attalie schaute und dank der ausgezeichneten Akustik dieses Kunsttempels auch klar verstand.

Bis nach 1 Uhr nachts saß ich im Café noch mit einer theaterbegeisterten Frau und ihrer erwachsenen Tochter in lebhaften Gespräch über das Geschaute und Gehörte. Ihrer Einladung auf das Gut folgte ich mit Rücksicht auf meine Vereinbarung mit Engmann nicht.

Mit Engmann und seiner Reisebekanntschaft, einem 45-jährigen kultivierten Amerikaner aus Philadelphia M. Wilkinson, verbrachte ich einige Tage im damals noch bezaubernden Avignon. Noch gab es den Massentourismus nicht und die Straßen mit den herrlichen Platanen, dem Wohnsitz tausender von Zikaden waren noch nicht durch lebensfeindliche Abgase verpestet. Unter die wenig zahlreichen Autos mischten sich noch Pferdedroschken und hin und wieder ein 2-rädriger Maultierkarren (charrette), dessen humorvoller Fahrer sich mit einer Autohupe den Weg bahnte. Abends, vor dem Café sitzend und englisch plaudernd, mussten wir zeitweise unsere Stimmen erheben, um nicht von den in den Platanen konzertierenden Zikaden übertönt zu werden.

Während ich mit Wilkinson das Musée Calvet besuchte, unterzog der Historiker Engmann die Remparts\footnote{Stadtmauern} einer eingehenden Besichtigung. Mit ihm fuhr ich früh um 4 Uhr mit dem Personenzuge nach Remoulin -- unterwegs ein ergötzliches Gespräch mit dem Schaffner, der August 1914 noch in den traditionellen roten Hosen an die Front gerückt war und wie die \enquote{boches}, nicht faul, ihre roten Hosen aufs Korn genommen hätten, pif-paf! Und die \enquote{satanés boches} schossen nicht schlecht\dots Ah! vous êtes des Allemands? Mais ça ne fait rien\dots!

Einen halben Tag verwandten wir auf die Besichtigung dieses gewaltigen römischen Viadukts, des Pont du Gard. Während Engmann zeitig zurückfuhr, sah ich mir nachmittags in Remoulin eine unblutige Corrida mit dem Stier Ramoneur (Schornsteinfeger) -- wegen seiner schwarzen Farbe -- an. Ein köstliches Geschicklichkeitsspiel mit dem Stier, dem es eine der drei Kokarden (in den französischen Nationalfarben) zu entreißen gilt. Als die fast rein dörflichen Zuschauer hörten, dass ich Deutscher sei aus Breslau, wollten sie wissen, ob man in Schlesien vom Ramoneur gehört habe. Meine bejahende Antwort erhöhte sichtlich ihre Festtagsstimmung. Die Corrida wurde am nächsten Nachmittag fortgesetzt: jetzt erprobten im gleichen Spiel Frauen und Kinder ihre Behendigkeit, doch nicht mit dem äußerst starken, sehr wendigen und nicht ganz ungefährlichen Ramoneur, sondern mit einer weit friedfertigeren Kuh.

Während Wilkinson seinem Ziel Florenz zustrebte, begleitete er mich zunächst noch ein Stück fort auf meiner Kulturreise durch die Provence und Côte d'Azur, mit Etappen in Nîmes, Tarascon, Arles, Camargue, in les Saintes Marines de la Mer, Marseille, Huyères und dem damals kleinen Seebade Huyères-les-Bains, schön im Pinienhain gelegen, mit gutem Sandstrand, einem Ausflug zur Insel Porquerolles, nach Cannes und Nizza, das damals von nur wenigen Sommergästen besucht wurde, so dass Hotels und Pensionen nur zum Teil belegt waren. Lange erging ich mich in der Altstadt, die damals noch ein sehr italienisches Gepräge mit italienischer Stadtfolklore und vorwiegend italienisch sprechender Bevölkerung hatte. Ausflüge nach Monaco und Monte Carlo beschlossen meine Kulturreise. Rückfahrt mit der Eisenbahn von Nizza über Cuneo nach Turin und über den Brenner mit kurzen Stationen in München und Nürnberg zurück nach Breslau.\\

Im nächsten Sommer 1930 strebte ich Oberitalien zu, unter anderem um meine italienische Sprechfertigkeit aufzufrischen. Die erste Etappe bildete das deutschsprachige Südtirol mit Bozen und Meran, wo ich die wirtschaftliche und politische Lage der Deutschen unter dem Faschismus Mussolinis nicht ohne Sorge betrachtete. Weiter ging's mit der Bahn über Rovereto nach Riva am Gardasee und dann im Vaporetto nach Malcesine, wo ich mich verabredungsgemäß mit Engmann und dem Ehepaar Prause traf. Goethes \enquote{Italienische Reise} war bestimmend für diese Etappe sowie auch für die Route bis Venedig. Noch fuhr man im Sommer im Schlitten die steinigen Steilhänge des Monte Baldo hinab. Als mir ein wahrscheinlich auf ungewohnter Nahrung beruhendes Jucken am ganzen Körper den Schlaf stark verkürzte, sah ich im ersten Morgengrauen vor meinem Fenster einen Fischer im Kahn wenige hundert Meter vor mir auf dem See, winkte und rief ihn herbei. Er holte mich, fuhr wieder hinaus und ich half ihm noch zwei Stunden beim Fischfang. Damals gab es noch zahlreiche Fischer in Malcesine -- 1961 bei meinem nächsten Besuch waren dieser Beruf und die Fische infolge Wasserschmutzes schon im Aussterben.

\marginpar{15.7.03}

\section{Intermezzo: Wiedersehen mit England nach 50 Jahren im Februar/März 1975}
Der Anlass zu dieser Reise war ein doppelter: Erstens die große allumfassende, von der internationalen Kritik sehr positiv aufgenommenen Ausstellung der Werke Turners in der Londoner Royal Academy of Arts und zweitens der Wunsch, Helga und ihre kleine Familie in York zu besuchen, wo ihr Mann an der University eine nichtbezahlte Studienstelle als visiting scholar vom Oktober 1974 bis Juli 1975 innehatte.

Da die Turnerausstellung bereits am 2. März ihre Pforten schloss, waren wir gezwungen, unsere Reise noch in der klimatisch ungünstigen dritten Februardekade anzutreten.

Wir flogen mit einer voll, besonders mit Jugendlichen, besetzten Maschine der englischen Fluggesellschaft Dan-Airs von Tegel nach dem noch neuen Flughafen Gatwick, ca. 50~km südöstlich vom Zentrum Londons entfernt. Der Flug selbst war, da ohne jede Sicht, völlig uninteressant. Das englische Personal verfügte, trotz seit langem fast täglichen Fluges zwischen der Bundesrepublik und London nur knapp über die bescheidenen deutschen Sprachkenntnisse, die erforderlich waren beim Einkassieren der Gelder für während des Fluges gekaufte Lebensmittel und sonstige Waren.

Eintreffen in Gatwick gegen 17 Uhr, eine reichliche halbe Stunde später startete der Omnibus mit der Mehrzahl der Fluggäste nach dem jeweiligen Hotelunterkünften in London.

Die Suburbs, die wir durchfuhren, machten einen, z.T. auch jahreszeitlich bedingten, tristen Eindruck: nur mäßig beleuchtet, eintönige ein- bis zweistöckige Reihenhäuser älterer Bauart, auf den, der London zum ersten Mal sah, recht provinziell wirkend. Hin und wieder ein etwas gepflegtes Wohnviertel gefolgt von unschönen Geschäftsstraßen oder Plätzen. Erst in der Nähe des Hydeparkes, nach mehr als einstündiger Fahrt und nachdem die meisten der deutschen Fluggäste bereits abgesetzt waren, wurde das Straßenbild großstädtisch und auch harmonischer. Schließlich waren wir beide allein übrig geblieben und wurden vor dem Hotel St. James, Buckingham Gate, abgesetzt. Eine männliche japanische Dienstkraft brachte dienstbeflissen unser Gepäck zum Zimmer 650 im 6. Stock. Das Hotelpersonal war, von den Schlüsselstellen abgesehen, mit Ausländern besetzt, vorwiegend mit Japanern und anderen Ostasiaten und -innen. Unser Hotel war zentral und verkehrsgünstig und doch sehr ruhig gelegen. \enquote{Continental breakfast} wurde von Japanerinnen im Zimmer serviert, zum \enquote{English breakfast} musste man sich ins Hotel-Restaurant im Nebengebäude begeben: es erwies sich um zwei weiche Eier und Cornflakes mit Milch reicher, wofür ein supplement zu entrichten war.

Ich vermisste gegenüber 1925 Fisch (die schöne frische Scholle), Porridge und frisches Obst, dagegen gehörte sowohl zum English- wie dem Continental breakfast ein Glas Saft (grapefruit, tomato). Der Tee erwies sich als sehr gut, wie eh und jeh.

Als wir am ersten Morgen das Hotel verließen, hatte gerade der stattliche livrierte\footnote{\enquote{Livree} bezeichnet hier die einer Uniform ähnlichen Bekleidung der Dienerschaft} und mit Zepter bewehrte Portier seinen Posten bezogen und grüßte distinguiert als Gentleman.

Wir fragten einen älteren, offenbar zum Dienst eilenden Clerk nach dem nächsten Weg zur Royal Academy und er geleitete uns sehr freundlich durch mehrere Straßen zum St. James Park und in diesen hinein, bis er die Gewähr hatte, dass wir uns nicht mehr verlaufen. Wir befanden uns inmitten einer sehr gepflegten herrlichen Parklandschaft im Frühlingsgewand: ringsum grüne bereits geschorene Wiesen bestückt mit gelben Osterglocken und bunten Krokussen, hier und da blühende japanische Sträucher; auf den Wasserflächen tummelte sich Wassergeflügel aller Art. Es war ein nicht sonniger, hell dunstiger Inselmorgen mit kühlem, aus Nordost wehendem Inselwind. Eine middle-aged woman vertrieb sich und einzelnen Spaziergängern die Zeit damit, dass sie Möwen mit Brotstücken fütterte, die diese ebenso gewandt wie dreist ihr vom Munde \marginpar{579} rissen. Vorbei am interessanten Tudorbau des St. James Palastes gelangten wir zum Hof der Royal Academy of Arts, wo jedoch schon eine mehrere hundert Meter lange Menschenschlangen auf Einlass warteten. Wir verschoben daher den Besuch der Turnerausstellung auf den nächsten Tag und besichtigten zu Fuß Teile von Westminster und der City. Londons Kern -- alle Kriegsschäden sind beseitigt -- ist nach wie vor imposant und wirklich weltstädtisch, architektonischer Ausdruck des einst so mächtigen, ja weltbeherrschenden British Empire, von dem noch 1944 [Roosevelt gesagt haben soll], es sei nach Zerschmetterung des Hitlerfaschismus der Weltfeind No 2\dots 

Roosevelt, der in Stalin einen befreundeten Gentleman erblickte\dots

Westminster und City sind nicht nur imposant, sondern an einigen Stellen sogar schön, natürlich \underline{nicht} die vielbesungenen Picadilly Circus und Leicester Square, beides, besonders der erste enge und unschöne Plätze, der überdies noch -- in London kein Einzelfall -- mit einem Zeugnis englischer Geschmacksverwirrung belastet ist: inmitten des Platzes steht das Shaftesbury Memorial, ein pyramidenförmiger Bronzebrunnen gekrönt von einer Erosfigur -- dem Symbol des englischen Philosophen, Moralisten und Philanthropen, des Earl of Shaftesbury.

Auch der Trafalgar Square, zweifellos Londons schönster Platz mit der Nelson Säule gibt trotz einiger eindrucksvoller Gebäude (wie National Gallery, Kirche St. Martin's in the Field) kein harmonisches Gesamtbild. Die künstlerischen Qualitäten der hoch oben (über 50 m) stehenden Statue des Seehelden sind von unten schwer zu beurteilen, was einen bösen französischen Kritiker zu der Bemerkung veranlasst: die Engländer hätten alle ihre Statuen auf solch hohe Säulen setzen sollen -- in der Entfernung wirken sie ganz interessant. (Vgl. Leonhard 77mal England Seite 414)

In der Poet's Corner von Westminster Abbey wiederholte sich übrigens mein negativer Eindruck des Jahres 1925 von der englischen Bildhauerkunst.

Imposant und gleichzeitig schön sind nach wie vor Regent Street mit dem Aldwick-Bogen und den Lettern \enquote{To the Friendship of The English Speaking Nations}, auch die neugotischen Law Courts, wo man wie vor 50 Jahren Juristen in Perrücken einherschreiten sehen konnte. Gleich östlich von den Law Courts das Temple Bar Memorial, das, von einem Drachen gekränzt, die Grenze zwischen Westminster und City of London am Beginn der Fleet Street kennzeichnet.

Im Tea Room eines großen kosmetischen Warenhauses machten wir eine Erfrischungspause um dann vorbei am \enquote{Temple}, Dr. Johnson's House zum Ludgate Hill mit St. Paul's Cathedral aufzusteigen: mir bot sich ein völlig verändertes Bild dar: die große Wren-Kathedrale, eine der größten Kirchen der Welt, völlig freistehend, so dass man diesen gewaltigen harmonischen Renaissancebau von allen Seiten betrachten kann; seine schöne, 111~m hoch ragende Kuppel ist das weithin sichtbare Wahrzeichen der City. Ganze Straßenzüge, besonders zwischen St. Paul's und Themse hat man nach den Zerstörungen im 2. Weltkriege nicht wieder aufgebaut bzw. abgerissen.

Zur Themse herab führen breite mit Blumenbeeten geschmückte Terrassen, die dem kalten Ostwind ungehindert Zutritt zum Viertel gewährten, sehr zum Schaden von Margots Gesundheit. Ein kleines altes, schlicht-gotisches Kirchlein dicht neben der Kathedrale hatte man stehen lassen: ein kleines nur mit einem Hemdchen bekleidetes und fröstelndes Kind neben der großen vornehmen Dame im Pelz\dots

Unser beider Eindruck vom Innern der Kathedrale mit einem Wort: kalte Pracht. Holman Hunt's berühmtes Bild: Christus mit der Laterne \enquote{the Light of the World} hing noch an dem alten Ort, ein in den Farben minderwertiges Bild dieses nun heute endlich auch in England umstrittenen Präraffaeliten. Die Kirche war von Touristen stark besucht. An mehreren Ständen verkauften Japanerinnen (Studentinnen) Ansichtspostkarten, Kunstbücher und Souvenire von der Kirche.

Fröstelnd bestiegen wir den Bus No 11 -- schon 1925 fuhr er die gleiche Strecke -- zu den Houses of Parliament und Westminster Abbey, der berühmten Kirche patriotischer Erbauung, der Krönungskirche der englischen Herrscher seit William the Conqueror, der einzigen Kirche Englands (außer George's Chapel in Windsor Castle) die weder dem Bischof von London noch dem Erzbischof von Canterbury untersteht, sondern \enquote{Königseigen} ist.

Dicht am Eingang im Fußboden eine große mit frischen Blumen geschmückte Gedenkplatte für Winston Churchill, unweit der Gedenkplatte für den Unknown Soldier des 1. Weltkrieges. Für unzählige Männer der Tat und des Geistes, denen England seine Macht und seinen Ruhm zu verdanken hat, ist Westminster Abbey die großartige vaterländische Gedenkstätte. Das künstlerische Niveau vieler Skulpturen ist freilich nur mittelmäßig. Wenn einem eine Büste durch seine künstlerische Qualität auffällt wie die des nordamerikanischen Schriftstellers Longfellow am Rande der Poet's Corner, von seinen englischen Freunden gestiftet, so stammt sie bestimmt nicht von einem englischen Bildhauer.

Westminster besitzt übrigens zwei im Freien stehende eindrucksvolle und gute Statuen jüngeren Datums die Churchills auf dem Parliament Square und vom King George V. im Old Palace Yard, gegenüber vom King's Tower: der barhäuptige Kopf eines Durchschnittsmenschen, dem ein gewaltiger vom Halse zur Erde fließender Umhang imperiale Würde verleiht.

Nach einer Umschau in Westerminster Hall fahren wir im steifen Ostwind mit dem Taxi zu etwa halbem Berliner Preis ins Hotel, wo wir im Restaurant ein nicht billiges, qualitativ unter dem italienischen und weit unter dem französischen liegendes Abendessen einnahmen, bedient von ostasiatischem Personal.

Am nächsten Morgen um halb 9 Uhr zählten wir im Hofe der Royal Academy of Arts zu den ersten Einlassbegehrenden. Ein heller, kalter Morgen, nur knapp über dem Gefrierpunkt. Hauptfront und die beiden Flügel der Akademie hatten edle Formen der Renaissance.

Die Ausstellung reich, ja eigentlich zu reich, besonders an Aquarellen, die allerdings Turner als großen Meister zeigten, vom strengen Realisten bis zum Gestalter des Farbrausches unter Auflösung jeglicher Konturen. Nach kaum einer halben Stunde erstickten die Säle fast von Besucherscharen. Turner, 1925 in England wenig, auf dem Kontinent so gut wie nicht bekannt, wurde nun im Bewusstsein der englischen Massen verdientermaßen in den Rang eines großen englischen Malergenies erhoben, der nicht nur der erste große Impressionist -- trotz französischen Widerspruchs -- ist, sondern die weitere Entwicklung bis zum Expressionismus vorweggenommen hat. Manche seiner großen Bilder verraten deutlich -- was mir schon 1925 in der Tate Gallery aufgefallen war -- seine tragische Weltanschauung.

Er ist zweifellos der größte englische Landschaftsmaler, weil größer als Constable, der mich 1925 fast ebenso stark beeindruckt hatte. In eine Reihe mit den ganz großen Genies wie Tizian, Rembrandt, Rubens u.a. gehört Turner jedoch nicht, dies zeigte ganz deutlich die Ausstellung. Er ist kein Porträtist, er hat zwar ein oder zwei gute Selbstporträts gemalt; alle Versuche jedoch, auf historischen Szenen Persönlichkeiten wie etwa Raffael im Vatikan darzustellen, bleiben ausgesprochen schwach, so auch der Kopf der schönen Shylock-Tochter Jessica. Schön ist nur der Dekor, das Kleid, ihr Gesicht hingegen ist weder schön noch ausdrucksvoll, nicht im Sinne Shakespeares.

Nach zwei Stunden wollten wir uns unten im großen Erfrischungsraum mit Tee aufmuntern. Der Neger-Aufseher machte mir klar, das gehe nicht, wer die Ausstellung verlasse, müsse eine neue Einlasskarte kaufen. Auf mein Verlangen führte mich der Neger zum Chief Inspector, von dem ich für uns beide eine Sondergenehmigung erwirkte. Als ich mich bei ihm zurückmeldete mit den Worten: You see, I really have my come back -- verzog er keine Miene. Der kleine Scherz war nicht angekommen. Als ich mich mit eben diesen Worten beim Neger zurückmeldete, schüttete dieser sich aus vor Lachen.

Übrigens hatten wir beim tea eine ca. 30-jährige englische Nachbarin, die ganz genauso geziert sprach wie die weibliche Jugend von 1925, was mich damals so gestört hatte; diese Untugend war also in England noch nicht ganz ausgestorben.

Gegen 16 Uhr verließen wir die Royal Academy und wanderten vorbei am prachtvollen Tudor Style St. James Palace durch den schönen St. James Park mit seinen einheimischen und exotischen Wasservögeln und Blumenteppichen inmitten von grünen Wiesen über Trafalgar Square und Whitehall zu den Houses of Parliament. Da es gerade seine Pforten schloss, nutzten wir die Zeit, um den Anblick dieser wahrhaft eindrucksvollen und für die einstige Weltmacht durchaus repräsentativen neugotischen Baues im Lichte der untergehenden Sonne, besonders vom rechten Themseufer, aus zu genießen. In diesem Londoner Bezirk waren, wie auch anderweits in den letzten 50 Jahren viele großzügige Fußgängertunnel gebaut worden. Das Denkmal der mutigen keltischen Königin Bondicra, der letzten keltischen Kämpferin gegen die römische Weltmacht, in der Mitte des 1. nachchristlichen Jahrhunderts, ziert nach wie vor die Westauffahrt zur Westminster Bridge.

Auf der Suche nach einem Dinner-Lokal landeten wir schließlich bei Lyons (Charing Cross) während uns andere Lyons Lokale zu stark proletarisiert erschienen.

Donnerstag war in erster Linie der National Gallery gewidmet, deren Schätze so groß sind, dass nur jeweils 1/3 gezeigt werden kann. Wir waren besonders beeindruckt von einigen sehr guten Rembrandts (u.a. Belsazars Fest) und anderen Holländern (Hobbema, Avenne von Middelharass, Ruisdal, Vermeers \enquote{Dame am Clavicord}), Dürers Portrait seines Vaters, Leonardo da Vincis \enquote{Madonna in der Felsengrotte}, Tintoretto \enquote{Ursprung der Milchstraße}, repräsentative Constables, Hogarths, Reynolds auch Turners! Gute Praeraffaeliten und franz. Impressionisten.

Eine Leihgabe von Thyssens Erben interessierte mich besonders: C.D. Friedrichs Ostermorgen; rückseits guter Kommentar seines ursprünglichen Besitzer Thyssen vom Jahre 1850! An in die Augen springender Stelle aufgestellt; England besitzt sehr wenige deutsche Romantiker. Übrigens ist die Zahl der Portraits großer und auch weniger bekannter Engländer erstaunlich groß.

Nach dem Lunch fahren wir zum Bahnhof King's Cross, kaufen Fahrkarten nach York, spazieren durch Teile des Hyde Park zum riesigen Verkehrszentrum Hyde Park Corner, von hier ermüdender Fußweg zum Hotel.

Am Freitag dem 28. Februar fahren wir zum mir von 1925 vertrauten Bloomsbury Viertel mit dem British Museum. Ich hatte auf der Woburn Place in der Nähe des Russel Square und Russel Hotel während der ersten Wochen meines England-Aufenthaltes gewohnt. In diesem angenehmen Wohnviertel schien sich nichts geändert zu haben.\\

Im riesigen British Museum beschränkten wir uns auf einige Säle:
\begin{enumerate}
	\item die griechisch-römischen Antiquitäten 
	\item die ägyptische Sammlung im Erdgeschoss; hier beachteten wir besonders den \enquote{Rosettastein}, eine schwarze Basalttafel aus dem Jahr 195 v. Chr. mit dreisprachiger Inschrift (ägypt. Hieroglyphen in der Priesterschrift, in weltlicher Schreibweise und in griechischer Übersetzung). Er wurde 1798 bei Rosetta im Nildelta gefunden und ermöglichte dem französischen Orientalisten Champollion einige Jahre später die Entzifferung der Hieroglyphen
	\item Ägyptische Mumien und Antiquitäten im Obergeschoss
\end{enumerate}

Leider haben wir die große Impressionisten-Galerie \enquote{Courtand-Gallerie} nicht gesehen.

Anschließend fuhren wir zum Tower, den ich Margot jedoch leider nur von außen zeigen konnte -- den Besuch des White Tower mit der ehrwürdigen normannischen Schöpfung der St. John's Chapel mussten wir einem späteren Zeitpunkt vorbehalten -- und gingen auch auf die Tower Bridge, von der aus sich ein schöner Blick auf die Themse und Uferpartien der City (Embarkments) im Scheine des letzten Tageslichtes und der schon aufgeflammten Nachtbeleuchtung bot.

Den Vormittag des letzten Londoner Tages nutzten wir für eingehende Besichtigung des höchst eindrucksvollen Parlamentsgebäudes.

Bemerkenswert, dass der englische Herrscher auf dem Gange zur Parlamentseröffnung vorbeischreiten muss nicht nur an den Statuen englischer Herrscher sondern auch an einer Gedenktafel über die Hinrichtung des Königs Charles I., der sich dem Volkswillen widersetzt hatte. Bemerkenswert auch, dass es außer dem prunkvollen \enquote{Ankleideraum} der Königin (bzw. des jeweiligen Monarchen), der langen Royal Gallery, dem House of Lords, dem House of Commons noch eine Peers' Lobby und eine Commons' Lobby gibt und zwischen den beiden noch eine Central Lobby. An den Seiten des \enquote{Churchill Arch} der Commons' Lobby sind zwei gute Bronzedarstellungen (Statuen) links von Churchill (in typischer Haltung) und rechts von Lloyd George angebracht.

Ein nochmaliger Blick in die Poet's Corner von Westminster Abbey verstärkte den Eindruck des Missklanges zwischen Bedeutung der Persönlichkeit, ihrem Ruhm und dem recht bescheidenen künstlerischen Können des Bildhauers.

Dann gingen wir durch die Victoria Tower Gardens und die Millbank Straße an der hier schon sehr breiten Themse entlang zur Tate Gallery. Reiseführern entnahm ich, dass dieses Museum in seinen Räumen zu gleicher Zeit nur einen noch geringeren Teil seiner mächtig angewachsenen Schätze als die National Gallery, nämlich 1/6, zeigen kann. Diese vorwiegend englischen Malern vom 16. Jahrhundert bis zur Gegenwart gewidmeten Galerie enthält auch bedeutende französische Gemälde des 19. und 20. Jahrhunderts aber auch ausländische Skulpturen wie z.B. den \enquote{Kuss} von Rodin. Mich interessierte u.a. besonders Romney \enquote{The Parsons Daughters} und \enquote{Mrs. Hamilton as Circe}.

Wir nahmen um 14 Uhr den Lunch im Tate Restaurant ein, fuhren dann mit der Taxe zum Hotel, luden unser Gepäck ein und fuhren weiter zum Bahnhof King's Cross. Fahrpreis etwa 1/3 des Berliner Preises. Der Bahnhof war ziemlich schmutzig und lud ebensowenig wie die anderen Londoner Bahnhöfe zu längerem Aufenthalt ein.

Der Intercity Zug, der 1. und 2. Klasse führte, fuhr nach etwa 20 Minuten durch geschlossene Stadtteile Londons, dann durch aufgelockerte Industriesiedlungen und hier und da noch einzelnen Tabantenstädte durch eine hügelige landwirtschaftlich genutzte Landschaft fast ganz ohne Wälder, doch umweltfreundlich: oft einzelne oder kleine Gruppen von sehr alten Bäumen, anders hier und da Knickähnliche Hocken. Die Rauchbekämpfung, die in London so erfreuliche, überzeugende Erfolge gezeigt hat, ist auch in der Provinz soweit fortgeschritten, dass man von den Fabriken nur noch selten eine dunkle Rauchfahne aufsteigen sieht.

Bei unserer Ankunft in York, es war mittlerweile Nacht geworden, erwies sich das Bahnpersonal auch ungebeten hilfreich: man trug uns das Gepäck über die Gleise, ließ uns Lifts benutzen, die sonst nicht für die Reisenden bestimmt waren und besorgte uns ein Taxi. Bald landeten wir vor dem letzten der Reihenhäuser der Universität Heslington, 1~km außerhalb der Stadt York.

Uns empfing Schwiegersohn Fritz; Enkel Niki schlief schon und Helga war Sonnabend früh unmittelbar vor Torschluss, zum Besuch der Turnerausstellung nach London gefahren. Noch am Abend kehrte sie unverrichteter Sache zurück -- sie hätte wegen des starken Andranges ca. 4 Std. auf Einlass warten müssen!

Unsere Kinder bewohnten ein sogen. Gästehaus der Universität, ein Reihenhaus mit Flachdach und -- wie alle englischen Reihenhäuser -- ohne Unterkellerung. Vor der Haustür befindet sich, von Nachbarn durch eine hohe Mauer getrennt, ein schmaler, winziger Gartenhof. Im Erdgeschoss mit sehr praktisch angegliedertem Essplatz und ein großes Zimmer mit Terrasse, ferner, rechts von der Haustür ein kleines Arbeitszimmer und eine Toilette, die jedoch nicht heizbar und daher in den kalt-nassen Wintermonaten nicht zu benutzen ist. Im 1. Stock drei Schlafzimmer und ein ebenfalls nicht heizbares Badezimmer. Um es im Winter benutzen zu können -- und die erste Märzhälfte erwies sich als besonders nass, kalt und dabei wie stets in England, windig, musste man die Tür zum Badezimmer Tag und Nacht weit offen lassen.

Der Warmluftkanal eines jeden Hauses wurde durch einen Thermostaten geregelt: er hatte in jedem Zimmer dicht über dem Fußboden eine etwa 20x30 cm große Öffnung, an der man sich bequem die Füsse wärmen konnte.

Die Wohnung, die hinreichend mit Mobiliar und allem notwendigen Hausrat ausgestattet war, kostete monatlich DM 300.-

Die Universität ist in den 60er Jahren mit all ihren Instituten und Studentenwohngebäuden auf dem Gelände eines ehemaligen Gutsbesitzes inmitten von Wiese- und Teichgelände im modernen Zementblockstil, jedoch ohne Hochhäuser und ziemlich geschmackvoll erbaut worden. Das alte Herrenhaus im roten Ziegel-Tudorstil dient als Verwaltungsgebäude. Auf den Teichen und Wasserläufen tummelt sich auch im Winter vielerlei Geflügel. Die Verbindungswege zwischen den einzelnen Universitätsgebäuden führen über eine Anzahl Holzbrücken. Einige Institute sind im Dorf Heslington selbst untergebracht, an dessen Rand das Gutshaus und Teile des Gutsparkes erhalten sind, mit Taxushecken und schönen alten Laub- und Nadelbäumen, an dessen Rande eine große abstrakte Plastik von Henry Moor steht, ich glaube treffend \enquote{Complexity} \enquote{Verflochtenheit} genannt, mit der ich mich allmählich anfreundete. Das Ganze gibt ein modernes und doch fast anheimelndes Gesamtbild.

Jeden Morgen kam über die grüne Wiese ein Entenpaar auf die betonierte Terrasse unseres Wohnhauses, schaute durchs Fenster ins Wohnzimmer hinein, Futter aus Nikis Händen heischend.

York ist keine eigentlich schöne, aber interessante Stadt; aus römischer Zeit, in der sie die Hauptstadt Britanniens war, bestehen keine nennenswerte Spuren.

Von den mittelalterlichen mit Zinnen bewehrten Mauern gekrönten Wällen des 13. Jahrhunderts, ist ein beträchtlicher Teil erhalten. Die Wälle erlebten wir als schönen grünen mit Hunderten von gelben Osterglocken bestickten Teppich. Aus etwa der gleichen Zeit stehen noch drei Stadttore, von den das Monk Bar, das Mönchstor, mit seinen zinnengeschmückten Türmchen das größte und eindrucksvollste ist.

Das Wahrzeichen Yorks ist sein Münster, ein gewaltiger Bau aus dem 13.-15. Jahrhundert im decorated Stil, einer englischen Spielart der Spätgotik. Da der Höhenunterschied zwischen Mittel- und Seitenschiffen gering ist und die großen Fenster trotz ihres reichen architektonischen Zierrats zumeist das Sonnenlicht fast ungebrochen hereinlassen, glaubt man in einer riesigen Hallenkirche zu sein, dessen Erbauer es vor allem auf eine erhebende Raumwirkung ankam. Bei den anderen mittelalterlichen gotischen Kirchen der Stadt ist das Ornament ebenfalls ein integrierendes Element der Architektur, jedoch zum Teil schon in der englischen spätgotischen Spielart des Perpendicular Style mit dem Fächergewölbe der Decke, das in dieser Vollendung meines Wissens nur in England anzutreffen ist.

Es ist nicht zu verwundern, dass keine der gotischen Kirchen Yorks hohe spitz auslaufende Türme besitzen -- es gibt nur wenige solche in England. Sie erscheinen auch hier abgeschnitten, jedoch wird der Vertikaltrend durch vier -- oder wie am Münster acht kleine, spitzauslaufende Ecktürmchen angedeutet. Der geräumige Münster Platz ist von z.T. altertümlichen oder wenigstens nicht störenden moderneren Bauten eingefasst. Die Stadt hat hier und da ein paar kleinere, einigermaßen harmonische Plätze wie den an der Town Hall und Partien altertümlicher Straßen mit vorgekragten Stockwerken, so besonders die \enquote{Shambles}.

Helga fuhr uns einen halben Tag lang durch die reizvolle Landschaft im Norden Yorks mit interessanten weiträumigen Kur- und Erholungsorten -- healthresorts. Leider wurde der Genuss dieser Fahrt durch Dauerregen beeinträchtigt.

Das anhaltend feucht-kalte und dabei stets windige Wetter wirkte sich nachteilig auf unsere Gesundheit aus, so dass wir unsere Abreise um 8 Tage verschieben und mehrere Tage das Haus hüten mussten.

Viel Spaß machte uns der verschmitzt-lebendige Niki, der sich besonders gern mit seinem Opa neckte oder ihn nachahmte; als ich in meine Nasenlöcher -- zur Verbesserung der Atmung -- mit Tinktur getränkte Watte stopfte, tat er das gleiche; als mir ein Artikel des \enquote{Telegraph} den Ausruf \enquote{großartig} entlockte, rannte er in die Küche und verkündete den dort versammelten Familienmitgliedern: \enquote{Opa sagt doosatig, doosatig}.

Mein Schwiegersohn führte ein \enquote{Interview} seines Kollegen David Rowe mit mir herbei.

Herr Rowe, ein 34 Jahre alter mittelgroßer Herr mit einem koketten dreieckigen ca. 1 cm x 1 cm großen Haarbüschel auf seiner linken Wange, begrüßte mich modern mit \enquote{hello!} (nicht mehr wie früher mit \enquote{how do you do}.) Ich hatte einen Fragenkatalog aufgestellt, der vor allem soziale, politische und künstlerische aktuelle Probleme enthielt. Im Brennpunkt des englischen Interesses stand damals der Beitritt Englands in die EWG und der Aufstieg von Mrs Thatcher zur Führerin der Konservativen Partei. Die fast 2-stündige Unterhaltung verlief lebhaft und interessant, offenbar auch für Mr. Rowe, der von sich aus den Besuch wiederholte. Ich erfuhr von ihm u.a., dass die Labour Party völlig in Konservatismus erstarrt, dagegen die Konservative Partei die wahre liberale Fortschrittspartei sei, in der die alten Tories zu einer kleinen Minderheit zusammengeschrumpft seien.

Überraschend erschien eines Abends der intelligente Mr. Anthony Culyer, 32 Jahre alt, in aussichtsreicher Stellung an der Yorker Universität, mit seiner deutschen (schwäbischen) Frau. Ein für mich sehr interessanter Abend. Auf meine Beobachtung angesprochen, dass wir als Ausländer überall sehr freundlich, ja liebenswürdig und hilfsbereit behandelt würden im Gegensatz zu den früheren \enquote{Reservedness} meinte Mr. Culyer, der Engländer habe erkannt, dass sich die Ausländer ausbeuten lassen.

Mein Eindruck von der Nostalgie der Engländer nach dem Zerfall des British Empire wurde bestätigt. Sie können sich begreiflicherweise nur schwer mit der betrüblichen Tatsache ihres wirtschaftlichen Niedergangs und des Verlustes an politischer Macht abfinden und der bescheidenen Stellung eines Mitgliedes der europäischen WG.

Lesefrüchte aus \enquote{77 mal England} von Rudolph Walter Leonhard
\begin{quote}
	In der englischen offiziellen Rangliste von heute kommt der Adel des Geistes erst an 59. Stelle\dots
\end{quote}

\begin{quote}
	Bei englischer Küche und englischem Regen sollen auch noch Besucher unserer Tage melancholisch geworden sein.
\end{quote}

Zum Englischen Geschmack auf künstlerischem Gebiet:
\begin{quote}
	Ein Franzose erblickt im Hyde Park das Denkmal Wellingtons, des Siegers von Waterloo, der als Achilles dargestellt ist und ruft zufrieden aus: \enquote{Wir sind gerächt!}
\end{quote}

Am letzten Tage, dem 13. März, ging der Regen in York in Schnee über; die Landschaft lag am nächsten Morgen, als wir in Heslington auf den Omnibus zum Bahnhof warteten, im weißen Kleide vor uns. Unsere stets hilfreiche Helga begleitete uns bis zum Flughafen Gatewick.
