\section{Einberufung nach Glogau}

Der Zustrom an Kriegsfreiwilligen bei allen Truppenteilen und aus \underline{allen} Schichten der Bevölkerung war gewaltig. Wenige Tage später wurde ich in Bunzlau zur Infanterie ausgemustert, jedoch erst am 28. November nach Lerchenberg b. Glogau zur Ausbildung einberufen.

Mein Vater war es zufrieden; er fürchtete, dass das Jägerbataillon als Elitetruppe an besonders heißen Brennpunkten eingesetzt und damit die Überlebenschance seines einzigen Sohnes verringert werde.

Die bevorstehenden Kriegsereignisse waren natürlich für alle, besonders in den ersten Wochen und Monaten, ein ungeheures Spannungsmoment. In einer Zeit, in der es noch \marginpar{174}kein Radio gab, war die Zeitung die wichtigste, ja meist einzige Informationsquelle. Ich fuhr mehrfach mit dem Rade die 15~km nach Bunzlau, um am Aushang des \enquote{Bunzlauer Stadtblattes} etwaige \enquote{letzte Depeschen} zu lesen. Mit Kriegsausbruch und Ausmusterung hatte sich mein Lebensgefühl mit einem Schlag gewandelt: ich lebte in Siegersdorf nur provisorisch, ja ephemer, auf den Einberufungsbefehl wartend, der ja jeden Tag eintreffen konnte.

Und doch kam es zu einer Liebesromanze, die sogar zu einer heimlichen Verlobung führte. Während des Berliner Sommersemesters hatten sich inzwischen meiner Familie und der des vis-à-vis in neugebauter Villa wohnenden Arztes Dr. Thamm, seiner Frau \enquote{Teti} und der zu Besuch weilenden unverheirateten Schwester Charlotte Steinberg (freundschaftliche Bande) gebildet. In Hause des Dr. Thamm hielt sich ein junges aupair-Mädchen aus La Chaud-de-Fonds auf; \enquote{Teti} und Lotte, die beide gute Vorkenntnisse besaßen, wollten ihre Sprechfertigkeit weiter verbessern. Man lud mich sofort ein, es wurde angeregt französisch geplaudert und Belletristik vorgelesen. Nach wenigen Tagen kehrte wegen des Kriegsausbruches die Schweizerin in ihre Heimat zurück, Dr. Thamm war als Assistenzarzt an die Westfront ausgerückt. Ich ersetzte gewissermaßen für die beiden Frauen das Mädchen der Suisse romande; und ich ging oft mit Lotte in unserem schönen Wald spazieren. Ich muss gestehen, dass, ehe ich's mir versah, Lotte mich eroberte. Sie hatte viel Humor, neckte sich reizend mit meinem Vater. Es blieb nicht aus, dass ich mich in sie verliebte. Sie war das jüngste von 4 Mädchen und einem Jungen des kürzlich in Breslau verstorbenen Geh. Rat Steinberg, der mit der Familie des \enquote{Großen Geistes} Kühnemann und anderen Universitätslehrern verkehrt hatte.

Lottes älteste Schwester war mit einem österreichischen Baron, die beiden Schwestern mit Ärzten verheiratet. Lotte war Lehrerin, hatte fremdsprachliche Interessen und ließ sich von mir in Latein fördern. Es bestand zwischen uns ein sehr gutes kameradschaftliches Verhältnis. Noch im September kam Felix, \enquote{Tetis} Mann, an Rheumatismus krankend mit einer herrlichen Gallé-Vase und anderen Beutestücken aus dem Felde zurück. Allerdings behauptete mein Vater, der mit ihm wöchentlich zweimal in der Bahnhofsgaststätte Skat spielte, dass Thamm auf dem Nachhauseweg sein rheumatisches Leiden völlig vergaß und überhaupt nicht hinkte.

Schon im Oktober hatte ich begonnen, in einigen meiner Fachgebiete wieder zu arbeiten, da kam endlich -- und doch plötzlich -- meine Einberufung nach Lerchenbach bei der Oderfestung Glogau.

Die Hoffnung auf einen siegreichen Blitzkrieg mit baldigem Kriegsende war nach der unglücklichen Marneschlacht und dem Übergang zum Stellungskrieg verraucht. Mit schmerzlichem Bedauern nur gedachte man der optimistischen Worte des Kaisers: \enquote{Bevor das Laub von den Bäumen gefallen ist, werdet ihr wieder zu Hause sein.} In den Winternächten tauchte \enquote{Frau Sorge, die grau verschleierte Frau} hier und da auf; man begann von einem bevorstehenden Eintritt Rumäniens an der Seite unserer Feinde zu munkeln, was dann auch einige Monate (Wochen?) später geschah.

Wir wurden in einem Barackenlager untergebracht, das man, wie wir von Dorfbewohnern erfuhren, 1870/71 für französische Kriegsgefangene geschaffen hatte. 52 Mann wurden in einem großen Raum in 2-geschossigen Drahtgestellbetten untergebracht. Die Bedürfnisanstalt befand sich außerhalb der Baracke in mehr als 50~m Entfernung. Infolgedessen befand sich jeden Morgen eine Pfütze vor der Eingangstür. Wachte man in der Nacht auf, bedrückte einen die dicke Luft, für deren Verbesserung das Feldküchenessen -- sehr häufig gelbe Erbsen und Kraut -- völlig ungeeignet war\dots

Die Umstellung von freiem Studentenleben auf den Militärdienst fiel mir schwer. Vormittags früh Instruktionsstunde durch Unteroffiziere der Reserve, von Beruf westfälische Bergleute; anschließend Exerzierdienst einschließlich Eindrillen des Parademarsches; nachmittags Felddienstübungen, schließend Putz- und Flickstunde.

In den ersten 14 Tagen fiel ich oft auf. \enquote{Busse, Kinn an die Binde}, \enquote{Busse, Kopf schneller herumreißen}, \enquote{Busse, Gewehr einziehen}, \enquote{Busse, zurück marsch, marsch!} \enquote{Vor, marsch, marsch; auf Kommando hinlegen!} War ich noch über die Pfütze hinweggesprungen, \enquote{Busse, zurück, marsch marsch -- hinlegen!} und ich warf mich befehlsgemäß in die Pfütze hinein.

Plötzlich fiel ich nicht mehr auf: ich war auf den Gedanken gekommen, den Unteroffizier im Anschluss an den Dienst in der Kantine zum Schnaps einzuladen (mit 1-2 Fortsetzungen). Fiel ich künftig mal auf, so wertete ich dies nur als Erinnerung: ich hatte den Unteroffizier seit fast 8 Tagen zu traktieren unterlassen!

An einem Sonntag musste ich 100 mal aufschreiben: Auf das Kommando \enquote{Achtung!} steht der Soldat still und verharrt in dieser Stellung so lange, bis das Kommando \enquote{Weitermachen!} erfolgt ist.

Am Montagmorgen, vor Beginn der Kompanie-Instruktionsstunde, die wir stehend in der Manege hören mussten, überreichte ich dem Feldwebelleutnant meine Strafarbeit. \enquote{Was sind Sie in Zivil? Warum haben Sie zuwidergehandelt?} \enquote{Ich war das Opfer einer Gehörshalluzination; ich glaubte, Herrn Leutnants Kommando \enquote{weitermachen} vernommen zu haben!} Jetzt hatte ich bei dem Feldwebelleutnant, wie sich bald zeigte, einen Stein im Brett gewonnen 1. weil ich die befohlenen Strafarbeit widerspruchslos ausgeführt und 2. weil ich eine gescheite, ihn beeindruckende Erklärung für mein Versagen formuliert hatte. Er bedachte mich bei Gefechtsübungen mit Sonderaufgaben wie Patrouillenführer, Führer eines Unteroffiziersposten u.a.m. Außerdienstlich redete er mich sogar mit \underline{Herr!} Busse an.

Ich benutzte jede Gelegenheit, um eine Stunde in die städtische Atmosphäre Glogaus zu entweichen, im Café in Zeitschriften und in Buchhandlungen zu stöbern und deutsche und französische Bücher zu kaufen, z.B. Bruno Willes \enquote{Offenbarungen des Wacholderbaums} -- doch es fehlte völlig an Zeit zur Lektüre. Mit dem Junglehrer Martin Haering, intelligent und belesen, temperamentvoll und witzig, verband mich Freundschaft; er meldete sich im März an die Front -- zu diesem Zeitpunkt war das schon erlaubt; ich habe nie wieder von ihm gehört. Zu Weihnachten gab es natürlich keinen Urlaub. Lotte Steinberg kam nach Glogau und wir konnten einige Stunden miteinander verbringen. Nach einigen Wochen der Felddienstübungen auf den Truppenübungsplätzen Wartelager bei Posen und Neuhammer am Queis unfern Siegersdorf, wo ich mich übrigens anhand von Toussaint-Langenscheidt mit den Schriftzeichen und anderen elementaren Dingen der russischen Sprache bekannt machte, wurde ich \enquote{Gefreiter} und für die Zeit vom 20. April bis 12. Juni zum Offiziersanwärter-Kursus auf dem Truppenübungsplatz Döberitz nordwestlich von Berlin abgeordnet. Voran gingen einige Wochen im Ersatzbataillon Bunzlau, wo ich einige Tage im Saale des Odeon verbrachte, in dem ich 3 Jahre zuvor die -- Tanzstunde besucht hatte; anschließend wohnte ich privat. Der Kompanieführer Hauptmann d.R. Beinhorn nahm sich fast väterlich meiner an und riet mir von einer freiwilligen Frontmeldung ab, es sei schließlich besser für mich als Porte-épée-Träger an die Front zu kommen.

Über dem Truppenübungsplatz Döberitz, einer Streusandbüchse, strahlte bei langsam bis zu 33° ansteigenden Temperaturen eitel Sonnenschein. Ausbildungsleiter war Hauptmann Watsak, Führer des 1. Zuges Oberleutnant Matthari, beide aktive Offiziere und Monokelträger. Der Führer des 2. Zuges trug am rechten Ärmelaufschlag ein blaues Band mit den Lettern \enquote{Gibraltar} in Gold -- er gehörte einem braunschweigischen Regiment an, das einst -- 1704 -- im Solde Englands Gibraltar erobert hatte. Die Lehrgangsteilnehmer waren Jahrgänge von 17-38 Jahren, also Oberschüler bis zu Regierungs- und Studienräten, teilweise waren sie schon an der Front gewesen und mit dem \ac{ekii} und einer Auszeichnung ihres Landesherrn geschmückt.

Matthari ließ sich von jedem im Gliede Zivilberuf und Beruf des Vaters sagen; bei \enquote{Zeitungsredakteur} fragte er nach dem Blatt, in einem Falle bemerkte er, leicht missbilligend: \enquote{Eine linksliberale Zeitung.} Als ich ihm sagte: \enquote{Vater königlicher Förster -- Hausfideikommiss der Königlichen Familiengüter}, meinte er anerkennend: \enquote{eine Elite}

Wadzak führte in der ersten Instruktionsstunde u.a. aus, dass wir die Chance hätten, in den vornehmsten Stand, den des preußischen Offiziers aufzurücken. Diesem Stande werde zu Unrecht von gewisser Seite Hochmut und sogar Überheblichkeit vorgeworfen. Herrgott, was ist schon dabei, wenn ein junger Leutnant, im stolzen Bewusstsein, des Königs Rock zu tragen, einem Stande anzugehören, dem Preußen \underline{alles} zu verdanken hat, gelegentlich mal über die Stränge haut?!

Als wir an einem Maimorgen mit anderen Verbänden antreten mussten zur Besichtigung durch hohe Militärs Bulgariens, das bald darauf an unserer Seite in den Krieg eintrat, bemerkte Wadzak: \enquote{Kerls, ich schäme mich, dass ich hier mit euch herumstehen muss, um von Kaffern besichtigt zu werden!}

\marginpar{183}
Der Kursus war im ganzen gesehen nicht langweilig und für den damaligen Stand der infantristischen Ausbildung einigermaßen modern. Einmal kanzelte Wadzak, noch zu Pferde sitzend, die Kompanie nach Kommando \enquote{stillgestanden!} scharf ab. Er erregte sich dabei so, dass er bei dem Worte \enquote{Schweinerei} seinem Roß einen Schlag mit der Reitgerte versetzte, so dass es sich hoch aufbäumte und seinen Reiter in den märkischen Sand abwarf. Vor uns liegend, tastete er nach dem Monokel, das ebenfalls abgestürzt war, raffte sich auf, ließ seinen Gaul durch seinen Burschen abführen, musterte streng-prüfenden Blicks die Kompanie und gab Befehl, in die Quartiere abzurücken. Es war sehr schwer, uns auf die Lippen beißend den Ernst zu wahren -- wir haben die Prüfung bestanden.

Der Kursus fand seinen Abschluss mit einer schriftlichen Prüfung, die allen denen, die das Abitur hatten, überraschend leicht erschien, und fast alle wurden zum Viezefeldwebel und Offiziersaspiranten befördert.


\section{Einsatz an die Front}
Nach einer kurzen Dienstzeit im Ersatzbatallion des Infanterie Regiments 19 in Bunzlau, wo ich jetzt als Offiziersaspirant am Mittagessen in der Offiziersmesse teilnahm, fuhr am 4. Juli der Truppentransportzug vom Bahnhof Bunzlau gen Osten ab. Die Bevölkerung hatte ihre \enquote{Feldgrauen} mit Blumen geschmückt, Eltern und Ella waren nach Bunzlau gekommen, überall feuchte Augen und ein letzter Händedruck und Kuss bei den Klängen der Militärkapelle \enquote{Muß i denn zum Städtelein hinaus.}

Die Fahrt ging mit Aufenthalten über Oels, Namstau, Kreuzburg; als wir die Reichsgrenze überschritten hatten, gab es Aufenthalt. Wir stiegen aus. Polnische Kinder bettelten um Brot \enquote{chleba, chleba}. Dies waren die ersten slawischen Worte, die ich hörte. Ein oberschlesischer Soldat übersetzte mir diese Lautgebilde, die ich als \enquote{Kleba} aufnahm. Am Abend des zweiten Tages wurden wir in der Gegend von Opatów östlich Kielze ausgeladen und bauten Zelte; man vernahm Kanonendonner von der 25~km entfernten Front. Am frühen Morgen brachen wir auf, etwa 2 Kompanien stark, in nordöstlicher Richtung auf den langsam vernehmlicher werdenden Kanonendonner zu. Wir marschierten auf einer breiten unbefestigten Landstraße durch bodenwelliges Gelände, hin und wieder Waldstücke, sonst Ackerland gemäß der Dreifelderwirtschaft betrieben, d.h. 2/3 mit Getreide, Hirse u.ä. bebaut, das restliche Drittel brach und verunkrautet. Beiderseits der Straße Spuren des Krieges, Holzkreuze für Gefallene, Postenlöcher, Schützengräben, Reste von allerlei Kriegsmaterial. Wir kamen durch von der Bevölkerung verlassene Dörfer, meist Straßendörfer, aus Holz und mit Stroh gedeckt. Hin und wieder vereinzelte alte Leute, die nicht mit den anderen geflohen waren. Aus den Aschenmassen verbrannter Wohnhäuser ragten als einzige massive Teile die riesigen Öfen mit den 3-4~m hohen Rauchfängern hervor.

\marginpar{185}
Die Truppe hatte damals noch kein einziges motorisiertes Fahrzeug, d.h. auch Regiments- und Brigadekommandeure ritten zu Pferde, nur die Chefs der großen Stäbe vom Divisionär an aufwärts verfügten über einen PKW. Die Feldküchen wurden von den kleinen osteuropäischen \enquote{Panjepferden} gezogen, die Chargen ließen ihr Gepäck in kleinen requirierten Bauernwagen, den sogen. Panjewagen befördern. Auch die Krankenträger mit roten Kreuzbinden am Arm erfreuten sich dieses Vorzuges. Aufklärungsflugzeuge waren damals im Osten noch eine Seltenheit; auf dem ganzen Tagesmarsch wie auch in den nächsten drei Wochen bekamen wir keines zu Gesicht.

Ein Feldlazarett, in Bauerngehöften untergebracht, und uns entgegenkommende verdreckte Fahrzeuge mit dem Rotkreuzzeichen auf dem Verdeck, aus denen man hin und wieder das Stöhnen Verwundeter vernahm, ließen die Nähe der Front erkennen. Die Kanonen schwiegen; Infanteriefeuer war noch nicht zu hören.

Am späten Nachmittag hielten wir in einem Dorfteil von Czekazewice, unweit der Kamienna, eines Nebenflusses der Weichsel, poln. Wista. Infanteriekugeln summten wie Insekten über uns hinweg, einer meiner Leute sang übermütig: \enquote{Kommt ein Vogel geflogen\dots}, sprang aber rasch beiseite, als eine Kugel laut klatschend in einen Holzpfosten neben ihm einschlug.

In Czekazewice war auch die Verwundetensammelstelle des LIR 19\footnote{LIR 19 = Landwehr-Infanterie-Regiment Nr. 19, Einheit der 20. Landwehr-Infanterie-Brigade, wiederum Teil der 35. Reserve-Division der Hauptreserve Thorn}, für das wir bestimmt waren als Ersatz. Von einem Hilfsarzt (Medizinstudent höheren Semesters) hörte ich, dass das Infanterieregiment am Vortage schwere Verluste an Toten und Verwundeten erlitten habe.

Nach Einbruch der Dunkelheit überquerten wir auf einer Behelfsbrücke den schmalen Kamienna-Fluss und schlugen im Schutze des linksufrigen Steilhanges Zelte auf. Ich war der 6. Kompanie zugeteilt, die am Vortage 80 Tote, darunter sämtliche Offiziere, bei einem aussichtslosen Angriff auf die russische vordere Stellung, die durch ein Astverhau geschützt war, verloren hatte. Angesichts der schweren Verluste lag die Kompanie mehrere Tage als Regimentsreserve in Ruhe.

Nach Tagesanbruch bot sich uns ein beeindruckender Anblick: die gefallenen Kameraden lagen in langer Reihe ausgestreckt, die Gesichter waren mit Zeltplanen verdeckt. Bedrückend unsere erste Arbeit an der Front: Gräber für die Toten ausheben. Sie waren die Opfer einer wahnsinnigen Taktik: durch opferreichen Scheinangriff -- ohne auch nur annähernd ausreichende Artillerievorbereitung -- den Feind von der wirklich geplanten Angriffsfront abzulenken.

Der Führer des 2. Bataillons Major d.R. Hilgendorf -- die Leute nannten ihn \enquote{Stehkragen}, weil er unter dem Rockkragen stets einen als schmaler weißer Rand sichtbaren steifen weißen Kragen trug -- hielt eine angemessene Grabrede. Hilgendorf war eine große, würdige Erscheinung mit gepflegtem, fast weißen Schnurrbart. Einmal, zwischen Weichsel und Bug, in einem Gelände platt wie der Tisch, ließ der Regimentskommandeur die beiden Bataillone auf riesigem Brachfeld Gewehre zusammensetzen und wegtreten; niemand durfte sich über 150~m entfernen. Viele holten Punkt 2 der zivilen Tagesordnung nach. \enquote{Stehkragen} schritt würdevoll, gefolgt von seinem Burschen, mit dem kurzen Feldspaten, dem \enquote{Schanzzeug} in der Hand, um Abstand von der Tuppe zu erlauben. \enquote{Mensch, kuck mal, Stehkragen geht sein Morgenei legen!} ging es von Mann zu Mann. Aller Augen folgten dem würdigen Herrn, um ihn menschliches, allzu menschliches verrichten zu sehen. Die Anweisungen, bzw. Kommandos konnte man nicht hören, man sah nur, wie der Bursche ein Loch grub, dann zurücktrat, eine plötzliche Kehrtwendung machte. Als es geschehen war, der gewichtige, stets die Form wahrende alte Herr sich langsam erhoben und adjustiert hatte, wurde das Produkt vom Burschen zugeschaufelt und beide traten in soldatischer Haltung den Rückmarsch an.
\marginpar{Heft 1 beendet 17.4.1998}



Wir hatten als Regimentsreserve am Ufer des Flusses mit dem klangvollen Namen Kamienna (\enquote{die Steinige}) dank dem Steilufer gegen Sicht geschützt, nur etwas Ordnungsdienst, Appelle und Instruktionen abzuhalten. Die Kugeln der russischen Infanterie pfiffen hoch über uns hinweg und die Feldartillerie beschoss Ziele im Hinterland, kaum dass russische Schrapnells über unseren Köpfen und damit für uns ungefährlich platzten, was mich nicht daran hinderte, täglich im Fluss zu schwimmen. Im Dorf fiel mir eine Art Obelisk mit einer längeren polnischen Inschrift auf. Ich rief einen vorbeigehenden, frontmäßig ungepflegten, lange nicht rasierten Gefreiten mit staubbedecktem Gesicht herbei, in dem ich einen vielleicht polnisch sprechenden Oberschlesier vermutete, und fragte, ob er mir die Inschrift übersetzen könne. Er verneinte -- und ich erkannte in ihm den Pfarrvikar Kölbig, der jetzt nur als Telefonist Frontdienst leistete. Einst war in Bunzlau aus dem Munde des damaligen Oberprimaners in der Pension Kattein für mich, den Untertertianer, reine Weisheit geflossen. Wochen darauf hatte ich Gelegenheit zu einer längeren Unterhaltung mit ihm. Ihn quälte unablässig die bohrende Frage: welche Sünden hat das deutsche Volk auf sich geladen, dass Gott ihm diese schwere Strafe eines weltumfassenden Krieges auferlegt habe. Und er stellte Mutmaßungen an. Das Gespräch war für mich peinlich und schwierig. Meinen schließlichen Einwand, Gott braucht ja überhaupt nicht seine Hand im Spiele zu haben, empfand er als lächerlich. Ich habe ihn nie wieder gesehen noch von ihm gehört. 

Dann rückte die 6. Kompanie in der Abenddämmerung durch einen sandigen Hohlweg in die Stellung. \enquote{Unterkunft} in einfachen Erdlöchern; fast völlig Stille; hin und wieder ein ferner Infanterieschuss. Legte man sich in sein Erdloch, vernahm man leise aber pausenlos die Erschütterung fernen Geschützfeuers. Trat man wieder hinaus ins Freie -- hörte man nichts; ab und zu flammten hüben und drüben flackernde Leuchtkugeln auf. Da kam plötzlich, als ich mich schon zum Schlafen eingerichtet hatte, der mündliche Befehl: Sofort zum Abmarsch antreten.

Wir marschierten mehrere Stunden durch die Nacht. Als ich sehr müde wurde und für Sekunden marschierend einzuschlafen drohte, verließ ich den mir als \enquote{Porteepée}-Träger vorgeschriebenen Platz und trat in die Marschkolonne \enquote{ins Glied} ein, um beim Einschlafen nicht auf die Erde, sondern auf den Vorder- oder Hintermann zu fallen. Das hat sich bewährt. Lange nach Mitternacht hielten wir in einem uralten Eichenwald. Die Mäntel wurden gelöst, man bereitete sich auf dem Erdboden zum Schlafen aus. Die meisten schliefen sofort ein. Ich konnte es nicht. Ich wusste, dass am Morgen eine Durchbruchsschlacht beginnt, mein erstes Gefecht. Werde ich am Abend noch am Leben sein? Zum Krüppel geschossen? Werde ich mich vor meinen Leuten, die größtenteils wesentlich älter als ich waren und schon viele Infanteriekämpfe bestanden hatten, wirklich bewähren? Die Erregung ließ mich kein Auge schließen. Da bemerkte ich erste Anzeichen der Morgendämmerung und wenige Sekunden später erdröhnte der Eichenwald vom Abschuss eines schweren deutschen Festungsgeschützes -- ich schaute auf die Uhr: Punkt 3 Uhr.

Minuten später begann einige Kilometer entfernt eine Kanonade. Bald erscholl das Kommando: \enquote{zum Abmarsch fertig machen!} In anderthalb Stunden gelangten wir -- es war strahlender Sonnenschein, ein herrlicher Julitag -- in die Höhe der deutschen Artilleriestellungen. Wir schauten den Artilleristen lange beim Feuern zu, es war eine Batterie Feldartillerie; etwa zweihundert Meter weiter keuchte eine schwere Batterie. Die Russen erwiderten das Feuer mit Schrapnells, die glücklicherweise nicht gut lagen. Wahrscheinlich lenkten Deutsche wie Russen das Artilleriefeuer von Fesselballons aus, die weit hinter der jeweiligen Front wie eine Leberwurst am Himmel hingen. Nach einiger Zeit verschärfte sich das Schusstempo unserer Artillerie, man sah, wie die Kanoniere ihr Äußerstes hergaben. Das dauerte wohl 20 Minuten, dann trat eine kurze Feuerpause ein, in der man gegnerisches Infanteriefeuer und die Explosion russischer Schrapnells frontwärts vernahm. Wir wussten: unsere vorderste Linie war zum Sturm auf die russische Feldstellung angetreten. Jetzt erhielten wir den Befehl zum Vorrücken in Schützenlinie, das hieß für mich 10 Schritt vor meinen etwa 60 im Abstand von 2~m ausgeschwärmten Leuten, zu meiner Seite je einen Entfernungsschätzer der Sonne entgegen rücken, durch leicht welliges Ackerbaugelände, durchsetzt mit kleinen Waldstücken. Bald kamen uns Truppen von russischen Gefangenen entgegen. Wir überschritten erst die deutschen, dann die russischen, jetzt von den Unseren besetzten Gräben, teilweise durch die deutsche Artillerie zerschossen; Krankenträger schafften Verwundete zurück. Jetzt waren wir vorderste Linie, es ging rasch vorwärts, die linke Flanke meines Zuges rückte am Rande eines Gehölzes vor. Unsere Artillerie beschoss weiter vor uns liegende Ziele. Da krepierte gut 150~m vor uns in der Luft eine Salve von Schrapnells, eine Reihe schöner weißer Wölkchen, etwa 25~m im Abstand von einander, ein schönes Bild auf dem blauen Sommerhimmel! dachte ich -- von einem Kugelregen war nichts zu spüren. Eben hatte ich Befehl gegeben, nach links auf das Gehölz zu verlängern, da krachten -- ich glaube, es waren sechs -- Schrapnellwölkchen unmittelbar, vielleicht 15~m vor und über unserer Schützenkette -- im selben Augenblick ein vielfältiges Schreien -- 6-8 meiner Leute waren auf einem Schlag mehr oder weniger schwer verwundet. Wir rannten jetzt so schnell wir konnten in der befohlenen Richtung vorwärts, kriegten aber noch einmal eine Ladung ab, waren einige Minuten ohne Beschuss, kamen aus dem Gehölz heraus in ein ausgedehntes Kornfeld, wo uns heftiges Maschinengewehrfeuer von einer Bodenwelle ca. 500~m vor uns empfing. Ich ließ hinlegen, von rechts kam der Ruf: Halten! Wir schützten uns gemäß Vorschrift durch eine kleine, mit dem Spaten rasch aufgeworfene Welle des trockenen und damit einen relativ guten Kugelschutz bildenden Sandbodens. Die Russen strichen das Kornfeld mit MG- und Schützenfeuer ab. Kaum hatte ich einen ca. 40 cm hohen Sandhügel fertig und mich platt hingedrückt, hatte ich das Gefühl, dass ich geschlafen habe vor Erschöpfung, ob 2 Minuten oder 1 Stunde, weiß ich nicht. Ich bemerke mit Entrüstung, dass sich mein linker Nebenmann, kaum 2 Meter von mir entfernt, noch keinen Sandschutz geschaufelt hat und schnauze ihn an -- da sehe ich den Blutfleck mitten auf seiner Stirn. Er war tot. Bald kam von hinten Verstärkung, wir rückten vor, nur noch hin und wieder von kleineren Nachhutabteilungen des sich zurückziehenden Feindes beschossen.

Die Russen zogen sich etappenweise zurück und setzen sich in einer mehr oder weniger gut ausgebauten Stellung fest. Wir mussten die Stellungen bei Helenow sichern -- bereits Werke zum Schutze der russ. Weichselfestung Iwangorod (poln. Modtyn).\\

Ein tolles Bild der Verwirrung bot ein deutscher schwerer Haubitzenzug: 2 Geschütze waren, \enquote{vierelang} (d.h. von 4 Pferden, d.h. 2x2 hintereinander), denkbar ungeschickt, da auf offenem Felde ohne Deckung gegen Sicht aufgefahren und sofort von russischer schwerer Artillerie mit Erfolg aufs Korn genommen worden: die Pferde scheuten und jagten wild in verschiedenen Richtungen mit den Geschützen durch unsere Schützenlinie hindurch und davon. Jetzt waren wir das Ziel der russ. Feldartillerie, die zunächst zu kurz schoss. Ich glaubte blitzartig das schachbrettartige Vorwärtstasten der russ. Artillerie erkannt zu haben, errechnete, dass sie die Mitte unseres Zuges in einer halben Minute erfasst haben werde, machte mir ein um ca. 75~m Zurückbleiben meines linken Flügels zunutze, um rasch dorthin zu rennen -- und schon krachte die Schrapnellsalve in das Zentrum meines Zuges -- ich hatte meine beiden Entfernungsschätzer verloren, wie ich hörte nicht allzu schwer verletzt, und zwei Mann waren tödlich getroffen.

Bald gerieten wir in das Schnellfeuer einer Batterie schwerer russ. Haubitzen, die uns mit Granaten beschossen. Da ich wusste, dass die Granate bei gleicher Zielsetzung wegen der steigenden Erhitzung -- und damit Ausdehnung -- niemals in denselben Trichter, sichtbar durch die riesige schwarze Wolke aus Pulverdampf und Erde, trifft, stürzte ich mich während des russ. Sperrfeuers beim Vorgehen oder richtiger Rennen immer in einen solchen frischen stinkenden und von glühend heißen Sprengstücken nicht freien Trichter, drückte den Kopf tief nach unten und schon zischte und fauchte die nächste Granate heran, um 50 oder auch nur 10 Meter entfernt zu explodieren, und auf diese Weise gelangten wir fast ohne Verluste ca. 300 Meter vor die russische Stellung. Sie lag unter deutschem Artilleriefeuer und die russ. Infanterie schon nicht lebhaft, aber, wie ich bald sah, gut gezielt: ca. 30~m schräg hinter mir stand im Schutze des Kamins eines abgebrannten Hauses ein Kompanieführer des Nachbarbataillons und gab laut rufend seinen Leuten das Visier, d.h. Ziel und Entfernung des Gegners an; er beugte sich wohl 2 Sekunden hinter der Deckung hervor -- und schon hatte ihn ein russischer Scharfschütze in den Leib getroffen. Immer wieder übertönte das Schreien der Unglücklichen den Gefechtslärm.

Plötzlich schossen die russischen Infanteristen lebhafter und boten mit ihren Köpfen selbst ein Ziel; ich gebot gezieltes Schützenfeuer. Auf einmal hörte ich ein paar Meter links von mir einen meiner westfälischen Landser laut fluchen und sah, wie er einen jungen, erst kürzlich eingetroffenen Schlesier ohrfeigte. \enquote{Was ist da los?} rief ich. \enquote{Feldwebel, der Junge zielt nicht richtig, er muckt\footnote{d.h. schließt beim Abdrücken die Augen} beim Schießen.}

Nach einiger Zeit wurde das deutsche Artilleriefeuer auf die russ. Infanteriestellung vor uns lebhafter, schwoll zum Schnellfeuer an und wurde plötzlich auf weiter hinten liegende Ziele des Gegners gelenkt. In diesem Augenblick scholl es aus dem Munde aller Porteépéeträger \enquote{Sprung auf! Marsch, marsch!} und wir stürmten gegen die russ. Stellung. Die Russen schossen anfänglich, traten aber, bevor wir ihre Feldstellung erreichten, aus dem Graben heraus und hielten die Hände hoch. Alle waren -- im Vergleich zu unseren Leuten -- prachtvolle große blonde Gestalten mit sympathischen Gesichtern, rassisch eine gute germanisch-finnisch-slawische Mischung: es waren Petersburger Gardeschützen, ein Eliteregiment. Alle freuten sich, dass für sie der Krieg zuende war. Sie zeigten uns die Photographien ihrer Bräute, bzw. Frauen und kleinen Kinder.

Die Nacht brachten wir, d.h. die Kompanieoffiziere und ich als Offiziersaspirant in einer gedielten sauberen und geräumigen polnischen Bauernstube zu; die Bäuerin war mit ihren kleinen Kindern im Dorf geblieben. Sie bedeutete uns freundlich, wir sollten uns Stroh holen. Der Kompanieführer Riedel drückte ihr seine Anerkennung aus, indem er das einzige polnische Wort, das er kannte, \enquote{dobrze} rief und dabei seine Arme sprechen ließ.

Wir übernachteten auf dem Vormarsch häufig in Bauernhäusern; dieses hier zählte zu den wenigen, die ungezieferfrei waren, d.h. weder Flöhe, Läuse oder Wanzen beherbergten. Je weiter wir nach Osten kamen, umso häufiger wurden diese unerfreulichen Insekten. Ich weiß nicht, was schrecklicher war, ein völlig verwanztes oder verflohtes Haus: in beiden Fällen habe ich es je zweimal vorgezogen, den zweiten Teil der Nacht im Freien zuzubringen. Der ständige Kampf gegen Kleiderläuse erwies sich als notwendig, ohne jedoch zu einer völligen Niederlage dieser Tierchen zu führen. In der Nähe der deutschen Reichsgrenze wurden für Heimaturlauber \enquote{Entlausungsanstalten} eingerichtet; die beim Grenzübertritt obligatorische Bescheinigung lautete in meinem Falle: \enquote{Es wird bescheinigt, dass Leutnant Busse entlaust und seuchenfrei ist.}

Wir näherten uns der Festung Iwangorod; das merkten wir daran, dass wir am Tage Granatfeuer aus schwerem Festungsgeschütz bekamen und nachts große Festungsscheinwerfer, die Erde nach marschierenden Kolonnen und den Himmel nach Zeppelinen abtasteten. Hierzu der Frontwitz: Petrus hat mit sofortiger Wirkung angeordnet, dass die Englein Schlüpfer zu tragen haben!

\enquote{Zeppeline} waren die riesigen zigarrenförmigen Starrluftschiffe im wesentlichen aus Aluminium konstruiert und mit Leuchtgas, später mit Helium, gefüllt. Den Namen verdanken sie ihrem Erfinder, dem Grafen Zeppelin. Sie konnten beachtliche Lasten tragen und wurden im 1. Weltkrieg zum (meist nächtlichen) Transport von Bomben eingesetzt. Infolge ihrer großen Verwundbarkeit wurden sie schon im Laufe des 1. Weltkrieges vom Flugzeug verdrängt. Sie brummten weithin vernehmbar in der Tonlage der Hirschkäfer, die an schönen Sommerabenden in den Eichenwäldern Wolhynows uns umbrummten.

Eines Abends kam der Befehl, ich solle mit meinem Zug im Morgengrauen über die sumpfige Wiese längs des linken Flussufers in Richtung auf das Dorf Malinkowa vorstoßen, im Falle starken feindlichen Feuers mich jedoch an geeigneter Stelle festsetzen und nach Einbruch der Nacht in die Ausgangsstellung zurückkehren, nur Sturmgepäck einpacken.

Die russ. Posten schliefen jedoch nicht. Kaum hatten wir den Fluss in der Morgendämmerung so geräuschlos wie möglich überquert, Schützenlinie gebildet und waren 200~m auf dem quarksenden Boden vorangekommen, setzte lebhaftes Infanteriefeuer ein, wir rannten nach vorn so schnell wir konnten, mit einem Krach riss mir eine Kugel den Spaten aus der Hand -- ich hob ihn auf, er war noch ganz; ein Mann kriegte einen Beinschuss und wurde vom Sanitäter zurückgebracht und wir gelangten ohne weitere Verluste über die völlig platte etwa 1~km lange Flusswiese hinweg an eine kleine, wohl 2~m ansteigende Böschung, wo wir Halt machten, uns sammelten und uns vor Feindsicht geschützt mit guter Sicht nach vorn hinlegen konnten. Ich sagte den Leuten, sie könnten das Mitgebrachte verzehren. Ich kaute einen schon ziemlich harten Brotkanten und bestrich ihn mit Sardellenpaste, das einzige, was ich im Sturmgepäck bei mir führte. Da hörten wir in der Ferne einen schweren Artillerieabschuss, das Geschoss kam auf uns zu, das Zischen und Heulen wurde immer lauter bis es etwa 20 Sekunden nach dem Abschuss mit fürchterlichem Fauchen unmittelbar über uns hinwegflog -- wir glaubten den Geschosswind zu spüren -- und jeder von uns glaubte sein Ende gekommen, boten wir doch in unserer Schräglage von ca. 45° den Granatsplittern ein großes Ziel. Da -- ein dumpfes Aufschlagen -- keine Explosion -- und ein etwa 1 Minute währendes Zischen: die Granate war ca. 75~m hinter uns in den Sumpf geschlagen ohne zu krepieren.

Wir waren diesmal gerettet, das Festungsgeschütz beschoss nun in ruhiger Schussfolge die Stellung aus der wir aufgebrochen waren und rückwärtige Ziele. Selten schmeckte mir und uns allen das Feldküchenessen so gut wie an diesem Abend im spärlichen Mondeslicht. \enquote{Was haben Sie da eben aus dem Kochgeschirr weggeworfen?} fragte ich den Landser neben mir. \enquote{Eine tote Maus.} Sie beeinträchtigte nicht im geringsten unseren Appetit.

Die erste Stadt, die wir passierten, war die polnische Land- und Kreisstadt Zwolen. Die Russen hatten Zeit gehabt, fast die ganze Bevölkerung auf die trostlosen Straßen der Flucht nach Osten zu treiben und sie dann durch die den Rückzug notdürftig deckende Kavallerie, die sogen. Kosacken, wie sie es zu tun pflegten, in Brand zu stecken. Was war geblieben? Ein riesiger Aschefleck, aus dem in der Mitte die Mauern von ein paar Steinhäusern ragten -- sonst nur ein Chaos von Hunderten der gewaltigen slawischen Kamine. Der Kamin war in Dorf und Landstadt der feste, aus Stein gebaute Kern des Hauses, der gleichzeitig als Herd, Backofen, Heizung und meist, besonders im russischen Bauernhaus, den Senioren oben im Winter als Schlafstelle diente.

Im Winter 1915/16, es war in Weißrussland südlich Daranowitschi, hatte der Kompaniefeldwebel die Schreibstube in einem der völlig unversehrten grauen Bauernhäuser eingerichtet. Auf dem mächtigen Ofen, der Isla, überwinterten zwei russische Bauern, Vater und Sohn; dieser war 63 Jahre alt, der Vater wusste sein Alter nicht, der Sohn sagte uns, er ist über 90. Sie verließen den Ofen nur, um unser Feldküchenessen einzunehmen, und der jüngere stieg öfters herab, um sich mit einem Fidibus die mit Zeitungspapier gedrehte Zigarette anzustecken. Die Kleidung, Tag und Nacht die gleiche, wechselten die Mushiks\footnote{Mushik = russ. (leibeigener) Bauer} im Winter nicht. Ich konnte damals noch nicht genügend slawisch, um mich nennenswert mit diesen gutmütig und sympathisch wirkenden Mushiks zu unterhalten.

Südlich Iwangorod hatten sich die Russen auf das rechte Weichselufer zurückgezogen, suchten aber unsere Pioniere beim Brückenbau mit Feldartillerie-Feuer zu stören. Unser Regiment in Kompaniekolonnen geordnet ca. 3~km vor der Weichsel. Ich hatte im Felde noch kein Flugzeug am Himmel gesehen. Da tauchte plötzlich eines am hellblauen Himmel von Osten her auf. Mit dem Eisernen Kreuz auf der Tragfläche -- ein deutsches. Es kreiste in etwa 1000~m über uns und warf ein trichterähnliches Gebilde mit einem langen roten Wimpel ab. Sofort jagten zwei Bataillonsadjutanten zu Pferde nach der wichtigen Beute. Wie ich von einem Offizier, der von der Westfront kam, [erfuhr], waren Fliegermeldungen ein begehrtes Sammlerstück -- genauer die Verpackungen der Meldungen. Nach wenigen Minuten erscholl der Befehl: Bataillone mit Gefechtsbagage sofort zum Abmarsch fertig machen. Man vernahm den Inhalt der Fliegermeldung: \enquote{Feind in vollem Rückzug nach Osten.}

Die Weichsel hatte ein sehr breites, flaches Flussbett; nur in der Mitte war eine etwa 50-70 Meter breite Schifffahrtsrinne, die von den Pionieren mit Pontons überbrückt war. Das Regiment watete durch das knietiefe Wasser oder fuhr auf Panjewagen zur Pontonbrücke. Ich selbst gelangte auf der\linebreak Kompanie-Feldküche hinüber. Wir befanden uns jetzt unweit von Maciejowice und Podzamcze, Namen, die vielen Polen Tränen in die Augen trieben: hier waren 1863 \num{30000} Polen, meist Bauern und nur mit Sensen bewaffnet, von ihren russischen Unterdrückern erbarmungslos niedergemetzelt worden.

Eines Nachmittags, es war schon im August, hielt das Regiment auf einer großen Waldlichtung. Major Hilgendorf ließ mich zu sich kommen. \enquote{Busse, Sie erhalten einen ehrenvollen Auftrag. Das Dorf Kolpenien hier hinter dem Wald ist nach Meldung von Kavalleriepatrouillen noch von den Russen besetzt. Sie nehmen Kolpenien ein. Im Notfall fordern Sie bei mir sofort Unterstützung an. Baldigst Meldung.} Ich besprach mit meinen Leuten die Situation; wir gingen in Linie, Mann an Mann, ich voran mit dem Kompass und kamen in einer halben Std. an den Waldrand; das Straßendorf lag auf einer kleinen Anhöhe vor uns, 500~m durch eine Wiese vom Waldrande getrennt. Ich bildete, gegen Sicht geschützt, Schützenkette und dann rannten wir gegen das Dorf. Beim Verlassen des Waldes kam aus einem Waldstück rechts hinter dem Dorf sofort lebhaftes Infanteriefeuer, wir gelangten im Nu ins Dorf und auf die Dorfstraße, ich ließ die Leute Deckung nehmen, besah mir kurz mit meinen Entfernungsschätzern und zwei Unteroffizieren das Dorf nach möglichen Überraschungen und hob bei einsetzender Dämmerung 100~m östlich des Dorfes einen Behelfsschützengraben aus, nachdem ich dem Major einen Lagebericht geschickt hatte.

Im Morgengrauen weckte mich der Major mit seinem Adjutanten Frömsdorf; er war des Lobes voll, wusste meine Meldung auswendig, hinter jedem Satz ein anerkennendes Wort. \enquote{So, und jetzt langen Sie mal tief in die Schachtel hinein}, und damit reichte er mir einen Karton mit Keksen und -- mit etwas leiserer Stimme --: \enquote{Der Regimentskommandeur war auch sehr zufrieden, das EK ist Ihnen sicher.} Ich war eigentlich erstaunt, dass man von der Sache so viel Aufhebens machte.

Eine Woche später, -- es war ein drückend heißer, fast wolkenloser Augusttag, wiederholten sich für das Regiment die taktische und örtliche Situation: diesmal sollte das 3. Bataillon ein Dorf nehmen. Der Angriff wurde in aller Stille im Walde vorbereitet, doch plötzlich stiegen aus allen Häusern des großen Dorfes Rauchwolken und züngelten Flammen auf, im Nu war alles ein riesiges Flammenmeer. Es war fast windstill, der heiße Rauch stieg senkrecht zum Himmel und türmte sich zu gigantischen herrlichen Gewitterwolken auf. Die Leute stritten sich, ob die Wolkentürme vom brennenden Dorf oder von einem aufziehenden Gewitter kämen. In wenig mehr als einer Stunde war das schöne Wolkengebirge dahin, geblieben nur ein rauchender Aschehaufen aus dem Dutzende von gespenstischen Kaminen ragte.

In einer Nacht hatten wir unsere österreichischen Bundesgenossen in einer Feldstellung auf einer Anhöhe abgelöst. Vor uns ein gleichmäßig, völlig deckungsloser zu einem 400~m entfernten Waldrande abfallender Hang. Am Waldrande erkannte man eine gut getarnte russische Feldstellung; sie war besetzt. Im Laufe des Vormittags suchte mich Major Hilgendorf auf. \enquote{Busse, Sie werden heute nachmittag, es ist der Wille des Brigadekommandeurs, mit ihrem Zug die russische Stellung am Waldrand stürmen.} \enquote{Mit oder ohne Artillerievorbereitung, Herr Major?} \enquote{Das steht noch nicht fest.} \enquote{Das Gelände ist für den Angriff äußerst ungünstig; wir stehen beim Sturm für die Russen gegen den hellen Himmel, bieten sehr gute Ziele; das Unternehmen wird viel Blut kosten; der Erfolg ist sehr zweifelhaft, die Katastrophe der 6. Kompanie bei Czekazewice kann sich leicht wiederholen.} \enquote{Ja, Busse, das Unternehmen wird viel Blut kosten.} \enquote{Darf ich Herrn Major bitten, Artillerievorbereitung sicherzustellen?} \enquote{Ich will es versuchen, Busse, doch die Entscheidung trifft die Brigade.} Ich rief die Unteroffiziere und Gruppenführer zur Besprechung. Man murrte: \enquote{Immer der 3. Zug! Warum nicht die anderen Züge?} sagte man mir mit vorwurfsvollem Blick. Wahrscheinlich dachte man innerlich: das haben wir dem Ehrgeiz dieses jungen Kerls zu verdanken!

Ich gab Auftrag, das russische Grabenstück und den Waldrand scharf zu beobachten, ohne meinen ausdrücklichen Befehl nicht zu schießen. Kurze Zeit darauf glaubte ich mit dem Fernglas eine Bewegung zu beobachten; da kam auch schon mein wachsamer Unteroffizier Wojat an, er glaube, man bringe ein Geschütz in Stellung -- an einer anderen Stelle als der von mir beobachteten. Bald konnte ich dem Regiment melden: Feind hat zwei Grabengeschütze in Stellung gebracht -- mit dem Schnellfeuer zweier Grabengeschütze -- Geschosse, Kartätschen -- kann man den Angriff auf ein Grabenstück wie das vor uns liegende fast ohne Infanteriefeuer blutig abwehren. -- Oberleutnant Mühlau, Führer der M.G.-Abteilung beim Regiment, kam zur Überprüfung meiner Angaben. Mit dem Scherenfernrohr bestätigte er meine Beobachtungen.

Am frühen Nachmittag beobachtete ich eine Abteilung russ. Infanterie in Zugstärke ausgeschwärmt -- Entfernung 1400~m -- auf das Waldstück zueilen. Ich ließ nicht schießen, gab aber Meldung an das Regiment. Unsere Spannung wuchs. Dann kam gegen 5 Uhr nachmittags Meldung vom Regiment: Angriff auf russische Feldstellung abgeblasen. Unsere Freude war groß. Ich war stolz, von meinem Zuge eine sichere Katastrophe abgewendet zu haben. 

Ein mir unvergesslicher Abend Ende August 1915: die 5. und 6. Kompanie lagen als Regimentsreserve auf einer Anhöhe; vor uns das Tal des Bug-Flusses nordwestlich Brest-Litowsk, hinter den sich der Feind zurückgezogen hatte; ringsherum loderten in der Dunkelheit brennende Dörfer, Gutshöfe oder Weiler; in etwa 6~km Entfernung brannte die Stadt Slonins, man vernahm einzelne Detonationen; hin und wieder hörte man die Explosion russischer Schrapnells über der Stadt. Wir hatten einen Holzstoß in Brand gesetzt, man empfand die Wärme schön angenehm. In dieser Szenerie drängten sich immer mehr Kameraden um einen jungen Geiger der 5. Kompanie, der bezaubernd spielte. Es war sublimste Kunst, klassische Themen, vielleicht phantasierte dieser Virtuose klassische Variationen; alles lauschte lautlos, gepackt von der Erhabenheit großer Kunst -- auf dem Hintergrund schrecklicher Zerstörung ringsum. Es war für mich die erhabendste Stunde im ganzen Weltkrieg.

Wir überquerten den Bug auf der von unseren Pionieren gefestigten Pontonbrücke und kamen bald in den Bereich des z.T. echten Urwaldes von Bialowic in der Bialowica Puschtscha, eines der bekanntesten Jagdgründe des Zaren, reich an Elchwild. Es kam vor, dass wir vorübergehend den Kontakt mit den Russen verloren, Kavallerie- und Infanteriepatrouillen standen oft vor schwierigen Aufgaben.

Eines Nachmittages, während gerade ein heftiges Gewitter mit Platzregen niederging, erreichte unser Bataillon der Befehl, den Österreichern, die auf heftigen Widerstand der Russen gestoßen waren, zu Hilfe zu kommen. Nach Einbruch der Dunkelheit betraten wir einen Eichenwald, der etwas geradzu Gespenstisches hatte: der schmale Weg -- wir mussten in Zweierreihen marschieren -- führte durch schwefelfarbene bis gelbgrünliche, gefleckte Stämme, die Leuchtflecken schienen sich zu bewegen, vom Boden bis über unsere Köpfe zu steigen oder umgekehrt von oben bald hier, bald da herabzurieseln. Jeder dachte: um Himmelswillen, in einem solchen Schreckenswald nicht allein sein; man suchte körperliche Berührung mit dem Nebenmann, viele fassten sich an der Hand. Was war das? Der Wald schien nicht so sehr alt zu sein -- modernde, vom Gewitterregen durchtränkte Holzteile? Alle Leute bestätigten, dass sie ähnliches noch nicht gesehen hätten und auch ich muss an meinem Lebensabend sagen, dass ich wohl in einigen Fällen einzelne kleine phosphoreszierende Flecken gesehen habe, aber nie einen ganzen phosphoreszierenden Gespensterwald. Wir atmeten auf, als nach etwa einer Dreiviertelstunde mit dem Verlassen des Eichenwaldes der Spuk ein Ende nahm.

Der Vormarsch ging weiter in fast gerader nordöstlicher Richtung. Eines Nachmittags wurde die Kompanie abkommandiert als Wachkompanie des Brigadegenerals von Albrecht; er hatte sich in einem polnischen Herrenhause inmitten eines großen schönen Parks niedergelassen. Die Züge hatten verteilt die Gewehre zusammengesetzt und bauten Zelte. Da hörte man plötzlich in der Nähe lebhaftes Schießen. Was war los? Die Russen? Alles rannte, ohne Befehl abzuwarten, zu den Gewehren und stürmte gegen den im Park ausgeschwärmten Feind an -- Dutzende von Schweinen, die aus dem Schweinestall ausgebrochen oder von einem gewitzten Landser herausgetrieben waren.

Alle Befehle, das Schießen sofort einzustellen, blieben vergebens, bis der erschrockene und wütende General selbst erschien, tobte und mit schweren Strafen drohte. Meine Leute brachten eine fette Sau angeschleppt. Da wir im Zuge keinen Fleischer hatten, begann sofort ein schreckliches unfachmännisches Gemetzel: jeder säbelte sich mit seinem Seitengewehr oder seinem Messer nach Gutdünken ein Stück heraus. Bald wurde gebraten, gekocht, geschmort. Wie viele andere hatte mein Bursche fette Kartoffelpuffer gebacken und ich zählte zu denen, die sich tagelang mit verdorbenem Magen und Stuhlverhärtung herumschleppten.

Anfang September machte sich die herbstliche Kühle unangenehm bemerkbar. Die Lage an der Kampffront gestattete oft weder die Einquartierung in Panjehäusern noch den Bau der kleinen satteldachförmigen Zelte -- jeder Soldat führte außer dem Mantel eine Zeltbahn mit sich, die beide vorschriftsmäßig um den Tornister gewickelt wurden. Dann grub man sich, einzeln oder zu zweit, eine flache rechteckige Grube in die Erde, so dass man ausgestreckt liegen konnte und von dem meist wehenden kühlen Nachtwind leidlich geschützt war. Doch das Lager war weder weich noch warm; man wachte in der Nacht mehrfach auf, rannte sich warm, schlug sich die Arme, boxte sich mit einem Kameraden, trank, falls vorhanden, einen Schluck Schnaps und legte sich wieder hin. Doch manch einer erkältete sich oder bekam ernste rheumatische Beschwerden, sodass jetzt außer den Russen das Herbstwetter der Kampftruppe Verluste zufügte. wusste man, dass man voraussichtlich 2 Tage am gleichen Standort bleiben werde, wurden sofort Unterstände gebaut. Wir erhielten wieder Ersatz: an Stelle des verwundeten Leutnant Riedel trat der 20-jährige aktive Leutnant Nord; Leutnant d.R. Stallbaum erhielt die Führung des ersten Zuges. Kompaniefeldwebel blieb weiterhin Kollmeier, 33 Jahre, von Beruf Steiger, ein intelligenter, anständiger, tüchtiger Westfale; wir hegten schon seit Wochen freundschaftliche Gefühle für einander. Ich versorgte ihn mit Lektüre; wir sprachen oft lange miteinander. Dagegen war mir Nord vom ersten Augenblick an unsympathisch; er trug stets wie auf dem Kasernenhof rote Glacéhandschuhe, spielte den feinen Herrn, war aber innerlich unsicher und, wie sich bald herausstellte, feige. Unter den Ersatzmannschaften waren vier Elsässer, von denen ich drei erhielt: Masson, nur französisch sprechend, Roos, nur deutsch sprechend und Pissaroni, ein windiger gebürtiger Italiener. Nord erklärte vor der angetretenen Kompanie in Gegenwart der Neulinge: \enquote{Feldwebel, die Elsässer dürfen selbstverständlich nicht als Posten eingeteilt werden}, was natürlich so viel heißt wie man kann ihnen nicht trauen. Ich hielt diese Äußerung coram publico für denkbar dumm und gefährlich, und verständigte mich mit Kollmeier; von ihm erfuhr ich, dass ich von den vier Elsässern auf sein Betreiben drei erhalten hätte, er wisse, dass sie bei mir am besten aufgehoben seien. Roos und Masson hingen wie Kletten aneinander trotz ihrer Verständigungsschwierigkeiten. Ich machte sie, zumal diese \enquote{Stelle} \enquote{vakant} war, zu meinen Entfernungsschätzern\footnote{das hieß praktisch soviel wie meinen Ordonnanzen}, und es gelang mir bald, sie an meine Person zu fesseln. Solange ich noch in der Kompanie war, zählten sie zu den treusten und zuverlässigsten meiner Leute. Den Windhund Pissaroni holte ich mir öfter in meinen Unterstand, wir spielten Schach und sprachen Italienisch -- mit Masson sprach ich französisch, er hat mir eine Reihe französischer Volkslieder textlich und musikalisch beigebracht. Dieses soziale Verhalten eines Offiziersaspiranten konnte damals einem genormten preußischen Berufsoffizier nicht gefallen. Nord fragte Kollmeier, was ich eigentlich in Zivil sei. \enquote{Student der Philologie}, worauf er antwortete, Studenten könne er überhaupt nicht leiden, höchstens noch Juristen.

Gegen Ende September erstarrte der Bewegungskrieg zum Stellungskrieg. Die Front kam etwa 10~km östlich Baranowiči zum Stehen und verlief fast genau nordsüdlich, die weißrussischen Kleinstädte Gorodischtsche und Stolowitschi dicht hinter unserer Front lassend. Der Schava-Fluss und Oginski-Kanal lagen zwischen den Fronten, die weiter südlich mitten durch die Pripjet\-sümpfe, ca. 20~km östlich Pinsk verliefen.

Die Bevölkerung, soweit solche in diesem Frontabschnitt und im Hinterland östlich des Bug zurückgeblieben war, war vorwiegend weißrussisch, jedoch überall, gen Osten langsam schwächer werdend, waren Polen eingestreut. Mit dem polnischen Bevölkerungsteil konnte ich mich allmählich dank meinem Studium der Langenscheidschen \enquote{Briefe zum Selbstunterricht in der polnischen Sprache} leidlich verständigen. Ich hatte bemerkt, dass sich sowohl Polen aus der Provinz Posen, altem polnischen Stammesgebiet, sowie polnisch sprechende Oberschlesier -- von beiden Gruppen gab es wohl ca. 10\% in unserem Regiment -- sich gut mit ihren russisch oder weißrussisch sprechenden \enquote{Vettern} verständigen konnten. So beschloss ich, zunächst die russische Sprache beiseite zu lassen und meine Kenntnisse und Fertigkeiten im Polnischen weiter voranzutreiben. Ich sagte mir, dank der relativ nahen Verwandtschaft der slawischen Sprachen untereinander werde intensive Erlernung der einen slawischen Sprache auch das Studium einer anderen -- des russischen -- erleichtern.

Die Natur schien uns für die Strapazen eines Bewegungskrieges mit einem goldigen [sic] sonnigen Frühherbst zu entschädigen. Der Soldat des 2. Bataillons, weiterhin unter Führung des Major Hilgendorf, genoss ruhige Tage in verschiedenen nicht verbrannten Dörfern einige Kilometer hinter der Front. Vorn lagen österreichische und junge deutsche Regimenter. Man machte sich's in den \enquote{Panjehäusern} gemütlich, man wusch seine Sachen, putzte, flickte, schrieb Briefe. Das Eintreffen der Feldpost war immer ein großes Ereignis. Man las Zeitungen der Heimat.

Östlich des Bug hatten die Dorfkirchen römisch-katholischen Stils, baulich kaum von denen Ostdeutschlands verschieden, mehr und mehr dem Typus der Ostkirche Platz gemacht: Grundriss das griechische gleicharmige Kreuz, Ein- oder Mehrkuppelbau, die Kuppeln farbig -- grün, silbern, golden, blau; der meist hölzerne Glockenturm, stets niedriger und unscheinbarer als die Kirche, von dieser räumlich stets getrennt, wie in Italien das Campanile.

In der zweiten Oktoberdekade rückten wir etwa 30~km weiter nach Süden auf den Bahnknotenpunkt Baranowiči zu. In den Abend- und frühen Nachtstunden arbeiteten wir an b- und c-Linien, d.h. an Auffangstellungen für die Infanterie. Unversehrte Bauerngehöfte in der Nähe der Front, d.h. bis ca. 2~km entfernt, wurden abgerissen und als Baumaterial für Gräben und Unterstände verwendet. Wir richteten uns in Bol. Kolpenica ein. Eines Tages fuhr ich mit Feldwebel Kollmeier -- ich weiß nicht mehr mit welchem Auftrag -- im Pajewagen nach Baranowiči; in dieser weiträumigen Stadt von (im Frieden) ca. \num{20000} Einwohnern lag unsere große Bagage. Hier -- die Stadt war, abgesehen vom Bahnhofsgebäude, nicht beschädigt -- befand sich die Feldpoststelle unserer Division, mehrere österreichische Dienststellen; mit Interesse vernahm ich, dass bald eine Feldbuchhandlung eingerichtet werde.

Der größte Teil der Bevölkerung, die sich aus einer weißrussischen und polnischen Minorität und einer jüdischen Majorität zusammensetzte, waren in der Stadt geblieben. Bei einer im Jahr 1911 vorgenommenen Volkszählung ergab sich, dass 85\% der Einwohnerschaft von Baranowiči Juden waren, die sämtlich deutsch d.h. frühneuhochdeutsch fränkischer Prägung mit hebräischen Einsprengseln sprachen. Der gesamte Handel, die Schenken, Teestuben und fast alle Handwerke befanden sich in jüdischen Händen. Die Juden hatten schon auf dem Vormarsch ihre Deutschfreundlichkeit bekundet, beim Durchmarsch durch einzelne nicht oder wenig zerstörte Kleinstädte uns mit Wasser erfrischt und vor allem mit Backwaren und manche Gebrauchsgegenstände zum Kauf angeboten. \enquote{Wir Juden sein alle Freinde von die Deutschen.}

Die Stadt hatte im Innern eine Reihe von breiten aus Ziegelhäusern bestehenden Straßen. Auf einem riesigen Platz stand eine große aber nicht sehr schöne russische Kirche. Als imposanteste Profanbauten sind mir zwei Banken und eine Apotheke in Erinnerung, alle drei im klassizistischen Stil, die Apotheke, wie ich es meist in Weißrussland sah, mit einem schlichten aber eindrucksvollen klassizistischen hölzernen Vorbau.

Der Klassizismus ist bis zur Oktoberrevolution der letzte paneuropäische Stil, der bis zur entferntesten russischen Kleinstadt oder großem russischen Dorf vorgedrungen ist.

Aus unserem geruhsamen, fast idyllischen Leben wurden wir am 20.10.1915 bei Sonnenaufgang durch eine heftige russische Kanonade auf den österreichischen Frontabschnitt östlich Baranowiči gerissen. Schon bevor der Befehl \enquote{höchste Alarmbereitschaft} eintraf, hatten wir uns marschbereit gemacht; bald kam der Abmarschbefehl mit Sturmgepäck. Der Russe hatte die Österreicher, meist rumänischer Nationalität überrumpelt und war in ihre Gräben eingedrungen. Wir rückten in der Dämmerung in die c-Linie ein, wurden aber alsbald, da man einen nächtlichen russischen Angriff fürchtete, weiter nach vorn gezogen, wo wir die Nacht bei -2°-3° auf dem gefrorenen Felde liegend und nur mit dem Mantel sehr mangelhaft gegen Kälte geschützt, verbrachten. Feldwebel Kollmeier hatte mir vor dem Abmarsch ein Fläschchen Schnaps zugeschoben, das mir in dieser Nacht gute Dienste leisten sollte. Um trotz der Kälte und dem Ostwind schlafen zu können, nahm ich einen kräftigen erwärmenden Schluck aus der \enquote{Pulle}, schlief daraufhin 1,5 Stunden fest, nahm wiederum einen Schluck und schlief erneut im Gefühl der Wärme ein, und dies wiederholte ich noch mehrmals in der Nacht mit dem gleichen Erfolg. So hatte mich 45\%iger Alkohol vor Schädigung der Gesundheit durch Unterkühlung bewahrt, während nicht wenige aus unserer Kompanie im Feldlazarett behandelt werden mussten.

Bei Sonnenaufgang griffen die Russen mit Artillerieunterstützung an. Wir stürmten nach vorn, um über eine Anhöhe hinweg die in einer Talmulde befindliche Auffangstellung zur Unterstützung deutscher und österreichischer Einheiten zu erreichen. Wir waren kaum 200~m vorangekommen, da legte schwere russische Artillerie unmittelbar vor uns Sperrfeuer mit Granaten. Mit ganzer Kraft rannten wir durch die schwarze dröhnende Rauch- und Druckwand, rund 90\% kamen unversehrt hindurch. Das Schlimme war, dass man keine Abschüsse hörte, die Granaten waren also schneller als der Schall, also mussten die Russen ihre Artillerie weit vorgezogen haben. Auf der Anhöhe verschnauften wir, zum Teil durch zerschossene Gehöfte und Bäume gegen Sicht gedeckt; ich sah in der Ferne den russischen Fesselballon, der uns wohl zuerst entdeckt hatte, einer meiner Leute rief: \enquote{Vor uns auf der Anhöhe die russ. Batterie!} Es stimmte, ich rief: \enquote{Visier 1800, Schnellfeuer!} Meine Leute schossen offenbar nicht schlecht, denn in weniger als 5 Minuten war die Haubitzbatterie von der Anhöhe verschwunden. Dies war das einzige Mal, dass ich im Weltkriege Direktschuss durch feindliche Artillerie erlebte. Die Russen hatten, offenbar ihres Erfolges gewiss, die Artillerie zu weit vorgezogen. Im ersten Weltkriege schoss die Artillerie fast stets aus Deckung gegen Sicht, und die ladenden und feuernden Kanoniere waren so gut wie ungeschützt, wenn sie unter infanteristischen Direktbeschuss gerieten.

Am frühen Nachmittag war die Vorbereitung zum Gegenangriff beendet, unser Regiment stürmte und eroberte die verlorengegangene österreichische Stellung ohne große Verluste zurück. Zum Nahkampf kam es nicht, die Russen traten im letzten Augenblick aus den Gräben und hielten die Hände hoch. Der Versuch der Russen, vor Winterbeginn den wichtigen Eisen\-bahn\-knoten\-punkt Baranowiči zurückzuerobern, war misslungen.

Eine Woche lag unser Regiment in Ruhe in der Stadt Baranowiči. Appells, Appells und Exerzieren, auch etwas Freizeit. Ich kam mehrfach mit Groth, einem Kameraden aus dem Lager Döberitz, zusammen. Er war Vizefeldwebel und Offiziersaspirant in der 2. Kompanie des 1. Bataillons. Sein Kompanieführer war Oberstleutnant d.R. von Heinz, ca. 32 Jahre alt, der mehrere Jahre in den afrikanischen Kolonien Dienst getan hatte, Ende Juli 1914 seinen Jahresurlaub in Hamburg antreten wollte und dort gleich, wie er sich ausdrückte, geschnappt wurde. Er verstand wenig vom Exerzierdienst, und wenn er vor der Kompanie stand, musste Zugführer Leutnant Böhm, meines Alters, ihm von hinten die Kommandos soufflieren -- anders ging es nicht. Groth machte mich mit ihm bekannt. Von Heinz blieb auf seiner Pritsche liegen und erklärte: \enquote{Ich bin Lieger, ich liege gern.} Genau so verhielt er sich jedem gegenüber, der nicht sein Vorgesetzter war; auch bei den dienstlichen Besprechungen mit seinem Feldwebel blieb er liegen. Er erklärte sein Verhalten nicht ohne Selbstgefälligkeit als persönliche Eigenart, so wie ein anderer etwa sagt: zum Frühstück trinke ich stets ein Glas Sekt und rauche eine Havanna.

Eines Morgens war um 8 Uhr Bataillonsexerzieren, er hatte vorher die Führung der Kompanie Leutnant Böhm übertragen, er könne nicht kommen, da er zwei jüdische Frauen auf seinem Lager habe. Mein Freund Groth erhielt am gleichen Vormittag die fast dienstliche schriftliche Weisung, er solle ihm eine seiner beiden Frauen abnehmen, zwei seien ihm zuviel.-- Der Vorfall kam dem Regimentskommandanten von der Chevallerie zu Ohren, von Heinz wurde zu ihm beordert. Nachher erzählte er mir, liegend wie immer: \enquote{Der Alte hat sich aufgeregt, ich habe mich für einen Adligen unwürdig benommen, usw. -- na ich von Heinz -- was ist schon dran an diesem Adel, \enquote{von Hinze} oder \enquote{von Heinz}? Ist das etwas besonderes? Was meinen Sie? Der Alte wird komisch.} Außer einem Anranzer des \enquote{Alten} ist dem Oberleutnant von Heinz offenbar nichts passiert.

Am 31. Oktober rückte die 6. Kompanie bei Kälte und Schneetreiben ab, um den Bahnschutz an der Strecke Baranowiči-Brest-Litowsk zu übernehmen. Jeder Zug erhielt eine Strecke von 10-12 km. Ich hatte die längste Strecke zu marschieren und langte am zweiten Marschtag nach Einbruch der Dunkelheit in einem relativ geräumigen Bahnwärterhause an. Kompanieführer Nord hatte mir die Anweisung gegeben, noch im ersten Teil der Nacht eine Patrouille auf dem Schienenstrang bis zum 9~km entfernten Domanowo, seinem Sitz, zu schicken und ihm Meldung zu erstatten. Da miserables Wetter und stockfinstere Nacht war, ignorierte ich zum Wohle meiner Leute unbekümmert diesen mündlichen Befehl und sandte erst nach Tagesanbruch meine Meldung.

Das war eigentlich ein bedenklicher Fall von militärischer Insubordination\footnote{Gehorsamsverweigerung im Dienst}, und so sah es auch Nord. Kaum hatte ich mich weisungsgemäß mit meinem Zuge in den ziemlich engen Bahnwärterhäuschen bei dem fast restlos verbrannten Dorfe Borki eingerichtet, brachte der Streifendienst, den ich mit Domanowo unterhielt, die Order, mich an dem und dem Tage nachmittags 4 Uhr bei ihm zu melden. Nord empfing mich in der Kompanieschreibstube, die in einem steinernen Hause neben den Trümmern des Bahnhofgebäudes und des Wasserturmes untergebracht war. Feldwebel Kollmeier hatte mir vor dem Betreten des Hauses gesagt: \enquote{Nord ist schlecht auf Sie zu sprechen. Er sagte vorgestern in der Schreibstube: \enquote{Viezefeldwebel Busse ist ein frecher Bube.}} Ich fragte Kollmeier, wer außer ihm zugegen gewesen sei. Er nannte mir die Namen, es waren außer Kollmeier noch vier Leute der Kompanie. \enquote{So, das war also vor versammelter Mannschaft!} Kollmeier meinte: \enquote{Ja, das war es.} \enquote{Darf ich Nord gegenüber von Ihrer Mitteilung Gebrauch machen?} \enquote{Ja, gewiss!} antwortete der tüchtige Feldwebel.

Nachdem ich mich stehend mit Helm auf dem Kopf in strammer Haltung vor dem sitzenden Nord wegen der Insubordination schlecht und recht herausgeredet hatte, sodass die Gefahr einer dienstlichen Meldung nach oben abgewendet war, forderte Nord mich auf, es mir bequem zu machen und mich zu setzen. Ich lehnte ab und bat, das dienstliche Gespräch fortsetzen zu dürfen. Er war erstaunt und erwartungsvoll. \enquote{Mir ist zu Ohren gekommen, dass Herr Leutnant mich einen frechen Buben -- und dies vor versammelter Mannschaft, genannt haben.} Hier und bei allen Wiederholungen dehnte ich den u-Laut im Wort Bube und bemühte mich, dem Wort durch Tonfall und Mienenspiel einen möglichst despektierlichen Sinn, so in der Nähe von Schuft oder Verbrecher, zu geben. Nord erblasste. \enquote{Wenn ich das gesagt habe, so habe ich es gesagt wie etwa \enquote{dreister Junge} oder \enquote{verwegener Mensch}\dots}\enquote{Nein, Herr Leutnant, \enquote{Bu-be, Bu-be ist etwas ganz anderes, eine schwere Beschimpfung, eine unerhörte Beleidigung.} Nord stotterte: \enquote{So sehe ich\dots das aber\dots nicht. Und wer hat Ihnen das überhaupt gesagt?} Hier sprang Kollmeier auf und rief: \enquote{Das bin ich gewesen, Herr Leutnant, ich bin nicht so feige, für das was ich sage und tue offen gerade zu stehen.} Nord (unsicher): \enquote{Also Sie, Feldwebel -- aber das war doch mehr privat, nicht vor versammelter Mannschaft.} Kollmeier stellte fest, dass außer ihm noch vier Soldaten der Kompanie bei der Beleidigung zugegen waren. Nord wand sich wie ein Wurm, ich gab aber nicht nach, verlangte Wiedergutmachung. Was ich darunter verstehe, wollte Nord wissen. \enquote{Entschuldigung vor versammelter Mannschaft.} Als Nord nicht darauf eingehen wollte, drohte ich, wenn ich erst Offizier sei, werde ich mir gemäß dem Ehrenkodex Genugtuung zu verschaffen wissen\dots}

Es versteht sich, dass ich mich mit einem anderen, besseren Offizier und Menschen gegenüber als Nord es war, mich zu einer solchen gewagten Drohung nicht verstiegen hätte.

Nord wurde jetzt weich in den Knien, beorderte die Zeugen des Vorfalls in das Geschäftszimmer und entschuldigte sich \enquote{vor versammelter Mannschaft} in aller Form bei mir.

Vierzehn Tage später sagte mir Kollmeier, an die Kompanie sei von oben die Weisung ergangen, umgehend die Papiere zwecks meiner Beförderung zum Offizier einzureichen. Nord mache der Bericht großes Kopfzerbrechen, vor allem weigere er sich, mir im Betragen die Note \enquote{sehr gut} zu geben, der unerlässlichen Voraussetzung für die Beförderung zum Offizier. \enquote{Nein}, so Nord, \enquote{das kommt nicht in Frage, ich kann nur schreiben \enquote{nicht immer einwandfrei} oder höchstens \enquote{nicht ohne Bedenken im ganzen gut}.} Kollmeier habe Nord klargemacht, dass jede von \enquote{sehr gut} abweichende Betragensnote einer eingehenden Begründung bedürfe. Daraufhin habe Nord ihn, Kollmeier, beauftragt, den Konduitebericht zu entwerfen. Das habe Kollmeier jedoch abgelehnt mit den Worten: \enquote{Einem Feldwebel steht es nicht zu, über einen Offizier oder Offiziersanwärter Konduiteberichte anzufertigen, das ist allein Aufgabe des Herrn Kompanie- bzw. Bataillonsführers.}

Als mich Nord nach einiger Zeit nach Domanowo zum Tee einlud, erfuhr ich von Kollmeier, dass Nord noch immer an dem Bericht über meine Person herumkaue, sich aber weder zum \enquote{sehr gut} noch zum eingehenden Bericht über mein angeblich nicht einwandfreies Betragen entschließen könne.

Acht Tage später erfuhr ich, dass Major Hilgendorf den fälligen Bericht angemahnt habe und dass Nord zuletzt widerstrebend in die noch offene Berichtslücke \enquote{sehr gut} geschrieben habe. Die Angelegenheit war also erledigt und wurde von Kollmeier und mir mit einigen Schnäpsen \enquote{aus der Heimat} besiegelt.

In der völlig verwanzten Bahnwärterbude hatte ich fünf Wochen lang genügend Gelegenheit, die Lebens- und Kampfesweise dieser Tierchen zu studieren. In der ersten Zeit, bevor wir wirksame Methoden zur Ausrottung der Insekten gefunden hatten, bewunderten wir ihre Intelligenz: obwohl wir morgens alle zahlreiche Schwellungen durch Bisswunden auf unserem Körper feststellten, hat keiner von uns eine Wanze beim Beißen gespürt und erwischt; und doch kam es bei unserer relativ faulen Lebensweise in Borki täglich vor, dass der eine oder andere noch lange nach dem Löschen der Petroleumlampe noch wach war. Wir glauben, dass die klugen Tiere ein Gespür dafür haben, ob ihr Opfer schläft oder noch wach ist; im letzten Fall warten sie geduldig, bis das Opfer eingeschlafen ist.

Zwei Wochen lang plagte mich ein Darmkatarrh, ich wollte nicht zum 30~km entfernten Bataillonsarzt gefahren werden, der mich wahrscheinlich ins Lazarett geschickt hätte. Als nichts mehr verdaut wurde, nach jeder kleinen Aufnahme von Zwieback heftige Schmerzen einsetzten und schließlich die Exkremente aus hellem Schleim bestanden, fastete ich streng 36 Stunden, aß dann hungrig ein paar Zwiebäcke -- und nach einer Stunde wieder dasselbe, Schmerzen etc. Anfang Dezember marschierten wir plötzlich ab, ich fürchtete, den Fußmarsch zur Front nicht zu schaffen -- aber es ging, und siehe da, wie durch ein Wunder war ich plötzlich ohne die geringste ärztliche Hilfe wieder gesund.

Ich durchstreifte die Landschaft mit meinen beiden elsässischen Ordonnanzen Masson und Roos, die Dörfer waren fast restlos niedergebrannt, selten hauste hier oder da kümmerlich ein Greis. Hier und da sprang ein abgemagerter Hund aus einer Grube hervor und attackierte uns, ein treuer Wachhund, der das bis auf den Kamin verkohlte Grundstück der geflohenen Herrschaft verteidigte. Ich erschoss sie, die sonst im Laufe des Winters elend verhungert wären. Spuren von Wild habe ich, mit Ausnahme von Wolfsfährten nicht im Schnee gefunden.

Das Bataillon Hilgendorf bezog einen Schützengraben südöstlich von Ljachowitschi (d.h. Lechendorf, genauer Sippe der Lechen-Polen) von den Russen getrennt durch den Sektschara-Fluss. Die Stellung war leidlich ausgebaut mit passablen Unterständen. Der Winter war zunächst für russische Verhältnisse sehr mild, vorwiegend Temperaturen um Null, Schnee mit Tauwetter wechselnd. Am zweiten Weihnachtsfeiertag stieg ich bei dichtem Nebel aus dem Graben um meinen und den angrenzenden Zugabschnitt mir von oben anzusehen. Da, auf einmal ein Schuss von russischer Seite, ich spüre so etwas wie einen leichten Klaps am Rücken, springe sofort in den Graben, sehe, dass sich der Nebel für einige Augenblicke ein wenig über die Flussmulder erhoben hat und ich so dem russischen Scharfschützen ein Ziel geboten hätte. Im Unterstand stellte ich mit meinem Burschen fest, dass der Mantel in der Kreuzgegend Ein- und Ausschuss hat und der Rock angekratzt war, 1-1,5 cm von der Wirbelsäule entfernt.

Zwischen Weihnachten und Neujahr erhielt ich vom Bataillon den Auftrag, mitternachts eine Patrouille gen Osten zur Schtschara zu unternehmen. Es war schwierig, Glatteis auf leicht zum Flusse hin abfallendem Gelände, wir gingen von einer durch niedergebrannte Gehöfte vorgezogenen Stellung, es war spannend, ich hatte Herzklopfen; ging eine Leuchtkugel hoch, boten wir ein gutes Ziel; bei Gefahr des Ausrutschens auf dem glatten, leicht abschüssigen Gelände. Ich war froh, als es stärker zu regnen begann und ich daher mit meinen vier Mann, allerdings ohne besondere Beobachtungen gemacht zu haben, den Rückweg antreten konnte.

In der Sylvesternacht, es war Neuschnee gefallen, hatte ein Kompanieführer des Nachbarregiments 7. R. 48 mit einem Offizierstellvertreter kräftig gezecht und sich gewaltigen Mut angetrunken. Nach dem \enquote{Prosit Neujahr!} waren die beiden nicht zu halten, sie machten auf eigene Faust eine Feindpatrouille, um \enquote{den Russen zu zeigen, was eine Harke ist} -- und wurden nicht mehr gesehen.

Nach Neujahr regnete und regnete es, der Graben verwandelte sich in einen Morast, die \nolinebreak{Schtschara} vor uns in einen See; die Leute wurden mir krank, ich machte den Vorschlag, die Truppe -- bis auf einzelne Posten -- auf b- bzw. c-Stellung \enquote{ins Trockene} zurückzunehmen; es wurde abgelehnt. Ich meldete mich daraufhin (lächerlich und unglaublich von mir!) krank, ging zum Bataillonsarzt: \enquote{Habe Gliederreißen}. Der Arzt gab mir morgens Tee zu trinken, und damit gelang es, kleine Übertemperaturen von 37,1° zu erzeugen. Hilgendorf war -- begreiflicherweise -- von meinem Verhalten enttäuscht, und sagte, sein ganzes Haupt missbilligend schüttelnd: \enquote{Busse hat Reißen -- was soll ich davon denken?}

Am übernächsten Tag wurde die 6. Kompanie -- jetzt unter dem langen, eingebildeten Jäger-Hauptmann Bühmann -- abgelöst; wir lagen in Blockhäusern -- von der Truppe gebaut -- im Walde in Reserve, bauten weitere Blockhäuser bei herrlichem Winterwetter. Nach Neuschneefall war die Temperatur auf -25° gesunken, aber dank völliger Windstille sehr erträglich.

Da kam schon nach einer Woche meine Ernennung zum Leutnant der Reserve mit Versetzung zur 3. Kompanie des 1. Bataillons. Das 1. Bataillon rückte alsbald in Reserve weiter südlich hinter den Oginski-Kanal. Hauptmann Hubert, Chef der 3. Kompanie, schickte mich zum Dorfe Omelno (deutsch etwa Dünenort), einem kleinen Dorf, ca. 3-4~km westlich des Oginski-Kanals, das ich \enquote{aufbauen} sollte: Dächer in Ordnung bringen, Öfen einbauen etc. zur Aufnahme der ganzen Kompanie.

In diesen Tagen hatte ich Zeit und Muße, meine Beziehungen zu Lotte Steinberg zu überdenken. Der Gedanke mich so früh, mitten im Kriege, mitten im Studium, an ein Mädchen gebunden zu haben, bedrückte mich mehr und mehr; ich fühlte mich unfrei, hielt meine heimliche Verlobung für übereilt und tat Lotte den, wie sich erwies, großen Schmerz, die Verlobung aufzuheben und unsere Beziehungen in ein \enquote{reines Freundschaftsverhältnis} zu verwandeln.

Nach 10 Tagen etwa erschien Hubert mit den beiden anderen Zügen; er war nicht mit allem, was ich geschaffen hatte, einverstanden; er war in meiner damaligen Sicht ein großer Pedant, in Zivil Ingenieur bei Siemens\&Halske, der sich, wenn er angeheitert war, seines großen Einflusses bei der Firma rühmte: \enquote{\num{40000} Mann -- ich wiederhole: vierzigtausend Mann tanzen nach meiner Pfeife! Was ist Moritz dagegen (gemeint unser Regimentskommandeur Moritz von der Chevallerie) mit seinen \num{2000} Mann!}

Hubert gab uns Zugführern Leutnant Niehaus, Leutnant Rühling und mir stets detaillierteste, sehr klar und langsam mit leicht beim z anstoßender Zunge vorgetragene Anweisungen, duldete weder Mitschreiben noch Unterbrechungen durch eine Frage -- geschah dies, so wies er uns zurecht: \enquote{Ich wiederhole alles noch th-thweimal.}

Unser Leben glich für einige Zeit dem der Eulen: wir schliefen am Tage, rückten bei Einbruch der Dämmerung mit Schanzzeug ab, marschierten 1,5 Stunden, hoben dort bei Frostwetter eine Reservestellung im tief gefrorenen Boden aus, rückten gegen 3 Uhr morgens wieder ab, langten nach 4 Uhr in unserem Bauernhaus wieder an, wo unsere von der Arbeit befreiten Burschen uns ein üppiges Mahl zubereitet hatten -- wir konnten ja in der Kantine damals, Anfang 1916, noch zusätzliche Lebensmittel, nämlich einige Fleisch- Gemüse- und Fischkonserven und Büchsenmilch zum Feldküchenessen hinzukaufen. In der ganzen Kriegszeit hat uns das Essen nicht so gut gemundet, wie morgens in der 5. Stunde, in dem herrlich warmen russischen Blockhaus.

Die vorderste Stellung, für die wir auf langsam ansteigendem Gelände in ca. 800-1000~m Abstand für die Infanterie eine Auffangstellung in den Nächten anlegten, verlief unmittelbar am Oginski-Kanal auf moorigem Boden; die Unterstände mussten z.T. über dem Erdboden gebaut werden. Schlug unweit eine russische Granate ein, schwankte der ganze Boden. Um den genauen Standort der russischen Batterie ausfindig zu machen und sie alsdann wirksam bekämpfen zu können, hatte man Artillerie-Lichtmesstrupps hinter der Front eingerichtet. Das Prinzip war höchst einfach: gegeben waren die Entfernung der einzelnen Messtrupps von einander und die beiden beim nächtlichen Aufblitzen des Mündungsfeuers gemessenen Nebenwinkel -- und damit war der Standort der Batterie trigonometrisch zu errechnen und auf der Karte genau einzuzeichnen. In den kalten Nächten wärmte ich mich gelegentlich beim benachbarten Messtrupp auf, zu dessen Leitung mein Freund, der Steiger Kollmeier, unter Beförderung zum Offizierstellvertreter kürzlich abkommandiert war. Er hatte mir vor dem Verlassen unseres Regiments die Kröner'sche Taschenbuchausgabe des Faust zurückgegeben mit Hinweis auf Textstellen, die sich auf das Soldatenleben bezogen. Faust und \enquote{Also sprach Zarathustra} waren die beiden Bücher, mit denen ich ins Feld gerückt war.

Wie schwer es ist, die Entfernung eines Feuerscheins oder eines Aufblitzens in der Nacht zu schätzen, das zeigte sich im Winter bei nächtlichen Bränden russischer Bauernhäuser (russ. Isbá - \textcyr{изба}) die mit Soldaten belegt waren. Alles bis auf den Kamin aus Holz bestehend, die Soldaten auf Stroh liegend; durch Unvorsichtigkeit (z.B. beim Rauchen) kam es nicht selten zu Bränden. Der Gegner pflegte auf die Feuerstelle zu schießen, wo er mit Recht eine Anzahl Soldaten beim Löschen oder Lokalisieren des Brandes vermutete. Als ca. 2~km hinter dem Oginski-Kanal ein Bauernhaus brannte, verfehlte die russ. Artillerie ihr Ziel um ca. 2 km.

Einmal war ich einige Tage zu einer anderen Kompanie abkommandiert, die etwa 6~km hinter der Front südlich Baranowiči in einem fast unversehrten russischen Dorf lag. Wir Offiziere hatten uns nach 10 Uhr abends auf unser Nachtlager begeben, das Licht gelöscht und plauderten. Da plötzlich Infanteriefeuer, ganze Salven von Schüssen, die unweit des Dorfes zu fallen schienen. Wir fuhren auf. Was war los? Ein russischer Stoßtrupp soweit hinter die Front vorgestoßen? Während wir zu Rock und Koppel mit Pistole und die Burschen zum Gewehr griffen, fiel uns ein Flackern auf: eines der Nachbarhäuser brannte, die Soldaten hatten das Weite gesucht unter Zurücklassung der Patronengurte, die jetzt im Feuer explodierten und das Geräusch von Infanteriefeuer hervorriefen. Das -- wie alle russischen Bauernhäuser -- strohgedeckte Haus verwandelte sich im Nu in eine riesige funkensprühende Flamme; unsere Leute stiegen eiligst auf die umliegenden Dächer, um sie zu begießen, die Wassereimer wanderten vom nächsten Ziehbrunnen über eine Menschenkette. Nach wenigen Minuten eröffnete die russische Artillerie das Feuer -- zu unserer Freude schoss sie mindestens 4~km zu kurz. In weniger als einer Stunde war vom Bauernhaus nur ein glühender Aschehaufen übriggeblieben, es wurden Feuerwachen eingeteilt und wir gingen schlafen. Die Bewässerung der Dächer hatte weiteren Schaden verhütet.

Ende März verschwand die Schneedecke und die Wege verwandelten sich für Wochen in Morastkanäle. Am Nordrand der Pripjetsümpfe lag unsere Kompanie einige Tage hinter dem im vordersten Graben stehenden Bataillon Major Hilgendorf. Ich hatte mit meinem Zuge nach Anweisung des Majors den Zufahrtsweg für die Feldküche usw. vom ca. 40 cm tiefen Morast mit Schaufeln zu befreien und das noch gefrorene Wegbett anschließend mit Stroh zu bedecken. Nachmittags, als ich schon fast fertig war, erschien der Regimentskommandeur und herrschte mich an: \enquote{Busse, was soll der Quatsch?} Ich erläuterte ihm die von Major Hilgendorf erhaltene Anweisung. \enquote{Alles Scheiße, alles Mist! Entfernen Sie sofort das Stroh und füllen Sie das Bett mit festem Material, Steinen, Ästen u.a. aus, verstanden?} Eine halbe Stunde später kam Major Hilgendorf, sah erschrocken mein Treiben. \enquote{Aber Busse, was machen Sie denn da?} \enquote{Ich handle auf Befehl des Herrn Regimentskommandeurs, er war vorhin hier und kritisierte die Anweisungen des Majors -- darf ich wörtlich zitieren?} \enquote{Tun Sie das, Busse!} \enquote{Er sagte: \enquote{Es ist alles Scheiße, alles Mist.}} Ich freute mich diebisch auf die Reaktion \enquote{Stehkragens}, eines Ästheten, der auch im Felde die gepflegte Form wahrte. Er war sichtlich betroffen. \enquote{So, das hat er also gesagt. (kurze Pause)} Es folgte der Übergang zu einer längeren, ins Philosophische mündenen Betrachtung: \enquote{Da sehen Sie, Busse, wie verschieden die Menschen sind\dots}

Eines Tages besichtigte Hauptmann Hubert mit mir die vorderste Stellung, in die wir bald einrücken sollten. Wir kehrten bei Leutnant Gölden ein, dem Kompanieführer der Fünften, einem phlegmatischen, gemütlichen Kölners, in Zivil Gerichtsassessor, der in jeder Situation zum Verdruss Hilgendorfs herrlich zu schlafen vermochte. \enquote{Ein juter Schnaps jefällig, Herr Hauptmann?} Drei Gläser wurden gefüllt, wir wollten anstoßen, da sagte Hubert, das Glas Schnaps missbilligend zurückschiebend: \enquote{Dieser Schnaps ist 13° warm; ich trinke nur Schnäpse, die 7 bis allerhöchstens 8° haben.} Gölden war erstaunt, mich belustigte dieser eigenartige, bald 50 Jahre alte Herr, der sich für sehr gescheit hielt und es wohl auch war, allerdings mit Einschränkungen: Aus den Ersatzmannschaften, die wir im Mai kriegten, hatte ich einen Münchner Koch, namens Gruber, zu meinem Burschen gemacht. Er bereitete zusätzlich mit viel Geschick für uns Offiziere den \enquote{Empfang} zu, seine Spezialität waren Saucen, ein paar mal machte er aus schon bedenklich riechenden Seefischen buchstäblich ein leckeres Mahl. Hubert konnte es nicht lassen, als er beim Regimentskommandeur zu Gaste war, mit den Qualitäten \enquote{seiner} Kompaniekochs zu renommieren. Schon zwei Tage später war im Regimentsbefehl zu lesen: Landsturmmann Gruber von der 3. Kompanie wird mit sofortiger Wirkung zum Regimentsstab versetzt -- wir waren unseren tüchtigen Koch und ich meinen aufmerksamen Burschen los.

Ein andermal hatte ich einen Laufgraben zwischen der a- und b-Linie anzulegen. Hubert bestimmte den Neigungswinkel der Wände mit 65° und Verkleidung der Wände mit Stroh. Ich hatte Bedenken, wollte die Wände mit Faschinen verkleiden. Hubert darauf: \enquote{Busse, Sie haben wenig technisches Verständnis. Hier spielt der Reibungskoeffizient eine entscheidene Rolle. Ich wiederhole: Sie machen etc\dots} Eine knappe Woche hielt die Verkleidung, dann wurde sie großenteils von einem Gewitterguss hinweggespült. Hubert dazu: \enquote{ein solcher Platzregen stand nicht im Programm.}

Im Zuge einer Verschiebung des Bataillons machte ich Quartier in einem weißrussischen Dorf im Pripjetgebiet. Das Straßendorf lag, wie stets in Russland, an einer für deutsche Verhältnisse ungewöhnlich breiten, jetzt im April versumpften, selbstverständlich ungepflasterten Straße. Viele der Bauernhäuser der \enquote{\textcyr{избы}} [isbüi] hatten keinen Kamin; in der \enquote{\textcyr{изба}} [isba] befand sich in dem einzigen Raum ein steinerner offener Herd, dessen Rauch durch ein Loch in der Decke auf den Boden und von dort durch ein Uhlenloch im Giebel nach außen ziehen konnte -- falls der Wind günstig war. Der Fußboden bestand, wie meistens, aus gestampftem Lehm. Die Familie lagerte auf Stroh, ich näherte mich ihnen gebückt und verhandelte in dieser Haltung mit ihnen, denn bis ca. 140 cm über dem Fußboden herab staute sich bei Windstille der Rauch. Dies waren die primitivsten Dauerwohnverhältnisse, die ich in Europa kennengelernt habe. Natürlich nehme ich die Höhlenwohnungen aus, die ich 1935 bei Budapest sah. Die spanischen Höhlenwohnungen, vor allem in Andalusien, zeigen ein hohes kulturelles Niveau.

Eine willkommene Abwechslung war für mich mein kurzes Kommando zu einem \enquote{Infanterie Fliegerlehrgang} nach Poryck, dem Sitz der Armeeabteilung. Es handelte sich um infanteristische Übungen in Verbindung mit Flugzeugen, die vor allem den Zweck verfolgten, für Flieger die eigenen Infanteriestellungen besonders die vordersten deutlich sichtbar zu machen. Das geschah vornehmlich durch Auslegen von weißen Tüchern. Mir schien dieses Verfahren aus verschiedenen Gründen fragwürdig und ich habe auch nichts von seiner praktischen Anwendung an der Front gehört. Ich hatte Gelegenheit, einen kurzen Flug in einem Doppeldecker zu machen: ich saß schon in der kleinen, offenen, für heutige Verhältnisse sehr primitiven Kiste, da ertönte Fliegeralarm, der Anflug zweier russischer Flugzeuge wurde gemeldet, und ich musste sofort wieder aussteigen. Dafür war ich mit zwei anderen Teilnehmern des Kurses zum Abendessen bei Kommandeur Generalleutnant von Riemann und seinem Stabe geladen. Es war im feudalen Schloss eines Großgrundbesitzers. Man aß gut und trank guten Wein. Der Generalleutnant -- Anrede Ew. Exzellenz -- saß mir schräg gegenüber, prostete mir zu und fragte, mit einem Blick auf meine nur mit dem \ac{ekii} und der österreichischen Tapferkeitsmedaille geschmückte Brust \enquote{Herr Leutnant, warum haben Sie noch nicht das EK I?} Ich antwortete etwas dreist aber sachlich richtig: \enquote{Ew. Exzellenz, diese Auszeichnung zu erwerben ist für einen jungen Leutnant äußerst schwierig, wenn weder der Herr Kompanieführer noch der Herr Bataillonskommandant sie besitzen.} Ich stieß beim General auf volles Verständnis; so etwas sei wohl leider ziemlich häufig, er missbillige es aber, wenn bei den Vorschlägen für Ordensverleihungen die Anciennität eine Rolle spiele.

Übrigens erfreute sich damals die Fliegerstaffel einer Freiheit, die im 2. Weltkriege schwerste Freiheitsstrafen zur Folge gehabt hätte: sie führte ein dickes Buch, in dem der tägliche Kriegsbericht eingetragen wurde: links stand unser Wehrmachtsbericht -- und auf der rechten Seite der französische, den der Eiffelturm morste.

Mitte Mai trat ich von Sčukontowšeisna westlich Stolswičica, ca. 25~km nördlich Baranowiči meinen ersten Heimaturlaub an. Ich machte 2 Tage im polnischen Kulturzentrum Warzawa Station. Bei einem Opernbesuch stellte ich mit Genugtuung fest, dass ich nicht nur kleine Gesprächsfetzen meiner Nachbarn, sondern hin und wieder auch schon ganze Zusammenhänge verstand; mein Bursche Maczenski, Bauernsohn aus der Posener Gegend und ein polnischer Abiturient aus Posen förderten meine Sprachfertigkeit und lehrten mich polnische Liedtexte. Zu Hause schaffte ich mir sofort einen Photoapparat an, studierte ein gutes Lehrbuch und machte nun Aufnahmen aller Art auf Platten- und Filmpacks. Mein Bestreben war immer, ein gutes Bild zu erreichen, gleich ob bei Stand- oder Momentaufnahmen, Außen- oder Innenaufnahmen, wobei ich mitunter 45 Sek lang belichtete (z.B. Aufnahmen eines Stolleninneren) Im ersten Jahr entwickelte ich die Bilder bei Rotlicht im Unterstand oder bei Tag blind in einer Tonne, deren Deckel durch ein dichtes Tuch ersetzt wurde.

Reichlich 14 Tage später holte man mich in Baranowiči wieder mit einem Panjewagen ab. Wir bauten weiter Truppenunterkünfte im Wald von Sukontowscina, einem Straßendorf mit vorwiegend weißrussischer Bevölkerung und einigen polnischen Bauernfamilien, mit denen ich wegen meiner fortgesetzten polnischen Sprachstudien Kontakt hatte. Auf ein selbstgewebtes Leinenhandtuch stickte mir das schlichte 16-jährige Töchterchen die verzierten Worte: Szeresliwic powroczié do domu! (Glückliche Heimkehr!) Entzückend war ihr 3,5-jähriges Schwesterchen, die in reizender Koketterie bei uns Offizieren, besonders aus sprachlichen Gründen bei mir, sich Schokoladenstücke zu ergattern verstand.

Vieles, u.a. die ständig in russischen Hinterland wachsenden Verladerampen, deutete auf einen russischen Angriff hin.

Eines Nachmittags erschien über unserem Dorf ein russischer Aufklärungsflieger, unter Beschuss unserer Feldartillerie, deren Geschosse sich mehr und mehr dem Ziel über unserem Kopfe näherten, bis der Russe getroffen zu sein schien, er schwankte ein paar Sekunden und schraubte sich dann senkrecht von ca. 1000~m auf ca. 300~m herunter, während die Artillerie den Beschuss einstellte und unsere Leute über den offensichtlichen Erfolg jubelten -- da flog der Russe plötzlich in niedriger Höhe mit Vollgas zur russischen Front zurück -- unbehelligt durch zu spät einsetzendes Infanterie- und Artilleriefeuer.

Mitte Juni griffen die Russen an, wurden aber verlustreich zurückgeschlagen; wir verblieben in höchster Alarmbereitschaft, wurden aber nicht eingesetzt. Ende Juni kam mein Regiment in ein von österreich-ungarischen Truppen gebautes Waldlager zwischen Gorodisče und der Front. Es hieß nach einem ungarischen General \enquote{Szendelager}. Am Ostrande des Waldes waren österreichische Artilleriestellungen. Wir verkehrten abends mehrfach mit den österreichischen Offizieren, die sich durch wirklich charmante Geselligkeit auszeichneten. Tagsüber hatte ich meist genügend Zeit, das Gelände nach Photomotiven zu durchstreifen; besonders gern photographierte ich schöne Wolkenbildungen, die ich gelegentlich in andere Aufnahmen mit langweiligem Himmel hineinkomponierte. Ich glaubte im Photographieren eine Art schöpferischer Betätigung entdeckt zu haben; dies war -- neben Sprachstudien bis ins letzte Kriegsjahr hinein -- mein \enquote{Hobby} (ein Wort, das damals in Deutschland noch unbekannt war; man sprach von Schachsport, Fotosport etc.)


\section{Schlacht bei Baranowiči}
Am 3. Juli frühmorgens weckte uns eine intensive russische Kanonade. Wir wurden sofort in höchste Alarmbereitschaft versetzt. Es dauerte nicht lange, da schoss auch die österreichische Artillerie aus allen Rohren Sperrfeuer. Wir wussten: die russische Infanterie ist zum Sturm angetreten. Wir wussten: der neue, als tüchtig geltende russische Oberfehelshaber Brussikow hatte die Offensive begonnen, die in die Geschichte als \enquote{Schlacht bei Baranowiči} einging. Während der 1-2 Sekunden währenden Feuerpausen vernahm man Infanteriefeuer. Unser Bataillon rückte mit Sturmgepäck ab: die 3. und 4. Kompanie an einem Flusslauf entlang, die 1. und 2. Kompanien trafen gegen Mittag am Gefechtsstand des K. und K. Infanterieregiments 2 ein, meldeten sich beim Kommandeur und baten um weitere Befehle. Der Regimentskommandeur zu unseren Offizieren: \enquote{Schau's, meine Herren, das Mittagsmohl wird grad' aufgetragen, bitt' schön, Platz zu nehmen, die Supp' wird gleich für Sie kommen und nach dem Moahl werden wir das weitere besprechen. Zur Zeit hat sich die Lage beruhigt, der Russ' sitzt in unserer b-Linie.} Unsere Offiziere, denen die österreichische Nochalance missfiel, hatten kaum die Suppe gelöffelt, als man eilige Schritte am Eingang des geräumigen Unterstandes vernahm. Der Regimentsadjutant sprang auf, es fielen hastig hervorgestoßene unverständliche Worte, wahrscheinlich auf ungarisch, der Adjutant sprach leise mit dem Regimentskommandeur und dieser wandte sich laut an die deutschen Offiziere in schönstem Österreichisch: \enquote{Situation hat sich geändert, der Russe ist durchgebrochen -- wird im Moment hier sein} -- und damit wandte er sich an seine Ordonnanzen mit Weisungen für die Flucht.

Unsere Kompanien schwärmten sofort im Walde aus, rückten gen Osten vor und hatten in wenigen Minuten Feindberührung. -- Es war unseren Offizieren nicht gelungen, von der österreichischen Regimentsführung Aufklärung über die Lage zu erhalten. Aber sie brachten in ihrem Abschnitt die Russen zum Stehen.

Ich hatte das seltene Glück beim Vorgehen mit meinem Zuge auf einen leicht mit Bäumen und Sträuchern abgedeckten Hügel zu gelangen, von dem aus ich bei schönstem klaren Juliwetter einen weiten Fernblick hatte. Es war ein wahrer Feldherrenhügel, wie ich es in alten Schlachtschilderungen las: vor mir \enquote{wogte die Feldschlacht}, d.h. vor uns in einer Entfernung von ca. 1-3 Kilometern bewegte sich in winkliger Linie flüchtende Infanterie -- im Fernglas erkannte ich Österreicher -- gefolgt in Abstand von 200-600~m von Russen schräg auf uns zu. Auf dem Hügel feuerte eine benachbarte deutsche Maschinengewehrabteilung ohne Offizier. Ich trat hinter das erste feuernde MG und fragte den Unteroffizier durch die Zielrichtung misstrauisch geworden: \enquote{Worauf schießt Ihr denn?} -- \enquote{Auf die vorderen Infanteriereihen.} -- \enquote{Ja seid ihr denn wahnsinnig? Das sind doch Österreicher!} -- \enquote{Nu, eben! Die wollen wir ja zum Stehen bringen.} Ich befahl auf die zweite Linie, die Russen zu schießen!

Die Kompanie erhielt Befehl südöstlich zum Skrobowa-Bach zur rücken. Unterwegs begegneten uns fliehende Österreicher. Ich sah folgende Szene: einer meiner Leute, ein westfälischer Landwehrmann von etwa 32 Jahren wandte sich unweit von mir an einen rücklaufenden österreichischen \enquote{Schnür\-schuh-Kameraden} (unsere Leute trugen Stiefel, genannt \enquote{Knobelbecher}): \enquote{Wohin Kamerad?!} Antwort: \enquote{Munition holen.} Da löste der Westfale aus seinem Patronengurt einen \enquote{Rahmen} von 5 Infanterie-Patronen (mit Nickelspitzen) und schleuderte sie ihm ins Gesicht, dass sofort das Blut aus seiner Backe spritzte \enquote{da haste Munition}. Bezeichnend für die Stimmung unserer Leute. Es hatte sich schon herumgesprochen, dass in der letzten Nacht, unmittelbar vor dem Angriff, ein ganzes tschechisches Bataillon mit dem Major an der Spitze, zu den Russen übergelaufen war und es ihnen so ermöglichte, hier die Front von den Flanken aufzurollen. Wir hatten übrigens einen kernigen Westfalen, Leutnant Duhst, der schon im Frieden gedient hatte, als Kompanieführer bekommen. Wir beide standen uns nicht gut, da ich seine Befehle nicht stramm und peinlich genug ausführte, überhaupt keine Begeisterung für Kadavergehorsam zeigte. Beispiel: ein russischer Aufklärungsflieger war am Himmel; Duhst kommandierte der Kompanie \enquote{hinlegen}, ich verharrte ein paar Sekunden in Hockstellung um zu sehen, ob der Befehl von meinem Zug prompt ausgeführt worden war, da herrschte Duhst mich an: \enquote{Herr Leutnant Busse, Sie sind nicht aus Glas!}

Bei Einbruch der Dunkelheit waren wir am Fuße einer Hügelkette angelangt, auf der die unbefestigte österreichische c-Linie, ein Graben ohne Draht -- und andere Hindernisse, der von der russischen Artillerie von Tag zu Tag mehr eingeebnet wurde. Die Russen hatten die österreichische a- und b-Linie auf breiter Front genommen und griffen aus der 120-150~m entfernten ehemaligen b-Linie fortgesetzt mit Artillerievorbereitung an. Im Gegensatz zu früher, wo die Offiziere 10-20~m hinter der stürmenden Infanterie blieben, um diese vorzutreiben, rannten jetzt die jungen Offiziere 10 und mehr Meter ihren Leuten voraus, ja sie brachten sogar leichte Maschinengewehre mit.

In den schweren Kampfestagen vom 3.-7. Juli drangen besonders nachts immer wieder von Offizieren geführte kleine Trupps in unseren Graben ein, wo es zu erbitterten Nahkämpfen kam -- der Mensch wurde zu einer wilden Bestie, man überwältigte (meist) die russischen Trupps und schlug sie tot. Beweis: in meinen Kriegsakten hatte ich in Breslau einen handschriftlichen Regimentsbefehl des Wortlauts:

\begin{quote}
Ich mache ausdrücklich darauf aufmerksam, dass Gefangene gemacht werden dürfen.
\raggedleft gez. von der Chevallerie
\end{quote}

Am 4. Juli wurde abends das nie völlig ruhende Artilleriefeuer heftiger und heftiger; nach Einbruch der Dunkelheit entlud sich über dem Walde, in dem wir uns befanden, ein schweres Gewitter, man wusste oft kaum, ob das Getöse und das Aufflammen von Blitzen oder explodierenden russischen Granaten stammte. Ich stand gerade mit dem Ostpreußen Leutnant Böhm, einem strengen Katholiken, zusammen, als ein Blitz in unmittelbarer Nähe mit mächtigem Donnerschlag in einen Baum fuhr. Böhm bemerkte: \enquote{Das menschliche Waffengetöse wird doch von der Stimme Gottes, wie sie sich in Blitz und Donner äußert, machtvoll übertönt}. Wie wäre der gute Böhm wohl mit den \enquote{Errungenschaften} des 2. Weltkrieges, vor allem der Atombombe, fertig geworden?

Am 6. Juli rückte die 3. Kompanie nach Einbruch der Dunkelheit in den vordersten Graben. Er war so zusammengeschossen, dass man sich im Flackern der Leuchtkugeln mehr oder weniger stark bücken musste, um nicht vom Feind bemerkt zu werden. Der Kompanieführer war beim 1. Zuge mit Leutnant Niehaus, es schloss sich an der 2. Zug unter Leutnant Rühling und ich schloss mich mit dem 3. Zug rechts an. Zur Verstärkung hatte ich an den Flügeln je ein Maschinengewehr. Für die Nacht ließ ich mich, nachdem ich mit den Unteroffizieren alles Erforderliche besprochen hatte, auf einer etwas erhöhten, leicht nachgiebigen aber trockenen Stelle zum Schlafe nieder; ich deckte mich mit dem hellgrünen Gummimantel, den ich aus dem Urlaub mitgebracht hatte, zu. Von längerem Schlaf war natürlich keine Rede, da die russische Artillerie Gräben, das unmittelbare Hinterland, vor allem die Laufgräben mit Störungsfeuer belegten. Wir hatten leider schon in der Nacht zwei Tote und einige Verwundete. Als es Tag wurde, besichtigte ich mein Lager, die Erde schien sich etwas erhöht zu haben und hatte Risse bekommen: ich hatte die Nacht auf einem, nur mit dünner Erdschicht bedeckten Toten gelegen. In meinem Grabenstück lagen so mehrere Tote, die man wegen des russischen Artillerie- und Infanteriefeuers nicht nach hinten geschafft hatte. Im Laufe des Vormittags, es war ein heißer schwüler Tag mit Neigung zu Regenschauern, konzentrierte sich das Artilleriefeuer mit langsam steigender Heftigkeit auf unseren Graben. Alle Telefonleitungen waren zerschnitten. Gegen Mittag waren meine beiden Maschinengewehre durch Artilleriefeuer außer Gefecht gesetzt. Ich begab mich zur Lagebesprechung, wobei ich weite Strecken auf allen Vieren kroch, zum Kompanieführer. Trotz aller Vorsicht hatte mich ein russisches Maschinengewehr entdeckt und jagte mir einen Feuerstoß haarscharf über meinen Rücken. Über meinen Kompanieführer war ich entsetzt: er war völlig niedergeschlagen, schien mir passiv und widerstandslos, sagte nur: \enquote{heute sind wir alle verloren, es kommt keiner davon, es hat alles keinen Zweck}. Ohne Erfolg versuchte ich ihm Mut zu machen; er hatte offenbar Todesahnungen, auch Niehaus war niedergeschlagen. Zu meiner eigenen Verwunderung war ich optimistisch, hatte keine schlimmen Ahnungen

\marginpar{erstes Heft gefüllt am 19.11.1998}






Auf dem Rückwege zum 3. Zug kam ich an meinem Siegersdorfer Nachbarn Tschirner vorbei: ein Granatsplitter hatte ihm ein fausttiefes und -großes Loch oben am Oberschenkel gerissen, das Blut strömte, ich sah die Schlagader zucken, er flehte mich an: \enquote{Leutnant, helfen Sie mir!} Ich konnte den Armen nur die Hand drücken, aussichtslos, einen Sanitäter herbeizuschaffen, der ja auch nichts hätte ausrichten können. Kaum war ich zu meinem Standort in der Mitte des Zuges gelangt, hatte ich eine Lichterscheinung und weiter nichts. Als ich, wie einer meiner Leute sagte war es nach wenigen Minuten, die Besinnung wiedererlangte, verspürte ich Nässe und leichten Schmerz unmittelbar über meiner linken Schläfe: vor mir war ein Schrapnell explodiert, eine Kugel war an meiner Stirn abgeprallt und hatte in meinen Helm zwei Löcher gerissen -- das war alles.\footnote{Anm. Helga: Diese Begebenheit hat uns Fritz häufig erzählt. Er hatte auch Fotos von dem Helm gemacht, die leider mit allen anderen Kriegsfotos in Breslau geblieben und somit unwiederbringlich verloren sind...}  Kurze Zeit darauf, um viertel vor 1 Uhr, legte die russische Artillerie das Feuer nach hinten, die Russen kletterten aus dem Graben und kamen auf uns zu gerannt. Auch ohne den Befehl: \enquote{Feuer, Standvisier!} feuerten meine Leute was sie konnten. Plötzlich hörte ich russische Laute schräg hinter mir: die Russen hatten den linken Flügel der Kompanie überrannt und waren uns in die Flanke und schon halb in den Rücken gelangt. Ich brüllte: Alle Mann schnell zurück!

Offenbar fiel ich mit meinem Regenmantel als Offizier auf, die frontal angreifenden Russen waren schon auf ca. 40~m heran, stießen Laute aus und schossen stehend auf mich; im Nu hatte ich den hellgrünen Mantel abgeworfen rannte, und, wenn ich nicht mehr konnte, warf ich mich für Sekunden in einen Granattrichter und gelangte schließlich mit ca. 25 meiner Leute heil den Hügel herunter bis zur 2. Kompanie, die zum Gegenstoß bereit lag. Außer Atem und nur mit Mühe redend, berichtete ich über die Lage soweit ich sie gesehen hatte, verschnaufte mich zwei oder drei Minuten, da tauchten auch schon die ersten Russen auf dem Hügel auf, die 2. Kompanie stürmte zum Gegenangriff vor, ich mit dem Rest meiner Leute mit ihnen. Wir blieben für Sekunden stehen, schossen auf die mit aufgepflanzten Bajonett Vordringenden, es kam zum Nahkampf, ein Russe holte gerade zum Bajonettstoß auf meinen linken Nachbarn aus, da ließ ich meinen Spaten mit solcher Wucht auf seinen Kopf sausen, dass ich Mühe hatte, ihn wieder herauszubekommen. Wir gelangten schließlich, Kompanien durcheinander, in unsern vor einer halben Stunde von den Russen genommenen Graben zurück. Nur wenige Russen wurden gefangen, einem Teil von ihnen gelang die Flucht. Das deutsche und österreichische Sperrfeuer -- die österreichische Artillerie war gut, die Offiziere und Mannschaften tüchtig, während man im ganzen russischen Offensivabschnitt die österreichische Infanterie durch deutsche Truppen ersetzte -- hatte offenbar gut gelegen. Die Russen wagten nicht, ihrer ersten Angriffswelle eine zweite nachzuschicken. Sie hatten aufgegeben und beschränkten sich auf Störungsfeuer und kleinere Scheinangriffe.

Am Abend wurde das Regiment 19 aus der vordersten Linie herausgenommen und in einen Wald in Reserve gelegt.

Die Bilanz war traurig: Bataillonskommandeur Haas gefallen, sein Adjutant Frömsdorf schwer verwundet. Leutnant Niehaus gefallen, Duhst mit einem Lungenschuss und Rühling mit schwerem Armschuss konnten ins Lazarett geschafft werden, desgleichen mit einem Beinschuss der Führer der 2. Kompanie, mit dem zusammen ich den Gegenangriff gemacht hatte.

Mein Freund Groth war schon am 2. Tag der russ. Offensive gefallen, desgleichen Leutnant Böhm. Über 1/3 der 3. Kompanie, darunter auch 15 Mann meines Zuges, waren in Gefangenschaft geraten. Die Russen waren auf Bajonettkampf gedrillt und hatten eine Anzahl meiner Leute im Graben erstochen. Der wackere Rektor aus Löwenberg war durch einen Bajonettstich in den Mund und das Gehirn getötet. Übrigens erhielt ich meinen hellgrünen Regenmantel zurück: 100~m hinter dem Graben hielt ihn ein toter Russe mit einem Blutfleck auf der Stirn im Arm.

Wir, d.h. Regiment 19, waren ein paar Tage lang Armeereserve. Die Chargen, soweit sie die Kämpfe vom 3.-7. Juli mitgemacht hatten, wurden vom (offiziellen) Oberkommandierenden Prinzregent Luitpolt von Bayern besichtigt. Ich stellte fest, dass ich als einziger Offizier des 1. Bataillons übrig geblieben war. Die Königliche Hoheit, ordengeschmückt, richtete an jeden Offizier die gleichen, eintönigen Fragen nach Alter, Frontdienst und Beruf -- und wir meinten, dass wir uns 8 Tage die rechte Hand nicht waschen dürften, die von der königlichen Hoheit berührt worden war.

Die kurze Zeit im Waldlager war nicht ohne Verluste, da die Österreicher keinen Stollen oder bombensichere Unterstände gebaut hatten; wir lagen in Blockhausbaracken. Einmal fuhr ganz in der Nähe eine russische Granate und tötete 11 Mann des 2. Bataillons, der Splitter einer anderen riss dem 51-jährigen zum Kommandeur des 2. Bataillons aufgestiegenen Hauptman Hubert den linken Arm weg.

Als ich am anderen Tage mit dem mir aus der Ausbildungszeit gut bekannten schlesischen Lehrer Petzold, jetzt Unteroffizier, plaudernd im Walde lag, gab es ein klatschendes Geräusch. Wir schauten um uns im Glauben, es sei ein Ast abgebrochen und plauderten weiter, bis Petzold auf einmal sagte: \enquote{Du, mir ist so komisch}, und fortfuhr: \enquote{Mensch, aus meinem Ärmel fließt ja Blut!} Eine müde Infanteriekugel von der rund 2~km entfernten vordersten russischen Linie hatte eine tiefe Steckschusswunde im Oberarm erzeugt, die ihm bislang keinen Schmerz verursacht hatte.

Jetzt bekam das Regiment Ersatz, viele neue Offiziere und Mannschaften. Bataillonsführer wurde Hauptmann Leitloff, in Zivil Güterdirektor beim Fürsten Pless, guter Reiter, praktischer, angenehmer Mensch mit gesundem Verstand, Mitte dreißig. Bataillonsadjutant wurde Mannich, \enquote{Postreferendar, aus einer der ältesten Beamtenfamilien Dresdens}, wie er betonte; mein Kompanieführer wurde ein 45-jähriger Bankangestellter der Deutschen Bank, Liedke. Der profilierteste, witzigste und intelligenteste war der heitere rheinländische Genießer Gerling, Subdirektor der Görlitzer Waggonfabrik und später, in den 20-er Jahren deren Generaldirektor. Wir bezogen wieder den Schützengraben, es gab russischerseits Kanonaden, denen jedoch nur Scheinangriffe folgten, Ablenkungsmanöver vom russischen Hauptangriffsziel etwas weiter nördlich.

Als vom Regiment eine Umfrage kam, wer von den jungen Offizieren ein vier-wöchiges Kommando zur Schweren Artillerie wünsche, meldete ich mich sofort und verlebte im Walde südlich von Gorodischtsche eine für mich interessante und abwechslungsreiche Volontärszeit. Ziel war das bessere gegenseitige Verständnis der beiden Waffengattungen.

Die Batterie des Hauptmanns Scholz wusste sich zusätzliche Lebensmittel zu verschaffen, die den Infanteristen in Staunen versetzten: \enquote{Aber Herr Kamerad, Sie werden sich doch das Brot -- hier ist auch Weißbrot -- nicht nur mit Butter bestreichen, darauf gehört Honig! etc.}.

Ich war Volontär, konnte praktisch machen, was ich wollte. Mich interessierte zunächst das stets indirekte Schießen der schweren Artillerie mit dem Richtkreis, saß stundenlang mit dem Batterieführer im Gefechtsstand, wenn er das Feuer auf bestimmte Ziele lenkte, bekam Verständnis für alle möglichen technischen Dinge. Eines Tages wurden die russischen Stellungen drei Stunden lang beschossen. Ich übernahm an diesem Tage am 1. Geschütz die Rolle des Kanoniers 5, des Abzugskanoniers. Auf das Kommando: Geschütze -- Feuer! zog ich mit obligatorisch geöffnetem Munde\footnote{zur Vermeidung eines Knalltrauma mit folgendem Tinnitus oder Gehörverlust}, das linke, dem Geschütz zugewandte Ohr mit dem Mittelfinger der linken Hand verstopft haltend, die Leine, und heraus jagte mit dröhnendem Knall pfeifend und zischend das 15 cm Kalibergeschoß, am blauen Himmel in der Verlängerung des Rohres sichtbar als kleine sich im Nu zu einem Punkt zusammenziehende schwarze Scheibe.

Als ich so im Rhythmus der 4 Geschütze eine Zeitlang geschossen hatte, flog mir auf einmal unter Krachen Dreck ins Gesicht. Die Russen erwiderten das Feuer, und im Getöse der eigenen Geschütze hörte man das Herannahen der Granaten nicht und konnte sich nicht, wie bei der Infanterie, zu Boden werfen um die Gefahr zu verringern.

Die Abende mit Scholz und seinen Offizieren waren feuchtfröhlich und oft ausgedehnt; trennten wir uns dann zu sehr vorgerückter Stunde, dann verspürte der Batterieführer das Bedürfnis, \enquote{das Sperrfeuer zu überprüfen} und ließ eine \enquote{Rollsalve!} gen Osten jagen -- jede Granate kostete 350 M, damals das Monatsgehalt eines 40-jährigen Regierungsrates.

\marginpar{290}
Ich begleitete den Batterieführer auf seinen Gängen zu den in den Schützengräben eingerichteten Artilleriebeobachtungsposten. In Infanterie hatte damals im August 1916, zunächst für die Grabenposten, Stahlhelme eingeführt. Scholz ließ sich von einem Wachposten den Helm geben und setzte ihn belustigt auf, ich fotographierte ihn mit diesem neuen Kopfschmuck, fragte die Leute nach der Regimentsnummer und wie lange sie schon hier seien u.ä. Als wir dann durch einen Laufgraben nach hinten gingen, hörten wir Schritte hinter uns. Wir wurden von einem Offizierstellvertreter in Begleitung zweier Wehrmänner mit aufgepflanztem Seitengewehr gestellt und für verhaftet erklärt. \enquote{Folgen Sie mir zum Herrn Bataillonskommandeur!} Wir folgten erstaunt und belustigt. Nachdem der Offizierstellvertreter dem Bataillonsführer Meldung erstattet hatte, während wir draußen unter strenger Bewachung verharrten, erhielten wir Weisung, den Bataillonsgefechtsstand zu betreten. Und da stand er, Major d.R. Hofmann-Scholz, der Vorgesetzte meines Vaters und unserer Familie wohlbekannt. Er erkannte mich sofort, seine Stirn glättete sich und sein erstes Wort war: \enquote{Ach, da ist ja der junge Busse!} Und dann folgte eine eingehende Belehrung, besonders an Hauptmann Scholz gerichtet, der unzulässige und in der Tat verdächtige Fragen an den Posten gerichtet habe; Nach einem kurzen persönlichen Gespräch mit mir wurden wir mit Händedruck verabschiedet.

Mit der Brussilow-Offensive war unser Regiment vorübergehend in den Verband der 201. Infanteriedivision getreten. Das österreichische Oberkommando richtete nach der Beendigung der Schlacht bei Baranowiči die Aufforderung an die 201. Infanterie Division, Vorschläge für österreichische Auszeichnungen für Offiziere der deutschen Division einzureichen. Kennzeichnend für die damalige Stimmung im deutschen Ostheer gegenüber unseren Bundesgenossen war die Antwort der 201. Division: \enquote{Die Offiziere der 201. Inf. Div. legen zur Zeit auf österreichische Auszeichnungen keinen Wert.}

\marginpar{292}
Übrigens war ich nach den Kampftagen im Juli für das \ac{eki} vorgeschlagen worden, doch hatte, wie ich später hörte, Moritz erklärt: \enquote{Busse ist noch zu jung}. Ich erhielt diese Auszeichnung im Jahr darauf.

Einige Wochen nach meiner Rückkehr in den Alltag des Grabenkrieges wurde unser Regiment mit einem anderen Landwehrregiment aus dem Armeeverbande des Generalobersten Woyrsch, der anlässlich der erfolgreichen Abwehr der Brussilow-Offensive zum Generalfeldmarschall befördert worden war, herausgezogen und übten -- Parademarsch! Als der wohlbeleibte alte Herr am Tage der Abschiedsbesichtigung die Front unserer Kompanie abschritt, bemerkte mein Landwehrmann Hentschel aus der Bunzlauer Gegend in prachtvollem Schlesisch: \enquote{Dam sitt man's oah dass es keene Marmelade nich frist.} Woyrsch drückte jedem einzelnen Offizier und zwei Mann von jeder Kompanie die Hand, nachdem er uns in bewegten Worten seinen Dank ausgesprochen hatte. Er konnte die Tränen nicht unterdrücken. Er galt als \enquote{nobler Herr}, war Rittergutsbesitzer bei Breslau und verzichtete während des ganzen Krieges auf sein hohes Generals- bzw. Marschallsgehalt zugunsten des Roten Kreuzes.

Wir wurden südlich Baranowiči, wie es hieß, nach Rumänien verladen; ich stellte jedoch anhand des Kompasses nach Passieren von Brest-Litowsk fest, dass wir nach Wolhynica fuhren, wo die Russen, denen hier nur Österreicher gegenüber standen, einen tiefen Einbruch erzielt hatten. In Iwaniče ausgeladen brachte uns ein Tagesmarsch nach Konincky, unweit Swinincky in einer fruchtbaren Gegend gelegen. Als wir nach 2-3 Ruhetagen einen Sonderzuschuss an Nahrungsmitteln und Alkohol erhielten, wusste jeder, dass wir angreifen würden. Am Abend vor dem Angriff hatte das \enquote{Offensivwasser} sichtbar seine Wirkung getan. Einige Landser riefen: \enquote{Ja, gibt man uns denn keine Handgranaten?} Nach 1,5-stündigem Nachtmarsch und sehr kurzer Nachtruhe schoss unsere Artillerie aus allen Rohren. Mittags kletterten wir aus den Gräben und erreichten gegen Abend bei nur mäßigen Verlusten den vorgeschriebenen russischen Graben, während die Russen ihre Stellung nur 60-70~m vor uns hielten. Von den Postenlöchern und kurzen Sappen\footnote{Laufgraben, Schützengraben} aus warf man in der Nacht Handgranaten. Die Leute lagen in der kühlen Nacht in dem schmalen Sturmgraben, ich saß in einer Nische. Als ich nach höchstens drei Stunden Schlaf erwachte, war ich, ohne dass ich es merkte, in eine warme Decke eingehüllt worden.

Ich habe den Wohltäter unter meinen Leuten nicht herausbekommen, mit freundlichen Gesichtern wollte es keiner gewesen sein.

Noch manches aus meinem Erleben während der Stellungskämpfe am sogen. Eierwald, unweit des völlig dem Erdboden gleichgemachten wolkynischen Dorfes Swinjucki, wäre zu berichten, das in meiner Erinnerung immer wieder auftaucht: der sinnlose, durch neugierige Unvorsichtigkeit verschuldete Tod am Morgen nach der Einnahme des russischen Sturmgrabens zweier Neulinge an der Front, die, obwohl gewarnt, den Kopf einen Augenblick im Lichte der Morgensonne über die Brustwehr erhoben und sofort von russischen Scharfschützen, die ja nur ca. 60~m entfernt waren, die tödliche Kugel in den Kopf erhielten, einer, junger \enquote{Musketier} und 35-jähriger Feldwebelleutnant, verheiratet und Vater von zwei kleinen Kindern -- oder wenige Tage später, als die Russen ohne Erfolg unsere Gräben zurückzugewinnen suchten und wir, die 3. Kompanie unter Leutnant Liedke, als Bataillonsreserve in einem 500~m hinter unserm vordersten Kampfgraben ausgehobenen b-Linie lagen in unmittelbarer Nähe des Bataillonsgefechtsstandes mit Hauptmann Uhlig als Kommandeur, und wie nach zweistündigem schweren Artilleriefeuer dieses plötzlich von der a- auf die b-Linie verlegt wurde, die russische Infanterie angriff und Uhlig der 3. Kompanie befahl, in offenem Gelände auszuschwärmen, wogegen ich, weil sinnlos, mich sofort bei Uhlig zur Wehr setzte und mir dieser sagte: \enquote{Aber von der Existenz dieser b-Linie weiß ich ja gar nichts!} Er hatte vier Tage lang seinen bombensicheren Unterstand nicht verlassen, Sekt und immer wieder Sekt getrunken. Jetzt kletterte er aus dem Graben, stand mit wehendem Mantel und beobachtete mit heroischer Geste, seinen Stock in der Hand, die in seiner Nähe krepierenden leichten Granaten -- wir, d.h. sein Adjutant und ich, hatten alle Mühe ihn soweit zu ernüchtern, dass er sich wieder in Deckung begab.

Oder noch folgendes von Uhlig, dem es als früherem aktiven Offizier mit organisatorischem Geschick gelungen war, zum selbständigen Führer eines Detachements zu werden, dem eine Zeitlang auch -- zu unserem Nachteil -- das 1. Bataillon 19 angehörte. Ich war bei ihm 2. Adjutant, verantwortlich für den ganzen Ausbau des Grabensystems seines Bereichs. Er hatte seinen Unterstand mitten in einem Eichenwald, an dem jedes Mal die Essenholer vorüberkamen. Bald, es war Winter mit beständiger Schneedecke, fragte man mich, was mit dem Bataillonsführer los sei. Grund des Staunens der Essenholer: wenn ihn die erforderliche Menge Sekt gehörig angeheizt hatte, trat er splitternackt vor den Unterstand, machte Freiübungen im Schnee und rief den verblüfften Essenholern zu: \enquote{Was? Da staunt ihr über euren Kommandeur!}

Im Juli 1933, 16 Jahre später, sah ich ihn auf der Terrasse eines Strandhotels in Laboe skatspielend sitzen. Ich nahm Gelegenheit, mit ihm gemeinsame Kriegserinnerungen auszutauschen. Er konnte oder wollte sich an diese beiden Vorfälle und manches andere überhaupt nicht erinnern. Ich hatte den Eindruck, dass der Alkoholgenuss das Gedächtnis des noch nicht 60-jährigen Hamburgers bedenklich geschwächt hatte.

Abwechslung bot mir, wenn wir in Reserve lagen, die Entenjagd. Ich hatte in einem der russischen Schützengräben eine großkalibrige einläufige Jagdflinte mit 3 Patronen erbeutet. Weitere Munition verschaffte ich mir über den Regimentsverpflegungsoffizier aus Warschau.

Einmal erlegte ich am flachen See von Swinjucki auf einen Schuss zwei Enten; um sie aus dem Wasser zu holen -- einen Jagdhund hatte ich natürlich nicht -- musste ich bis weit über die Knie in das oktoberkühle Naß steigen. Ein älterer Offizier hatte mich, wohl 500~m entfernt vom Seeufer aus, beobachtet und brüllte mich unter Namensnennung \enquote{Hauptmann X} an, das sei sein Jagdgebiet, ich solle das Weite suchen. Natürlich, nach der Rangordnung hatte er Hackrecht -- was ich jedoch nicht tragisch nahm und mich von der Ausübung der Wasserjagd nicht abhielt.

Wir verspeisten die eine Ente -- sie schmeckte vorzüglich. Die zweite, ein Erpel, verbreitete schon bei der Zubereitung einen üblen Geruch und erwies sich als eine Tranente aus dem hohen Norden.
Unvergesslich Zawidowo, 16~km hinter der Front, ein Bauerndorf mit zum Teil weißgetünchten Häusern, einer schönen orthodoxen Kirche mit vielen Heiligen in bunten Farben, dem Lettner, russ. Ikonen -- vor denen die Offiziere des Lehrganges abends zechten\dots

Hier überwinterten zwei Schwadrone, deren Offiziere fast ausschließlich adlig und alle Monokelträger waren. Zwei Kilometer nördlich des Dorfes begann ein ausgedehnter Wald, fast ein Urwald mit Luchsen und anderen Wildarten. Ihn lernte ich sechs Wochen lang von November bis gegen Weihnachten 1916 auf den fast täglichen Übungsritten in Verbindung mit taktischen Aufgaben kennen. Ich hatte das Glück, als Einziger des Regiments zu diesem Kursus für Infanterietaktik kommandiert zu werden. Leiter war ein Hauptmann polnischen Namens von Koscielski. Der Reitkurs, unter Leitung von Kavallerie-Wachtmeistern, war aufgeteilt in zwei Gruppen, eine für Anfänger, die andere für Fortgeschrittene. Da ich ein paarmal auf einem Küchenpferd gesessen hatte und ich mir bei den \enquote{Fortgeschrittenen} intensive Förderung versprach, meldete ich mich zur 2. Gruppe. Es gelang mir schließlich, mich vom schlechtesten zum viertschlechtesten emporzuarbeiten. In der ersten Woche kam es mehrfach vor, dass mein gutgedrillter Kavalleriegaul, des ungeschickten Reiters müde, aus der Manege ausbrach und im Galopp auf seinen Stall zu rannte, mich auf den gefrorenen holprigen Acker abwarf, aber im gleichen Augenblick stehen blieb, auf den unglücklichen Reiter mitleidig herabschauend. Während der ausgedehnten Übungsritte mussten wir auch Geländeskizzen anfertigen -- hierbei bedauerte ich erneut, wie schon auf der Universität in kunstgeschichtlichen Übungen, dass auf dem humanistischen Gymnasium das Fach \enquote{Zeichnen} nur bis zur Tertia gelehrt worden war.

Den Reitkursus beschloss eine Besichtigung durch den bekannten Kavalleriegeneral von der Marwitz. Ich fiel zweimal unangenehm auf. Gleich zu Anfang kommandierte er: Steigbügel hoch -- rechts aufgesessen! Das war nicht geübt worden, ich kam rechts nicht hoch, kroch fix unter dem Gaul durch, um von links aufzusteigen. Der General bemerkte es und herrschte mich an: \enquote{Herr, ich lasse Sie sofort abführen, wollen Sie wohl rechts aufsitzen?!} Schließlich gelang es mir. Bald auf \enquote{Terráb} folgte \enquote{Linksgalopp!} Aber mein Gaul wollte nicht, er verharrte im Trab trotz angestrengter Schenkel- und Fußarbeit. Der temperamentvolle General rannte auf den Gaul zu, rief unwillig \enquote{Ist denn das ein Kavalleriepferd?!} und versetzte ihm einen kräftigen Fußtritt, worauf er dann ansprang.

Die Kritik des Reitergenerals war streng und nicht ohne beißende Ironie. Zusammenfassend formulierte er: \enquote{Meine Herren, Sie haben gezeigt, dass Sie das Pferd als Tragetier verwenden können.}

Und nun ritt ich auf meinem braven instinktsicheren Bataillonspferd \enquote{Liese}, das ich zum Kursus mitbekommen hatte, mit meinem Burschen Maczenski in den tiefverschneiten Frontabschnitt Uhligs zurück.

Spannend und aufregend waren jetzt nachts die Arbeiten am Drahtverhau, für die ich als \enquote{Grabenoffizier} verantwortlich war. Sorgfältig in Schneehemden waren wir beim Verlassen des Grabens gehüllt, beim Abschuss einer Leuchtkugel erstarrte jede Bewegung, zumal der Russe wenig mehr als 200~m entfernt war. Man war jedes Mal froh, nach Beendigung der Arbeit heil in den Graben zurückzukehren. Die Artillerie schwieg oft mehrere Tage lang, unangenehm war nachts der gegenseitige Beschuss mit Minen aller Kaliber.

Ein Museumsstück befand sich 200~m hinter meinem Unterstand, dessen Tür durch Minendetonationen oft in der Nacht aufgerissen wurde: ein 10 cm Feldgeschütz Modell 1871. Es besaß noch keinen Rohrrücklauf, bei jedem Schuss flog das Geschütz um 1,5~m zurück und musste neu gerichtet werden, sodass bei \enquote{Schnellfeuer!} mindestens 1,5 Minuten zwischen den einzelnen Schüssen vergingen.

Schließlich kehrte das 1. Bataillon wieder in den Regimentsverband zurück; Leiter: Hauptmann Leitloff, Adjutant: Mannich. Ich war stellvertretender Adjutant und \enquote{Grabenoffizier} und gehörte jetzt zum Bataillonsstab, mein Unterstand lag ca. 2~m unter der Erdoberfläche, geschützt durch mehrere Lagen dicker Baumstämme, gekrönt von einer Schicht von Betonbalken, auf der, wie sich bald erwies, die russischen Granaten explodierten ohne die Balkenlagen zu beschädigen. Ich richtete ihn wohnlich ein, er wurde um den Fensterplatz herum getäfelt, einige leidlich gerahmte Landschaftsbilder unseres Bataillonsmalers zierten die Wände wie, in noch besser Ausführung, den behaglichen Unterstand des Bataillonskommandeurs. In unmittelbarer Nähe hatten unsere Vorgänger, die Österreicher, sogar ausgehend vom Laufgraben zur Stellung, eine ca. 8 qm große Laube ausgehoben mit Tisch und Sitzgelegenheiten. Die Winterabende, sofern nicht durch die Russen gestört, waren bei Leitloff gemütlich. Man trank mehrere Glas Tee mit Rum, die Unterhaltung wurde sehr lebhaft und Leitloff fing an zu pfeifen. Dann kurbelten wir unser Grammophon und legten meist \enquote{Schmalzplatten} auf, z.B. Sonaten von Torelli u.a. worauf wir uns in unsere Gemächer zurückzogen.

\marginpar{306}
Der Verlauf unserer Stellung hatte seine Tücken: am linken Flügel waren die Postenlöcher bzw. Sappenköpfe nur 40~m voneinander entfernt. Auf beiden Seiten wurden Minenstollen vorgetrieben, es wurden Horchgeräte zur Überwachung verwandt. Der Bau der Minenstollen sowie der Infanteriepanzertürme wurde von Pionierunteroffizieren ausgeführt, ich jedoch trug die Verantwortung für beides, was praktisch bedeutete, dass ich mich in beide Techniken, theoretisch wie praktisch, hineinknien musste.

Ein 14-tägiger Heimaturlaub zeigte, dass zu Hause nicht alles zum Besten stand. Mein Vater hatte der dreiwöchige Kummer infolge meines angeblichen Gefallenseins\footnote{Anm. Helga: ein Soldat aus Siegersdorf hatte unmittelbar nach den Kampftagen im Juli den Tod von Fritz Busse gemeldet. Der Vater ist daraufhin ergraut. So wurde es uns erzählt.} und Sorgen um Ellas Schicksal sehr zugesetzt.

Da platzte die Nachricht vom Sturz des Zaren und Ersatz durch die Kerenski-Regierung, was viele auf ein baldiges glückliches Ende des Krieges hoffen ließ. Der Regimentsadjuntant telegraphierte mir die Verleihung des \ac{eki}.

Ich besuchte noch die Rabenaus in Reisicht; Tante Lisa, die Tochter eines polnischen Professors, jubelte über die am 5. November 1916 erfolgte Proklamation eines neuen selbständigen polnischen Staates, allerdings nur in den Grenzen Kongreßpolens und von Gnaden der beiden Monarchen Deutschlands und Österreichs.

Im Felde ging alles zunächst seinen gewohnten Gang weiter, hatte doch die neue russische Regierung die energische Fortsetzung des \enquote{Krieges bis zum siegreichen Ende} verkündet. Zwischen vorderster Linie und dem \enquote{Breslauer Lager} in dem die Unterstände des Bataillons-Stabes lagen, wurde ein b-Graben ausgehoben. Ich hatte bei den nächtlichen Arbeiten übersehen, dass ein Grabenstück kein Schussfeld hatte; und schon berichtigte ein Hauptmann die neue Stellung. \enquote{Aber Herr Leutnant, was ist denn das? Hier ist ja kein Schussfeld, keine Sicht zum Feind?!} \enquote{Hinterhangstellung, Herr Hauptmann!} \enquote{Ach so, Sie haben die neuesten Berichte über die taktischen Fortschritte im Grabenkrieg der Westfront studiert und gleich hier die Nutzanwendung gemacht. Sehr gut!} Die Ausrede hatte Erfolg, meine Arbeit war \enquote{gut und einfallsreich.}

Die Nachricht von der am 6. April 1917 erfolgten Kriegserklärung der USA an Deutschland (als Folge des unbeschränkten U-Boot-Krieges) wirkte an der Front bedrückend.

Wenige Tage später führte ich den Generalstäbler der Division, den Major Graf von Westarp durch die Stellungen des Bataillons. Er fragte zwei Posten der vordersten Linie: \enquote{Na, was sagt ihr zur Kriegserklärung Amerikas?} Man erwiderte ihm, das sei eigentlich schlimm, Deutschland habe es jetzt noch schwerer. Von Westarp belehrte sie: \enquote{Ach was, wir wollen froh darüber sein, jetzt haben wir sie (die Feinde) alle beisammen, jetzt können wir richtig zuschlagen!} Dies war offenbar damals die Meinung des Generalstabs, der fest von der Niederringung Englands binnen weniger Monate durch den uneingeschränkten U-Boot-Krieg überzeugt war.

Hoffnung erweckte indes der Zusammenbruch der neuen Brussilow-Offen\-sive im Frühsommer 1917. Unser Abschnitt blieb verschont, wir hatten einen relativ ruhigen, schönen Sommer. Ich übernahm zeitweilig zusätzlich die Bataillons Adjutantur. Es hatte sich herumgesprochen, dass ich Kenntnisse in slawischen Sprachen besaß -- damals etwas völlig ungewohntes bei einem deutschen Offizier, und so wurde ich auf Vorschlag der Division von der österreichischen Heeresgruppe, zu der die Division gehörte, zum Parlamentäroffizier der Division ernannt.

Der österreichische Oberleutnant Diepold, ein kräftiger, gesunder, charmanter Wiener, war zwei Monate bei uns als Austauschoffizier; er volontierte. Er ist für mich der Typus des K und K-Offiziers geworden. Zunächst beanstandete er die seiner Meinung nach völlig unzureichende Verpflegung der deutschen Offiziere. \enquote{Ein Offizier \enquote{fasst} in der KuK-Armee das Doppelte des \enquote{gemeinen Mannes}}. Er setzte es durch, dass auch er bei uns doppelte Ration erhielt -- sonst würde er sich zu seiner Truppe zurückmelden. Die neue Brussilow-Offensive stimmte ihn pessimistisch. \enquote{Denn schaun's wenn der Russe wi\underline{e}rklich oangreift, da konn man nix mache. Der hot Kräfte!!} Er erzählte, wie er als Kompaniechef tief unten in einem bombensicheren Unterstand gesessen habe und die russische Artillerie hat gesch\underline{o}ssen und gesch\underline{o}ssen, die Erde hat gedröhnt \enquote{und ich hob doag'sess'n und g'zietert und g'zietert\dots}

Er wunderte sich ehrlich, wie wir das gemacht haben, der russischen Offensive 1916 zu widerstehen. Nach seiner Meinung hoben sich unsere Offiziere nicht mehr so scharf wie nötig von den Gemeinen ab. Aber auch von den ernsten Schwierigkeiten mit den fremdsprachigen Nationen der Donaumonarchie wusste er zu erzählen, glaubte aber, dass die Donaumonarchie den Weltkrieg überlebe.

Wir spielten ein paarmal Schach in meiner Laube -- aber am liebsten plauderte er, und er plauderte charmant von seinem Wien.

Hauptmann Leitloff ließ mich ein paarmal unter seiner Aufsicht auf seinem prachtvollen Reitpferd üben. Ich kam aber nicht mit ihm zu Rande, er reagierte auf die geringste Gewichtsverschiebung, den leisesten von mir nicht beabsichtigten und nicht gemerkten Schenkeldruck -- und ritt so z.B. eine Volte, zu meinem Erstaunen.

Nach dem Scheitern der russischen Offensive und mit der Ausstrahlung der politischen Machtkämpfe in Petersburg machten sich langsam Auflösungserscheinungen beim Gegner bemerkbar.

\marginpar{Abschrift von Heft 2 beendet 24.11.1998}

Die russische Artillerie legte zwar noch geringes Störungsfeuer auf Stellungen, Laufgräben, unser \enquote{Breslauer Lager} und das Hinterland, doch ließ das russische Schützenfeuer mehr und mehr nach. Im Frühherbst erfuhr ich von ersten Unterhaltungen russischer Grabenposten mit ihren deutschen \enquote{Kollegen}, soweit diese polnisch sprachen, d.h., aus der ethnographisch rein polnischen Provinz Posen oder gemischtsprachigen Provinzen (Oberschlesien, Westpreußen) stammten. Eine Kompanie meldete das Verschwinden dreier Landser, die auf Posten gestanden hatten und deren Muttersprache das oberschlesische Polnisch ist. Es wurde sofort vereinbart, die Sache streng vertraulich zu behandeln und nicht nach oben weiterzugeben. Im Laufe der 4. Nacht kehrten sie zurück: sie waren von ihren russischen \enquote{Kollegen}, den Posten, die an dieser Stelle nur 40-50~m von ihnen entfernt lagen, eingeladen worden, mit ihnen an einem \enquote{Fest} ca. 10~km hinter der russischen Front teilzunehmen. Man habe ihnen ehrenwörtlich zugesichert, dass sie selbstverständlich wieder zurückkehren würden. Es sei ein lustiges Schlachtfest mit Tanz u.s.w. gewesen. Die russischen \enquote{Kollegen} seien selbstverständlich auch auf ihre Posten zurückgekehrt. Wir, d.h. Kompanieführer, Bataillonskommandeur und ich kamen überein, die Sache nicht zu melden, was für die Landser eine höchst gefährliche Kriegsgerichtsverhandlung zur Folge haben würde. Strengste Verschwiegenheit sei von allen Beteiligten und Wissenden zu bewahren -- in diesem Sinne wurden auch die Delinquenten belehrt.

Vom Oktober 1917 suchten die Russen mit uns ins Gespräch, vor allem über unsere Kriegsziele, zu kommen. Sie schnitten hier und da Durchgänge durch ihr Drahtverhau und kamen z.T. in größeren Scharen an unser Drahtverhau, sprachen über unsere Kriegsziele, wir hätten Annektionsabsichten an der Ostsee und den Ostseeprovinzen, fragten, was wir mit Belgien vorhätten u.ä. Es wurde sehr bald deutlich, dass unter ihnen geschulte Propagandisten waren, die unsere Leute verunsichern und ihren Widerstandswillen lähmen sollten, und nach der Oktoberrevolution tauchten deutlich erkennbare kommunistische Propagandaleute auf. Das deutsche Oberkommando fürchtete nicht ohne Grund für die \enquote{Moral} der deutschen Soldaten und gestattete nur besonders ausgewählten Trupps solche Zusammenkünfte mit den Russen. Ich bildete für das ganze Regiment einen \enquote{Propagandatrupp} aus Leuten, die keinesfalls für kommunistische Verbrüderung anfällig waren. Er bestand großenteils aus Bauernsöhnen und einigen im Bürgertum fest verankerten Feldgrauen. Mein 30 Mann starker Propagandatrupp bewegte sich innerhalb des ganzen Divisionsabschnittes, während die eigentliche Grabenbesatzung allmählich verdünnt werden konnte. Das Gesprächsverlangen wurde beiderseits durch Schwenken einer weißen Fahne angezeigt. Beim ersten Treffen mit einer etwa 30-40 Mann starken russischen Gruppe an unserem Drahtverhau kam es zu einem kleinen Zwischenfall. Ich photographierte die Gruppe, in deren Mitte sich eine politisch sehr aktive Frau, vielleicht eine Krankenschwester befand, die uns mit allen möglichen Fragen über unsere Kriegsziele, das von uns beabsichtigte Schicksal Belgiens u.a.m. bedrängte, Fragen, auf die wir laut Armeebefehl nicht eingehen durften. Als sie sah, dass ich ein Foto gemacht hatte, überschüttete sie mich mit Worten des Protestes und der Empörung, ich hätte sie, die Frau, zumindest vorher um Erlaubnis fragen müssen, wie sich das in Russland und hoffentlich in ganz Europa gehöre. Nur allmählich, als ich mich wiederholt entschuldigt hatte, während sich die russische \marginpar{315} Männerwelt \enquote{neutral} verhielt, beruhigte sie sich. Posten hatten mir schon vor Wochen gemeldet, dass ihnen gegenüber ein neues Maschinengewehr eingebaut sei; die Besatzung bestehe, in einem Unterstand vereint, aus 5 männlichen und zwei weiblichen Soldaten.

Unter den Russen, die aus Drahtverhauen zu Gesprächen, dann auch zu Tauschgeschäften kamen, befanden sich immer einige deutsch sprechende. Sie setzten sich zusammen a) aus deutschen Kolonisten, vorwiegend aus den deutschen Wolgakolonien, den ukrainischen Kolonistendörfern und dem Baltikum, und b) aus russischen Juden, die sämtlich \enquote{jiddisch} sprachen, d.i. ein mittelfränkischer Dialekt aus frühhochdeutscher Zeit mit hebräischen Einsprengseln. Diese zweite Schicht trat zur Zeit der russischen Revolution Okt/Nov. 1917 und nach dem Waffenstillstand zwischen Deutschland und Russland immer stärker hervor. Schon im November konnte ich am Drahtverhau aus dem Juden Weissbrodt, in Zivil Inhaber eines Schuhgeschäftes in Odessa, so nebenbei manches über die Zersetzungserscheinungen in der russischen Armee herausbekommen, z.B. dass viele kommunistisch gesinnte großrussische Soldaten gruppenweise von der Front wegliefen, um zur Revolutionsarmee zu stoßen.

Eines Tages ließ ich die weiße Fahne schwenken und rief zu ihm als erwünschten Unterhändler hinüber: \enquote{Weißbrodt, Weißbrodt!} Da kommt ein Divisionsstabsoffizier durch den Graben und fällt mir ungehalten ins Wort: \enquote{Aber Herr Leutnant! Sie können doch die Russen nicht um Weißbrot betteln!} Das lustige Missverständnis war bald geklärt.

Weißbrodt erklärte mir eines Tages übrigens: \enquote{Herr Leitnant, wenn Se komm nach Odessa -- de Deitsche werd'n ja woll die Ukraine besetzen -- dann komme Se mich besuchen. Ich verkauf Se scheene Schuhe und Stiefel, billig, zum Sonderpreis!}

Es war wohl um die Zeit, wo der Waffenstillstand abgeschlossen wurde, da überbrachte mir der Bataillonsadjutant am Drahtverhau die Einladung eines der uns gegenüberliegenden Bataillone, ein Major Sowieso, die Einladung zum Tee nachmittags 4 Uhr. Ich fragte die Division an, wie ich mich verhalten solle. Graf Sponlatz\footnote{Anm. Helga: oder Sposslatz}, Generalstäbler, sagte mir: \enquote{ich muss mich eigentlich wundern, dass Sie mich anrufen; selbstverständlich müssen Sie ablehnen!} Zu meinem Verdruss brachte mich der \enquote{Kleinkarierte} (so beurteilte ich ihn ob seiner Entscheidung) um ein reizvolles kleines Abenteuer -- Tee beim Kommandeur der Truppe, mit der man sich über ein Jahr schwer duelliert hatte!

Der Waffenstillstand war das Signal für Verminderung der russischen Präsenz an der Front und für stärkere Lockerung der (russischen) Disziplin. Juden und Nichtjuden boten uns vor allem Pferde und leere Artilleriemunition zum Kauf an, was der Heeresleitung im Hinblick auf die Westfront sehr willkommen war, zumal es uns gelang, sehr billig einzukaufen.

Ich hatte als neuen russischen Bataillonsführer einen Oberleutnant Erschke -- er schrieb mir seinen Namen in lateinischen Buchstaben auf -- kennengelernt. Er war in Zivil Besitzer einer Streichholzfabrik in Sibirien, war deutscher Abstammung, sein Großvater war aus Deutschland eingewandert und er sprach noch leidlich deutsch, übrigens auch ziemlich gut französisch. Er war ein gebildeter Mensch, wir verließen bald die leidigen Themata Krieg und Politik und sprachen von Kunst und Literatur u.ä. Eines Tages machte er mich mit seinem Adjutanten Fähnrich bekannt, einer mittelgroßen sehr hübschen Erscheinung in gepflegter gut sitzender Uniform. Sein ebenmäßiges Gesicht mit gerader schmaler Nase war beherrscht von lebhaft und intelligent dreinschauenden dunkelbraunen Augen. Seine Stimme war klangvoll, lag etwas höher als bei jungen Männern üblich; da er offenbar aus einem sehr guten Nest kam, sprach ich ihn französisch an, worauf er sofort einging, und er sprach fließend und -- bis auf das russische Zungen --R-- fast völlig akzentfrei mit fast weiblichen Charme. Ich wurde stutzig als ich seine gepflegten Hände sah und meine Augen über seine Figur glitten -- der Fähnrich schien jetzt unsicher zu werden, sprach ein paar Worte mit seinem Kommandeur und verschwand -- ich habe ihn nie wieder gesehen -- erfuhr aber von russischen Soldaten, dass der Fähnrich\dots die Freundin des Oberleutnants Erschke war. Näheres, z.B. seit wann Erschke einen weiblichen Adjutanten an der Front hatte usw. wollte ich nicht erfragen; Erschke selbst sprach kein Wort darüber. Eines Tages überreichte er mir eine gepflegte Schachtel mit drei Stück vorzüglicher Gesichtsseife, die ihm, wie er sagte, seine Frau für mich geschickt habe. Eine sehr willkommene Gabe, da wir uns 1917 an der Front wie in der Heimat schon mit Tonseife bescheiden mussten.

Mittlerweile war der Abschnitt der Division nur noch von zwei Kompanien unseres Regiments unter Hauptmann Schmidt, in Zivil Amtsgerichtsrat, übrig geblieben. Das Gros war schon in der Etappe, viele von den jüngeren Offizieren waren schon an die Westfront abgegeben worden. Nur wenige von ihnen haben die deutsche Heimat wiedergesehen.

Zur Verbesserung des äußerst fleischarm gewordenen Feldküchenessens schoss ich öfters mal eine Krähe -- faute de mieux. Eines Tages fiel etwa 50~m von mir auf dem gefrorenem, gerade schneefreien Erdboden eine Schar Sperlinge ein. Um möglichst viele von ihnen zu erlegen, zielte ich mit meinem einläufigen Jagdgewehr ein paar Meter vor sie auf die Erde, in der Erwartung, dass sich durch Losreißen von gefrorenen Erdklümpchen die Wirkung der Schrotkörner verdoppeln werde. Ich hatte mich nicht verrechnet, auf der Strecke blieben insgesamt ein Viertelhundert Spatzen! Das mag vielleicht wie Jägerlatein klingen, ist aber \marginpar{322} ungeschminkte Wahrheit. Ich lud Hauptmann Schmidt zum Abendessen ein, dessen einziger Gang aus einem Braten von einer Krähe und 25 Sperlingen bestehen sollte. Mein Bursche Maczysirki hatte mit dem Rupfen und Ausnehmen der 25 Spatzen verdrießliche Arbeit; nach Entfernung des Federkleides blieb nicht viel Wildbret übrig, die Tierchen waren bis Ende Januar sehr abgemagert! Unglücklicherweise ließ mein Bursche das Wildbret etwa zwei Minuten zu lange braten, was genügte, um die Spatzen beinahe zu verkohlen -- der einzelne Sperling war nicht viel größer als ein Maikäfer! Schon während des Diners fragte Schmidt, ob es nicht noch einen zweiten Gang gebe! Nun, den hielt ich bereit in Gestalt von zwei Flaschen guten Weins, die seine Stimmung ungemein hoben.

\section{Krieg gegen das bolschewistische Russland}
Am 9. Februar hatte das Deutsche Reich mit den nicht bolschewistischen Vertretern der Ukraine in Kiew, der Rada, einen Separatfrieden abgeschlossen, während die russischen Unterhändler in Brest-Litowsk die Friedensverhandlungen verschleppten. Da die Russen schließlich die deutschen Friedensbedingungen nicht annahmen, kam es zum Abbruch der Verhandlungen und am 18. Februar zum Wiederbeginn des Krieges gegen das bolschewistische Russland, dessen Heer effektiv großenteils demobilisiert war, während die Schaffung der Roten Armee noch in den Anfängen steckte.

Unser Regiment war inzwischen an die Demarkationslinie des Waffenstillstandes etwa 10~km östlich von Pinsk im Polesje, zu deutsch in den Rokitno -- oder Pripjetsümpfen verlegt worden.

Am Morgen des 18. Februar begann unser Vormarsch längs der Bahnlinie Pinsk-Moeyr bei schneearmem mäßigem Frostwetter. Es war eine kuriose Kriegsführung: voran längs der Bahnstrecke eine schwache Kavallerieeinheit, die Infanterie und Artillerie folgte streckenweise im Güterzug. Zu kurzen Gefechten kam es nur auf und an der Bahnstrecke, wobei die Russen vor der Lokomotive auf dem Fahrgestell eines Güterwagens ein Geschütz mit behelfsmäßiger, anscheinend hölzerner Panzerung mit sich führten, während auf dem Tender, der in Russland nicht mit Kohle, sondern mit Eichenkloben beladen war, ein oder zwei Maschinengewehre untergebracht waren. Wir führten am Ende des Truppentransportzuges zwei Geschütze mit. Unsere Taktik bestand darin, dass wir MG- oder Artillerieduells auf dem Schienenstrang nach Möglichkeit vermieden, während die Infanterie versuchte, dem Gegner in Flanke und Rücken zu kommen. Zunächst stießen wir auf keinen Widerstand. Am Abend des ersten Marschtages gelangten wir in ein weißrussisches Dorf, in dem sich das russische Feldlazarett sowie Teile des Trains befunden hatte. Bei der Unterbringung meiner Leute merkte ich, dass die Dorfbevölkerung wie das im Lazarett noch verbliebene, vorwiegend weibliche, zahlenmäßig ganz geringe Personal uns keineswegs unfreundlich gesinnt war; man lud mich sogar zu einer Abendtafel in den Räumen des Lazaretts ein, wohin ich mich auch mit einem anderen Offizier begab. Ich kam sofort ins Gespräch mit einem polnischen Assistenzarzt, der auf russischer Seite gekämpft hatte und von dem furchtbaren Blutbad, das die Bolschewisten in Kiew angerichtet hatten -- nach seinen Angaben waren \num{2000} russische Offiziere und viele Bürger auf den Straßen Kiews umgebracht worden; er hatte sich durch die Flucht retten können und war noch ganz erfüllt von dem Furchtbaren, das er durchlebt hatte. Da er nur gebrochen deutsch und ich nicht fließend polnisch reden konnte, sprachen wir französisch, was ihm glatt von der Zunge ging. Die Grausamkeiten hatten, so ergab sich bald, die ganze weißrussische und besonders die ukrainische Bevölkerung derart in Schrecken versetzt, dass man uns geradezu als Retter ansah.

An der Abendtafel saß mir schräg gegenüber eine merkwürdige, schwarzgekleidete Gestalt mit Vollbart und -- zu meinem Staunen -- mit langem Zopf. Der Pole klärte mich auf -- es sei der Dorfpope. Auf meine Frage, ob wir ihn nicht ins Gespräch ziehen könnten, entgegnete er, Dorfpopen seien völlig ungebildete Bauern. Ich bat ihn, doch vorsichtiger zu sprechen, vielleicht könne der Pope doch etwas verstehen? Der Arzt meine: Der? Er kann nur mit Mühe seinen Namen schreiben, er versteht nur sein mundartliches Weißrussisch, was haben Sie für eine Vorstellung vom russischen Popen!

Nach dem Abendessen lud er mich zu sich in seine nicht ungemütliche Blockhauswohnung ein. Wir spielten bis 4 Uhr morgens Schach, unterbrochen durch angeregte Gespräche. Er äußerte sich, wie fast alle Polen, nur geringschätzig über die Russen und ihre Kultur, denen sie sich weit überlegen fühlten. Er meinte, die Deutschen und Polen eine -- trotz der politischen Gegnerschaft -- doch ein Band: die Zugehörigkeit zum europäisch-humanistischen Kulturkreis und einem etwa gleichhohen Bildungsniveau. Er setzte mir Preiselbeeren und Wodka vor, wir rauchten russische Zigaretten.

Die Russen hatten sich, wie sich herausstellte, bis hinter den Eisenbahnknotenpunkt Luninez, ca. 80~km östlich Pinsk zurückgezogen. Dieser wichtige Ort befand sich in der Hand der polnischen Legionäre Pilsudskis, die auf der Seite der Mittelmächte kämpften. Ihr Führer befand sich allerdings seit Juli 1917 in Magdeburg in Haft, u.a. weil er nicht den Eid auf die Mittelmächte leisten wollte.

Als sich die Spitze unseres Regiments noch 10~km vor Luninez auf dem Marsche befand, erhielt ich den Auftrag, nach Luninez vorzureiten und für ein Bataillon in Absprache mit den polnischen Legionären Quartier zu machen. Da mein Bursche kein Pferd bekam, ritt ich allein auf dem nur leicht verschneiten Bahndamm entlang, dessen zweites Gleis die Russen entfernt hatten. Die eine Hälfte des Bahndammes, die des abmontierten Gleiskörpers, sah einem verschneiten schmalen Kartoffelfeld nicht unähnlich; zwischen ihm und dem anderen Gleis lag ein knapp 1,5~m breiter Steg, auf dem sich gut reiten ließ.

Ich mochte schon die Hälfte der Strecke in der winterlichen Ebene durchtrabt haben und glaubte schon in der Ferne Außenbezirke von Luninez wahrzunehmen, da kommt vom Zielort meines Ritts her ein Zug mit rasch anschwellendem Gezisch und Gefauch herangefahren, ein Ausweichen auf den tief rilligen Boden des demontierten Gleises schien mir zu gefährlich. Ich nehme mein Roß straff an die Kandarre, aber in dem Augenblick, wo das schwarze Ungetüm bedrohlich neben uns faucht, bäumt sich mein Brauner auf, macht kehrt und jagt fast in Tuchfühlung neben der Lokomotive her, sucht seitlich etwas von der Lokomotive loszukommen, gerät zwangsläufig auf die \enquote{Furchen} des Nebengleises, stürzt, ich fliege über den Kopf des Pferdes hinweg, schlage mit dem rechten Oberschenkel mit voller Wucht gegen die steinharte Seitenwand einer Furche und der Pferdeleib rollt über meinen Thorax hinweg. In diesem Augenblick glaube ich mein Ende gekommen, mir schwinden die Sinne, der schwere Pferdekörper drückt mir alle Luft aus dem schmerzenden Brustkorb. All das -- es hätte weit schlimmer enden können -- spielte sich in wenigen Sekunden ab. Das Pferd liegt hilflos neben mir, ich versuche aufzustehen, der rechte Oberschenkel schmerzt zu sehr, es will nicht gelingen, da hilft mir jemand, zieht mich am Arm auf. Es ist der russische Bahnwärter, der aus seinem etwa 1,5~km entfernten Wärterhäuschen herbeigeeilt ist. Er bekommt das offenbar nicht ernstlich verletzte Pferd hoch und hilft mir, dem entschlossen Weiterlebenden, schließlich in den Sattel.

Mein Pferd trägt mich im Schritt in die Stadt, auf deren Straßen ziemlich viel russische Zivilisten sowie betriebsame polnische Soldaten zu Pferde unterwegs sind. Ich entledige mich meines Auftrags vom Pferde aus und bekomme schließlich selbst von den Polen Quartier in dem bescheidenen Holzhaus eines russischen Bahnangestellten angewiesen. Am nächsten Tage untersuchte mich der Bataillonsarzt und glaubte, dass ich wahrscheinlich außer einem Bluterguss im rechten Oberschenkel keinen ernstlichen Schaden erlitten hätte, ordnete aber für mehrere Tage Bettruhe an.

In dem kleinen Holzhaus mit den kleinen Zimmern wohnte ein junges kinderloses Ehepaar zusammen mit der ebenfalls jungen Schwester des Mannes. Man gab sich alle erdenkliche Mühe, mich zu zersteuen, sang vor allem viele russische Lieder zur Balalaika. Die stattliche junge Frau war ausgesprochen hübsch, eine blonde russische Dorfschönheit mit klaren, kalten Augen. Ihre Schwägerin, dunkelblond, auch Russin, hatte weichere Züge. Auf meinen Wunsch schrieben sie mir die Texte der gesungenen Volkslieder auf, ich prägte mir die schönen Melodien ein.

Am dritten Tag drängte mich die hellblonde Schöne, ich solle doch Gehversuche machen, sie wolle mich begleiten und mir die Stadt zeigen. Sie machte sich besonders \enquote{fein}, zog einen schönen schwarzen Samtmantel an und führte den stark humpelnden deutschen Leutnant durch die verkehrsreichsten Straßen der in schönes Neuschneegewand gekleideten Stadt. Die Schöne, die ebenso wenig wie ihr fremder Gast auf ein Liebesabenteuer aus war, wollte offensichtlich von möglichst vielen Mitbürgern und -innen mit dem deutschen Offizier gesehen werden. -- Warum? Wahrscheinlich hatte sich ihr Mann mit den Bolschewiki kompromittiert und hoffte auf diese Weise die nicht ausbleibenden Denunziationen unwirksam zu machen. Ob ihr das gelungen ist, zumal nachdem Generalfeldmarschall von Eichhorn, der Oberkommandierende der deutschen Besatzungsarmee in der Ukraine, einem kommunistischen Attentat in Kiew zum Opfer gefallen war?

Unvergesslich ist mir eine der nun folgenden Nächte, als große Teile des Regiments im Güterzug in Richtung Gomel fuhren, durch Sumpfgebiete mit vielen Brücken über z.T. sehr stattliche Nebenflüsse des Pripjet. Würden nicht Brücken von den Bolschewiki angebohrt oder mit Sprengladungen versehen sein? Das fürchteten nicht nur wir Deutsche sondern auch die russischen Lokomotivführer und Heizer. Wie leicht konnten sie, deren meist kommunistische Gesinnung bekannt war, in der nächtlichen Dunkelheit allein oder zusammen abspringen und den Truppentransport dem Verderben aussetzen. Es hatte daher stets ein Offizier mit schussbereiter Pistole Dienst auf der Lokomotive. Meine Zeit lag zwischen 1-3 Uhr nachts. Ich suchte mit den beiden nicht gerade Vertrauen erweckenden Gesellen, dem Lokomotivführer und dem Heizer mit aufmunternden kurzen Bemerkungen in Kontakt zu kommen. Sie arbeiteten beide konzentriert; immer bevor wir eine Brücke passierten, verlangsamte der Lokomotivführer die 40 km/h kaum überschreitende Fahrgeschwindigkeit und schien durch das Zischen und Fauchen der Maschine hindurch nach Geräuschen vom Gleiskörper zu lauschen, ich passte genau auf wie man den Zug bremste und starrte dann in die schwarzen Fluten unter uns, und der Heizer schob einen riesigen Eichenkloben nach dem anderen, mit kurzen Atempausen von höchstens einer halben Minute, in den gefräßigen feurigen Bauch der Maschine. Hatte man die Brücke glücklich passiert, atmeten wir drei erleichtert auf.

Angenehm war die Wärme auf der Lokomotive. Trotzdem war ich froh, auf mein Strohlager im kalten Güterwagen zurückkehren zu können. Es war alles gut gegangen, nicht einmal die Holzladung auf dem Tender hatte Feuer gefangen. Lokomotiven mit brennendem Tender habe ich in der Ukraine mehrfach fahren sehen: um das Feuer zu löschen, suchte man den nächsten Wasserturm zu erreichen.

In der folgenden Nacht näherten wir uns Rjetschiza am Dnjepr. Zwischen 3-4 Uhr morgens fuhr ich von meinem Strohlager auf: plötzliches Halten des Zuges hatte mich geweckt. Ich schaute hinaus: wir standen auf einem verdunkelten Bahnhof. Kaum hatte ich mich wieder ausgestreckt, höre ich das Geräusch eines herankommenden Zuges: er kommt auf dem benachbarten Gleis aus entgegengesetzter Richtung und hält neben uns. Nach kurzer Pause von hüben und drüben Maschinengewehrstöße. Ein paar Kugeln durchschlagen die Wände unseres Wagons. Dann setzt sich der Zug auf dem Nebengleis wieder gen Osten in Bewegung. Nach zehn Minuten Pause werden wir ausgeladen. Wir befinden uns in einem Vorort von Rjetschiza, dessen Bevölkerung während der Schießerei geflohen war. Wir beziehen Quartier in den Häusern unweit des Bahnhofs. Ich richte mich mit meinem Burschen in einem Hause ein, das seine Bewohner, Angestellte, etwa Telefonistinnen, eben erst verlassen hatten: der Schusswechsel hatte sie bei der Zubereitung des Morgenkaffees überrascht. Das kam uns sehr zu passe. Kaum hatten wir gefrühstückt, uns beim Kaffee aufgewärmt und wollten uns für den Rest der Nacht ausstrecken -- da hörten wir Artillerieabschüsse und gleich darauf Krach! Krach! die Detonation von Granaten mit flacher Flugbahn im Dorf. Der sowjetische Panzerzug konnte höchstens 3~km entfernt sein! Mein Bursche annektierte für mich noch rasch eine den Mädchen gehörende Wolldecke und schon traf der Befehl zum Abmarsch ein. Ziel: den Panzerzug aus Flanke oder von hinten angreifen! Nach längerem Marsch durch den Wald entdeckten wir ihn vom Waldrande aus in ungefähr 600~m Entfernung; man musste uns bemerkt haben, denn man war dabei, das Geschütz auf dem behelfsmäßigen Panzerwagen auf uns zu richten. Also Kommando: hinlegen, Visier 600 Schützenfeuer. Zwei Granaten sausten über unsere Köpfe weg -- zu weit geschossen, unsere gezielten Schüsse schienen gut zu liegen, denn der Panzerzug gab eilig Volldampf und fuhr von unseren Kugeln begleitet gen Osten ab.

Wir rückten zur Dnjeprbrücke, die hier ein über 1~km breites Urstromtal überquert, mit dem schon in seinem Oberlauf mächtigen Strom, mindestens fünf mal so breit wie die Elbe bei Magdeburg. Es handelte sich um eine Eisenbahnbrücke, die nur links am kleinen Geländer einen etwa 50 cm breiten Bohlensteg mit ungeschützter rechter Seite aufwies. Zwischen den Schwellen, auf denen die Schienen ruhten, gähnte der Abgrund mit den Fluten des Dnjeprs in der Tiefe. Ein Pfad nur für absolut schwindelfreie Fußgänger. Aber dies schien gering gegenüber den tödlichen Gefahren, um die jeder von uns wusste: erschien jenseits der Brücke der Panzerzug, so genügten zwei gut gezielte Schuss Feldartillerie, um eine ganze im Gänsemarsch vorrückende Kompanie in den Abgrund zu schleudern! Aber wir mussten hinüber. Ich eilte an der Spitze meiner Leute in äußerster Anspannung und mit klopfendem Herzen vorwärts, nur vereinzelt einen Blick in die schwindelnde Tiefe wagend. Und wir schafften es schließlich, und wie wir wieder festen Boden unter uns spürten, stürzten wir uns, in gewohnter Weise ausschwärmend, auf den oberen Rand des Urstromtales, warfen uns hin und richteten das Gewehr mit Standvisier auf die knapp 500~m entfernte Biegung des Bahndamms. Wir mochten wohl nur 5 Minuten verschnauft haben, da tauchte der russische Panzerzug auf. Wir nahmen ihn sofort unter Feuer, er schoss zwei Granaten über unsere Köpfe und drehte schleunigst ab und verschwand hinter der Biegung.

Dies war mein letztes Gefecht im Osten; es fand statt am 1. März 1918.

\section{Frieden von Brest-Litowsk}
Am 3. März wurde der Frieden von Brest-Litowsk geschlossen. Deutschland hatte seine Kriegsziele im Osten erreicht. Wir blieben als Besatzungstruppe in der Ukraine.

\marginpar{340}
Zunächst waren wir vom LIR 19 (Landes Infanterie Regiment) noch kurze Zeit im weißrussischen Gebiet. Am 4. März wurde die dritte Kompanie, zu der ich zurückgekehrt war, im Walde 5~km von Pribor, ca. 30~km westlich Gomel, ausgeladen. Im Schnee längs der Bahnstrecke fielen mir zahlreiche Spuren großer Hunde auf -- doch nein, die Fußballen waren nicht, wie bei Hunden, glatt sondern infolge der Behaarung rauh und von unschärferen Konturen. Es waren Wolfsfährten. Gleich beim Eintritt in den Wald kamen uns einige Hausschweine entgegen, viel schmaler als in Deutschland, darunter ganz dunkelbehaarte, Produkte von Kreuzungen mit Wildschweinen.

Hausschweine laufen frei in Wald und Feld herum; ihr Auftauchen kündigt die Nähe eines Dorfes, nicht weiter als etwa 5~km an. Sie benehmen sich in der freien Natur ungezwungen und völlig ungeniert. Den Menschen beachten sie, haben aber keine Furcht vor ihm.

Pribos war ein mittelgroßes Straßendorf, die Eingänge in die Geschäfte nur von der breiten ungepflasterten Dorfstraße aus. Alle Bauerngehöfte sind nach den Außenseiten durch kräftige Latten oder Weidengeflechtzäune miteinander verbunden. Grund: die Wölfe, die hier noch die gefährlichen Feinde von Mensch und Tier waren. Eine Kirche besaß Pribor nicht, wohl aber eine Schule mit sogar zwei Lehrerinnen. Der Dorfälteste, der Starostá, war uns bei der Unterbringung behilflich.

Leutnant Liedke, seit Herbst 1916 mit Unterbrechungen Führer der 3. Kompanie, war wieder zu uns gestoßen. Er wohnte bei einer verwitweten Bäuerin in einer großen gedielten Stube mit zwei richtigen Betten, von denen eines des Nachts öfter von einem Liebespaar aus dem Dorf benutzt wurde. Ich hatte ebenfalls ein passables Bauernhaus erwischt; das große Zimmer hatte in der Krasnyj Ugol (der schönen Ecke) an der Längs- und Schmalseite des Raumes eine Holzbank an der Wand, davor einen Tisch und das Bild der Schwarzen Muttergottes von Czestochowa mit dem \enquote{Ewigen (roten) Lämpchen} darüber. In dieser Ecke nahm der Gast des Hauses, in diesem Fall ich, Platz. Der Krasnyj Ugol diagonal gegenüber befand sich die \enquote{Pjetsch}, der gewaltige Ofen, zugleich auch Herd und Backofen, und darüber die Schlafstelle der Alten.

Diese Isbá (Bauernstube) hatte noch einen durch eine Tür verbundenen ebenfalls gedielten Nebenraum; hier wurden auf zwei Sägeböcke Bretter gelegt, darauf kam eine Decke und fertig war meine -- trotz der Decke -- harte aber ungezieferfreie Schlafstelle.

Morgens brachte man mir eine Schüssel mit Wasser, die Leute waren durchaus freundlich. Als ich nach dem \enquote{Örtchen} fragte, wurden ihre Gesichter zu Fragezeichen; auch eine Latrine oder Grube mit einer Stange darüber war nicht vorhanden. Was tun? \enquote{Wir gehen immer hinter den Stall, auf das Feld}. Ich fürchtete, dort mit einer eindrucksvollen Fäkaliensammlung konfrontiert zu werden. Aber nichts davon, größte Sauberkeit! Ich hatte mein Geschäft noch nicht vollendet, da vernahm ich des Rätsels Lösung: das angeregte sympathische Grunzen der freiherrlichen Schweine, die sich mir näherten, ungeduldig auf die Beendigung meiner Aktivitäten wartend, um ihres Amtes als Gesundheitspolizei und Dorfreiniger zu walten.

Dieses Erlebnis schockierte mich schon nach wenigen Tagen nicht mehr. Die freundlichen Bauersleute luden mich zum Essen ein: Schweinebraten mit Kapusta\footnote{Anm. Helga: Kohl}. Es schmeckte mir besonders gut.

Am 2. Tag suchte mich der Starostá auf; er bat um Stacheldraht zur Anfertigung von \enquote{spanischen Reitern}, die russischen Soldaten hätten sie ihnen weggenommen; der Dorfeingang bzw. -ausgang müsse unbedingt jeden Abend versperrt werden, die Wölfe kämen jede Nacht ins Dorf.

Das war eine willkommene Beschäftigung für unsere Leute. Jede Nacht hörte man die Wölfe bellend heulen; sie umkreisten das Dorf, wie die zahllosen Wolfsfährten im Schnee bewiesen, suchten nach schwachen Stellen in den Bretter- oder Faschinenzäunen, um in die für sie ungemein attraktiv riechenden Viehställe einzudringen. Die Bauern erzählten mir viel von den Missetaten der Wölfe, wieviel Vieh sie in den letzten Jahren gerissen hatten; vor zwei Jahren hätten sie einen Bauern unweit des Dorfes buchstäblich zerfleischt.

\marginpar{346}
Am 3. Tag kam ein stattlicher, hochgewachsener blonder junger Mann mit klaren freundlichen Augen ins Haus -- ungeheure Freude: es war der Sohn, er hatte bei den Petersburger Grenadieren die zweite Hälfte des Krieges mitgemacht und war nach der Demobilisation per Bahn und später große Strecken zu Fuß marschiert.

Unvergesslich ist die Hochachtung die er vor deutscher Bildung hatte; er sagte einmal wörtlich: \enquote{ich kann weder lesen noch schreiben; wir wachsen auf wie die Bäume im Wald, keiner kümmert sich um uns; jedes 8-jährige deutsche Kind ist uns überlegen}.

Man begreift den leidenschaftlichen Bildungswillen, ja Bildungssturm, der Russland in den zwanziger Jahren (2. Hälfte), wo die Erwachsenen einschließlich unzähliger Alter abends zu Schulen strömten\dots

Psibors Schule war, wie ich vermutete, erst kurz vor dem Weltkriege gegründet worden. Die beiden Lehrerinnen klagten sehr darüber, dass in Russland noch keine Schul\underline{pflicht} bestehe: selten kämen mehr als ein Drittel der Kinder zur Schule, zur Zeit der Ernte oder im Winter bei schlechtem Wetter noch weniger. Die Elementarkenntnisse und Fertigkeiten blieben daher sehr gering.

An die russische Bevölkerung erging der Befehl des deutschen Oberkommandos, alle Waffen einschließlich der Jagdgewehre an den jeweiligen deutschen Ortskommandanten abzuliefern. Ich gab den Bauern einen beruhigenden Kommentar: vorübergehende Vorsichtsmaßnahme, Waffen würden ihnen nicht weggenommen, Rückgabe an Besitzer zu gegebener Zeit. Mit dem, was in meiner Bauernstube zusammengetragen wurde, hätte man ein schönes kleines Jagdflinten Museum einrichten können: vorwiegend Vorderlader aus der 1. Hälfte des 19. Jahrhunderts, ein- und zweiläufige, meist 12 mm Kaliber. Auch meine Wirtsleute holten unter der Diele der\enquote{isbá} 3 bemooste Flinten hervor.

Der Starostá schlug mir eine gemeinsam von unseren Leuten mit den russischen Bauern zu unternehmende Wolfsjagd vor. Ich begeisterte mich für den Gedanken, ließ mir von der Methodik der Wolfsjagd und diesem Wild erzählen, doch ehe wir der Verwirklichung des freudig aufgenommenen Planes nähertreten konnten, kam der Befehl zum Abmarsch bzw. zur Verladung über Gomel, einer größeren Stadt im ukrainisch-weißrussischen Grenzgebiet nach Niskowka an der Strecke Gomel-Bachmatsch. \marginpar{348} Von Niskowka ging eine Zweigstrecke nach dem 30~km entfernten Karjukowka, wo sich eine der größten ukrainischen Zuckerfabriken befand. Dorthin wurde ein Zug der Kompanie verlegt, um den normalen Betrieb der für Deutschlands Ernährung wichtigen Fabrik sicherzustellen und insbesondere einen Streik der angeblich kommunistischen Belegschaft zu verhindern. In den etwa 14 Tagen meines Aufenthalts in Niskowka als Divisionsverbindungsoffizier kontrollierte ich einmal den Betrieb in Karjukowka. Meine Haupttätigkeit bestand in der telefonischen Aufnahme und Weitergabe der Divisionsbefehle aus Gomel, dem Sitz der Division, an die bis zu 200~km weit auseinanderliegenden Formationen. Ferner war ich \enquote{Bahnhofschef} und hatte u.a. dafür zu sorgen, dass stets eine bestimmte Anzahl Lokomotiven und erforderliche Leerwagons bereitstanden.

Eines Tages, an dem ich vier Lokomotiven unter Dampf hielt, kam der Bahnhofsassistent atemlos angerannt mit der Meldung, dass soeben eine Lokomotive befehlswidrig in Richtung Gomel abgedampft sei. Ich bekam schließlich den folgenden Sachverhalt heraus: der Lokomotivführer war von einem jüdischen Händler bestochen worden, ihn mit einer Ladung von 400 Hühnern, auf dem Tender verstaut, nach Gomel zu fahren! \textcyr{Нечего делать!} [nitschewo delatch] Da war nichts zu machen.

Ich wohnte etwa 50~m vom Bahnhof entfernt im geräumigen Hause des Bahnhofvorstehers, einem wie in allen kleineren Ortschaften, einstöckigen Gebäude mit flachen Dächern und hohen Zimmern. Man betrat das Haus von der Schmalseite, gelangte durch einen sehr kleinen Flur in den großen Salon, angefüllt mit zahlreichen Polstersesseln und einem Sofa, die, abgesehen von besonderen Festlichkeiten, stets mit weißleinenen Schutzbezügen bedeckt waren, umgeben von erstaunlich zahlreichen hohen immergrünen Pflanzen wie Gummibäumen u.a. Vom Salon aus führte eine Tür zu den Wohn- und Schlafräumen der Beamtenfamilie; auf der anderen Seite lag mein nicht sehr großes Zimmer, früher von den Hausangestellten, dann von den ältesten der Kinder bewohnt. Der Bahnhofsvorsteher Kurowski war, wie in der Ukraine nicht selten Menschen in etwas gehobener Stellung, polnischer Abstammung, die sich alle mehr oder weniger als Nationalpolen und den Russen kulturell überlegen fühlten. In der Familie -- und mit mir -- sprachen sie polnisch. Kurowski machte den Eindruck eines ordentlichen, korrekten Beamten. Seine Frau, betont Dame mit Geltungsbedürfnis, lud mich zum Tee ein und machte mich sofort mit ihrer mannbaren Tochter bekannt, die mir jedoch wegen ihres eitel-eingebildeten Gehabes und ihrer Oberflächlichkeit nicht gefiel. Als man mich eines Nachmittags mit der \enquote{schönen Müllerstochter}, einer russischen Blondine im eleganten schwarzen Samtmantel auf der Dorfpromenade -- das war zur Zeit der Schneeschmelze und des Polnovodje der Bahnkörper -- hatte spazierengehen sehen, wurde ich von Mutter und Tochter geschnitten. Bahnhof und ein paar hundert Meter des Gleiskörpers ersetzten hier den Markt, die Piazza, Pleaza oder die zentrale Hauptstraße anderswo. Wer nichts zu tun hatte, flanierte hier, besonders wenn die Ankunft eines Zuges zu erwarten war. Sonntag nachmittag promenierte das ganze Dorf, vor allem die weibliche Jugend im Sonntagsstaat, auf den Schienen. Erst auf das Tuten der Lokomotive -- sie tönte in der Ukraine tief wie ein Hochseedampfer -- wich man auf das unbefahrene Gleis aus. Jeder russische Zug fuhr von der Station erst nach dem 3. Glockenzeichen ab: nach dem 1. Glockenzeichen dachten die Fahrgäste an die Bezahlung des am Buffet genossenen Trunkes oder Imbisses, der meist aus einem getrockneten Fisch, Wodka oder Kwas bestand, beim 2. Zeichen verließ man das Bahnhofsgebäude oder beendete das Gespräch allmählich vor dem Wagen, beim 3. Glockenton stieg man eilig ein oder verharrte auch nach dem Anfahren des Zuges noch eine Zeitlang auf dem Trittbrett.

Einige Abende war ich im Hause des dörflich-wohlhabenden Müllers zu Gast. Es war noch während des 8-wöchigen Velikij Poot, der \enquote{Großen Fastenzeit}, in der die orthodoxen Christen streng fleischlos aßen. Aber welch prachtvolle Eigerichte wusste die Müllerin herzustellen! Nach dem schmackhaften Abendmahl noch längeres Zusammensein bei Samovar, Wodka und Tanz zu Grammophon. Inmitten dieser netten, nur russisch sprechenden Familie bedauerte ich, dass meine Sprachfertigkeit im Russischen noch unbefriedigend war.

Der Müller hatte Feinde am Ort: der Sohn des Bahnmeisters, Student in Kiew, der seine Osterferien in Niskowka verlebte, machte ihn für die seiner Meinung nach zu hohen Brotpreise verantwortlich und bemühte sich, von mir die Genehmigung zur Einberufung einer Dorfversammlung zu erhalten, in der er sprechen wollte\dots, worauf ich natürlich nicht einging. Wir unterhielten uns viel über deutsche Philosophie, besonders Feuerbach, Hegel und Marx interessierten ihn; er war auch sonst belesen und diskutierfreudig, sprach auch etwas deutsch und ziemlich gut französisch.

Alle Grade unserer Truppe fühlten sich während der Besatzungszeit in der Ukraine sehr wohl: die Verpflegung war ungleich besser als an der Front. In Niskowka erhielt jeder täglich 0,5-1 l roten kräftigen und sehr wohlschmeckenden Krimwein, von dem ein großes Lager in Gomel in die Hände unserer Division gefallen war. Köstlich war der vorzügliche echte Tee, den man preiswert kaufen konnte. Und man konnte schließlich die seit Anfang 1917 in der Heimat bedenklich karg gewordenen Lebensmittelrationen durch Paketsendungen an die Angehörigen, wie z.B. Sonnenblumenöl, aufbessern.

Eines Tages erschien der Divisionsstab in einem Sonderzug, bestehend aus russischen 1. und 2. Klasse-Wagen. Sie waren sehr komfortabel ausgestattet, verbrachte doch die höhere Beamtenschaft angesichts der riesigen Entfernungen schon im europäischen Teile Russlands oft 2-4 Tage ohne Unterbrechung in ihnen. So gehörten Schreibtisch, Polstersessel, Sofas, Chaiselongue, Regale und Schränke zu ihrer Ausstattung.

Als der Divisionsstab am nächsten Tage weitergefahren war, mussten auch wir unsere Zelte abbrechen und in Etappen weiter ins Innere der Ukraine vorrücken -- jetzt ausschließlich mit dem Zuge. Man konnte es sich auch in den Holzklassewagen so bequem machen: durch Herunter- bzw. Hochklappen der Rücklehnen ergaben sich zwei übereinanderliegende Lagerflächen, auf denen es sich gut ruhen ließ.

Das Landschaftsbild wandelte sich allmählich: die riesigen Nadelwälder des Pripjetgebietes und Weißrusslands gingen in Mischwälder und in reine Laubwälder über, in denen die Buche und besonders die Eiche vorherrschte bis schließlich die Wälder fast ganz der ukrainischen Tschernosjom Kultursteppe wichen und die fruchtbare Ebene, soweit der Blick reichte, mit Sonnenblumen, Weizen oder Arbusen -- rotfleischigen Wassermelonen -- und Zuckerrüben bedeckt war. In den Dörfern hatten die weiterhin meistens strohgedeckten Häuser weißen Anstrich. Wenn Gogol in seinen ukrainischen Erzählungen von den herrlichen taghellen Nächten seiner Heimat schwärmt, so dürften wohl die weißen Häuser der Wirkung des Mondscheins zugute gekommen sein. Die Bauernhäuser wurden unten gern bis zu etwa 80 cm mit einer schrägen Lehmschicht gegen Nässe abgedichtet, die man oft rötlich, gelblich oder bläulich einfärbte und an deren Erneuerung und Verschönerung damals in der Zeit vor Ostern emsig gearbeitet wurde. Wir fuhren eines Tages an Dörfern vorbei, denen eine Fülle von Windmühlen verschiedener Größen und Bauformen ein ganz eigenes Gepräge gaben: neben winzig kleinen, nur ca. 3-4~m hohen, gab es mittlere und große, jedoch nicht so große wie die aus Deutschland oder Holland bekannten. Das Merkwürdigste jedoch an ihnen war die Zahl der Flügel: viele hatten 5 oder 6, manche sogar 8 oder 9 Flügel.

Einen märchenhaften Anblick bot vom Zuge aus die Stadt Ssumy (russ. \textcyr{Сумы}). Über einem Eichenwald, den der Eisenbahnzug auf hohem Damm halbkreisförmig umfuhr, glänzten und funkelten im Licht der Abendsonne zahlreiche Kuppeln von einer Reihe orthodoxer Kirchen in grünen, roten, blauen, silbernen und goldenen Farben. Es war ein Bild wie aus 1001 Nacht, das unseren Landsern laute Rufe der Bewunderung entlockte. Das Innere der über \num{40000} Einwohner zählenden Stadt war dann, von den schönen meist 5-kuppligen Kirchen abgesehen, für die Mitteleuropäer etwas enttäuschend: die sehr breiten Straßen waren ungepflastert, glichen breiten Feldwegen. Zu beiden Seiten, entlang an den Backsteinhäusern, liefen schmale Fußstege. Sie bestanden aus 3-4 Bohlen, die auf in die Erde eingerammten Pfählen ruhten. Diese Straßenform wiederholte sich in allen kleineren und mittleren Städten. Alle Hauptstraßen verliefen mit dem Blick in die unendlichen Weiten Russlands -- besonders im Gegensatz zu den Stadtplänen des einstigen römischen Weltreiches -- ohne Blickfang, ohne Abschluss für das Auge. Eine Kanalisation kannten alle diese Städte damals noch nicht, auch keine Wasserleitung.

Berichten könnte ich auch von großen und wohlhabenden Bauerndörfern wie Tamarowska, das über \num{10000} Einwohner zählte. Hier wohnte ich als Divisions-Verbindungsoffizier im patrizisch-stolzen Haus eines Dorfreichen. Das weiße Gebäude mit Flachdach, zu dem eine 10-stufige steinerne Treppe mit Balustrade hinaufführte, zeigte außen klassizistische Form mit Pilastern. Der Salon war wieder der bei weitem größte Raum des Hauses voller Polstermöbel, die im Alltag weiße Bezüge trugen, und voller großer Wintergartengewächse. Dieses Haus stand an der Hauptstraße, breit wie eine spanische Rambla, unweit des großen Dorfplatzes, der in der Mitte von einer mehrkuppligen grün und silbern leuchtenden Kirche beherrscht wurde. Auf ihm fand täglich zwischen 6 und 7 Uhr nachmittags das Gulangje statt, der Bummel der Dorfjugend: im Kreise bewegten sich je zwei Burschen und zwei Mädchen, vereinzelt auch Paare, ungestört durch den aufgewirbelten Staub. Die Burschen neckten, auch mit Gesten, die vor ihnen gehenden Mädchen; Scherzworte flogen hinüber und herüber. Ich beteiligte mich ein paar mal an diesem lustigen Treiben, was von der ukrainischen Jugend beifällig aufgenommen wurde.

\marginpar{361}
Alles freute sich auf das größte Fest der Ostkirche, das Osterfest, das schon am Gründonnerstag abends beginnt. Jung und Alt umschreiten nach Einbruch der Nacht mit brennenden Kerzen und grünen Weidenzweigen die Kirche. Oft läuteten an den Tagen der Vorosterzeit die Glocken, doch anders als in den Ländern der Westkirchen: es war mehr ein Bimmeln, mehrere Glocken verschiedener Tonhöhen wurden nacheinander oder auch gleichzeitig angeschlagen. Fragte ich, warum geläutet werde, erhielt ich stereotyp die Antwort: \textcyr{маленкий} [malenki], gelegentlich auch \textcyr{болъшой праздник} (d.i. ein kleiner bzw. großer Feiertag). Ich hatte den Eindruck, dass die Zahl der ostkirchlichen Feiertage, an denen nicht oder nur begrenzt gearbeitet wurde, höher als in der katholischen Kirche war.

Großartig war die Mitternachtsmesse zum ersten Osterfeiertag, die ich weiter südlich in Isjum erlebte. Herrlich inbrünstig der byzantinische Chorgesang mit dem abschließenden Jubelruf: Christós voskrése -- vistinij voskrese! (Christus ist auferstanden -- in Wahrhaftigkeit auferstanden!) Damals wurde in der Ukraine der alte Brauch noch geübt, er hatte noch volle Geltung: wem man am Ostertage \enquote{Christos voskrese} zurief, der antwortete mit \enquote{vistinij voskrese} und durfte den dreifachen Kuss nicht verweigern -- eine Sitte, an der besonders die Jugend fromm festhielt.

Toll war dann -- nach achtwöchiger Fastenzeit -- die kulinarische Seite des Osterfestes. Es begann mit dem Frühstück: Gänsebraten, leckere Pasteten (russ. pirogi oder piroschki), reiches Gebäck, herrlicher Tee beim dampfenden Samowar, in den man große geschlagene Stücke Zuckers, oft auch Marmelade tat\dots Nur wer von unseren Leuten einen besonders kräftigen Magen hatte, vermochte das Osterfest ungetrübt zu überstehen.

\marginpar{363}
Ich muss kurz noch einiges über die dörfliche Kultur in der Ukraine berichten. Sie ist oder war -- entsprechend der größeren Fruchtbarkeit des Tschernosjomgebietes, des milderen südlichen Klimas und des daraus resultierenden größeren Wohlstandes der weißrussischen und wohl auch der ostpolnischen -- jenseits der Weichsel -- überlegen: ich sah nur gedielte Bauernhäuser, viele bereits mit einem Komfort an Möbeln ausgestattet. Ich erlebte Bäuerinnen, deren Koch- und Backkunst beachtlich war. Eine alte Bäuerin brachte mir ein paarmal Proben selbstgebackener Pirogen, \enquote{piroški}, in Weißbrot eingebackenes Kalbfleisch mit Reis, auch Fisch, alles recht schmackhaft. Vor allem muss ich der Sauna gedenken, die ich zum ersten Mal auf einem mittelgroßen Bauernhof kennenlernte. Isoliert auf dem großen Hofe stand ein kleiner Holzbau, in dessen Mitte sich ein mit Rillen versehener Lehmkegel befand. In ihm war ein Ofen eingebaut, der auch gleichzeitig der Erhitzung des Wassers zum Gießen diente. An der Wand waren in einer Länge von etwa 3~m Holzstufen angebracht, auf denen mehrere Personen Platz hatten. Ich saß auf der obersten Stufe, während ein 15-jähriger Junge Wasser auf die Kegelspitze goss, das im Nu in den Rillen verdampfte. Dann erhob ich mich, stieg hinab und wurde von dem Jungen mit Birkenruten gepeitscht, bis die Haut krebsrot und empfindlich wurde. -- Dieses Peitschen im Badehaus ist übrigens ein sehr alter russischer Brauch. Schon in den altrussischen Heldenepen aus der Kiewer Zeit -- also zwischen 900-1200 n. Chr. -- wird erzählt, dass die Helden nach der Rückkehr aus der Schlacht zunächst das Badehaus betraten, sich mit heißem Wasser reinigten und dann gegenseitig peitschten.

Die Lebensfreude der Dorfjugend fand ihren Niederschlag im Musizieren -- Balalaika und Ziehharmonika herrschten vor --, im ein- oder zweistimmigen Gesang mit und ohne Begleitung und nicht zuletzt im Tanz. Die Dorfjugend tanzte in den kleineren und mittelgroßen Dörfern, die nie einen Wirtshaussaal besaßen, in dieser oder jener der großen Bauernstuben. Man tanzte auch einige der alten bekannten europäischen Tänze neben slawischen, folkloristischen. Aber immer nur einige wenige Tänze hintereinander, dann trat ein Paar in den Kreis und improvisierte ein kleines Ballett, eine reizende getanzte Liebesromanze: ein junges Paar lernt sich kennen, der Bursche umwirbt das Mädchen, es gibt Missverständnisse, das Auftreten einer zweiten Tänzerin führt zu einer Eifersuchtsszene, der Vater mischt sich ein, u.s.f. Bemerkenswert war die Grazie und das Minenspiel der Tänzer(innen).

Von menschlichen Begegnungen möchte ich die mit einer jüdischen Familie, ich glaube, es war in der Stadt Konotop, erwähnen. Die Familie, bestehend aus dem Elternpaar und zwei jüngeren, etwa 17-19 Jahre alten Töchtern, sprach jiddisch, d.i. jene auf mittelfränkischer Basis der spätmittelhochdeutschen bzw. frühneuhochdeutschen Zeit beruhenden germanischen Sprache mit hebräischen Einsprengseln. Man lobte die Deutschen sehr, bei denen es die Juden zu so angesehenen und einflussreichen Stellungen gebracht haben. Man erkundigte sich nach Frankfurt und Breslau. Das Bemerkenswerteste war: die Mädchen sangen im Verlauf des Abends zur Balalaika ein fünfstrophiges Heideröslein mit mir unbekannter Melodie; leider habe ich den Text nicht aufgezeichnet.

Um Pfingsten herum lag der Bataillonsstab mit Leitlof, Mannich und mir in Sol, am Rande des Donjezbecken. Leitlof wohnte in der städtisch-modernen Villa mit Flachdach des Direktors eines florierenden Salzbergwerks. Herr von Starnawski war Pole, desgleichen seine elegante Frau, deren Hand mit einem großen funkelnden Diamant geschmückt war. Es waren zwei Kinder in der Familie, eine hübsche und intelligente 17-jährige Tochter und ein 10-jähriger Junge. Die Eltern sprachen außer ihrer Muttersprache und russisch auch fließend deutsch und französisch. Auch die beiden Kinder waren 4-sprachig erzogen, das Mädchen sprach vorzüglich französisch und leidlich deutsch. Sie erzählte mir ausführlich mit großer Lebendigkeit auf Französisch den Inhalt eines von ihr verfassten Dramas à la Sardon mit Liebesintriguen. Mannich verliebte sich alsbald in diese \enquote{junge Dame von Welt}, wie er meinte, und ritt viel mit ihr aus. Pfingsten waren wir abends zum Diner geladen. Zugegen waren außer uns Deutschen zwei Herren des Aufsichtsrats mit ihrem Vorsitzer, einem gut hochdeutsch sprechenden Juden Meissel und die französische Gouvernante. Die deutsche war mit dem Eintreffen der deutschen Truppen nach Deutschland zurückgekehrt.

Das Essen war üppig und mehr mittel- als osteuropäisch und wurde erst gegen Mitternacht beendet. Alle geladenen Gäste erhoben sich nach polnischer Adelssitte und drückten der Herrin des Hauses den Dankeskuss auf die Hand. Dann rief diese aus: \enquote{Mademoiselle, chantez-nous quelque chose}. Und Mademoiselle ging zum Flügel, spielte und sang dazu ein paar französische Chansons.

Die etwa 30-jährige Gouvernante beklagte sich bei mir bitter über ihr Los: sie gehöre nur zum Personal, sie werde keineswegs als gleichberechtigter Mensch behandelt.

Erwähnenswert ist noch, dass ein Teil des Aktienkapitals des Bergwerks sich in belgischen Händen befand. Man erzählte uns von der Schreckenszeit unter den Bolschewisten, deren Funktionäre nicht das Geringste von der Wirtschaft verstünden und um ein Haar das Werk ruiniert hätten. Man hätte allgemein das Eintreffen der Deutschen begrüßt.

Mannich unternahm mit der schon fast weltgewandten jungen Polin einen Tagesausflug nach Charkow, kam aber etwas bedrückt wieder: sie habe in Charkow für ihn weit weniger Interesse gezeigt, als für die Auslagen der Juweliere\dots

In dieser Gegend war es auch, als ich einmal auf einem langen Ritt in die schier unendliche Kultursteppe auf der Straße, d.h. einem etwa 60~m breitem Feldwege, einem Jungen begegnete und ihn russisch fragte, wieweit es noch zum nächsten Dorfe sei. Der 10-jährige Junge antwortete in schwäbelnder Mundart: \enquote{Zu Fuß ne guete Stunde}. Es war der erste Deutsche in Zivil, dem ich in der Ukraine begegnete, und so glaubte ich im ersten Augenblick, er sei mit seiner Familie erst kürzlich während der deutschen Besatzung hierher gelangt. Auf meine Frage: wie lange seid ihr denn hier? kam prompt die Antwort: 128 Jahre! Nun ging mir ein Licht auf: der Junge gehörte einer Familie an, die zur Zeit Katharinas~II. aufgrund deren Werbung in Süddeutschland und der Schweiz nach Russland ausgewandert waren. Und nun forschte ich weiter und erfuhr, dass es sich um ein deutsches Kolonistendorf von 300 Einwohnern handelte, alles nur Deutsche, und ringsum Russen, also eine Insel im russischen Meer. Die Jahreszahl wusste er vom Lehrer, auch einem Deutschen.

Ich war begierig diese deutsche Insel kennenzulernen und ritt an einem der nächsten Tage hinüber. Das Dorf unterschied sich von außen nicht von den ukrainischen Dörfern der Gegend: strohgedeckte Häuser mit geweißten Wänden -- innen jedoch Stuben deutschen Gepräges, das sie aus der Heimat mitgebracht hatten. Die Familien des Dorfes waren alle miteinander verwandt, es herrschte Inzucht. Man stellte mir ein junges Brautpaar vor, der Junge 8, das Mädchen 9 Jahre; die Kinder wussten, dass sie sich heiraten würden. Auf meine Frage, warum die Kinder so zeitig einander versprochen worden seien, erhielt ich die recht bäurische Antwort: \enquote{Die Äcker der beiden Familien liegen günstig zu einander}. Dieses Inselleben und, wie ich damals glaubte, auch die Inzucht hatten diesen Menschen den Stempel einer spürbaren Schwermut aufgedrückt, die im Gegensatz zu der ukrainischen Fröhlichkeit stand. Man sagte mir, dass es im Dorfe keinen einzigen Fall von Vermischung mit Ukrainern gebe. Viele von ihnen konnten überhaupt nicht russisch sprechen. Kontakte mit der Außenwelt -- es gab ja damals weder Radio noch Fernsehen -- vermittelte allein der Lehrer, der jedes Jahr Pfingsten zu einem Treffen der deutschen Lehrer nach der Krim fuhr. Der Traum eines jeden war, doch einmal im Leben nach Deutschland reisen zu können. Die schüchterne Frage: ob die deutschen Truppen ihnen das wohl ermöglichen würden? Oft habe ich später an diese deutschen Bauern gedacht -- welches mag ihr Schicksal im 2. Weltkrieg gewesen sein?

Charkow war damals schon die große Verwaltungsmetropole des unweit gelegenen Donjezbecken und das Zentrum einer in den Anfängen steckenden Elektroindustrie. Das öffentliche Verkehrsmittel, das auf der Hauptstraße, der Jekaterinoslawskaja, die Verbindung von Hauptbahnhof und Stadtzentrum um die orthodoxe Kathedrale herstellte, war allerdings damals -- Herbst 1918 -- eine primitive Pferdebahn, wie ich sie 1900 als 7-jähriger Junge noch auf einigen Berliner Strecken gesehen hatte. Abseits einiger weniger großer Prachtstraßen, in denen auch zwei oder drei elektrische Straßenbahnlinien verkehrten, unterschied sich der Großteil der Häuserfronten dieses schon damals weit über einer halben Millionen zählenden Stadt kaum von denen durchschnittlicher russischer Mittelstädte. In dieser auf dem 50. Breitengrade\footnote{wie z.B. auch Frankfurt/Main} gelegenen Stadt erlebte ich übrigens einen jähen Witterungswechsel, den man dort nicht für ungewöhnlich hielt: am 28. Mai hatten wir vormittags eine Stunde lang Schneetreiben, und am 2. Juni seufzen wir unter 32° Hitze!

Noch im Juli hatte mir Herr Meissel in einem Charkower Café, wo ich ihm zufällig begegnete, seiner Freude über große deutsche Erfolge in Nordfrankreich Ausdruck gegeben -- und in der 2. Augustdekade berichtete der stellvertretende Regimentskommandeur Major Eck, Generalstäbler, der zum großen Generalstab Verbindung hatte, uns d.h. dem Bataillonsstab des in Bjelgorod liegenden 1. Bataillons unter dem Siegel der Verschwiegenheit von dem Konferenzergebnis im Hauptquartier zu Spa: Krieg nicht zu gewinnen, Waffenstillstandsangebot an Feinde werde dem Kaiser vorgeschlagen, Elsass-Lothringen sei verloren.

\marginpar{375}
Inzwischen war -- schon im Juli -- der Oberkommandierende der deutschen Besatzungstruppen Generaloberst von Eichhorn einem kommunistischen Attentat in Kiew zum Opfer gefallen und hier und da riefen ukrainische Eisenbahner zum Streik auf. Dies war für uns natürlich eine ernste Gefahr und so wurde deutscherseits auch mit drakonischen Strafen nicht nur gegen Streikpropagandisten, sondern auch gegen jeden Teilnehmer an einer Streikversammlung eingeschritten. Auch in Bjelgorod hatte eine verbotene Streikversammlung stattgefunden und ein Kriegsgericht, zu dem ich als Beisitzer kommandiert wurde, hatte über die Bestrafung der Delinquenten zu befinden. Als Mensch verwahrte ich mich gegen die Mitverantwortung an meiner Meinung nach ungerechten Blutstrafen, forderte für mich wenigstens das Recht der Stimmenthaltung, wurde aber vom Vorsitzenden Kriegsgerichtsrat in heftigen Auseinandersetzungen über einschlägige Kriegsartikel belehrt und erreichte wenigstens, praktisch als Verteidiger fungierend, den es im Kriegsgericht nicht gab, dass überall die mildere Strafe \enquote{lebenslängliches} Zuchthaus, bis \enquote{drei Jahre} Zuchthaus verhängt wurde. Als ich einmal das Gerichtsgebäude verließ um Luft zu schöpfen, umfassten junge Frauen und kleine Kinder meine Knie mit tränenerstickten Rufen \enquote{pomilnitje, gaspadin, pomilnitje} (Erbarmen, Herr, Erbarmen) Ich bin solcher Szenen nicht gewachsen.

Der Besuch des Div. Pfarrers, Korpsstudent, Schmisse, geschätzt von den Herren der höheren Stäbe, eine Welt, in der er sich sichtlich wohl fühlte. Den Unwillen des Korps rief er mit der herablassend-süffisanten Bemerkung hervor: die kleinen Stäbe leben eigentlich hier auch ganz gut. Offenbar ein Pfarrer -- aber bitte nur für die Oberschicht.

Erneut musste das Regiment Offiziere und jüngere Mannschaften an den Westen abgeben. Ich war schließlich der jüngste Offizier geworden. Weswegen hielt man mich fest, obwohl ich mich bestimmt keiner besonderen Gunst von der Chevallerie erfreuen konnte? Es waren nicht nur meine slawischen (wie auch französischen) Sprachkenntnisse sondern auch meine sonstigen Spezialitäten: Erfahrung in der Anlage und im Bau von Stellungssystemen, Bataillonsadjutants u.a. zu denen jetzt in Bjelgorod getreten waren Fernmeldewesen, Leitung des Regiments-Nachrichtenzuges (ich hatte mir u.a. Fertigkeit im Morsen angeeignet) und Leitung der Regiments-Reitertruppe von 30 Pferden. Es wurde gebildet im Rahmen von Felddienstübungen und Abwehrmaßnahmen gegen einen möglichen Durchbruchsversuch tschechoslowakischer Einheiten von Osten her. Wir hatten Nachrichten, dass ein Großteil der während des Krieges von den Österreichern zu den Russen übergelaufenen bzw. von den Russen gefangenen tschechischen Soldaten sich formiert hätten zu einem Verband, der über Artillerie und Maschinengewehre verfügten und die die Absicht hätten, durch die Ukraine kämpfend in ihre Heimat zurückzukehren. Schließlich verlautete im September, sie hätten sich der ukrainischen Demarkationslinie bis auf 30~km genähert. Man wollte unweit der provisorischen Grenze schon Kanonendonner gehört haben.

\marginpar{376}
Die letzten Wochen in der Ukraine verbrachten wir unweit der östlichen Demarkationslinie in der Gegend von Sloviansk-Kupiansk. Es war ein goldener September. Die Fronten vieler Bauernhäuser wurden mit langen Ketten roter Paprikaschoten geziert. Die Jugend kaute fortwährend Sonnenblumenkerne, \enquote{ukrainische Schokolade} wie unsere Leute sagten, wobei sie es im Ausspucken der Schale zu Rekordgeschwindigkeiten brachten, die schon an das MG-Tempo erinnerten.

Die Ausbildung meines Reitertrupps wurde zu meinem Hobby. Neben theoretischer Unterweisung, Übungen im Kartenlesen, Orientierungsübungen mit und ohne Karte, mit und ohne Kompass bei Tag und bei Nacht lagen mir besonders weite Ritte ins Land, mit eingestreuten kleinen didaktischen Übungen wie: aus einem Waldstück 1200~m vor uns Infanteriefeuer. Lösung: im Galopp hinter die nächste Bodenwelle, vor uns oder rückwärts -- 2 Mann bleiben bei den Pferden, alles andere ausgeschwärmt, am Rand der Bodenwelle mit Schussfeld hinwerfen, Feuerbefehl abwarten\dots

Am 28. September wurde das Regiment verladen, und zwar für den Einsatz in Frankreich.

\section{Verlegung nach Frankreich}
Für den Großteil der ukrainischen Bevölkerung war unser Fortgang schmerzlich, wussten sie doch, was ihnen mit der Wiederbesetzung durch die Bolschewiki bevorstand. Skoropatzki der Hetmann\footnote{in Polen vergleichbar mit dem Feldmarschall, vermutlich war aber \enquote{Hauptmann} gemeint, da der Hetmann am Ende des 18. Jahrhunderts abgeschafft wurde} von Kaiser Wilhelm~II. Gnaden, seines Zeichens Latifundienbesitzer, hatte keine Wehrmacht aufgestellt und auch in der Ukraine keinen Anhang. Das Schicksal dieses unglücklichen Landes schien besiegelt. Stalin deportierte über 10 Millionen Ukrainer nach Sibirien und siedelte sie dort an. Damit war diesem slawischen Volke das nationale Rückgrat gebrochen. Nach weniger als 40 Jahren eroberte der gebürtige Ukrainer Chruschtschow die Führung der Sowjetunion -- nicht als Ukrainer, sondern als panslawistischer Kommunist.

Auf der Durchfahrt über Poltava-Kiew-Korosten -- unvergesslich der Rückblick von dem hohen rechten Ufer des mächtigen Dnjepr auf die schöne, fast westlich anmutende Barockstadt -- wurden uns noch Beweise einer geradezu rührenden Gastfreundschaft zuteil.

In Breslau verschwand mein Bursche Maczénski, Bauernsohn aus dem Posenschen -- Nationalpole wie alle Polen dieser ehemaligen preußischen Provinz. Ich erhielt später von ihm eine polnisch geschriebene Karte, dass er sich an der Erstürmung des deutschen Militärflugplatzes bei Posen beteiligt habe. Das war das Ende unserer Beziehungen.

In Bunzlau, wo es nachmittags einen längeren Aufenthalt gab, waren meine Eltern auf dem Bahnhof erschienen. Mein Vater litt unter dem bevorstehenden Zusammenbruch des Bismarck-Reiches, zudem unter der Einwirkung einer, wie Dr. Thamm diagnostiziert hatte, \enquote{hartnäckigen Grippe bei unzureichender Ernährung}.

Hatten wir vom Verladebahnhof bis Breslau über 80 Stunden gebraucht, so ging es jetzt innerhalb der Reichsgrenzen schneller, und schon am nächsten Mittag wurden wir auf französischen Boden bei Mars-la-Tour, dem berühmten Schlachtfeld des Krieges 1870/71, ausgeladen. Alsbald erschienen am Himmel ein Dutzend alliierter Flieger und warfen bei geringer und unwirksamer deutscher Abwehr auf unsere Bagage Bomben ab, was diese \enquote{Kühen- und Bagagenhengste}, wie sie verächtlich vom Frontsoldaten genannt wurden, im Osten noch nicht kennengelernt hatten. Es drängte sich uns der betrübliche Schluss auf, dass die Angelsachsen den Luftraum beherrschten. Zudem erfuhr ich, dass unsere strategische Lage sehr ernst war, da wir, d.h. unser Regiment, die ganze Reserve einer Heeresgruppe darstellte\dots

\marginpar{381}
Wir wurden sofort im unter amerikanischem Druck zurückgenommenen St. Mihiel-Bogen eingesetzt. Als Kompanieführer im vordersten Graben bekam ich vier Tage lang noch einen Vorgeschmack von den Kampfmethoden an der Westfront, obwohl wir in unserem Abschnitt \enquote{nur Amerikaner} vor uns hatten. Eine halbe Nacht lang lagen wir unter Gasbeschuss. Die Grabenbesatzung wurde durch Anschlagen an Schienenstück alarmiert, die Unterstandseingänge mit in Wasser getauchten Decken abgedichtet, im Taschen- und Karbitlampenlicht prüfte man gegenseitig den Sitz der Gasmaske. Ich merkte bald, dass man innere Ruhe und damit ruhigen gleichmäßigen Atem bewahren musste, was nicht leicht war, besonders als ich schon nach wenigen Minuten die schreckliche Nachricht erhielt: zwei meiner Leute waren bei der Rückkehr vom Essensempfang auf eigene Mienen geraten und zerfetzt worden.

Es war eine Erlösung, als wir kurz nach Mitternacht die quälende Maske abnehmen und wieder frei atmen konnten. Zwei Tage später, als ich von einem Kurzlehrgang bei Bily über die infanteristische Kampftaktik gegen den schnellen und immer geländegängigeren englischen Tank zurückkehrte, erreichte mich der Befehl meiner Versetzung zum Regimentsstab. Auf dem Wege dahin, er lag auf einem Gutshof namens Belvedère-les-Barraques, kam ich an einer Verwundetensammelstelle mit Opfern des Gasbeschusses vorbei: Spritzer des im Augenblick der dumpf klingenden Detonation flüssigen Gases hatten kreisrunde bis zur Größe eines Pfennigstückes Löcher in das Fleisch der Wangen gefressen.

Zu meinen Obligenheiten beim Regimentsstab gehörten vor allem: Auswertung der Fliegerbilder besonders das eigene Stellungssystem betreffend (Frage: was ist aus der Luft vom Feind erkennbar? Was muss besser getarnt werden?), ferner Leitung des Nachrichtenzuges, insbesondere der Nachrichtenmittel, die verbleiben, wenn alle Telefonleitungen durch Artillerie und Fliegerbomben zerstört sind; dann blieben damals, wo die unteren und mittleren Stäbe wie Bataillon und Regiment noch nicht über drahtlose Telegraphie verfügten, nur Blinkverbindungen und Brieftauben. In Technik und Problematik dieser beiden Nachrichtenmittel arbeitete ich mich in Kürze ein. Vor allem jedoch war ich der persönliche Adjutant Chevalleries im Frontdienst und bei Kampfhandlungen: ich war sein ständiger Begleiter, wenn er, im Wechsel mit anderen Kommandeuren, 24 Stunden im Gefechtsstand Dienst hatte, der zwischen Belvédère-les-Barraques und der vordersten Linie, etwa 1,5~km von dieser entfernt, lag. Mehrfach ritt ich früh im ersten Morgengrauen zum Regiments- bzw. Bataillonsgefechtsstand zur Übermittlung und zum Empfang taktischer Nachrichten, die aus Furcht vor feindlicher Abhörtätigkeit nicht telephonisch durchgegeben werden konnten, bzw. zur Führung taktischer Besprechungen im Auftrage des Kommandeurs. Auf dem Wege dorthin musste ich über eine Chausseekreuzung reiten, die in den frühen Morgenstunden regelmäßig unter amerikanischem Artilleriebeschuss lag; ein Ausweichen auf die vom Regen aufgeweichten Felder zu beiden Seiten der Chaussee war nicht möglich. Alle 2,5 Minuten kam aus dem Marinegeschütz eine Granate mit ziemlicher Treffsicherheit angeheult. Ich hielt etwa 200~m vor der Kreuzung, wartete den Einschlag ab, jagte im Galopp über die Kreuzung und fühlte mich schon in Sicherheit, wenn die nächste Granate über mir fauchte. Das Kriegsglück war mir hold, auch mein sehr wachsames Pferd wich den Granattrichtern geschickt aus und kam nicht zu Fall.

Moritz von der Chevallerie, in Friedenszeiten Linienoffizier bei den Potsdamer Grenadieren, Junggeselle, Endvierziger, war stolz auf seine Herkunft aus altem bretonischen Adel, trug den Uniformrock stets mit dem Johanniterkreuz geschmückt. Schon am ersten Tage in Frankreich äußerte er: Sie müssen zujeben, dass doch schon ein französisches Bürgerhaus was besseres ist als ein russisches Château. Sein etwas nörglerisches, im Grunde wenig geselliges Wesen, das Fehlen geistiger außermilitärischer Interessen waren wohl der Grund dafür, dass er nicht sonderlich beliebt war. -- Nach dem Abendessen wurde in der Regel Skat gespielt. Hatte er Glück und etwas Geld gewonnen, wurde er gegen 11 Uhr abends müde und ging schlafen. Im gegenteiligen Falle spielte er ausdauernd und mehr oder weniger missgelaunt bis 2 Uhr oder 3 Uhr morgens, wobei er dem Mitspieler gern tatsächliche oder angeblich gemachte Fehler vorwarf (\enquote{Busse, wie konnten Sie den König zujeben!!}) Da ich mir meine schon oft genug durch Artilleriebeschuss und Flucht in bombensicheren Unterstand gestörte Nachtruhe nicht noch durch Kartenspiel schmälern wollte, zog ich die Telefonisten ins Vertrauen. Einer von ihnen erschien zu vereinbarter Zeit: \enquote{Herr Leutnant Busse ans Telefon}. Moritz: \enquote{Wer ist am Apparat?} Telefonist: \enquote{Die Division.} Moritz: \enquote{Tja Busse, da müssen Sie natürlich gehen!} Ich verschwand und ging schlafen. Wurde ein Mitglied des Stabes während einer Mahlzeit von einem Offizier unseres Regiments verlangt, rief Moritz dem Telefonisten zu: \enquote{Sagen Sie dem Herrn Hauptmann, dass wir essen!} Bequemlichkeit und die Jagd, sein Hobby, waren ihm wichtig. Als an die Regimenter der Divisionsbefehl erging, je einen Reservegefechtsstand zu bauen, wies mich Moritz an: \enquote{Machen Sie eine Stelle ausfindig, wo ich jute Gelegenheit habe, Karnickel zu schießen.}

Es ging indess auf das für Deutschland so trübe Ende des Krieges zu. Der deutsche Widerstandswille begann zu erlahmen, gleichzeitig mit Bewusstwerden, dass der Sieg nicht zu erringen sei. Beklommen hörten wir von der Meuterei der Matrosen und Aufständen in deutschen Städten, schließlich von der Bildung von Arbeiter- und Bauernräten. Von der Chevallerie kam blass, verstört und verschreckt zum Abendessen. Auf unsere Frage, was passiert sei, legte er den Finger auf den Mund und wartete, bis die Ordonnanz die Tür hinter sich geschlossen hatte und presste dann halblaut heraus: \enquote{Ich gehe vorhin vorm Stabsquartier auf und ab, kommt da ein Artillerist vorbei, ohne mich zu grüßen. Ich sage: von welchem Truppenteil sind Sie? Wissen Sie, was der Kerl mir antwortet? Von welchem Regiment ich bin, wollen Sie wissen? Das geht Sie 'nen Scheißdreck an, verstehen Sie?} \enquote{Was haben Herr Oberleutnant getan?} \enquote{Was sollte ich tun. Nichts, der Kerl konnte ja gemeingefährlich sein}, war seine Antwort. Dieser jeder Disziplin hohnsprechende Vorfall war wirklich bedrückend.

Am Ende der ersten Novemberwoche erschien der in unserem Regiment gebildete Soldatenrat in Belvédère-les-Barraques und teilte dem Regimentskommandeur die Absetzung und Vertreibung mehrerer Offiziere, darunter eines Bataillonskommandeurs, mit. Von der Chevallerie darauf in gedämpftem Tone: \enquote{Aber Sie können doch nicht meine Offiziere verjagen!} \enquote{Doch}, so der mir unbekannte intelligent aussehende Sprecher des Soldatenrates, \enquote{Sie haben nichts anderes verdient. Neben Ihnen steht Leutnant Busse, bei ihm herrscht Ordnung und Disziplin, aber er hatte auch ein Herz für seine Leute -- kein Soldatenrat würde ihn verjagen.} In trauriger Stunde sah ich in diesen Worten so etwas wie eine Bestätigung meiner selbst.

\section{Waffenstillstand und Heimkehr}

Am 11. November begann die Waffenruhe, die unter für Deutschland äußerst harten Bedingungen ausgehandelt worden war.

Als letztes Regiment der Division traten wir am 13.11. den Rückmarsch an. Alle Offiziere mit Ausnahme der Kommandeure mussten auf Verlangen des Soldatenrates unter Verzicht auf das Reitpferd zu Fuß marschieren. Die Soldaten hatten sich toll mit roten Bändern geschmückt -- wer weiß, woher diese Mengen an Revolutionssymbolen plötzlich gekommen waren. Als meine frühere Kompanie vorbeizog, rief ich aus: \enquote{Na, ihr seht ja wie Pfingstochsen aus! Schämt ihr euch nicht? Schließlich haben wir den Krieg \underline{verloren}!} Man nahm meine Worte ohne Gegenrufe hin, und am zweiten Marschtag hatten viele die bunten Bänder abgelegt.

Mit dem Regimentsstab ging auch Leutnant Fürstenheim zu Fuß, Führer der Großen Bagage und als solcher während des Krieges stets weit hinter der Front; seines Zeichens Jude und Rechtsanwalt. Während wir uns in trüben Betrachtungen über die wahrscheinlichen Folgen des verlorenen Krieges ergingen, gab er sich sehr optimistisch und meinte: \enquote{Was ist schon -- Deutschland verliert Elsaß-Lothringen und weiter nichts. Und mir ist es egal, ob ich in Frankfurt, Brüssel oder Warschau, übrigens eine nette Stadt, sitze, ich werde immer oben schwimmen.}

Am Abend tranken wir in engem Kreise eine Flasche Champagner. Chevallerie sagte: \enquote{Die letzte Nacht auf französischem Boden, prosit, meine Herren!}

Die nächste Nacht verbrachten wir im letzten lothringischen Dorf vor der Grenze des Saarlandes. Das Dorf hatte rein deutschsprachige Bevölkerung. Der Regimentskommandeur quartierte sich in der Pfarrei ein, ich im Hause des Dorfschullehrers nebenan. Der Lehrer und sein Sohn waren aus dem deutschen Heeresdienst noch nicht entlassen bzw. zurückgekehrt. Seine Frau und Tochter erzählten von unerfreulichen Dingen, die ihnen unter der deutschen Herrschaft und ganz besonders in der Kriegszeit widerfahren waren, Misstrauen und militär- und zivilbehördliche Schikanen. Die Erzählung der Lehrersfrau gipfelte in den niederschmetternden Worten: \enquote{Wir sind rein\linebreak deutsch. Hier im Dorf spricht keiner französisch. Ich kann diese Sprache auch nicht. Wir kommen jetzt an Frankreich und man wird uns nicht nach unserem Wunsche fragen. Wir wissen genau, dass uns die Franzosen keine gebratenen Tauben in den Mund stecken werden. Aber wenn man uns die Wahl zwischen Deutschland und Frankreich zugestehen würde, unsere Entscheidung wäre Deutschland? Nein. Wir haben zuviel Bitteres erfahren.} Ich war nicht erstaunt, aber tief traurig. Das war der Bankrott der Bismarckschen Reichslandpolitik unter ostelbischer Führung, die auch unter Wilhelm~II. bestehen blieb. Die preußische Verwaltung hatte sich überhaupt nicht bemüht, zumindest die deutsch bzw. niederallemannisch sprechende Mehrheit Elsass-Lothringens im deutschen Reich zu integrieren.

Bei Bons überschritten wir als letztes Regiment der Division die Saar, dicht gefolgt von französischen Truppen, deren Musikkorps auf dem linken Saar\-ufer, für uns vernehmlich, französische Märsche spielten. Wir machten zwei Tage im Industrieort Bons halt. Von der Chevallerie wohnte in der Villa des Mannesmann-Direktors Noack, ich daneben in einem Beamtenhaus. Noack führte von der Chevallerie und mich durch das große noch in den meisten Abteilungen voll arbeitende Röhrenwerk. Mächtige stählerne Zylinder durchstießen dicke, weißglühende, im Arbeitsvorgang Funkenhagel sprühende Vierkant-Eisenbalken. Der Eindruck war nicht minder gewaltig als die Szenerie in Menzels Walzwerkbild.

Während das Regiment den Marsch nach Würzburg, wo es demobilisiert wurde, fortsetzte, durfte ich mit der Bahn nach Bunzlau, dem Sitz des Regi\-ment-Ersatzbataillons mit bestimmten Weisungen vorausfahren. Die Fahrt ging in überfüllten Zügen, mehrfachen Umsteigen über Bingerbrück, das rührend zur Begrüßung der heimkehrenden Krieger in ein Fahnenmeer getaucht war, über Frankfurt, wo auf dem Bahnsteig ein Mann wie ein Schwerverwundeter aufschrie, dass man seine Brieftasche mit den Ersparnissen seines Lebens gestohlen hatte, über Halle nach Sagan / i. Schlesien. Der Anschlusszug der Strecke Siegersdorf-Hirschberg ging erst in zwei Stunden, also machte ich mich auf den Weg zur Stadt. Doch bald steuerten mich drei Landstürmer, d.h. Soldaten der Jahrgänge 40-45, mit Karabinern bewaffnet und mit roten Armbinden an und forderten mich auf, ihnen meine Waffen -- Offiziersseitengewehr und Armeepistole -- abzuliefern. Da ich ein solches Ansinnen für ehrenrührig hielt und mir obendrein diese Revoluzzer nicht den Eindruck von Mut und Kampfentschlossenheit machten, legte ich die Hand auf meine Pistolentasche mit den Worten: \enquote{Nur mit Gewalt!} Das genügte. Die drei sahen sich einige Sekunden verlegen an und kehrten zur Stadt zurück.

Am Spätnachmittag traf ich in Siegersdorf ein. Auf dem Bahnsteig stand mein Vater; er sah erschreckend schlecht aus. Unser Nachbar, der Arzt Dr. Thamm, Schwager Lotte Steinbergs und Jagdfreund meines Vaters, meinte: \enquote{Er quält sich bei unzureichender Ernährung mit einer hartnäckigen Grippe herum, wie schließlich wir alle mehr oder weniger}. Dieses leichtfertig über den Daumen gepeilte Urteil empörte mich innerlich. Ich nahm meinen Vater schon am nächsten Tag mit nach Bunzlau. Ergebnis der Röntgen-Untersuchung: schwerer Tbc-Schaden der rechten Lunge. Ein anderer als tüchtig geltender Arzt sagte mir nach gründlicher Untersuchung: Heilungsaussichten nicht mehr als 30\%. Einem tüchtigen Breslauer Facharzt, der sich zu einer Reise nach Siegersdorf bereit erklärt hatte, musste ich in letzter Stunde abtelegraphieren. Mein Vater widersetzte sich hartnäckig wegen der zu hohen Kosten. Tuberkulinspritzen, Ruhe und gute Verpflegung und die aufopfernde Pflege durch meine Schwester Ella versagten schließlich. Er ertrug sein Leid tapfer und starb am 31.1.1920 -- zu seinem Leidwesen erlebte er nicht mehr die examensmäßige Absicherung der Existenz seines Sohnes und hinterließ seine geliebte Ella ohne berufliche Ausbildung. Eine Beruhigung gab ihm das feierliche Versprechen seines Freundes Bonfils, \enquote{er werde sich um die beiden Kinder kümmern, bis sie flügge sind}.

Mein militärisches Verhalten war damals wohl nicht mehr ganz korrekt: ich betrachtete den Militärdienst als für mich erledigt und fuhr nach Breslau, um dort meine Studien beschleunigt fortzusetzen. Den Regimentskommandeur setzte ich von Breslau aus schriftlich in Kenntnis. Ein Telegramm Chevalleries rief mich nach Bunzlau zurück, jedoch, wie sich herausstellte, in der Hauptsache zu einem Abschiedsessen im \enquote{Hotel zum Kronprinzen} in kleinem Kreise; der Stabsarzt fehlte. Es wollte begreiflicherweise keine Heiterkeit aufkommen. -- Ich habe von der Chevallerie nie wiedergesehen.

