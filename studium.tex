\section{Greifswald}
Greifswald ist eine alte Hansestadt am Steppenflüsschen Ryk [slaw. reká] das von Greifswald bis zum 4,5~km entfernten Fischerdorf Wieck am Greifswalder Bodden seit altersher kanalisiert ist. Das Stadtbild ist noch heute von drei mächtigen gotischen, von Dohlen umflatterten Backsteinkirchen aus dem\linebreak 13./14. Jahrhundert beherrscht, die gleichzeitig das meilenweit sichtbare Wahrzeichen Greifswald bilden, jedem von den Bildern C. D. Friedrichs bekannt. Greifswald war vor dem 1. Weltkriege eine ausgeprägte \enquote{Sommeruniversität} mit nur rund \num{1000} Studenten im Winter, jedoch über \num{1500} im Sommer. Die Studenten beherrschten das Leben der Landstadt mit ihren wenig mehr als \num{20000} Einwohnern. Die damals ausschließlich mit der Bahn eintreffenden \enquote{Muli} wurden an der Sperre von Verbindungsstudenten abgefangen und \enquote{gekeilt}. Mir wurde an der Sperre der Koffer von einem Couleurstudenten aus der Hand genommen, der sich alle erdenkliche Mühe gab, mich für seine Turnerschaft zu \enquote{keilen}. Wir landeten zunächst in seinem Verbindungshaus, und er ließ mich nicht los, obwohl ich ihm klar sagte, dass ich nicht die Absicht habe, Couleurstudent zu werden, sondern begleitete mich weiter auf meiner Suche nach einer Studentenbude. Jedes 3. oder 4. Haus zeigte ein Schild: \enquote{Zimmer frei} oder \enquote{Zimmer für Studenten frei}. Bald fanden wir ein passendes im Erdgeschoß der \enquote{Langefuhrstraße}, von der man in 3 Minuten durch das klassizistische Steinbecker Tor zum Hafen mit den Bootshäusern der studentischen Verbindungen gelangte sowie zur Anlegestelle der kleinen Dampfer nach Wiek und damit auch der benachbarten, durch C. D. Friedrich berühmt gewordenen Klosterruine Eldena. Bald begegnete ich meinen Konabiturienten Menzel, Mahling und Richter, die sämtlich evangelische Theologie studierten. Die Greifswalder Fakultät galt als konservativ-orthodox, deren Hauptvertreter, wie man mir erzählte, u.a. unbedingt an dem Dogma festhielt, dass der Heilige \underline{Geist} in der \underline{Gestalt} einer Taube auf die Erde gekommen sei.

Inzwischen hatten mich Vertreter der \enquote{schwarzen}, d.h. nicht farbentragenden Verbindung A.T.V\footnote{Akademischer Turner Verein} mit Erfolg gekeilt. Sie besaßen in einer Villenstraße ein schönes Vereinshaus, in dem man gemeinsam ein gutes, kräftiges Mittagessen einnahm und anschließend auf dem \enquote{Paukboden} sich im Fechten, zunächst mit Schlägern, übte. Ein netter humoriger Bursche des 5. Semesters hatte es bald auf mich abgesehen, mich zu seinem Leibfuchs zu machen. Obwohl er ein tüchtiger Biertrinker war, fand er sich, wie auch die Mehrzahl des Konvents, damit ab, dass ich kein Bier trank. Ich durfte ausnahmsweise an den Kneipabenden Apfelsaft trinken, hatte mich aber zu strengster Verschwiegenheit verpflichten müssen. Doch schon auf der 1. Semesterkneipe -- es war Pflicht für uns Füchse, die Tafel nicht vor 2 Uhr morgens zu verlassen -- geriet der Abstinenzler in ein Dilemma: Schon bald nach Beginn der \enquote{Fidulität} um 10 Uhr abends belebte sich die Stimmung unter dem Einfluss des Bieres mehr und mehr, schon vor Mitternacht waren alle mehr oder weniger angebläut, und als einziger Nüchterner empfand ich bald das Unmögliche meiner Lage: Ich entsagte dem Apfelsaft und arrangierte mich mit dem Bier.

Der Heimweg war lustig: viele Studiker, die nicht mehr laufen konnten, fuhren johlend und singend in einer Mietsdroschke nach Haus. Bald musste ich seitlich ausweichen: aus den Fenstern der Droschken entleerten die Insassen in riesigen Bögen nach beiden Seiten ihre bierüberfüllten Blasen -- man nannte solche Kutschen \enquote{Sprengwagen}. Die Polizei verschloss meist die Augen. Mein Freund Menzel hatte allerdings einmal Pech: Als er sich mitten auf der Straße zu erleichtern begann, kam ein Polizist um die Ecke und forderte ihn barsch auf, das Urinieren sofort einzustellen. Er wollte und konnte wohl auch seine Beschäftigung nicht plötzlich abbrechen und vermaulte sich. Statt der damals üblichen Geldstrafe von 3 Mark brummte man ihm 9 Mark auf; zusätzlich wegen \enquote{Widerstandes gegen die Staatsgewalt}. Studenten, die erst um 3 Uhr oder später nach Haus gingen -- im nördlichen Greifswald ist es im Sommer um diese Zeit schon taghell -- klopften gern Bäcker heraus, und man brachte bereitwillig Tisch und Stühle sowie frische Butterbrötchen auf die Straße, wo man noch bis nach Sonnenaufgang saß.

Obwohl Greifswald im Rufe einer \enquote{Arbeitsuniversität} stand, war es doch auch Schauplatz so manchen Studentenulks. Eines Sonnabendnachmittags fand ich die Hauptstraße, die \enquote{Lange Straße}, völlig blockiert, man konnte sich nur mühsam auf den Gehsteigen durchquetschen: Verbindungen hatten die Pferde von sämtlichen Fahrzeugen abgespannt und die Wagen mit Deichseln ineinandergeschoben. Die Mehrzahl der Bevölkerung teilte die Freude der Studenten an diesem \enquote{Fez} und die \enquote{Herren Wachtmeister} wurden mit Schnäpsen bestochen.

Der tollste mir bekannte Studentenulk spielte sich gegen Abend auf dem damals noch üblichen Nachmittagsmarkt ab, wo der Student sich Lebensmittel zum Abendbrot kaufte: einen Hering oder Bückling, Zwiebeln, einen Zipfel Wurst oder Käse u.ä. Der weithin als Spaßvogel bekannte \enquote{Dr. Uhu} -- der Dr.-Titel war nicht echt, in Greifswald wurde jeder Student vom 1. Semester an von der Bevölkerung mit \enquote{Herr Doktor} angeredet, was so manchem Jungen imponierte -- erschien mit einem Gefolge auf dem Markt und man kaufte ein. Zuvor hatte er sich ein Kuheuter um den Leib gebunden und ließ eine Zitze zum Schlitz heraushängen und trug darüber, da der Mai noch kühl war, einen Sommerpaletot. \marginpar{89} Beim Zahlen schlug er den Paletot zurück, worauf die Marktfrauen auf das heraushängende Etwas zeigten und mehr belustigt als entrüstet riefen: \enquote{Herr Doktor! Herr Doktor!} Dr. Uhu schaute herunter: \enquote{Verflucht! Das Ding hat mich lang genug geärgert!} Zog ein Messer, schnitt die Zitze ab und warf sie vor die Marktweiber hin, die aufschrien und kreischten: \enquote{Ach, wie schade!} u.ä.

Sehr feierlich war der Amtsantritt des neuen Rektors. Die Chargierten aller studentischen Corporation(en) fuhren in Wicks in Pferdedroschken am Universitätsgebäude vor. Nur die drei Corps kamen vornehm in drei Autos, zu deren Sitzen man damals auf drei bzw. vier Stufen gelangte. Was sollte das vornehmste Corp der \enquote{Pommern}, Söhne des Landadels und reicher bürgerlicher Rittergutsbesitzer machen? Ebenso ein Fahrzeug benutzen wie andere Verbindungen? Unmöglich! Also ging man die 1,5~km vom Corpshaus zur Universität zu Fuß!

Ich belegte im 1. Semester viel mehr Vorlesungen und Übungen als ich auf die Dauer wirklich wahrnehmen konnte: Neben Fachvorlesungen wie Rehmkes Einführung in die Psychologie, Thuraus Abschnitte aus der französischen Syntax, Ehrismanns 4-stündige Vorlesung über Goethe und des franz. Lektors Vorlesung \enquote{poètes d'hier et d'aujourdhui} hörte ich noch Vorlesungen über gotische Kirchenkunst (ich nahm an einer sehr interessanten Studienfahrt nach Stralsund teil) und einige Vorlesungen für Hörer aller Fakultäten, so besonders die über \enquote{nationalökonomische Probleme der Gegenwart}. So pilgerte ich wochenlang jeden Morgen gegen 7 Uhr durch die Hunnenstraße mit dem imposanten Blick auf den gewaltigen und schönen Westturm des \enquote{Schlanken Nikels} in das Auditorium Maximum zu Rehmke, einer Bismarck ähnlichen Gestalt. Alle Studiker gingen mit Spazierstock zur Uni, den man mit der Mütze an den langen Reihen der Kleiderhaken aufhängte. Es tauchten in den Hörsälen vereinzelt Studentinnen auf -- an der ganzen Universität dürften es kaum über 20 gewesen sein, wenig attraktive, auf mich ältlich wirkende Gestalten. Ich belegte für 150.- M Kolleggebühren, die ich mir stunden ließ -- ein Betrag, der fast dem Doppelten meines Monatswechsels entsprach. Voraussetzung für die Stundung der Kolleggelder war das Bestehen mindestens einer Fleiß -- oder wie es damals hieß -- Diligenzprüfung über eine wöchentliche 4-stündige Vorlesung. Der Professor prüfte die Bewerber 1 Stunde lang im Hörsaal gegen Ende des Semesters. Bedingung war das Bestehen der Prüfung mit \enquote{sehr gut}.

Ich merkte bald, dass meine französische Vorkenntnisse besonders im Verstehen und Sprechen der Fremdsprache hinter den Leistungen der Abiturienten von Realgymnasien und Oberrealschulen zurückblieb und so beschloss ich, in den Semesterferien nach Frankreich zu gehen, angeregt von einem älteren Bunzlauer Studenten, der mit Erfolg drei Monate in Dijon gewesen war.

Der ATV besaß ein eigenes Bootshaus mit mehreren Ruderbooten, und ich lernte unter strenger Zucht auf dem Ryk rudern. Himmelfahrt machten wir zu 8 Mann in einem Fischerboot mit Segel eine Spritztour nach Rügen, wo, nach Rast im Jagdschloss Puttbus, einige Stunden flott gewandert wurde. Abends ging es bei steifer Briese heimwärts; streckenweise wurde gegen den Wind gekreuzt. Wir mussten zu zweien abwechselnd das eingedrungene Wasser auspumpen, und der Fischer und sein Gehilfe ließen die Schnapsflasche kreisen.

Die Umgegend von Greifswald mit ihren bescheidenen Reizen durchwanderte ich mit Menzel, oder ich fuhr gelegentlich auch allein nach Wiek an den Bodden. Am 1. Mai schon hatten wir uns an dem landesüblichen \enquote{Maiklopfen} im Walde von Eldenau beteiligt: man schlug z.T. singend mit Stöcken an die Bäume, um sie zum Grünen zu veranlassen. Auf dem dörflichen Tanzboden in Wieck tanzte man den entfernt mit der Quadrille zu vergleichenden pommerschen Gruppentanz \enquote{Kegel}. Er dauerte mindestens 20 Minuten. Ich tanzte ihn mit einem niedlichen Fischermädchen, das mich am Ende des Tanzes bat, sie doch -- wenn ich wollte -- beim nächsten Tanz schnell zu engagieren; ein Student stelle ihr nach, aber sie möge ihn nicht. Nach dem übernächsten Tanz drückte mir mein Rivale seine Karte in die Hand: auf ihr standen unter seinem Namen und dem Namen seiner schlagenden Verbindung die mit Bleistift geschriebenen Worte: \enquote{Mein Herr, Sie haben mich fixiert, ich ersuche Sie, sich nach dem nächsten Tanz auf die Toilette zu begeben.} Ich hielt mich jedoch an die Weisungen des Fuchsmajors, auf solche Carambolagen \underline{nicht} einzugehen.

Zu Pfingsten unternahmen wir selbfünfe eine herrliche Ostseewanderung; außer meinen Klassenkameraden Menzel und Richter nahmen auch noch zwei ATV-er, einer von ihnen mein Leibbursche in spe, daran teil. Wir fuhren mit der Bahn nach Rostock, dann mit dem Dampfer nach Warnemünde und wanderten von dort an der Ostsee entlang bis nach Stralsund, setzten dort im Fischersegelboot nach der Südspitze von Hiddensee über, durchwanderten die Insel nach Norden, setzten von der Ostspitze des Dornbusches nach Rügen über, fuhren schließlich mit der Inselkleinbahn nach Bergen und von dort ebenfalls 4. Klasse mit der Bahn über den \enquote{Streta Sund} (Fähre) nach Greifswald zurück.\\

Am Pfingstsonnabend entdeckten wir abends in den Dünen eine kleine Hütte voller Netze. Es gelang uns, das Fenster herauszunehmen, wir kletterten alle Mann hinein und pennten in und unter den Netzen. Ohne eine mögliche unliebsame Störung brachen wir vor Sonnenaufgang auf und badeten, wie jeden Tag, in dem noch frühlingsmäßig kühlen Meere.

Nach Landung in Hiddensee besuchten wir auf dem Gjellen eine Wegstunde von Neunkirchen, in seinem, heute würden wir sagen Bungalow, einen sonderbaren Heiligen, einen Seher, der sich selbst im Besitze ungewöhnlicher Seelenkräfte glaubte. Er hatte sich früher auch als \enquote{Vortragskünstler} betätigt und zeigte uns darauf bezügliche Dankschreiben u.a. von Bismarck und Bebel.

\marginpar{96}
Photos zeigten ihn in einem langen weißen bestickten Mantel mit ausgestreckten Armen \enquote{das (sehr unruhige) Meer beschwörend}!! Die Unterhaltung war lebendig, sein \enquote{Tick} wenig spürbar. Übernachtung in Vitte; der hügelige Dornbusch war herrlich; man sah am Horizont deutlich die Kreidefelsen der dän. Insel Mörn. Überfahrt nach Rügen, mit Regenschauern. Ich hatte Küchendienst: es gab Erbsen mit Spiegeleiern. Die Erbsen lagen uns noch stundenlang im Magen, da sie der Schauer wegen nur 40 Min. gekocht hatten und daher nicht gar waren. Auch die Spiegeleier waren nicht wie bei Muttern geraten, sondern zäh und ledern; ich hatte kein Fett in den Tiegel getan. Wir übernachteten mit Genehmigung des Wärters in einer Scheune neben dem Leuchtturm auf Kap Arkona, gruben uns ins Heu ein, standen aber alle schon vor 4 Uhr auf: in der zugigen Scheune vermochte uns das Heu nicht genügend vor der Kälte zu schützen. Menzel und ich -- die anderen waren schon zeitig zur nächsten Station der Inselbahn aufgebrochen, um nach Greifswald zu fahren -- wanderten noch einen Tag und stellten bei unfreundlichem Wetter fest, dass unser Geld gerade noch ausreichte, um 4. Klasse nach Bergen, der winzigen Hauptstadt Rügens, zu fahren. Dort berieten wir, wie wir völlig mittellos aus der Klemme kommen könnten. Menzel meinte, wir sollten uns als obdach- und mittellos in Polizeigewahrsam begeben; mir schien dieser Vorschlag in seiner Auswirkung problematisch und ich ging mit ihm zum Uhrmacher des Ortes. Nach Prüfung meiner Uhr und unserer Studentenkarte erklärte er mir, er könne gegen die Taschenuhr als Pfand 2 Mark geben, und er werde mir die Uhr nach Eingang der 2 Mark per Nachnahme schicken. Das genügte uns, wir wussten schon, dass die Fahrkarte Bergen-Altefähr-Stralsund-Greifswald 4. Klasse 90 Pf. kostete. Da blieben uns also noch 20 Pf., für die jeder von uns vier oder fünf Semmeln für unseren hörbar knurrenden Magen kaufen konnte.

Alles ging planmäßig, die Uhr langte zwei Tage später in Greifswald an.

Bald fühlte ich mich im ATV nicht mehr wohl; nicht nur die Beschränkung meiner persönlichen Freiheit war es -- man sah es ungern, dass ich Verkehrsgast im Akademisch-literarischen Bund war und manche der kulturellen Darbietungen Greifswalds wahrnehmen wollte -- auch der Geist der Vereinsbrüder entsprach wenig meinen freischärlichen Neigungen. Ich zog dem ATV-Convent mit anschließender Kneiperei einen Abend mit Kothe, dem Lautensänger alter Volkslieder meist im Zupfgeigenhansel vor, was mir nur ungern gestattet wurde.

So trat ich denn aus -- zum Leidwesen meines \enquote{Leibburschen}\\

Anfang Juli war ein 6 Tage dauerndes Volksfest in Greifswald, von den Studenten \enquote{Schwedenulk} genannt, eine Erinnerung an die Befreiung durch die Schweden oder von den Schweden während bzw. nach dem 30-jährigen Krieg. Im abendlichen Rummel erteilte mir eine robuste Bänkelsängerin einen Denkzettel. Sie stand auf einer Bretterestrade, neben sich die Drehorgel, dahinter eine riesige Leinwand im Format etwa 3x5 m, auf der sich im Carrés 9 große bunte Bilder aufregenden Inhalts befanden: die Darstellung einer tragédie larmoyante, die Mutter von 7 Kindern, deren auf jedem Bilde eines Hungers starb. Die Frau sang nicht sehr klangvoll, dafür sehr laut, während sie mit der Linken die Orgel drehte und in der Rechten einen über 2~m langen Rohrstock hielt und mit ihm auf das jeweils aktuelle Bild zeigte. Ich stand im 4. Glied der dicht gedrängten Zuhörermasse; drei Söhne waren bereits gestorben; als sie nun mit heroisch klagender Stimme zum 4. Kinde überging, konnte ich mich des Lachens nicht enthalten: Sofort sauste auf meinen Kopf der Rohrstock nieder mit den Worten: \enquote{Jrins nich, jrüner Junge, du hast keen Varständnis for dat trarische Drama.}
Ich habe später nur noch einmal, im Sommer auf dem Blücherfest in Löwenberg, Niederschlesien (zur Hundertjahrfeier des Sieges an der Katzbach 1913) eine echte Bänkelsängerin erlebt.

Nachdem es mir gelungen war, meinen Vater von der Notwendigkeit eines Frankreichaufenthaltes zu überzeugen -- es bedeutete ja für ihn erhebliche zusätzliche Kosten -- verlagerte ich meine Arbeit stärker auf Französisch.

Mein Onkel 2. Grades Horst von Rabenau, der mit seiner Frau und seinem Sohn Eberhard in Göhren/Rügen Badeferien machte, lud mich eines Julisonntages ein. Ich fuhr mit dem Dampfer hinüber. Der Onkel holte mich auf der Landungsbrücke ab, führte mich zu einer schönen Villa mit Garten und herrlichem Seeblick. Nach kurzer Unterhaltung mit der nicht gerade schönen, humorlosen und wortkargen Tante erklärte der Onkel, der bei schnellem Sprechen leicht anstieß: ich geh jetzt zum F-Frühschoppen, Fritz, u-unterhalte die Tante! Das fiel mir gesellschaftlich ungewandtem Jüngling nicht leicht; ich suchte nach Gesprächsthemen und Einfällen, aber meine unerfreuliche Gesprächspartnerin ging auf nichts ein, kaum dass sie ein paar einsilbige Worte äußerte.

Es war für mich eine Erlösung, als man ins Nebenzimmer zum Mittagessen aufbrach, zu dem sich auch der Onkel in bester Stimmung wieder einfand. Nachmittags fuhren wir mit einem Segelboot nach Binz, dem damaligen Modebad der Insel. An einigen der eleganten Hotels bemerkte ich Schildchen mit Aufschriften wie \enquote{Semitischer Besuch unerwünscht} oder auch \enquote{höflich verbeten}. Man begleitete mich auf die große Landungsbrücke, von der das Dampferchen mich nach Greifswald zurückbrachte.

Genau zwei Sonntage später stürzte die Brücke ein und begrub zwanzig Menschen unter sich -- ich las es in Dijon in der noch heute erscheinenden Tageszeitung \enquote{Bien public}.


\section{Dijon}

Am 19. Juli 1912 verließ ich ohne den Semesterschluss abzuwarten, mit dem Frühschnellzug Greifswald in Richtung Berlin und stieg dort auf den Anhalter Bahnhof um. Ich erlebte es, wie im Wartesaal einem am Nebentisch sitzenden Manne von einem Kriminalbeamten in Zivil mit blitzschnellem Zugriff Handschellen angelegt wurden. In Halle stieg ich nochmals um, besichtigte die Innenstadt deren architektonisches so erheblich vom Charakter der Hansestädte abweicht und fuhr im Personenzug weiter; ich wollte auf diese Weise einen besseren Eindruck von Land und Leuten gewinnen. Das Kommende schien für mich fast eine Traumfahrt zu sein. In Bebra hatte ich zwei Stunden Aufenthalt. Ich begab mich zu Fuß in ein etwa 3~km entferntes hessisches Dorf, \enquote{genoss} die mir neuen Fachwerkbauten, erfreute mich in der Dorfschenke an der Mundart und fuhr mit dem Schnellzug weiter über Frankfurt nach Straßburg; in der Nacht wurde ich mehrfach, während ich döste, von meinem sich schlafend stellenden Nachbarn, einem gepflegten, ca. 45-jährigen Herrn, handgreiflich belästigt. Um 4 Uhr morgens stieg ich in Straßburg aus und stellte erstaunt fest, dass hier im Gegensatz zu Greifswald, wo die Sonne um 3:20 Uhr aufging, kaum die Morgendämmerung zu spüren war.

Das Straßburger Münster war mir dank der Jugendarbeit Goethes, Kunstbüchern, Photos und Lichtbildern einigermaßen vertraut. Doch wie mächtig war der Eindruck des Originals: die gewaltige schöne Fassade, wie differenziert die einzelnen Bauabschnitte im Gegensatz zu der gewiss großartigen Einförmigkeit des Kölner Domes; die herrlichen Kirchenfenster und Skulpturen -- all das fesselte mich lange Stunden. Natürlich war ich auf der Terrasse, von der der junge Goethe der untergehenden Sonne zugetrunken hatte, ich fand die eingemeißelten Namen Goethes und Voltaires, des wohl größten Deutschen und des größten oder \enquote{französischsten} Franzosen. Manche meinten damals: \enquote{Wie sinnig, dass der Blitz vom Namen des großen Spötters das \enquote{Vol} abgeschlagen hat -- er ist nur -- taire, \enquote{ver}schweigen übriggeblieben.}

Abends war ich in einem deutschen Variété; unter gemischtsprachigem Publikum waren sich sehr selbstbewusst und arrogant gebärdende Studenten deutscher schlagender Verbindungen. Als eine Pariser Sängerin besonderen Applaus erntete, schienen die deutschen Studenten offensichtlich zu Anrempeleien mit den französisch sprechenden Besuchern geneigt.

\marginpar{103}
Am nächsten Morgen brachte mich der Eilzug über Belfort und Besançon nach Dijon. Die Landesgrenze verlief damals, glaube ich, eine Station hinter Altmühl. Die Grenzkontrolle ging sehr rasch vonstatten: ich zeigte meinen Studentenausweis und der Beamte sagte: \enquote{étudiant? -- Passez!}, indem er auf den Koffer, ohne ihn öffnen zu lassen, mit Kreide ein Zeichen machte.

Der Eintritt in das fremde Land war sinnfällig: im Abteil wurden die deutschen Laute durch französische mit anderem hellerem timbre verdrängt; das Sprechtempo und Gesten der Reisenden wurden lebhafter; draußen ging es etwas lauter zu: während der Zug auf einer Station hielt, schrillte eine Glocke; man vernahm während der Fahrt grelle Lokomotivenpfiffe, die Stationsschilder waren hellblau auf weißem Felde. Auf dem Gange sprach mich ein etwa 22-jähriger Franzose deutsch an -- er kehrte von einem mehrmonatigen landwirtschaftlichen Studienaufenthalt in Stuttgart zurück. Einladung.

Eine Pferdedroschke -- Autotaxis gab es noch nicht wie überhaupt das Auto damals im Straßenverkehr noch eine sehr seltene Erscheinung war -- brachte mich zu meiner \enquote{Pension bourgeoise} in der rue Chandronnerie, wo ich angemeldet war. Die Verständigung mit dem Droschkenkutscher misslang infolge meiner mangelhaften Gehörschulung; erst später begriff ich, dass ich ihm ungewollt ein höchst generöses Trinkgeld gegeben hatte, das er mit einer überaus höflichen Verbeugung quittiert hatte.

Der 13-jährige Sohn der Wirtin, in langer schwarzer am Hals locker geschlossener Schürze, in der er auch die Tischgäste bediente, öffnete mir; die Hausfrau begrüßte mich und führte mich in mein kleines über dem Essraum gelegenes Zimmer. Unten waren die \enquote{habitués} schon beim Käse angelangt, alles Studenten, einige Franzosen, sonst Österreicher, Ungarn, Deutsch-Böhmen, zwei Japaner. Der französischen Unterhaltung vermochte ich nur mit Mühe zu folgen, und ich machte zunächst bei Tisch keine Sprechversuche.

\marginpar{105}
Das Essen war für deutsche Vorstellungen vorzüglich, mittags und abends je 1/2 l roter (auf Wunsch auch weißer) Burgunder im Pensionspreis inbegriffen. Man war etwa 1,5 Stunden bei lebhafter Unterhaltung zu Tisch. Bis in die 20er Jahre hinein war, mindestens in den Weingegenden Frankreichs, die Nichtgewährung von 1/2 l Wein zu den Mahlzeiten für Hausangestellte ein legitimer Kündigungsgrund. Auch in Bistrots und kleinen Restaurants war der vin de table im Preis einbegriffen, und der Kellner fragte nur: rouge ou blanc? Der gute Wein wirkte anregend, allerdings stieg er mir besonders mittags leicht zu Kopf, so dass ich das Bedürfnis zu einer Siesta empfand. Es erscheint mir möglich, dass die Lebhaftigkeit des Franzosen, mindestens zum Teil, auf den Weingenuss zurückzuführen ist.

Dijon, die alte Kulturstadt, die als Residenz der mächtigen Herzöge von Burgund im 15. Jhd sogar die Chance hatte, zur Hauptstadt Frankreichs aufzusteigen, büßte nach dem Tode Karls des Kühnen (1488) allmählich von seinem Glanz und seinem Einfluss ein, entlockte aber noch Franz I. auf seiner Reise nach Italien auf einem Hügel vor Dijon den Ausruf: Welch herrliches Stadtbild! Es ist nächst Paris die schönste Stadt meines Landes!

Dijon war und blieb bis nach dem 1. Weltkrieg eine bedeutende Festung, sie war im 19. Jhd. und während des 1. Weltkrieges der südliche Eckpfeiler des Festungsgürtels und somit eine Garnisonsstadt von Rang.

Dijon kann sich aber noch heute rühmen, mehr als eine der durchschnittlichen französischen Provinzhauptstädte zu sein, wenn sie auch nach dem 1. Weltkrieg ihre Kunstakademie verloren hat und einige der schönsten und wertvollsten Gemälde seines sehr reichen Kunstmuseums nach Paris gewandert sind (so ein Bildnis von Rubens und mehrere anderer schöner flämischer Meisterwerke.) Der Rang der Universität, von deren Disziplinen sich einige früher in Abhängigkeit von Lyon befanden, ist gestiegen. Dijon ist ein geistiges Zentrum. Die Dijoner Akademie hatte in Europa in der Mitte des 18. Jhd. von sich reden gemacht, als sie J. J. Rousseau den Preis für seine epochemachende Arbeit über den negativen Wert von Wissenschaft und Kunst erteilte.

Dass Flandern einst zu Burgund gehörte, machte sich noch heute in seinen Kunstdenkmälern bemerkbar, nicht zuletzt in Mosesbrunnen und den prächtigen Grabdenkmälern seiner letzten Herzöge.

Einige von den zahlreichen bemerkenswerten Baudenkmälern sind nicht nur schön sondern einmalig, so besonders die Kirchen Notre Dame (besonders die spezifisch burgundische Fassade) und St. Michel mit seiner schönen Renaissancefassade während der übrige Körper gotisch ist.

Mit dem Ende des regulären Sommersemesters verschwanden die französischen Studenten aus der Pension, an ihre Stelle traten fremde Studenten, besonders Deutsche, die zum Ferienkursus für ausländische Studenten erschienen.

Inzwischen hatte ich mich schon etwas in Dijon umgesehen, die Wochenmärkte, besonders die Trödler, Antiquitätenverkäufer auf der malerischen rue Musette -- in der sich auch echte Dudelsackpfeifer, joueurs de cornemuse, herumtrieben -- besucht und mich hier und auch in den Warenhäusern im Hören und Sprechen geübt. Ein besonderer Reiz waren für mich die Straßensänger, die damals noch, wie in René Claires Film \enquote{Sous les toits de Paris} aus dem Jahr 1930, keine seltene Erscheinung waren. Ich stand öfters unter der lauschenden und bald mitsingenden Menge, den 2-Sous Text mit Noten in der Hand haltend, und lernte so manchers \enquote{Chanson des rues}. Noch gab es auf den Straßen fahrende, mit gesangesähnlichen Lautgebilden werbende \enquote{Marchands des quatre saisons} und auch die netten alten, auf Plätzen und Straßen singenden Frauen \enquote{Du mouron [vogelmiere] pour les petits oiseaux}.

Mir gefiel die Atmosphäre dieser Stadt, die leichte, angenehme Art zu leben, die sociabilité der Franzosen, die positiven Auswirkungen der franz. Revolution wie das gänzliche Fehlen des Standesdünkels, dass jede Frau, gleich ob Reinemachefrau oder Präsidentin mit \enquote{Madame} angeredet wurde.

\marginpar{108}
Die Ferienkurse wurden in enger kameradschaftlicher Zusammenarbeit vom Universitätsprofessor Lambert und einem Volksschullehrer Martenot geleitet und vorwiegend bestritten -- so etwas wäre im kaiserlichen Deutschland nicht denkbar gewesen, fühlte sich doch schon der Gymnasialoberlehrer wissenschaftlich und sozial als weit über seinem Kollegen der Volksschule stehend.

Das Ausland und die deutsche Sozialdemokratie prangerten den angeblichen Militarismus des kaiserlichen Deutschlands an. Wie mir nun in Frankreich schien nicht zu Unrecht: Die Offiziere, in Deutschland \enquote{der vornehmste Stand}, waren es in Frankreich nicht. Ohne Monokel und arrogantes Auftreten besuchten sie -- im Gegensatz zu Deutschland -- in Uniform gut bürgerliche Cafés -- was in Frankreich allerdings die Angehörigen des dort als besonders vornehm geltenden Standes, die Richter -- die noblesse de robe -- nicht taten.

Das in Deutschland und Österreich herrschende, geräuschvolle studentische Verbindungsleben mit seinen Kommerzen, Mensuren, seinen Farben und seiner doppelten Moral war in Frankreich unbekannt. Es gab in Dijon zwei lose \enquote{associations des étudiants}, eine mehr konservativ-kirchliche und eine andere liberale \enquote{Association générale des étudiants}, die auch ausländische Studenten gegen eine geringe Eintrittsgebühr aufnahmen. Ihr trat ich mit einem anderen Ausländer bei. Man versammelte sich in einem großen Raum des Café de Paris, bestellte sich eine Citron nature oder \enquote{un boc}. Man beriet über den nächsten Sonntagsausflug; ein Franzose fragte, ob er seine \enquote{petite amie} mitbringen dürfte, was bejaht wurde. (Eine \enquote{petite amie} war für die damalige Zeit in Frankreich selbstverständlich eine Nichtstudentin)

Plötzlich stand irgendein französischer Student auf und hielt uns Fremden eine gewandte Begrüßungsansprache. Ich hatte das Gefühl, hier mit ein paar Worten des Dankes antworten zu müssen, überlegte mir ein paar Sätze, wurde aber derart aufgeregt, dass ich mich nicht zu einer Antwort zu erheben wagte. Es zeigte sich, dass jeder französische Student auch im größeren Kreise völlig frei und ungehemmt reden konnte, ganz im Gegensatz zu uns deutschen Studenten, sofern wir nicht eine entsprechende hemmungslösende Schule meist in einer studentischen Verbindung hinter uns hatten. Ich nahm mir fest vor, diesen Mangel in meiner Persönlichkeitsbildung -- so empfand ich mein Versagen -- in der Heimat zu bekämpfen.

Herr Knauer aus Prag, 8. Semester und deutscher Burschenschaftler mit mehreren Schmissen, hatte ein amoureuses Verhältnis mit einer zarten, hübschen blonden russischen Studentin, offenbar aus gutem Hause, die sehr in ihn verliebt war. Er erklärte mir, dass eine Ehe natürlich aus nationalen Gründen mit einer Russin nicht in Frage komme. Übrigens waren noch mehrere russische Studentinnen im Kursus, die bei den frei zu wählenden Aufsatzthemen fast ausschließlich Fragen des Klassenkampfes, der Revolution und der Frauenemanzipation wählten.

Eines Tages stellte mich Herr Knauer zur Rede: \enquote{Ich habe Sie gestern auf der rue de la Liberté (der Hauptstraße Dijons) mit einem tschechischen Prager Studenten spazieren gehen sehen. Wie können Sie als Deutscher so etwas tun? Haben Sie denn kein deutsches Nationalbewusstsein? Sie sollten sich schämen und das künftig unterlassen. Ich bin in Prag geboren, zur Schule gegangen und habe auch an der Karlsuniversität studiert, aber Worte mit einem Tschechen wechseln -- so etwas gibt es für mich nicht. Ich kann, Gottlob, von mir sagen, dass ich kein einziges tschechisches Wort kenne oder gar in den Mund genommen habe.} Ich setzte ihm auseinander, dass ich solch engstirnigen Nationalismus ablehne, dass für mich auch auf politischem Gebiet der römische Grundsatz gelte: Audiatur et alteram pars. Wir standen künftig miteinander nur noch im Grußverhältnis.

Der 2. September brachte ein besonderes Erlebnis: eine deutsche Sedanfeier in Frankreich! Vor den Toren der Stadt war unter einem Berg von 300 gefallenen Preußen auch die einzige preußische Regimentsfahne im Kriege 70/71 verloren gegangen. 1912 existierte noch das große umzäunte und gepflegte Massengrab. Wir deutschen Studenten, wohl 60 an der Zahl, erhielten die polizeiliche Genehmigung zu einem geschlossenen Zuge von der Place d'Arcy durch die Stadt zu einer Gedenkfeier am Massengrab. In Zweierreihen, im Knopfloch ein schwarz-weiß-rotes Bändchen, gefolgt von kleinen Gruppen österreichischer und italienischer Studenten, marschierten wir zur Gedenkstätte. Der Kranzniederlegung folgte die gute Gedenkrede eines älteren Studenten. Während der Feier tummelte in taktvoller Entfernung von etwa 300~m ein berittener Polizist sein Pferd.

Und zurück ging es zur Stadt mit der Parole: Auf zur Dreibundkneipe in Gevrey-Chambertin, dem berühmten Weindorf an der \enquote{Route des grands vins}. Der Ort, 14~km von Dijon entfernt, war mit der Stadt durch eine am Rande der Côte d'Or entlang geführten Straßenbahn verbunden. Wir stiegen am Coin du miroir ein; unterwegs -- es war sehr schwül -- gerieten wir unter ein aufziehendes Gewitter, und schon fielen die ersten Hagelkörner. Da setzte plötzlich eine wahre Kanonade ein: aus Dutzenden von in den Weinbergen verteilten mörserartigen, senkrecht stehenden Schlünden zuckten Blitze und Rauchwolken unter donnerartigem Krachen. -- Es war das damals in besonders berühmten Weingegenden betriebene Hagelschießen. Wir waren auf den Erfolg gespannt -- die Straßenbahn wartete auf den Gegenwagen und tatsächlich: der kaum begonnene Hagelfall hörte auf -- einige klatschten Beifall. Es war also anscheinend gelungen, das Hagel-Unwetter, wenn zwar auch nicht aufzulösen, so doch zu den weniger bemittelten Nachbarn abzuschieben.

In Gevrey-Chambertin saßen wir mit einem halben Dutzend Österreichern und einigen Italienern und Italienerinnen, tranken rouge, plauderten und sangen nach Nationalität abwechselnd Volkslieder der Heimat -- es war nett.
Am nächsten Tage brachte die noch heute, 1973, existierende Tageszeitung (konservativ-kirchlich) Le Bien Public die Notiz: die deutschen Studenten haben das Beispiel einer würdigen und taktvollen patriotischen Gedenkfeier gegeben -- eineinhalb Jahre vor Ausbruch des 1. Weltkrieges.

Hauptsächlich um mein Gehör zu schulen besuchte ich an einigen Sonntagen den Gottesdienst im Temple Protestant, der kleinen, im Inneren völlig kahlen, nur mit einigen in die Steine gemeißelten Bibelsprüchen geschmückten Kirche calvinistischer Prägung. Der Pastor liebte es, die Gläubigen mit \enquote{mes pauvres pécheurs} (meine armen Sünder) anzureden.

Im Ferienkurs machte ich die Bekanntschaft eines 23-jährigen Persers, der schon in Lausanne studiert hatte und sehr gut französisch sprach. Er wie etwa 10 andere Perser gehörten der obersten persischen Gesellschaftsschicht an. Er war der Sohn eines reichen und unvorstellbar luxuriös lebenden Khans, während er selbst, Philanthrop mit einem starken sozialen Gewissen, betont einfach mit relativ bescheidenen Mitteln lebte. Er hatte literarische Neigungen und entwickelte mir seine Pläne eines orientalischen Dramas. Mit ihm -- Habibolah Khan Cheab Zadek war sein Name -- verband mich bald eine Freundschaft etwas romantischen Gepräges. Habibolas Bruder studierte in Berlin und arbeitete bereits an einer Faustübersetzung, während ein jüngerer Bruder in St. Cyr zum französischen Leutnant aufgerückt war; ihn lernte ich später in Paris kennen. Habibolas große warme braune Augen bezauberten manch junge Französin. Leider war er, wie sich später im Winter herausstellte, bisexuell veranlagt. Da ich seine anomale Neigung nicht erwidern konnte, wurde ich ihm gegenüber allmählich zurückhaltender. Wir haben noch bis zum 1. Weltkrieg in brieflichem Kontakt gestanden.

Die Universität Dijon hatte im Winter 1912/13 ca. 350 Studenten -- heute (1973) annähernd das Zehnfache.

Zu den interessanten Vorlesungen, die ich im Wintersemester hörte, zählte vor allem die von Professor Eisenmann: Das deutsch-französische Verhältnis von 1871 bis 1912. Mir gefiel unter anderem die gründliche Abwägung des französisch-russischen Bündnisses vom französischen Standpunkt. Den Wert des Bündnisses beurteilte er für den Ernstfall skeptisch, vom Unterschied der Regime Russlands und Frankreichs abgesehen, hätten Deutschland und Russland eines, das sie beide verbinde und von einer kriegerischen Konfrontation abhalten werde: die polnische Frage. Beide betrieben -- anders als Österreich -- dem annektierten Polen gegenüber die gleiche Politik: Germanisierung bzw. Russifizierung. -- Leider kam es eineinhalb Jahre später anders!

\marginpar{117}
Der Philosophieprofessor Rey, ein reiner Relativist, sagte seinen Studenten zu Beginn seiner Vorlesung: wer ernsthaft Philosophie studieren wolle, müsse zu diesem Zweck unbedingt Deutsch lernen. Übrigens behandelte er, wie auch die anderen Professoren, die Studenten wie es mit Schülern der Oberstufe der Gymnasien geschah: Monsieur X., wenn Sie weiterhin meine Vorlesung so unregelmäßig besuchen, werde ich Ihrem Vater schriftlich berichten; ähnliches war natürlich besonders in Übungen zu hören.

Literaturprofessor Lambert, der persönlich die ausländischen Studenten bei der Bücherentleihung beriet, warnte mich vor der Lektüre Verlaines: das sei kein Dichter französischen sondern fremden Geistes. Grob sagte man damals: \enquote{c'est écrivain est emboché.}

Im Winter las oder auch schrieb ich französisch oft in dem angenehmen Arbeitssaal der Stadtbibliothek, die -- wie auch die Seminarbibliotheken -- bis 10 Uhr abends geöffnet war. Der etwa 35-jährige, stets hilfsbereite Bibliotheksbeamte Regnandin, Junggeselle, verbrachte abends mit mir -- zuweilen war noch ein anderer Deutscher dabei -- so manche Stunde auf der Rue de la Liberté oder auch gelegentlich im Café \enquote{Lion de Belfort} bei einer Tasse Kaffee oder einem boc. Ich nutzte gern die Gelegenheit zu längerem Französischsprechen mit diesem reifen Menschen wie auch mit dem älteren (8. Semester) bayrischen Studenten Bachhuber, mit dem ich ebenfalls viel zusammen war, auch in seiner Wohnung. Zusammen mit einem anderen, schon vor der Prüfung stehenden Bayern, wohnte er bei Madame Beaugey zur Untermiete. Es wurde geplaudert; zuweilen las Frau Beaugey mit Schwung und Pathos vor.

In diesem Kreis feierte ich auch Sylvester; wenn das neue Jahr eingeläutet wird, umarmen sich die Franzosen und tauschen bises aus. Dies taten wir mit Ausnahme des ebenfalls bei Frau Beaugey wohnenden Engländers Wicliff. Um der Umarmung zu entgehen, flüchtete er unter den Tisch, und als Frau Beaugey ihn hervorziehen wollte, drohte er mit dem Taschenmesser. \enquote{Monsieur Georges}, fragte sie, \enquote{vous n'aimez donc pas les Français?} \enquote{J'aime les Français, mais pas les Françaises}, war seine Antwort in sehr starkem, von allen in seiner Abwesenheit viel belachten englischen Akzent.

Ein Gespräch, das ich mit ihm über Fragen des christlichen Glaubens hatte, beschloss er mit den Worten: We (the English) say: \enquote{Heaven for music, hell for company.}

Er war ein typisch englischer Student der damaligen Zeit, Individualist, Monokelträger, der mir seine nächsten Jahre folgendermaßen beschrieb: ich bleibe in Frankreich noch ein Semester, dann gehe ich zwei Semester nach Heidelberg (für England damals die deutsche Modeuniversität) dann gehe ich zwei Jahre nach Indien, und dann arbeite ich in einer englisch-indischen Gesellschaft, wo mein Vater Aktionär ist.

Er speiste nicht in französischen Lokalen oder Pensionen sondern in einem spanischen Café -- um nebenbei Spanisch zu lernen, wie er sagte.

Übrigens kannte man damals den Weihnachtsbaum noch nicht; nur die aus ihrer Heimat wegen der deutschen Annexion nach Dijon geflüchteten ca. \num{1000} Elsässer feierten Weihnachten gemeinsam unter einem großen Tannenbaum. Inzwischen ist der erstmalig in Straßburg im Jahr 1605 belegte Brauch in Frankreich weit verbreitet.

Vor Weihnachten hatte sich Greifswald in Erinnerung gebracht: Vom Universitätsrichter traf ein Einschreiben am 12.12. ein, ich solle mich am 11.12. um 11 Uhr vor ihm verantworten, weshalb ich keine einzige der im 1. Semester belegten Vorlesungen abtestiert und im 2. Semester keine Vorlesung belegt habe. Im Falle meines Nichterscheinens habe ich der folgenden Disziplinarstrafen zu gewärtigen etc. Man kümmerte sich noch um den einzelnen Studenten.

Der Konflikt wurde schriftlich beigelegt: ich erhielt sogar alle von mir belegten großenteils aber nicht von mir besuchten Vorlesungen abtestiert.

Weihnachten fuhr ich für einige Tage nach Lyon, natürlich mit der Bahn. Auf der Hinfahrt machte ich für einige Stunden Halt in Chalôns-sur-Saone und in Cluny, das aus seiner mittelalterlichen Glanzzeit wenigstens einige Bauten in die Gegenwart gerettet hat. Hier aß ich auch zu Mittag. Der treffliche Wein machte mich so müde, dass ich mich bei annähernd 10° am Rande des Städtchens auf eine grasbedeckte Böschung hinlegte. Nach einer Stunde wachte ich auf und setzte die Besichtigung fort.

Lyon, die zweite Stadt Frankreichs mit über einer halben Million Einwohnern lockte mich als stolze Großstadt mit einer immerhin \num{2000}-jährigen Geschichte, seiner schönen Lage zwischen Bergen, an der \enquote{mariage} zwei bedeutender Flüsse. Da werde man sicherlich mit dem Dampfer nach Marseille und damit ans Mittelmeer fahren können, dachte ich. Schon am Abend meines Eintreffens durchschlenderte ich neugierig die Stadt von Bahnhof Perrache bis zum Herz der Stadt, dem Place des Terreaux, mit dem stolzen Rathaus, mit Opernhaus und Musée des Beaux Arts. Es war feuchtes, leicht dunstiges Wetter, hier und da saßen noch Menschen auf dem Bürgersteig vor Cafés; aus einzelnen von ihnen drang weiblicher Gesang -- es gab damals, lange vor Rundfunk und Fernsehen, noch viele Cafés chantants. Eine Lichtreklame fand ich toll: über die Fassade eines 5-stöckigen Gebäudes zuckte, die Dunkelheit jäh zerreißend, ein bläulicher Blitz -- und nach weiteren zwei Sekunden Dunkelheit erschien in großer Leuchtschrift \enquote{Royal Hôtel}.

Ich ließ mich in einem kleinen Hotel in der Nähe des Rathauses nieder; die Tapeten waren damals wie in Deutschland dunkel, statt der heutigen Waschvorrichtungen standen -- ganz wie in Deutschland -- Schüssel und Krug mit Wasser auf dem Waschtisch mit der Marmorplatte. Das Zimmer kostete 2 Fr; der Wirt versicherte mir auf meine Frage mit kaum merklichen Lächeln, dass ein Handtuch im Preis inbegriffen sei.

Bei der Besichtigung der Primatenkathedrale St. Jean führte mich ein Abbé. Schließlich bedankte ich mich bei ihm für die Führung und er fragte mich: \enquote{D'ou êtes-vous mon fils?} \enquote{Je suis de Dijon, mon père}, war die Antwort des Frankreich- und Franzosenfreundlichen Jünglings.

Die römischen Ausgrabungen auf dem Hügel der Notre Dame de Fourvière, auf dem damals noch ein Fort der Festung Lyon stand, waren längst nicht so weit gediehen wie heute.

Ich genoss weniger die künstlerisch fragwürdige Kirche in neobyzantinischen Stil als den herrlichen Blick auf Lyon. Im Musée des Beaux Arts, einem schönen ehemaligen Benediktiner Kloster, interessierten mich besonders die Fresken oder freskenartigen Bilder des Puvis de Chavannes, von dem sich auch monumentale Fresken im Pariser Pantheon befinden.

Gern verweilte ich bei den vielen Buchläden unter den Colonnaden des Stadttheaters und schaute von einer der großen Rhônebrücken dem Spiel der Möwen über den grauen temperamentvollen Wogen zu.

Meine aus den folgenden ersichtliche geographische Naivität möchte ich meinem völlig unzureichenden, nur bis Tertia durchgeführten Erdkundeunterricht auflasten: Ich begab mich zum Zusammenfluss von Saône und Rhône, das Büro des Dampferverkehrs Lyon-Marseille suchend -- es existierte nicht. Die Rhône hatte zwar im Hochmittelalter Zehntausende von Pilgern aus ganz Europa auf Kähnen und Flößen nach Süden transportiert -- sie zogen dann weiter nach dem berühmten Wallfahrtsort Santiago de Compostella -- aber in der Zeit des ersten Weltkrieges war die Rhône ein urwüchsiger Fluss mit zahllosen Inseln und kleinen Buchten, noch nicht von der modernen Technik gebändigt. Ich musste also auf die erträumte Dampferfahrt gen Süden verzichten.

\marginpar{125}
Warum ich mich nicht in einem \enquote{Syndicat d'Initiative} erkundigt habe? Die gab es damals noch nicht, ich weiß nicht einmal, ob Cook damals schon eine Vertretung in Lyon besaß\dots Nicht entdeckt habe ich damals das alte Lyon, nicht die canuts, auch nicht den schönen \enquote{Parc de la Tête d'or}.

Inzwischen war der 1. Balkankrieg ausgebrochen. Ein österreichischer monokeltragender Student und Baron äußerte sich sehr besorgt um die weitere politische Entwicklung; er, der offenbar einer Diplomatenfamilie angehörte, erhielt eines Tages ein Telegramm, in dem er zu sofortiger Rückkehr aufgefordert wurde. In Dijon zogen gelegentlich Gruppen Jugendlicher unter Absingen nationalistischer und revanchistischer Lieder umher. Die Franzosen glaubten damals auf dem Gebiet der Luftfahrt einen Vorsprung zu haben; die Nationalisten versprachen sich viel -- in Liedern schlug sich diese Hoffnung nieder -- von den \enquote{Oiseaux de France}.

Eines der vielgesungenen nationalistischen Lieder schloss mit den Worten: \enquote{Et vous (d. i. die Deutschen) n'aurez plus l'Alsace et la Lorraine}.

Im Dijoner Stadttheater wurde eine Dramatisierung des Elsässer-Romanes \enquote{Les Obelés} [obelés?] gespielt. Als auf der Bühne ein deutscher Unteroffizier in Uniform auftrat, der noch dazu sich Elsässern gegenüber grob und plump benahm, schrie der ganze Olymp: \enquote{A l'Ouche, à l'Ouche}, d.h. schmeißt ihn ins Wasser -- Ouche ist der Name eines Nebenflusses der Saône, an dem Dijon liegt. Es wetterleuchtete also damals schon am politischen Himmel. Übrigens beteiligten sich in Dijon Studenten anscheinend nicht an diesen Demonstrationen.

Ich machte einen Monôme der \enquote{Association Générale des Etudiants} mit, d.h. einen etwas geräuschvollen Umzug in der Stadt mit, wobei ausschließlich unpolitische Lieder gesungen wurden, darunter: Ils sont dans les vignes, les moineaux. Ils ont bouffé les raisins, ils ont chié les pépins. Si cette chanson vous emmerde, merde, merde\dots nous allons la recommencer. Histoire de vous emmerder.

In einem Punkt unterschieden sich die Anschauungen über die Lebensführung der studentischen männlichen Jugend in Frankreich und Deutschland: während man dort über sexuelle Dinge sehr nüchtern dachte und offen über die \enquote{petite amie} oder sogar den Bordellbesuch sprach -- was in Deutschland im allgemeinen anstößig und in der Gesellschaft tabu war -- wurde in Frankreich der seltene Fall, dass ein Student trunken nach Hause kam, in der Öffentlichkeit und auch in der Studentenschaft scharf verurteilt -- während es in Deutschland geradezu besungen wurde: \enquote{Wer noch nicht besoffen war, der ist kein Student\dots} hieß es im Kommersbuch. In Frankreich fand man das eine sehr natürlich, das andere dagegen fast als widernatürlich, was zu einer deutlichen Minderung des gesellschaftlichen Ansehens führte.

Mit dem Februar ging schließlich auch das Wintersemester in Dijon zu Ende. Mein Vater erwartete meine Rückkehr und sandte mir das Reisegeld. Für mich war jedoch eine Beendigung meines Frankreichaufenthalts ohne Paris gesehen zu haben undenkbar.


\section{Umweg über Paris}

Das reichlich bemessene Reisegeld konnte ich für Paris jedoch nicht voll nutzen, da ich, vor allem beim Bücherkauf, über meine bescheidenen Verhältnisse gelebt hatte. Habibolah schuf Rat: er schrieb seinem jüngeren Bruder, dem Leutnant an der Militärakademie in St. Cyr, er solle mich zunächst auf seine Kosten 14 Tage im Hotel unterbringen und mir dann eine Stelle etwa als répétiteur, in einer Jesuitenschule verschaffen. Sein Bruder habe Verbindungen, er könne und werde das machen.

Ich war hochbeglückt: mir winkte ein längerer bezahlter Aufenthalt in der glanzvollen Metropole Frankreichs, dem geistigen und künstlerischen Zentrum wie es kein anderes Land aufzuweisen hatte. Ich würde so, ohne meinem Vater auf der Tasche zu liegen, meine französischen Studien an der denkbar ergiebigsten Quelle fortsetzen können.

Habibolah und Bachhuber begleiteten mich nach Mitternacht zum Abschied auf den Bahnhof.

Der Pariser Nachtzug war stark besetzt, mein Abteil war voller Matrosen, die die ganze Nacht pausenlos und sehr lebhaft plauderten, nur wenig konnte ich verstehen, vermochte aber auch bis zur Ankunft auf der Gare de Lyon im Morgengrauen keine Minute zu schlafen. -- Ich begab mich, wie verabredet, ins Hôtel Montparnasse, wo man mich in ein Zimmer im 2. Stock führte. Hier fand ich eine Karte von Habibolahs Bruder, ich solle ihn am nächsten Tage, einem Sonntag, um 10 Uhr im Salon des Hotels erwarten.

Und nun stürmte ich los, die Stadt meiner Träume zu besichtigen. Der Fußgängerverkehr auf den Trottoirs war lebhaft, der Fahrdamm von Pferdedroschken und sie kreuzenden Fußgängern beherrscht, hin und wieder wetteiferte ein Automobil in hellen Tönen hupend mit dem Tempo der Pferdedroschken, ja zuweilen wurden diese sogar behutsam von einem Automobil überholt. Meinen Magen beruhigte ich, wie ich es von Dijon gewohnt war, en passant mit einem Croissant und einem Stück Schokolade zu einem Sou (ein Sou entsprach damals 4 Pf). Das Quartier Latin fand ich noch schöner als ich es mir vorgestellt hatte: die rue Soufflot mit dem Blick auf das Panthéon, den Jardin du Luxembourg mit dem Brunnen der Cathérine de Médicis, die Place Richelieu und dann die Ile de la Cité mit Notre Dame und dem Palais de Justice und dem Pont Neuf mit dem Reiterstandbild Henri IV! Und dann der Louvre mit den sich weit öffnenden Tuilerien und auf einer Erhebung in der Ferne der Arc de Triomphe! Ich konnte mich nicht sattsehen an diesen großartigen Bildern, die, wie mir schien, so plan- und geschmackvoll angelegt waren.

Zu Mittag kaufte ich in einer der zahlreichen Garküchen für 2 Sous eine Tüte Pommes frites und verzehrte sie auf einer Bank im herrlichen Jardin du Palais Royal. Anschließend trank ich zur Belebung eine Tasse Café noir für zwei Sous. Im gleichnamigen Theater kaufte ich mir für den Abend eine Eintrittskarte. Dann wanderte ich weiter, ging vom Arc de Triomphe über Trocadéro zur gewaltigen Anlage um Tour Eiffel und Dôme des \marginpar{132} Invalides -- alles zu Fuß. Schließlich musste ich auch die Métro besichtigen und stieg unter der Seine auf der Station Cité aus. Unversehens war es Zeit geworden, sich ins Theater zu begeben -- vorher trank ich noch am Automaten einen kräftigen Café noir. Es war ein schönes, elegant und intim wirkendes Haus des 17. Jahrhunderts. Ich hatte einen Platz im 2. Rang, konnte die Bühne gut übersehen, auch die Akustik war gut. Die erste Szene des 1. Aktes mochte gerade zu Ende sein -- da hatte ich auf einmal das Gefühl einen Augenblick eingenickt zu sein -- ich verstand den Zusammenhang nicht mehr -- und fragte flüsternd meine Nachbarin, welche Szene das sei. Sie erwiderte, mild lächelnd, der 4. Akt sei gleich zu Ende. Ich begab mich, ohne den Schluss des Stückes abzuwarten, zu Fuß ins Hotel -- schließlich benötigte ich dazu ja noch eine Dreiviertelstunde!

Pünktlich erschien am nächsten Morgen der persische St. Cyrien, eine schöne mittelgroße Erscheinung, den Shako geschmückt mit der prächtigen rot-weißen Feder des australischen Kasuarvogels, seit Napoléon III traditioneller Kopfschmuck des St. Cyriens.

\marginpar{133}
Die Unterredung währte nicht lange, ihr Ergebnis zerfetzte im Nu meine Illusionen: in elegantem, völlig akzentfreien Französisch führte er aus, er könne seinen Bruder nicht verstehen, dass er in mir so unrealistische Hoffnungen auf eine Lehr- oder Hilfslehrertätigkeit, noch dazu in einem kirchlich-katholischen Institut geweckt habe, zumal ich doch Protestant sei und ohne ein irgendwie abgeschlossenes Studium. Vor allem müsse ich doch selbst wissen, dass es zwischen Frankreich und Deutschland gewisse ernste Differenzen gebe. Eine solche Maßnahme, wie sie sein romantischer Bruder sich vorstelle, sei allenfalls unter befreundeten Nationen praktikabel, aber nicht zwischen Deutschland und Frankreich. Ich solle mir Paris ansehen, er bezahle mir für 2 Wochen mein Quartier im Hotel, und solle dann nach Deutschland zurückkehren. Voilà.

In einem Café an der nächsten Ecke bedachte ich betrübt meine Lage. Ich machte einen Kassensturz, stellte fest, dass ich etwa 10-12 mal ins Theater gehen könnte, vorausgesetzt, dass ich für meinen Lebensunterhalt täglich nicht mehr als 35-40 Pfennige ausgäbe. Und es ging: früh 2 Croissants = 10 Pf., mittags und abends je eine Tüte pommes frites = 2x 8 Pf = 16 Pf., 5 Pf. pain blanc.

Das Schlimmste war: ich musste meinem Vater beichten, dass ich das Geld zur Heimfahrt in Paris verbraucht hatte. Diese unangenehme Sache schob ich über 8 Tage vor mir her.

Inzwischen genoss ich am Tage die Stadt Paris, die selbst ein Kunstwerk ist (oder war?) und seine Kunstmuseen, und abends war ich fast regelmäßig im Theater; vor allem in den \enquote{klassischen} Theatern: Comédie française, Odéon und Gymnase.

Da ich stets den ganzen Tag Paris durchstreifte und nicht ein zweites Mal das Debakel meines ersten Theaterbesuchs in Paris erleben wollte, begab ich mich am Spätnachmittag ins Hotel, stieg die Treppe hinauf in meine Dachkammer, in die man mich schon am 2. Tag verwiesen hatte, pustete die dünne Rußschicht von meiner Bettdecke und ruhte eine Stunde, bevor ich ins Theater aufbrach. Ich sah und hörte -- auch die Gehörsübung spielte damals, wo es noch kein Radio und Fernsehen gab, immer noch eine Rolle -- klassische und moderne Stücke. Modern waren damals besonders \enquote{le théâtre d'idées}, die \enquote{Thesenstücke} wie z.B. \enquote{Les Avaries} par Brieux.

Mich beeindruckten besonders die großen Fresken, speziell die von Puvis de Chavannes im Panthéon, im Luxembourg und Louvre genoss ich Klassik (Poussin und Ingre) und die Impressionisten, vor allem Monet. Ich suchte und fand auch einige von den Originalen, deren Reproduktionen im Wechselrahmen die Wände meines Jugendzimmers in Siegersdorf geziehrt hatten oder noch schmückten, u.a. Leonardos Gioconda (Mona Lisa) und Bauernbilder von Millet.

Im Pariser Stadtbild begeisterten mich immer von neuem die Straßen mit den schönen architektonischen oder skulpturalen Blickfängen am Ende wie die Avenue de l'Opéra, die Rue Royale mit der Eglise de la Madeleine, das Reiterdenkmal Ludwigs XIV., das, auf der Mitte eines Platzes stehend, das Blickfeld mehrerer Straßen begrenzt und viele andere mehr. Beim Schlendern durch eine Reihe von Straßen hatte man ein lockendes Ziel vor Augen, das einen anzog und die wirkliche Entfernung kürzer erscheinen ließ. Besonders gern durchstreifte ich das mir überaus sympathische Quartier Latin, die Cité, die Halles und Montmartre, machte eine kurze Dampferfahrt mit einer der \enquote{mouches} und besuchte auch Versailles und das Bois de Boulogne.

Als sich der Inhalt meines Geldbeutels beängstigend dem Nullpunkt näherte, schlich ich eines Nachmittags wie ein verliebter Kater um das Gebäude des Deutschen Generalkonsulats, den Rettungsring des Ertrinkenden: von dort werde man mich als Muster ohne Wert nach Hause schicken. Da traf Vaters Geldsendung ein. Der Begleitbrief begann mit dem ominösen Satz: \enquote{Fritz, es ist das erste und letzte Mal, dass ich Deine Schulden bezahle.}

Am folgenden Tag nahm ich fast wehmütig von Paris Abschied, stieg von der Métro auf die Gare du Nord um -- da zerbrach mein schwerer, hauptsächlich mit Büchern angefüllter Koffer in zwei Teile; Bücher und Kleidungsstücke lagen auf den Treppenstufen. Die Sache war peinlich: ich hatte weder Riemen noch Bindfaden bei mir. Würde ich wohl, wenn ich mich hilfesuchend von der Unfallstelle entfernte, nachher angesichts des Menschenstromes Klamotten und Bücher noch wiedersehen? Während ich dies bedachte, erschien der rettende Engel in Gestalt eines etwa 40 Jahre alten Eisenbahnbeamten: ich solle mich nicht von der Stelle rühren, er komme in wenigen Minuten wieder. Gesagt, getan. Er kam mit einem starken Riemen wieder, wir packten gemeinsam die Sachen ein, er schnürte den Koffer zusammen und wünschte mir mit freundlichem Händedruck \enquote{bon voyage}, nachdem er es abgelehnt hatte, wenigstens für den schönen Riemen eine Geldentschädigung anzunehmen.

Am nächsten Morgen kurz nach 7 Uhr lief der Schnellzug nach 9-stündiger Fahrt über das flammende Charleroi in Aachen ein; während der Grenzformalitäten nahmen die Reisenden ein im Wartesaal 1. Klasse schon bereitstehendes reichliches und gutes Frühstück für 80 Pf. ein.

In Köln machte ich 3 Tage Station, um mir diese alte \enquote{heilige} und damals noch heile und schöne Stadt anzusehen. Großartig der Dom mit seinen gewaltigen Ausmaßen und seiner künstlerischen Geschlossenheit, aber auch bald ermüdend infolge der vielhundertfachen Wiederholung des gleichen hochgotischen Motivs -- im Gegensatz etwa zu Straßburg. Von den anderen künstlerisch bedeutenden Kirchen befreundete ich mich ganz besonders mit St. Aposteln, dieser schönen, stilreinen romanischen Kirche aus dem Beginn des 11. Jahrhunderts. Im 2. Weltkrieg wurde Köln zu 72\% zerstört, darunter fast alle großartigen kirchlichen und weltlichen Bauten des Mittelalters von denen einige, darunter der Dom, St. Aposteln, Groß St. Martin, das Alte Rathaus mit dem Hanse-Saal und Gürzenich wiederhergestellt wurden. Die schönen alten Häuser am Rhein mit den vorgekragten, teils breiteren, teils engbrüstigen Giebeln, in deren einem ich Quartier gefunden hatte und das ganze schöne alte Stadtviertel unweit des Rheines sind verschwunden. Die meisten der wiederentstandenen Bauten stehen heute traurig und verloren da inmitten einer oft hypermodernen Zement Wüstenei -- mit der sie nichts gemein haben. 1958 sah ich Köln wieder -- ein erschütterndes Bild.

Ein Erlebnis war damals für mich die erste Begegnung mit Leibl im Wallraf-Richartz-Museum. Unvergesslich, mehrfach genossen, der Blick von Deutz auf dem anderen Rheinufer auf das alte Köln.

Ich war damals so begeistert von der deutschen, heimatlichen Atmosphäre Kölns, dass ich auf Ansichtskarten, die ich meinen französischen Freunden schickte, laut über das Wiedersehen mit der \enquote{Heimat} jubelte.

Am dritten Nachmittag fuhr ich mit der Straßenbahn in 40 Minuten in das schöne, ebenfalls ehrwürdige, dabei elegante Bonn und genoss vor allem den Blick auf den Rhein mit dem Siebengebirge im Abendsonnenschein.

\marginpar{140}
Mein kleiner Taschenführer empfahl seinen Lesern u.a. eines der alten kleinen, echtkölnischen Lokale in den winkligen Gassen unweit des Rheins aufzusuchen und dort \enquote{ne halwe Hahn un'n Glösge Köllsch}, d.h. eine Käsestulle und ein Glas Bier zu verlangen. Das tat ich vormittags, durchschritt den vorderen verrauchten Raum, in dem Klavier gespielt wurde, setzte mich in den hinteren aufs Sofa und bestellte die empfohlene Kölner Spezialität. Nach ein paar Minuten erschien der Kellner mit dem Gewünschten und einem jungen hübschen Mädchen, auf das er mit den Worten: \enquote{hier ist auch so'n halwe Hahn} hinwies. Sie setzte sich zutraulich zu mir aufs Sofa und bestellte sich ein 2. Frühstück. Sie plauderte reizend mit leicht schwäbischem Akzent von ihrer süddeutschen Heimat, die sie erst vor kurzem verlassen habe, kam auf ihre hiesige Tätigkeit zu sprechen, die der Liebe gewidmet war. Sie erzählte so beinahe unschuldsvoll-offen, dass sie, wenn sie abends einen Mann mit sich nehme, den sie nicht möge, sie sich \enquote{danach} umdrehe und die ganze Nacht durchschlafe; wenn sie aber einen habe, den sie gern möge, dann schlafe sie die ganze Nacht nicht! und dabei streiften mich sehr freundlich ihre schönen, fast unschuldsvollen Augen. Sie raffte ihr langes, damals noch fast bis zu den Knöcheln reichendes Kleid bis ans Knie, fuhr mit der Rechten bis weit oben hinein -- wobei mich wieder ein sehr freundlich ermutigender Blick traf, und zog ein Taschentuch hervor. Sie lud mich ein, abends nach 8 Uhr wiederzukommen, wir könnten dann zusammenbleiben, es sei denn, sie sitze schon mit einem anderen älteren Herrn zusammen, dann müsse ich eben an einem anderen Abend wiederkommen. Schließlich kassierte der Kellner ihr und mein Frühstück bei mir ab. Draußen, der verführerischen jungen Circe entronnen, stellte ich fest, dass dieses nette Plauderstündchen mich 6 normale Studentenmahlzeiten gekostet hatte.

\marginpar{142}
Weder Düsseldorf noch Essen lockten mich zum Besuch, dafür aber die Welfenresidenz Hannover, die damals noch einiges Sehenswerte aufzuweisen hatte, so außer Schloss und Park Herrenhausen, einem stattlichen Kunstmuseum, eine Anzahl spätgotischer Backsteinhäuser und schöne Sandsteinhäuser aus der Renaissance, dem 17. Jahrhundert vor dem 30-jährigen Kriege, ferner alte Fachwerkhäuser -- fast alles 1939-1945 der Vernichtung anheimgefallen.

Auch Magdeburg, die große geschichtsträchtige Stadt, nur 20~km von meiner Heimat Niegripp entfernt, fesselte mich zwei Tage. Noch standen damals stattliche Reste der alten Festungsmauer mit Glacis und zahlreiche alte Kirchen, soweit sie die fast völlige Zerstörung im 30-jährigen Krieg 1631 durch Tilly überlebt hatten; das ehrwürdige, künstlerisch bedeutsame Denkmal Kaiser Otto~I., des Gründers der Stadt, zierte den Alten Markt gegenüber dem Rathaus. Von all dem ist nach der furchtbaren Bombennacht vom April 1945 fast nichts übrig geblieben, und nur der stark beschädigte gewaltige Dom und die alte romanische Klosterkirche Unserer Lieben Frauen wurde wiederhergestellt. Vor dem 2. Weltkriege besaß Magdeburg auch ein sehr beachtliches modernes Kunstmuseum, in dem u.a. der Schöpfer heroischer Landschaften modernen Geistes, Bracht\footnote{Eugen Bracht, 1842-1921}, meine besondere Aufmerksamkeit erregte.

Nun verlangte es mich, meine alte Heimat, die Örtlichkeit und ihre Menschen wiederzusehen. Ich war ein paar Tage bei Lehrer Seehaus in Burg zu Gast, meinen einstigen Pensionseltern und Freund meiner Eltern. Mit meiner gleichaltrigen Gespielin Grete Seehaus wollte die alte Unbefangenheit nicht wieder aufkommen; sie schien untätig zu Hause zu sitzen und \marginpar{143} auf den Freier zu warten. Der kam auch wenige Jahre später in Gestalt des Sohnes eines Burger Goldschmiedes, während Paul Seehaus irgendwo als Bankangestellter tätig war. Ich mietete mir ein Fahrrad, fuhr zu der mit meinen Eltern befreundeten Gutsbesitzerfamilie Raue nach Schartau und vor allem nach Niegripp; es war ein sentimentaler Ausflug. Schon unterwegs traf ich meine frühe Gespielin Luise Mamai; sie war ein sehr hübsches Mädchen geworden. (Frau Kantor bemerkte etwas boshaft beim Kaffee, sie habe schon einiges hinter sich!) In 9 Jahren hatte sich vieles geändert: Schleusenmeister Zültz und Zolleinnehmer Wagner waren im Ruhestand und verzogen, Timmermann tot, Gustav Lüdde Tischlergeselle in Burg, mein Schulkamerad, Sohn des Fährmanns Boese, arbeitete als Gehilfe des Vaters. Emma Lüdde war die Frau eines Landarbeiters geworden; dieser sprang während des Besuchs etwa alle 5 Minuten auf einen Stuhl, griff nach der auf dem Schrank stehenden Schnapsflasche und tat einen Zug. Als ich 1931 wieder nach Niegripp kam, war Emma Sommerfeld Witwe.

Unsere einstige langjährige Hausangestellte Marie Tucher hatte einen ernsten, wortkargen Maurer geheiratet.

In Berlin übersprang ich nur einen Zug zu einer Stippvisite bei Martha Dehnicke.

Zu Hause fiel ich durch meine Lebhaftigkeit und die starke Beteiligung der Hände \marginpar{145}beim Reden auf. So stark und sichtbar war also die Einwirkung der fremden Umwelt auf mich.

Überall musste ich von Frankreich erzählen. Herr Bonfils, Direktor der Siegersdorfer Werke und jetzt Generaldirektor einer Gruppe gleichartiger Unternehmer, missbilligte die deutsche Außenpolitik: man müsse aus wirtschaftlichen Gründen eng mit Frankreich zusammenarbeiten und eine Union dieser beiden sich hervorragend ergänzenden Länder erstreben -- ein Gedanke, den ich im Winter 1958/59 in einer veränderten Welt von französischer Seite in Paris zu hören bekam.

Pastor Brückner, dessen nie langweiligen Sonntagspredigten man gelegentlich hörte, wetterte gegen \enquote{unsere lieben Vettern jenseits des Kanals} und gegen \enquote{die fetten Maden aus dem alten Testament} -- er meinte damit die jüdischen [Kaufleute?]. Seine Predigten überstiegen oft etwas den geistigen Horizont der Landbevölkerung, wie auch seine Bemühungen in dem Gemeindekirchenblättchen \enquote{kleines Glöckchen}, die Bauern in die Kant'sche Philosophie einzuführen, keinen Erfolg haben konnten. Er konnte in Gesellschaft und auch in Predigten eine verblüffende, gelegentlich als recht anstößig empfundene Offenheit an den Tag legen; so gestand er in einer Predigt über den Zweifel und Unglauben, dass er (im Amt!) zwei Jahre nicht an Gott geglaubt habe. Er machte aber auch Späße, die nicht selten das Missfallen mancher Amtsbrüder hervorriefen. Einmal saßen wir in der Altmann'schen Gaststube mit ihm beim Kaffee; er ließ dem Kaffee alsbald einen \enquote{Laubauer} Korn folgen. Da sprang er plötzlich auf und ging zum Fenster: \enquote{Da kommt ja ein lieber Amtsbruder!} Und scheinbar enttäuscht zurückkehrend: \enquote{es war der Schornsteinfeger.} Er nahm sich gern aus dem Gasthaus eine Flasche \enquote{Laubauer} mit nach Haus, um sich damit, wie er sagte, die Beine gegen den Rheumatismus einzureiben.

Eines Abends sprach ein Redner des als aggressiv bekannten Evangelischen Bundes. Er verteufelte den Katholizismus in einer Weise, die auch dem Predigtamtskandidaten Kölbing, den ich dort zufällig wiedertraf, zu starker Tobak war. Er vermochte dem Hetzredner auch theologische und historische Fehler nachzuweisen. Dies beleuchtet den damals oft sehr hochgespielten Gegensatz der beiden christlichen Konfessionen. Vor dem sogenannten \enquote{Kulturkampf} unter Bismarck lebten die beiden feindlichen Brüder weit friedlicher nebeneinander. In Nieder- und Mittelschlesien gab es aus der Zeit der Gegenreformation noch viele (fast) rein evangelische neben (fast) rein katholischen Gemeinden. So war Siegersdorf rein evangelisch, das nur 5~km entfernte Birkenbruck jedoch katholisch. Die alten Leute in Siegersdorf wussten noch zu berichten, dass in ihrer Kindheit gelegentlich der katholische Priester aus Birkenbrück seinen evangelischen Kollegen auf der Kanzel vertrat und dieser zuweilen dafür in Birkenbrück die Messe las.

Übrigens traf ich 1915 den Kandidaten des Predigeramtes Kölbing an der Ostfront als Telefonisten wieder. Er, zweifellos ein intelligenter Mensch, plagte sich mit dem Gedanken ab: wodurch hat sich das deutsche Volk so schwer versündigt, dass Gott ihm als Strafe diesen harten Zweifrontenkrieg auferlegt hat? Ich hörte im Winter 1918/19, nach dem Ende des ersten Weltkrieges aus dem Munde zweier Klassenkameraden, evang. Theologiestudenten, dass sie sich mit dem gleichen Problem herumgeschlagen hatten. Auch ihnen vermochte ich keine sie befriedigende Lösung zu ihrem Konflikt anzubieten. Mahlings Angebot, meinen Namen unter eine Glückwunschadresse zum 27.1. (Kaisers Geburtstag) an Wilhelm II. nach Doorn zu setzen, lehnte ich ab.

Übrigens überlebten alle vier geistlichen Klassenkameraden, drei evangelische und ein Katholik, den Krieg ohne schwere Verwundung.

Von meiner Tante Lisa von Rabenau eingeladen, verlebte ich in Reisicht\footnote{20~km von Bunzlau} angenehme Stunden. Die künstlerisch leicht talentierte Tante (sie musizierte und machte auch passable Verse, sie war Tochter eines polnischen Professors) bereitete mir einen netten musikalischen Empfang: Sie saß am Klavier, die 18-jährige hübsche Anneliese sang, ihre 5 Jahre jüngere Schwester Gretel geigte und der kleine 7-jährige Wolfgang rührte, den Blick fest auf die Noten gerichtet, haargenau und kräftig die Trommel. Es gefiel mir -- bzw. uns, d.h. Ella und mir -- bei ihnen immer gut, sofern der etwas nüchtern-prosaische Hausherr Karl von Rabenau nicht durch seine Anwesenheit das erfreuliche Niveau angenehmer Geselligkeit herabdrückte. Als sich später die Tante spürbar Hoffnungen machte, mich enger an eine der beiden Töchter zu fesseln, zog ich, der keine Neigung weder für Anneliese noch für Grete verspürte, \marginpar{149}mich allmählich zurück. Nach dem 1. Weltkriege verzogen sie nach Oberbayern, und wir haben den Kontakt mit ihnen verloren. Anneliese heiratete einen Herrn von Drebben und die anderen Familienmitglieder klagten über den Dünkel des Paares.


\section{Breslau}

Die beiden anstehenden Semester studierte ich an der Schlesischen Friedrich-Wilhelm-Universität gegr. 1810 zu Breslau.

Breslau, ausgestattet mit einigen hervorragenden gotischen Kirchen und Profanbauten aus der Zeit der Piastenherzöge, hatte eine weitere starke Prägung zur Zeit der Gegenreformation unter österreichischer Herrschaft erfahren. Matthiaskirche und Universität, ehemals\dots mit der herrlichen barocken Aula Leopoldina und dem Musiksaal.

Im Sommer 1919 war hier die Hundertjahrfeier der Befreiungskriege vom Napoleonischen Joch begangen. Zu diesem Zweck hatte man im Scheitinger Park den riesigen Kuppelbau mit einer Pergola und verschiedenen weiteren modernen, z.T. im kleinen die Kuppelform variierenden Ausstellungsbauten errichtet. Gerhard Hauptmann hatte für dieses Jubiläum ein Festspiel in Form eines Puppenspiels verfasst, das von flotten nationalistischen Gedanken à la Wildenbruch frei, vielmehr manche pazifistischen Passagen enthielt, was ihm die Gegnerschaft der Nationalisten und Alldeutschen eintrug, voran des Kronprinzen. Dieser kam zur Eröffnung der Feierlichkeiten im kaiserlichen Sonderzug; die Breslauer Bevölkerung umsäumte den vier Kilometer langen Straßenzug vom Hauptbahnhof bis zur Jahrhunderthalle, den der Kronprinz im Tempo 60 km/h -- entgegen der damals zugelassenen Geschwindigkeit von 30 km/h durchsauste, ohne den Menschenmassen besondere Aufmerksamkeit zu schenken.

\marginpar{150}
Den Juli über gastierte das Berliner Lessingtheater vor allem mit Stücken Ibsens und Gerhard Hauptmanns, von denen ich nur wenige versäumte. Spielleitung und Bühnenbilder waren ganz von naturalistischem Geist geprägt, die Interpretationen und die schauspielerischen Leistungen waren gut. In den Massenszenen des Festspiels beteiligten sich die Studentenschaft und die Kunstakademie, so auch meiner Freunde Kauf (Medizin) und der Kunstschüler Leo Loch, beide aus Neißen. Der verdienstvolle Münchener Simplizissimus brachte ein ganzseitiges Bild von Gulbansson, das G. Hauptmann auf einem Pfade im Park in einem Buch lesend darstellte, während ihm der ein Polopferd reitende Kronprinz einen Ball mitten ins Gesicht schlägt. Schon vorher hatte der Simplizissimus eine Aufforderung des redefreudigen Kaisers, die Schwarzseher (wegen Wilhelms Flottenpolitik u.a.m) sollten das Land verlassen, mit einem ganzseitigen Titelbild glossiert; rechts oben auf einem Hügel der Kaiser und wenige Getreue in rosigem Licht, während ungeheure schwarze Menschenmassen zum Meeresufer hinabwallen. Unterschrift: \enquote{die Schwarzseher verlassen das Land.}

Im Ausstellungspark der Jahrhundertfeier befand sich auch -- es war ja noch die Zeit des noch kaum angeschlagenen Kolonialismus -- befand sich auch ein echtes Kongonegerdorf mit etwa 100 Negern (im Familienverband), dessen Folklore man besichtigen konnte. Es war ein Stamm, dem man beachtliche körperliche Schönheit zubilligen musste. Am Eingang stand ein schöner, stattlicher Neger in weißer, damals auch in Europa im Hochsommer üblichen Bäderkleidung. Er hatte in Paris das Kunstgewerbe der Elfenbeinschnitzer erlernt, sprach fließend französisch. Ich unterhielt mich mehrfach mit ihm längere Zeit über folkloristische, religiöse und andere Fragen. Eines Tages bat er mich, eine größere Anzahl Briefe von Frauen an mich zu nehmen; er fürchte, dass seine Manager argwöhnisch werden könnten, wenn ihnen diese Briefe in die Hände fielen. Ich steckte sie ein und las sie zu Hause: alles Liebesbriefe, meist aus der gebildeten Damenwelt, darunter bekannte Namen des schlesischen Adels z.T. auch französisch und englisch abgefasst. Der Tenor reichte von \enquote{Nur eine Nacht mit Dir} bis \enquote{An Deiner Seite will ich mein ganzes Leben verbringen, ich bin sehr vermögend, Du brauchst nicht zu arbeiten, Du kannst bestimmen, wo wir glücklich sein wollen}, etc. z.T. mit Fluchtplänen!

Ist es verwunderlich, dass meine Gefühle gegenüber dem anderen Geschlecht skeptisch waren?\\

Die beiden anstehenden Semester studierte ich in Breslau. Ich hatte mich entschlossen, in den Lehrfächern Deutsch und Französisch die Oberstufe und in Englisch -- obwohl ich hier reiner Autodidakt war -- die zweite (mittlere) Stufe zu erwerben. Der Germanist Siebs und der Literarhistoriker Koch waren weithin bekannt, der erste u.a. durch seine \enquote{Deutsche Bühnenaussprache} und Mundartforschung, der zweite durch seine \enquote{Deutsche Literaturgeschichte der Neuzeit}. Seine Spezialität war der Motivvergleich. Prof. Appel, der Romanist, stand im Rufe besonders hoher Prüfungsanforderungen; sein Name war auch in Frankreich bekannt durch seine Forschungen über die provenzalische Sprache und Literatur. Große Anziehungskraft, besonders auf jüngere Semester, übte der Literaturkritiker Eugen Kühnemann aus. Er war ein eindrucksvoller, ja mitreißender Redner, der es verstand, den jungen Studenten für den Gegenstand seiner stets völlig frei, d.h. ohne schriftlichen Unterlagen, gehaltenen Vorlesungen zu begeistern. In seinen Vorträgen für Hörer aller Fakultäten behandelte er gern Werke großer Dichter und Denker, besonders solche des deutschen schöpferischen Geistes. Aber auch Werke Ibsens, Tolstois, Dostojewskis u.a. analysierte er fesselnd, darunter kaum eins, dem er nicht \enquote{unerhörte Tiefe} zubilligte. Brennpunkte seiner Weltanschauung waren Plato, Kant und der deutsche Idealismus. In den beiden vordersten Bankreihen saßen auffallend viele Studentinnen. Böse Zungen behaupteten, es gebe unter ihnen solche, die durch ihn nicht nur geistig befruchtet worden seien.

\marginpar{151}
Die Breslauer Universität bot auch viele andere gute Vorlesungen für Hörer aller Fakultäten an, die ein echtes \enquote{Studium generale} ermöglichten. Vorzüglich waren die Einführungen von Kautzsch in die Malerei des 19. Jahrhunderts, Landsbergs in die Kunst des Expressionismus, in die Kunststadt Venedig, William Sterns (später Hamburg) in die differenzielle Psychologie und Psychologie der frühen Kindheit u.a.

In eine Korporation trat ich nicht ein, schaute mich aber in der Organisation \enquote{Freie Studentenschaft} und ihre Arbeitsgruppen um und arbeitete im weiteren Verlauf hier und da mit, so im Asta \enquote{Allgemeinen Studenten Ausschuss} und als Leiter der studentischen Wohnungskommission. In dieser Eigenschaft berichtigte ich selbst einige Dutzende von Zimmern und lernte die Problematik der Studentenbuden kennen. Im ersten Drittel des 20. Jahrhunderts war ein großer Teil -- nicht nur der Studentenwohnungen -- in den europäischen Großstädten verwanzt, so auch meine erste Wohnung in der Kohlenstraße in Breslau, später in Berlin im alten Hansaviertel, ferner in den zwanziger Jahren in Paris, London, Grenoble (Hôtel d'Angleterre). Als ich einmal im Breslauer Universitätsviertel die Vermieterin vorsichtshalber fragte, ob sie \underline{viel} Ungeziefer habe, lautete die Antwort prompt: viel nicht!

Im Sommersemester lernte ich interessante, nicht alltägliche Studenten in der Wandergruppe und im Literaturzirkel der Freien Studentenschaft kennen. Ein Münchner Mediziner, Kraus, mit schriftstellerischer Neigung -- er hatte schon einiges in der Berliner Illustrierten veröffentlicht, ein Neuromantiker, machte im Sommersemester 1913 Station in Breslau, um dann 2 Semester in Moskau zu studieren. Er sagte, er habe geglaubt, Breslau sei Grenzstadt und so die Übergangsatmosphäre nach Russland. Seiner Anregung hatten wir den prächtig gelungenen Märchenabend auf der \enquote{Liebesinsel} bei Wilhelmshafen, dreiviertelstündige Dampferfahrt oberhalb Breslaus zu verdanken.

\marginpar{153}
Einer netten Pfingstwanderung zu viert im Eulengebirge folgten in den Sommerferien, in der ersten Augusthälfte, eine romantisch angehauchte Wanderung zudritt mit dem Theologiestudenten Schulz und dem Mathematiker und Musiktheoretiker Worbs von Meißen über Dresden, die Sächsische Schweiz, die Elbe aufwärts bis Leitmeritz und dann das letzte Stück der Reise bis Prag mit der Bahn, durch ausgedehnte Hopfenfelder. Wir übernachteten in den damals noch verhältnismäßig neuen und noch wenig frequentierten Jugendherbergen. Als wir in der Meißener Gegend keine Jugendherberge fanden, versuchten wir, die Nacht in einem Bretterhaufen liegend zu verbringen; der Versuch misslang. Bald schmerzte auf dem harten Lager der Rücken, und wir begaben uns zum nächsten Bahnhof, wo wir die Nacht im Wartesaal sitzend bis zum ersten Personenzug nach Dresden verbrachten. Wir drei kunst- und kulturgeschichtsbeflissenen Jünglinge genossen 3 Tage lang Dresden, vor allem natürlich die Galerie, das schöne Elbufer mit der Brühlschen Terrasse, mit seinen Bauten und Sammlungen und fuhren dann streckenweise mit dem Dampfer stromaufwärts, durchstreiften die schönsten Gegenden des Elbsandsteingebirges und der angrenzenden böhmischen Gebirgszüge. Stromaufwärts von Aussig fuhren wir natürlich im Kahn über die Elbe zum Schreckenstein -- allerdings hatten die Fahrgäste nicht mehr das romantische Gepräge wie auf Ludwig Richters berühmten Bild. Oft sangen wir zum Gitarren-Schrumschrum unseres Musiktheoretikers Worbs, mit dem es manche Kunstdiskussionen gab, Lieder aus dem Zupfgeigenhansel, der natürlich in unserem Rucksack nicht fehlte.

Eines Sonntagnachmittags besuchten wir einen Dorfschwoof in der Nähe von Aussig. Die Mädchen waren differenziert in Tracht gekleidet, z.T. ähnlich der Dirndl-Tracht, die später in den 20-er Jahren Mode war. Man tanzte beim Walzer, der vorherrschend war, Backe an Backe (sic), d.h. die Mädchen legten beim Tanz ihre linke Backe (sic) auf die rechte des Tänzers -- ich fand es ganz lieblich. Nur ein Mädchen war städtisch gekleidet und schloss sich zögernd und auf mein Zureden diesem netten Brauch an. Wir tanzten viel und kamen nett ins Gespräch, dass wir unsere Adressen austauschten und noch bis zum Ausbruch des 1. Weltkrieges öfters Kartengrüße austauschten.

Die schöne alte Kaiserstadt Prag, herrlich an der Moldau gelegen, mit dem gewaltigen hochragenden Hradschin und seinen für Deutsche und Tschechen so bedeutende historische Erinnerungen und Denkmälern, beeindruckten mich stark. Es war nächst Paris die schönste Stadt, die ich bisher gesehen hatte. Das deutsche Kunst- und Kulturleben konzentrierte sich damals um die Karlsuniversität, das Neue deutsche Theater am Wenzelplatz und das Kleine (Kammer) Theater Graben~26.

In der tschechischen Oper sahen wir eine schöne, temperamentvolle Aufführung von Carmen. Auch die junge tschechische Kunst im -- ich glaube -- Manes Museum fand unsere Beachtung. Alle Tschechen sprachen in Prag auch deutsch, man gab uns freundlich Auskunft, mehrfach mit dem Zusatz: \enquote{ah, man merkt, dass Sie Reichsdeutsche sind!} Darin versteckte sich der jahrhundertalte nationale Gegensatz zwischen Tschechen und Deutschböhmen, die ihre slawische Landsleute meist mit Herablassung behandelten.

Übrigens wohnten wir bequem in der deutschen Jugendherberge im Zentrum Prags. Wir fuhren mit der Bahn nach Haus, meine beiden Kameraden über Dresden, ich über Reichenberg-Polaun-Hirschberg-Löwenberg.

Übrigens hatte ich, um die Kolleggelder gestundet zu bekommen, wie immer, am Semesterende mich Diligenzprüfungen unterzogen, u.a. hatte mich der französische Lektor auf meinen Wunsch einer 1-stündigen neufranzösischen Grammatik- und Literaturprüfung unterzogen und mir bescheinigt, dass mein Wissen und Können den Durchschnitt erheblich übertraf und hinter dem Können der sechsten Semester nicht zurückstehe.

Während der Universitätsferien besuchte mich mein Freund Menzel; wir machten Radtouren bis ins Isergebirge, und einmal nahm uns der schon erwähnte Direktor Bonfils im Auto mit nach Haynau und Liegnitz. Wir fuhren, wie damals üblich, im 40-km Tempo durch die Landschaft, hinter uns eine große Staubwolke zurücklassend. Das war damals für uns immerhin ein beachtliches Erlebnis; u.a. stellten wir fest, dass wir beim Autofahren in größerem Umfange als beim Radfahren das Relief der Landschaft körperlich zu spüren bekamen.

\marginpar{156}
Die kunsthistorische Vorlesung \enquote{Venedig} sowie meine frühere Beschäftigung mit Feuerbach und Böcklin hatten in mir den Wunsch nach einer Italienreise im kommenden Sommer 1914 keimen lassen. Also trieb ich Italienisch anhand eines französischen Lehrbuches der italienischen Sprache, zumal dies auch meinem romanischen Hauptfach zugute kam. Der Erfolg war nicht schlecht, denn als ich im Wintersemester in Breslau bei der italienischen Privatlehrerin Adelina Marencci Konversationsstunden nahm, war diese nur schwer davon zu überzeugen, dass ich noch nicht längere Zeit in Italien gewesen war; insbesondere bezeichnete sie meine Aussprache als durchaus rein. Das konnte ich damals immerhin als eine Leistung werten, da ich Autodidakt war, keine italienischen Lehrer geschweige denn Radio, das es damals ja nicht gab, zur Korrektur hatte.\\

Die politischen Spannungen schienen mit der Beendigung der beiden Balkankriege durch die Friedensschlüsse von London im Mai 1913 im wesentlichen behoben. Es war Deutschland gelungen, seinen österreichischen Verbündeten vom bewaffneten Eingreifen auf dem Balkan abzuhalten. Russland hatte sein Nahziel erreicht: die Türkei war aus dem weitaus größten Teil des Balkans vertrieben, die südslawischen \enquote{Brüder} befreit worden. Nicht erreicht war das jahrhundertealte Ziel Russlands, die Gewinnung der Dardanellen und Konstantinopels infolge des Widerstandes der anderen Großmächte. Russland hielt den Augenblick, den Kampf um Konstantinopel zu beginnen, noch nicht für gekommen. Pflegte man früher in Petersburg zu sagen: \enquote{Der Weg nach Konstantinopel führt über die Wiener Hofburg}, so erfuhr 1908 diese Formel eine Änderung. In jenem Jahr hatte Wien Bosnien und die Herzegowina annektiert, was von Petersburg mit lautem Säbelgerassel beantwortet worden war. Damals sandte Kaiser Wilhelm II. an Kaiser Franz Joseph das berühmte Telegramm: \enquote{Ich stehe mit meiner gesamten Streitmacht hinter Ew. Majestät.} In Petersburg wurde es stiller und die neue Formel der Kriegspartei lautete: \enquote{der Weg nach Konstantinopel führt durch das Brandenburger Tor und die Wiener Hofburg}, ein bedenklich weiter Umweg, zumal sein Verbündeter Frankreich keinen Krieg wegen der Dardanellen wünschte. Die freundschaftlichen Bande, die Deutschland mit der Türkei geknüpft hatte, waren durch den italienischen Raubkrieg gegen die Türkei vom Jahre 1911 und durch die Balkankriege nicht erschüttert, sondern eher gefestigt worden. Der Bau der von Deutschland im Einvernehmen mit der Türkei finanzierten Bagdadbahn erregte das Misstrauen der beiden westlichen Großmächte. Bedenklicher schien es manchen, dass es im Dreibundgebälk wegen der immer stärker werdenden Irredenta-Bewegung an der Adria, in Oberitalien und Südtirol knisterte. Ein kaum verstecktes gegenseitiges Misstrauen zwischen der Donaumonarchie und Italien machte den Bündniswert des dritten Partners im Dreibund für den Ernstfall -- wie sich auch bald erwies -- problematisch.

Trotzdem schien die balance of power nicht ernstlich gefährdet und man sprach nicht vom Krieg -- trotz mancher unbedachten Worte des Kaisers und besonders des Kronprinzen, so. z.B. Deutschland werde noch um den ihm unter der Sonne gebührenden Platz kämpfen müssen u.ä. Niemand jedoch dachte an Krieg.

Durch meinen Freund Rogier aus Bromberg lernte ich die Ziele des Humbolt-Vereins kennen: die Kluft zwischen Gebildeten und den Arbeitern überbrücken durch kostenlose geistige Förderung der Arbeiter in Abendlehrgängen. Diese Lehrgänge, die sich auf die verschiedensten Gebiete erstreckten, wurden unentgeltlich und streng ehrenamtlich in wöchentlich 1-2 Doppelstunden erteilt, vorwiegend von Studenten und jungen akademisch gebildeten Beamten oder freien Berufen (Pädagogen, Ärzten, Juristen, Künstlern, etc.). 1-2 mal monatlich schloss sich an den Unterricht ein geselliges Beisammensein bei einem Glase Bier.

Allerdings musste ich in Breslau sowohl wie im Sommersemester in Neukölln feststellen, dass wir unter unseren Hörern nur verschwindend wenige echte \enquote{Handarbeiter} und Fabrikarbeiter hatten -- kein Wunder in einer Zeit, in der die Arbeitszeit von 6 Uhr morgens bis 6 Uhr abends dauerte und sie dann weder körperlich noch geistig leistungsfähig waren.

Der Gedanke des Humboldvereins wurde dann nach dem 1. Weltkrieg vom Staat (Volkshochschule) und den Gewerkschaften (Arbeiterbildungsverein etc.) übernommen.

\marginpar{160}
Im Winter verkehrte ich u.a. mit dem Kunstmakler und Graphiker Leo Loch, besuchte mit ihm und seiner Freundin das urgemütliche Künstlerlokal \enquote{Mönchskeller} am Dom, wurde gegen Semesterende Zimmernachbar Lochs am alten Martinikirchlein im Schatten der gewaltigen gotischen Doppel\--Kreuz\-kirche, in der schönen stillen Oase der Dominsel inmitten des Großstadtgetriebes. Die obere und Hauptkirche barg u.a. das wundervolle Grabdenkmal Heinrichs IV., des Minnesängers. An einem der Türme bemerkte man hoch oben einen ehernen Adler -- es war der polnische Orzet biaty aus alter Zeit. Vom Haupteingang der Kreuzkirche -- ihr gegenüber befand sich ein schöner Profanbau in maßvollem Barockstil, ein Stift für arme schlesische adlige Damen -- führte die schöne harmonische klassizistische Domstraße zum Hauptportal des alten Doms; rechts von ihm erstreckte sich der ebenfalls klassizistische Palast des Fürstbischofs, von dessen Garten, flankiert von der Oder, man einen sehr schönen Blick auf die vieltürmige Altstadt Breslaus hatte. Eine englische Quelle aus dem Ende des 18. Jahrhunderts bezeichnete Breslau als \enquote{well-churched}.

Hier, im Palast des Fürstbischofs -- sofern er nicht bei den feudalen Breslauer Kürassieren abstieg -- pflegte Wilhelm II. bei seinen Besuchen in Breslau zu übernachten. Kein Wunder, dass die Vatikan-freundliche Haltung des Kaisers in päpstlichen Verlautbarungen besonders durch Pius XII. gut honoriert wurde. Die Breslauer Bürgerschaft dagegen mit dem Oberbürgermeister an der Spitze wurden von Seiner Majestät ignoriert.


\section{Berlin}

Jetzt lockte die Reichshauptstadt, das große weithin ausstrahlende Kunst- und Kulturzentrum mit den großen Museen und Theatern, den Wissenschaftlern und Künstlern von Rang. Auch die studentische Vereinigung der \enquote{Freischar}, von der ich einzelne Vertreter bzw. \enquote{Verkehrsgäste} in Breslau kennengelernt hatte, zog mich an.

Ich bezog Quartier in der Lessingstraße im alten Hansaviertel, 3 Minuten vom Stadtbahnhof Bellevue am Rande des Tiergartens entfernt. Die Stadtbahn wurde damals noch mit Dampf betrieben, hatte eine 2. (Polster) und eine 3. (Holz) Klasse. Die Züge fuhren dicht und ziemlich flott, hielten nur kurz, und man stieg oft in das nächste Abteil ein. Couleurstudenten fuhren 2. Klasse; gerieten sie in eine 3. (Holz-) Klasse, auch auf der Straßenbahn, so durften sie sich gemäß Comment nicht setzen. Einmal stiegen zwei Couleurstudenten, ein Bursche mit seinem Leibfuchs in letzter Sekunde in mein Abteil 3. Klasse ein. Der Leibfuchs trat dabei einem der Fahrgäste auf den Fuß und entschuldigte sich, die Mütze ziehend, höflich: \enquote{Verzeihen Sie gütigst!} Der Bursche zog ihn ans Fenster und belehrte ihn: \enquote{So sagt man in der 2. Klasse; in der 3. Klasse heißt es einfach \enquote{hoppla}!}

Auch diese meine Studentenbude in Berlin war nicht wanzenfrei; in der warmen Jahreszeit sorgten die Tiere für zahlreichen Nachwuchs; ich wurde nicht selten in der Nacht von ihnen geweckt und ging auf Jagd; die Strecke\footnote{Strecke = die Anzahl der erlegten Tiere} war dann beträchtlich.

Meine Vorlesungen und Seminare fanden sämtlich im Hauptgebäude der Universität Unter den Linden statt, das Tor zum Vorplatz war flankiert von den Denkmälern der beiden Brüder Humboldt; einige Institute, so das Romanische, waren gegenüber im Annex des Alten Palais untergebracht. In den 80er Jahren des 19. Jahrhunderts wurde noch jeden Freitag aus diesem Nebengebäude eine Badewanne in das Alte Palais getragen, damit Kaiser Wilhelm I. seine wöchentliche Gesamtreinigung vornehmen konnte. Zwischen Universität und Altem Palais stand zwischen den letzten Linden das große, eindrucksvolle Denkmal Friedrichs des Großen.

Ich besuchte regelmäßig das mittelhochdeutsche Proseminar bei Gustav Roethe, dem Nachfolger auf den Lehrstuhl Erich Schmidts. Als solcher hatte er sich verpflichten müssen, er, der ein entschiedener Gegner des Frauenstudiums war, in Vorlesungen und Seminaren auch Frauen aufzunehmen. So musste er in seinem Proseminar im Sommer 1914 auch drei Frauen -- unter ca. 45 männlichen Hörern -- ertragen. Was tat der Kämpe Roethe? Er machte, wo der mhd.\footnote{mhd. = mittelhochdeutsch} Text nur dazu Gelegenheit bot, anzügliche Bemerkungen, so dass die Studentinnen erröteten und im Laufe von 14 Tagen eine nach der anderen auf die weitere Teilnahme an Roethes Proseminar verzichteten. So etwas war also noch im Sommer 1914 möglich!

Der führende Romanist war Prof. Morf, ein Schweizer. Ich besuchte sein überfülltes Proseminar -- ca. 50 ständige Mitarbeiter. Wir studierten den altfranzösischen Urtext des Parsifal von Chrétien de Troyes. Die altfranz. Sprachkenntnisse hatte ich mir autodidaktisch anhand von Voretzsch \enquote{Einführung in das Altfranzösische} angeeignet. Bei dem Literaturhistoriker Hermann hörte ich eine Vorlesung über Wagner als Dichter, bei Frischeisen-Köhler Philosophie der Kunst, bei Erdmann die Einführung in die Psychologie und bei dem damaligen Privatdozenten Erhard Lommatzsch \enquote{Interpretationen ausgewählter Texte aus Dantes \enquote{Divina Commedia}}.

Ich hospitierte noch gelegentlich bei einer Anzahl wissenschaftlicher Größen, wie dem großen Altphilologen von Milamowitz-Moellendorf, dem schon senilen 85-jährigen Philosophen und wagte mich in das Proseminar des Kunsthistorikers Goldschmidt.

Ich wurde \enquote{Verkehrsgast} in der studentischen \enquote{Freischar} -- als Mitglied konnte man erst nach 1-2 semestriger Probezeit aufgenommen werden. Es war eine sehr geistige, anregende Atmosphäre; wir hatten sehr interessante Diskussionen, z.B. im Anschluss an einen ausgezeichneten Vortrag über Damaschkes Bodenreform, über Zionismus und Judenproblem u.a.

Die Sonntage bzw. Wochenenden verliefen ähnlich wie beim Wandervogel; annähernd 15\% der Mitglieder und Verkehrsgäste waren schon Studentinnen. Herrlich waren die sonntäglichen Wanderungen; fast stets wurde an einem See gelagert und gebadet. Einmal waren wir bei den weltanschaulich und ernährungsmäßig fortschrittlichen Obstzüchtern in Oranienburg mittags zu Gaste. Unter Absingen des Liedes (s. Zupfgeigenhansl) \enquote{Steckt an den Schweinebraten} wurde ein leckres vegetarisches Mahl auf den Rasen gebracht: Reis mit Tomaten, die ich hier meines Wissens nach zum ersten Mal gegessen habe, und als Getränk Obstsaft. Für die Pfingstferien plante ich anhand von Fontanes \enquote{Wanderungen durch die Mark Brandenburg} eine \enquote{Fahrt} gegen Norden bis hinein nach Mecklenburg. Zur Vorbereitung las ich im riesigen Kuppelsaale der Staatsbibliothek fleißig die plattdeutschen Dichtungen Fritz Reuters. Da ich für meine Marschroute keinen Gefährten fand, wanderte ich allein; ich habe es nicht bereut.

Die Fahrt führte mich nach Rheinsberg, Müritzsee mit Mirow, Röbel und Waren, nach Stevenhagen und Neubrandenburg. Ich übernachtete meist bei Bauern; zweimal konnte man mir abends die Dorfchronik zu lesen geben.
Ich bemühte mich, niederdeutsch zu sprechen, womit ich allerdings nicht immer Erfolg hatte. Im Norden der Müritz erzählte ich den jungen Bauersleuten in Reuters Platt: die Frau verstand mich, der Mann ließ sich aber mein Platt in seine Dorfmundart auf die Frage \enquote{Wat seggt de Jung?} übersetzen.

\marginpar{167}
Ein ganz großes Erlebnis war für mich Reinhardts Bühnenkunst im Deutschen Theater. Ich sah vor allem zahlreiche Stücke Shakespeares, vom Stehparkett aus, das damals 3 Mark kostete, was dem Normalpreis von 6 soliden Studentenmahlzeiten entsprach. \enquote{Was ihr wollt} \enquote{erstand} ich dreimal hintereinander -- es war mein größtes Theatererlebnis überhaupt, das mich zu heller Begeisterung hinriss. Fast jeden Abend war die kaiserliche Loge besetzt mit Mitgliedern der kaiserlichen Familien: Geltung hatte der Satz: Pünktlichkeit ist die Höflichkeit der Könige. Zwei Minuten vor 7 Uhr, im Augenblick, wo der oder die Prinzen sämtlich mit Gefolge in Galauniform in der Loge platzgenommen hatte, ging der Vorhang hoch.

Am Nachmittag des 28. Juni schlug die durch Extrablätter gemeldete Nachricht von der Ermordung des österreichischen Thronfolgers in Sarajewo wie eine Bombe ein. Trotzdem wollte noch niemand an einen bevorstehenden Krieg glauben. Offenbar auch die Regierung nicht: der Kaiser trat die gewohnte Nordlandreise auf S.M. Hohenzollern an. Das Leben nahm, mindestens scheinbar, das studentische Leben tatsächlich, seinen gewohnten Fortgang.
Am 16. Juli war ich abends mit sechs anderen Studenten der Romanistik bei Dr. Lommatzsch in seiner Grunewaldwohnung zu Gaste. Es gab zu belegten Brötchen Bier. Nach langer lebhafter Unterhaltung stieg ich mitternachts auf der Heimkehr schon am Bahnhof Tiergarten aus, um noch etwas Luft zu schöpfen. Mich überholte im Tiergarten eine Gestalt, die mir im Lampenlicht irgendwie bekannt schien. Ich beschleunigte meine Schritte, holte sie ein und erkannte in ihr einen Bekannten aus Dijon, den serbisch-jüdischen Belgrader Studenten namens Deutsch. Lange gingen wir im Park auf und ab. Es gab nur \underline{ein} Thema: Krieg. Deutsch setzte mir den Gang der Dinge auseinander, wie er auch tatsächlich eintraf: auf österreichischer Seite zu hohe Sühne- und Garantieforderungen an Serbien; in Belgrad verletzter Nationalstolz mit russischer Rückendeckung, in 14 Tagen Krieg zwischen Österreich und Serbien-Russland. Und nun das enge Bündnis Deutschland-Österreich und Russland-Frankreich? Ich hoffte auf Berlins mäßigenden Einfluss auf Wien. Deutsch zweifelte daran. \enquote{Wir werden uns nicht wiedersehen; ich kehre nächste Woche nach Belgrad zurück. Leben Sie wohl!}

Erst um 3 Uhr morgens betrat ich meine Studentenbude im alten Hansaviertel. Indessen wuchs die politische Spannung, die Extrablätter häuften sich: Seine Majestät bricht Nordlandreise ab -- Berlins mäßigender Einfluss in Wien -- Österreichisches Ultimatum an Serbien -- Kaiser in Berlin eingetroffen -- Mobilisierung der österreichischen Armee -- Jetzt begannen nationalistische Massenkundgebungen in Berlin. Mengen sangen in den Abendstunden vor dem Dom: \enquote{Ein' feste Burg ist unser Gott} -- militärische Kundgebungen vor dem Schloss. --

In den letzten Julitagen bestieg ich, wie jede Woche einmal, gegen 19:30 Uhr am Bahnhof Friedrichstraße das Oberdeck des Omnibusses No. 10 nach Neukölln, wo ich im Humboldtverein zwei Stunden deutsche Literatur unterrichtete. (Lektüre: Mörikes \enquote{Mozart auf der Reise nach Prag}.) Das Oberdeck war damals, genau wie in London noch bis Ende der 20er Jahre, völlig offen. Große Leute gingen mit eingezogenem Kopf bzw. gebückt auf ihren Platz aus Angst, an die Starkstromleitung zu stoßen. Auf der Rückfahrt rief der Schaffner: \enquote{Alles aussteigen!} \enquote{Was ist los?} \enquote{Unter den Linden ist eine Schlacht im Gange.}

Ich arbeitete mich durch die in der Friedrichstraße angestauten Massen bis zum Café Bauer, um in der Menschenmasse von berittener Polizei zurückgedrängt zu werden: Sozialdemokraten, einige Gewerkschaften und antimilitaristische Gruppen waren vom Brandenburger Tor bis über die Kreuzung Linden-Friedrichstraße vorgedrungen und wurden von einem starken Aufgebot berittener Polizei abgedrängt und schließlich von den \enquote{Linden} vertrieben. Der \enquote{Vorwärts} wurde am nächsten Morgen beschlagnahmt.

Am 30. Juli war ich zum letzten Mal in der Universität, bei Erdmann, Psychologie. Nur ca. 1/3 der Hörer war im Auditorium. Erdmann: \enquote{Meine Herren, nach den mir vorliegenden Informationen besteht kein Anlass zu ernster Besorgnis (!!).} Und draußen Unter den Linden jagte ein Extrablatt das andere: \enquote{Des Kaisers Telegramm an den Zaren}, \enquote{In Russland allgemeine Mobilmachung}, \enquote{Monitoren der Donaumonarchie beschießen Belgrad}, \enquote{Kriegserklärung Russlands an Österreich}, enquote{Ultimative Anfrage an Frankreich} (Sie lautete: Was gedenkt Frankreich im Falle eines deutsch-russischen Krieges zu tun?), bis zum Extrablatt am Abend des 31.7.: \enquote{Allgemeine Mobilmachung}, nachdem schon um 17 Uhr von einem Auto des Generalstabes aus ein Offizier, mit weißem Taschentuch winkend, \enquote{Allgemeine Mobilmachung} gerufen hatte. Es folgten die Kriegserklärungen Bethmann-Hollwegs an Russland und Frankreich -- und das Extrablatt \enquote{Jaurès in Paris ermordet.}

Am 1. August 1914 mittags stand ich inmitten einer riesigen Menschenmenge vor dem kaiserlichen Schloss, um die durch Extrablätter angekündigte Ansprache Wilhelms II. zu hören. Seine Majestät betrat den Balkon des 2. Stockwerkes. Hurrahrufe. Dann atemlose Stille. Der Kaiser sprach ohne Lautsprecher (in Berlin \enquote{Flüstertüte} genannt.) Letzter Satz in der Rede -- mit entsprechender Geste: \enquote{Und jetzt wollen wir sie dreschen!} -- Kein einziger störender Zwischenruf. Jeder war überzeugt, dass wir Deutschen das Opfer einer Aggression sind, dass wir einen gerechten Verteidigungskrieg führen.

Den Nachmittag verbrachte ich in Charlottenburg bei Tante Martha Dehnicke und ihrem Mann, den sie stets \enquote{Hänschen} anredete. Man sprach nur vom Kriege und seinen Aussichten. Am 2. August wollte ich nach Hause fahren -- unmöglich, pausenlos rollten die Truppentransporte. Am 3. August fuhr ich, meist in einem entsetzlich vollgepfropften Zuge vom Görlitzer Bahnhof aus in ca. 8 Stunden nach Siegersdorf. Am nächsten Tag, dem 4. August, kam die Kriegserklärung Englands. Mein Vater war darüber sehr bedrückt; viele empfanden das gleiche. Am folgenden Tag fuhr ich auf der eingleisigen Strecke von Siegersdorf nach Hirschberg, um mich bei dem dort stationierten Jägerbataillon, bei dem einst mein Vater gedient hatte, als Freiwilliger zu melden. In Löwenberg stieg mein einstiger Klassenkamerad Martin Richter zu, in der gleichen Absicht. Während er angenommen wurde, lehnte man mich nach Prüfung meiner Augengläser (-2,5, -3,0 Dioptrien) ab.

